\renewcommand{\theequation}{\theenumi}
\begin{enumerate}[label=\thesection.\arabic*.,ref=\thesection.\theenumi]
\numberwithin{equation}{enumi}
	
\item In $\triangle AME \cong \triangle AXD$  
\begin{enumerate}
\item $\angle DAX = \angle EAM $------Same angle
\item $\angle AME = \angle AXD $------Right angles
\item  $\angle AEM = \angle ADX $------Corresponding angles
\end{enumerate}
%
By AAA criteria for similarity of triangles,
 $$\triangle AME \sim \triangle AXD$$
\item  In similar triangles, Corresponding sides are in equal ratio.
\begin{align}
\frac{AD} {AE} = \frac{AX} {AM}
\\
\implies \frac {AE}{ED} = \frac {AM} {MX}
\label{eq:sim1}
\end{align}
Similary
$$\triangle BNF \sim \triangle BYC$$\\

\begin{align}
 \implies \frac {BF}{FC} = \frac {BN} {NY}
\label{eq:sim2}
\end{align}

\item From \eqref{eq:constr_a}, \eqref{eq:constr_m} and \eqref{eq:constr_n},
%
%
\begin{align}
\brak{\vec{A}-\vec{M}}^T
\brak{\vec{M}-\vec{N}} &= \myvec{0 & (h-k)}\myvec{-a \\ 0} = 0
\\
\implies MA \perp MN
\label{eq:perp1}
\end{align}
%
Similarly from \eqref{eq:constr_b}, \eqref{eq:constr_n} and \eqref{eq:constr_m},
%
%
\begin{align}
\brak{\vec{B}-\vec{N}}^T
\brak{\vec{N}-\vec{M}} &= \myvec{0 & (h-k)}\myvec{a \\ 0} = 0
\\
\implies NB \perp NM
\label{eq:perp2}
\end{align}
%


\item From \eqref{eq:constr_a}, \eqref{eq:constr_b}, \eqref{eq:constr_m} and \eqref{eq:constr_n},
\begin{align}
\brak{\vec{B}-\vec{A}} &= \myvec{a\\ 0}
\\
\brak{\vec{N}-\vec{M}} &= \myvec{a \\ 0}
\\
\implies AB \parallel MN
\label{eq:parallel1}
\end{align}
%
Similarly from \eqref{eq:constr_a}, \eqref{eq:constr_b}, \eqref{eq:constr_m} and \eqref{eq:constr_n},
\begin{align}
\brak{\vec{M}-\vec{A}} &= \myvec{0\\ \frac{h}{2}}
\\
\brak{\vec{N}-\vec{B}} &= \myvec{0 \\ -\frac{h}{2}}
\\
\implies AM \parallel BN
\label{eq:parallel2}
\end{align}
%

\item From \eqref{eq:perp1}, \eqref{eq:perp2},\eqref{eq:parallel1},\eqref{eq:parallel2}

Quadrilateral ABCD is a rectangle. Also, In a rectangle opposite sides are equal
\begin{align}
AM = BN
\\
MX = NY
\label{eq:rect}
\end{align}

\item From \eqref{eq:sim1}, \eqref{eq:sim2},\eqref{eq:rect}
\begin{align}
\frac {AE} {ED} = \frac {AM} {MX} = \frac {BN} {NY}  = \frac {BF} {FC}
\\
\implies 
\frac {AE} {ED} = \frac {BF}{FC}
\label{eq:soln}
\end{align}
Hence proved
\end{enumerate}
