%Code by GVV Sharma
%December 7, 2019
%released under GNU GPL
%Proof of Baudhyana Theorem


\begin{tikzpicture}
[scale=2,>=stealth,point/.style={draw,circle,fill = black,inner sep=0.5pt},]

%Triangle sides
\def\a{4}
\def\c{3}
\def\b{sqrt(\a^2+\c^2)}

%Trigonometric ratios
\def\ct{\a/\b}
\def\st{\c/\b}

%perp distance
\def\r{\a*\st}

%Section Ratio
\def\k{1.2}

%Labeling points
\node (A) at (0,\c)[point,label=above left:$A$] {};
\node (B) at (-\a, 0)[point,label=below right:$B$] {};
\node (C) at (0, 0)[point,label=below left:$C$] {};

%Foot of perpendicular
\node (D) at ($({-\r*\st}, {\r*\ct})$)[point,label=above left:$D$] {};
%Coordinates of point E (foot of perpendicular DE on BC)
\coordinate (E) at ($(B)!(D)!(C)$);
%Labeling point E
\node [point,label=below right:$E$] at (E) {};


%Drawing triangle ABC
\draw (A) -- node[left] {$\textrm{c}$} (B) -- node[below] {$\textrm{a}$} (C) -- node[above,xshift=2mm] {$\textrm{b}$} (A);

%Drawing perpendicular DE
\draw[dashed] (D) -- node[right] {} (E);

%Adding label for DE
\node [right] at ($ (D)!0.5!(E) $) {$ y $};

%Adding label for CE
\node [below] at ($ (C)!0.5!(E) $) {$ x $};


%Drawing and marking angles
\tikzset{my angle/.style={fill=#1!40, size=0.5cm, mark=}}
\tkzMarkRightAngle[fill=blue!20,size=.2](A,C,B)
\tkzLabelAngle[pos=0.65](C,B,A){$\theta$}
\tkzMarkAngle[fill=orange!40,size=0.5cm,mark=](C,B,A)
\end{tikzpicture}


