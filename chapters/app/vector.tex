%\renewcommand{\theequation}{\theenumi}
%\begin{enumerate}[label=\arabic*.,ref=\theenumi]
\begin{enumerate}[label=\thesection.\arabic*.,ref=\thesection.\theenumi]
\numberwithin{equation}{enumi}
	\item A {\em matrix} of the form 
\begin{align}
  \vec{A} \triangleq  \myvec{a_1\\a_2} 
\end{align}
		is defined be  {\em column vector}, or simply, vector.
	In 
\figref{fig:tri_right_angle}	
		the point vectors $\vec{A}, \vec{B}, \vec{C}$ can be defined as
\begin{align}
\vec{A} = \myvec{0 \\ c},\,
\vec{B} = \myvec{0 \\ 0},\,
\vec{C} = \myvec{a \\ 0}
% \norm{\vec{D}} \define \sqrt{\vec{D}^{\top}\vec{D}}, 
%\label{eq:tri_norm_def}
\end{align}
\item 
\begin{align}
  \lambda\vec{A} \triangleq  \myvec{\lambda a_1\\ \lambda a_2} 
\end{align}
\item For 
\begin{align}
	\vec{B} &= 	\myvec{b_1 \\ b_2},
	\\
	\vec{A}+ 
	\vec{B} &= \myvec{a_1 \\ a_2}+	\myvec{b_1 \\ b_2} = 
	 \myvec{a_1 +b_1 \\ a_2 +b_2} 
\end{align}
  \item  The transpose of $\vec{A}$ is the {\em row vector} defined as
\begin{align}
  \label{eq:transpose2d}
  \vec{A}^{\top}  = \myvec{a_1 & a_2}
\end{align}
%
\item The {\em inner product} or {\em dot product} is defined as
  \label{prop:dot2d}
\begin{align}
  \label{eq:dot2d}
  \vec{A}^{\top} \vec{B} &\equiv \vec{A} \cdot \vec{B} 
  = \myvec{a_1 & a_2} \myvec{b_1 \\ b_2}= a_1b_1+a_2b_2 
\end{align}
%
	In 
\figref{fig:tri_right_angle},	
\begin{align}
  \vec{A}^{\top} \vec{C}  = 0
\end{align}
\item The {\em norm} of $\vec{A}$ is defined as
\begin{align}
  \label{eq:norm2d}
	\norm{\vec{A}} 
  &= \sqrt{\vec{A}^{\top} \vec{A}}= \sqrt{a_1^2+a_2^2}
\end{align}
\item In 
\figref{fig:tri_right_angle},	
it is easy to verify that 
\begin{align}
\norm{\vec{A}-\vec{C}}^2  
  = \myvec{-c & a} \myvec{-c \\ a}
= a^2 + c^2 = b^2
\end{align}
from 
	\eqref{eq:tri_baudh}.
Thus, 
	the distance betwen any two  points $\vec{A}$ and $\vec{B}$ is given by 
\begin{align}
  \label{eq:norm2d_dist}
\norm{\vec{A}-\vec{B}} 
\end{align}
\item Show that 
\begin{align}
  \label{eq:norm2d_const}
  \norm{\lambda \vec{A}} 
  = \abs{\lambda} \norm{\vec{A}}
\end{align}
\end{enumerate}
\section{Collinear Points}
\begin{enumerate}[label=\thesection.\arabic*.,ref=\thesection.\theenumi]
\numberwithin{equation}{enumi}
\item The direction vector of the line $AB$ is
\begin{align}
	\vec{A}-
	\vec{B} \equiv
	\vec{B}-
	\vec{A} \equiv \kappa \myvec{1 \\ m},
\end{align}
where $m$ is defined to be the slope of $AB$.
	In 
\figref{fig:tri_right_angle}, 
\begin{align}
\vec{A}-\vec{C}
	=  \myvec{-c \\ a} \equiv \myvec{1 \\ -\frac{a}{c}} = \myvec{1 \\ -\tan \theta}
\end{align}

		the slope of $AC$ is $-\tan \theta$
  \item Points $\vec{A},\vec{B}$ and $\vec{C}$ are on a line  if they have the same direction vector, i.e.
\begin{align}
	  \label{eq:collinear}
	p\brak{\vec{B}-\vec{A}}
	+ q\brak{\vec{C}-\vec{B}} = 0
	\implies p, q \ne 0.
\end{align}
	$\brak{\vec{A}-\vec{B}},
	 \brak{\vec{C}-\vec{B}}$
		are then said to be {\em linearly dependent}.
		\item If points $\vec{A}, \vec{B}$ and $\vec{C}$ are collinear, 
  \begin{align}
	  \vec{B}&= \frac{k\vec{A}+ \vec{C}}{k+1}
	  \label{eq:section_formula}
  \end{align}
  \solution From  
	  \eqref{eq:collinear},
	\begin{align}
	p\brak{\vec{A}-\vec{B}}
		+ q\brak{\vec{A}-\vec{C}} &= 0
		\implies \vec{B} = \frac{p\vec{A}+q\vec{C}}{p+q}
\end{align}
yielding 
	  \eqref{eq:section_formula}
	  upon substituting
\begin{align}
	k = \frac{p}{q}.
\end{align}
		This is known as {\em section formula}.

  \item Consequently, points $\vec{A},\vec{B}$ and $\vec{C}$ form a triangle  if 
	  \label{prop:two-tri-indep}
  \begin{align}
	  p\brak{\vec{A}- \vec{B}} +q\brak{\vec{C} -\vec{B}} 
	  \\
	  =\brak{p+q}\vec{B}- p\vec{A} -q\vec{C} = 0
	  \\
	  \implies p=0, q=0
	  \label{eq:two-tri-indep}
  \end{align}
  \item In 
	\figref{fig:tri_med_meet}, 
	\begin{align}
	AF = BF, \,
	AE = BE, 
	\end{align}
	and the medians $BE$ and $CF$ meet at $\vec{G}$.
	Show that 
%	Using Fig. \ref{ch2_median_ratio_val}, 
	\begin{align}
\label{eq:tri_med_centroid_ratio}
	\frac{GB}{GE} = \frac{GC}{GF} = 2
	\end{align}
%
\begin{figure}[!ht]
	\begin{center}
		\resizebox{\columnwidth}{!}{%Code by GVV Sharma
%December 10, 2019
%released under GNU GPL
%Drawing the median

\begin{tikzpicture}
[scale=2,>=stealth,point/.style={draw,circle,fill = black,inner sep=0.5pt},]

%Triangle sides
\def\a{5}
\def\b{6}
\def\c{4}
 
%Coordinates of A
\def\p{2.25}
\def\q{{sqrt(\c^2-\p^2)}}

%Labeling points
\node (A) at (\p,\q)[point,label=above right:$A$] {};
\node (B) at (0, 0)[point,label=below left:$B$] {};
\node (C) at (\a, 0)[point,label=below right:$C$] {};

%Foot of median

\node (D) at ($(B)!0.5!(C)$)[point,label=below:$D$] {};
\node (E) at ($(A)!0.5!(C)$)[point,label=right:$E$] {};
\node (F) at ($(B)!0.5!(A)$)[point,label=left:$F$] {};

%Drawing triangle ABC
\draw (A) -- node[] {} (B) -- node[below, yshift=-5mm] {$\textrm{a}$} (C) -- node[] {} (A);

%Drawing medians AD, BE and CF
\draw (B) -- (E);
\draw (C) -- (F);
\draw (A) -- (D);

%Drawing EF
\draw [dashed] (E) -- (F);

%Centroid
\node (G) at ($(B)!0.67!(E)$)[label={[shift={(0.8,-0.5)}]$G$}] {};

%Labeling sides
\node [right] at ($(A)!0.5!(E)$) {$\frac{b}{2}$};
\node [right] at ($(C)!0.5!(E)$) {$\frac{b}{2}$};
\node [left] at ($(B)!0.5!(F)$) {$\frac{c}{2}$};
\node [left] at ($(A)!0.5!(F)$) {$\frac{c}{2}$};
\node [below] at ($(E)!0.5!(G)$) {$p$};
\node [below] at ($(B)!0.5!(G)$) {$2p$};
\node [above right] at ($(F)!0.5!(E)$) {$P$};

%\node (G) at ($(B)!0.67!(E)$)[label={[shift={(-0.8,-0.5)}]$G_1$}] {};

%
\end{tikzpicture}

}
	\end{center}
	\caption{$\frac{GA}{GD} =2$}
	\label{fig:tri_med_meet}	
\end{figure}
\solution From 
	  \eqref{eq:section_formula},
  \begin{align}
	  \label{eq:section_formula-G}
\vec{G} = 
	   \frac{k_1\vec{E}+ \vec{B}}{k_1+1}
	  &= \frac{k_2\vec{F}+ \vec{C}}{k_2+1}
	  \\
	  \implies 
	   \frac{k_1\brak{\frac{\vec{A}+\vec{C}}{2}}+ \vec{B}}{k_1+1}
	  &= \frac{k_2\brak{\frac{\vec{A}+\vec{B}}{2}}+ \vec{C}}{k_2+1}
	  \\
	  \implies 
	\brak{k_2+1}   \cbrak{k_1\brak{{\vec{A}+\vec{C}}}+ 2\vec{B}}
	  &= \brak{k_1+1}\cbrak{k_2\brak{{\vec{A}+\vec{B}}}+ 2\vec{C}}
  \end{align}
  which can be expressed as
  \begin{align}
	  \cbrak{2 + k_2- k_1k_2 }\vec{B}-\brak{k_2-k_1}\vec{A}  - \cbrak{k_1 +2 - k_1k_2}\vec{C}
	  =0
  \end{align}
  and is of the form
	  \eqref{eq:two-tri-indep}
	  with 
  \begin{align}
	  p = {k_2-k_1}, q = {k_1 +2 - k_1k_2}.
  \end{align}
  Thus, from 
	  \eqref{eq:two-tri-indep}
  \begin{align}
\label{eq:tri_med_centroid_ratio-1}
	  k_2-k_1 &= 0,
	  \\
	  k_1 +2 - k_1k_2 &=0
\label{eq:tri_med_centroid_ratio-2}
  \end{align}
  Thus, from 
\eqref{eq:tri_med_centroid_ratio-2}
  \begin{align}
	  k_1=k_2
  \end{align}
  and substituting the above in 
\eqref{eq:tri_med_centroid_ratio-2} results in the quadratic
  \begin{align}
	  k_1^2 - k_1-2 &=0
	  \\
	  \implies 
	  \brak{k_1-2}\brak{k_1+1} &=0
  \end{align}
  admitting $k_1=k_2=2$ as the only possible solution.
  \item Substituting $k_1 =2 $ in 
	  \eqref{eq:section_formula-G}
  \begin{align}
	  \vec{G}=\frac{\vec{A}+\vec{B} + \vec{C}}{3}
	  \label{eq:centroid-G}
  \end{align}
\item $AG$ is extended to join $BC$ at $\vec{D}$.  Show that $AD$ is also a median.
	\\
	\solution Considering the ratios in \figref{},
  \begin{align}
\vec{G} = 
	  \frac{k_3\vec{D}+\vec{A} }{k_3+1} 
	  \\
	\vec{D}  =\frac{k_4\vec{C}+\vec{B} }{k_4+1} 
  \end{align}
  Substituting from 
	  \eqref{eq:centroid-G}
	  in the above, 
  \begin{align}
	  \brak{k_3+1}\brak{\frac{\vec{A}+\vec{B} + \vec{C}}{3}}
 = 
	  {k_3\brak{\frac{k_4\vec{C}+\vec{B} }{k_4+1}} +\vec{A} } 
	  \\
	  \implies \brak{k_3+1}\brak{k_4+1}\brak{{\vec{A}+\vec{B} + \vec{C}}}
 = 
	  {3} \cbrak{ {k_3\brak{{k_4\vec{C}+\vec{B} }} +\brak{k_4+1}\vec{A} }} 
  \end{align}
  which can be expressed as
  \begin{multline}
	  \brak{k_3k_4+k_3-2k_4-2}\vec{A}
	  \\
	-  \brak{-k_3k_4-k_4+2k_3-1}\vec{B}
	  \\
	  - \brak{-k_3-k_4 - 1 
+2k_3k_4} \vec{C} = \vec{0}
  \end{multline}
  Comparing the above with 
	  \eqref{eq:two-tri-indep},
  \begin{align}
	  p = {-k_3k_4-k_4+2k_3-1}, q = {-k_3-k_4 - 1 
+2k_3k_4}
  \end{align}
  yielding 
  \begin{align}
	  \label{eq:centroid-G-meet-1}
	   {-k_3k_4-k_4+2k_3-1} = 0
	   \\ {-k_3-k_4 - 1 
+2k_3k_4} = 0
	  \label{eq:centroid-G-meet-2}
  \end{align}
  Subtracting 
	  \eqref{eq:centroid-G-meet-1}
	  from
	  \eqref{eq:centroid-G-meet-2},
  \begin{align}
	  3k_3\brak{k_4-1} &= 0
	  \\
	  \implies k_4&=1
  \end{align}
  which upon substituting in 
	  \eqref{eq:centroid-G-meet-1}
	  yields
  \begin{align}
	  k_3 = 2
  \end{align}
  \iffalse
  \item In $\triangle ABC$, if $\vec{D}, \vec{E}$ divide the lines $AB, AC$ in the ratio $k:1$ respectively,  then $DE \parallel BC$.
	  \label{prop:two-tri-bpt}
	  \begin{proof}
		  From 
	  \eqref{eq:section_formula}, 
  \begin{align}
	  \vec{D}&= \frac{k\vec{B}+ \vec{A}}{k+1}
	  \\
	  \vec{E}&= \frac{k\vec{C}+ \vec{A}}{k+1}
	  \\
	  \implies 
	  \vec{D}-	  \vec{E}&= \frac{k}{k+1}\brak{\vec{B}- \vec{C}}
  \end{align}
  Thus, from 
		  Appendix \ref{prop:two-dir-vec}, $DE \parallel BC$.

	  \end{proof}

  \item In $\triangle ABC$, if $DE \parallel BC$, $\vec{D}$ and $\vec{E}$ divide the lines $AB, AC$ in the same ratio.  
	  \label{prop:two-tri-bpt-conv}
	  \begin{proof}
If $DE \parallel BC$,
		  from 
 \eqref{eq:two-par-dir-vec}
  \begin{align}
	  \label{prop:two-tri-bpt-conv-1}
	  \brak{\vec{B}- \vec{C}} = k\brak{\vec{D}-	  \vec{E}}
  \end{align}
Using   
	  \eqref{eq:section_formula}, 
let 
  \begin{align}
	  \vec{D}&= \frac{k_1\vec{B}+ \vec{A}}{k_1+1}
	  \\
	  \vec{E}&= \frac{k_2\vec{C}+ \vec{A}}{k_2+1}
  \end{align}
	  Subtituting the above in 
	  \eqref{prop:two-tri-bpt-conv-1}, after some algebra, we obtain 
	
  \begin{align}
\brak{p+q}\vec{A}- p\vec{B} -q\vec{C} &= 0
  \end{align}
  where
  \begin{align}
	  p = \frac{1}{k} -  \frac{k_1}{k_1+1},
	  q = \frac{1}{k} -  \frac{k_1}{k_1+1}
  \end{align}
  %
From 	  
	  \eqref{eq:two-tri-indep},
  \begin{align}
	p = q = 0
	  \\
	  \implies k_1 = k_2  = \frac{1}{k-1}
  \end{align}

	  \end{proof}
	  \fi
\end{enumerate}
\section{Matrices: Cosine Formula}
\begin{enumerate}[label=\thesection.\arabic*.,ref=\thesection.\theenumi]
\numberwithin{equation}{enumi}
\item The determinant of the $2 \times 2$ matrix 
\begin{align}  
  \vec{M} = \myvec{\vec{A} & \vec{B}}=\mydet{a_1 & b_1\\a_2 & b_2}
\end{align}
is defined as
\begin{align}
  \label{eq:det2d}
  \mydet{\vec{M}} &= \mydet{\vec{A} & \vec{B}} 
  \\
  &= \mydet{a_1 & b_1\\a_2 & b_2} = a_1b_2 - a_2 b_1
\end{align}
%
\item Let 
\begin{align}
  \vec{A} = \myvec{a_1\\a_2 \\ a_3},\, 
  \vec{B} = \myvec{b_1\\b_2 \\ b_3},\,
  \vec{C} = \myvec{c_1\\c_2 \\ c_3}
\end{align}
and 
\begin{align}
  \vec{A}_{ij} = \myvec{a_i\\a_j}, \,
  \vec{B}_{ij} = \myvec{b_i\\b_j}, \,
  \vec{C}_{ij} = \myvec{c_i\\c_j}. 
\end{align}
Then, \begin{align}  
	\mydet{\vec{A} & \vec{B} & \vec{C}} = a_1 \mydet{\vec{B}_{23} & \vec{C}_23} +a_2 \mydet{\vec{C}_{23} & \vec{A}_{23}} +a_3\mydet{\vec{A}_{23} & \vec{B}_{23}} 
\end{align}



\item
In Fig. \ref{fig:tri_cosine_formula}, show that
%
\begin{equation}
\label{eq:tri_cos_mat}
\begin{pmatrix}
0 & c & b \\
c & 0 & a \\
b & a & 0
\end{pmatrix}
\begin{pmatrix}
\cos A \\
\cos B \\
\cos C
\end{pmatrix}
= 
\begin{pmatrix}
a\\
b\\
c
\end{pmatrix}
\end{equation}
%
%
\begin{figure}[!ht]
	\begin{center}
		
		%\includegraphics[width=\columnwidth]{./figs/ch2_triang_ar}
		%\vspace*{-10cm}
		\resizebox{\columnwidth}{!}{%Code by GVV Sharma
%December 7, 2019
%released under GNU GPL
%Drawing a triangle given 3 sides

\begin{tikzpicture}
[scale=2,>=stealth,point/.style={draw,circle,fill = black,inner sep=0.5pt},]

%Triangle sides
\def\a{6}
\def\b{5}
\def\c{4}
 
%Coordinates of A
%\def\p{{\a^2+\c^2-\b^2}/{(2*\a)}}
\def\p{2.25}
\def\q{{sqrt(\c^2-\p^2)}}

%Labeling points
\node (A) at (\p,\q)[point,label=above right:$A$] {};
\node (B) at (0, 0)[point,label=below left:$B$] {};
\node (C) at (\a, 0)[point,label=below right:$C$] {};

%Foot of perpendicular

\node (D) at (\p,0)[point,label=above right:$D$] {};

%Drawing triangle ABC
\draw (A) -- node[left] {$\textrm{c}$} (B) -- node[below] {$\textrm{a}$} (C) -- node[above,xshift=2mm] {$\textrm{b}$} (A);

%Drawing altitude AD
\draw (A) -- node[left] {$\textrm{h}$}(D);

\tkzMarkRightAngle[fill=blue!20,size=.2](A,D,B)

\node [below] at ($(B)!0.5!(D)$) {$x$};
\node [below] at ($(C)!0.5!(D)$) {$y$};

\end{tikzpicture}
}
	\end{center}
	\caption{The cosine formula}
	\label{fig:tri_cosine_formula}	
\end{figure}
\solution From Fig. \ref{fig:tri_cosine_formula}, 
%
\begin{align}
	a &= x + y = b \cos C + c \cos B = \myvec{  \cos C & \cos B } \myvec{ b \\ c }
	\\
&=\myvec{0 & b & c } \myvec{ \cos A \\ \cos C \\ \cos B } 
\end{align}
%
Similarly,
%
\begin{align}
b &= c \cos A + a \cos C 
=\myvec{c & 0 & a } \myvec{ \cos A \\ \cos C \\ \cos B } 
	\\
c &= b \cos A + a \cos B
=\myvec{b & a & 0 } \myvec{ \cos A \\ \cos C \\ \cos B } 
\end{align}
%
The above equations can be expressed in matrix form as
\eqref{eq:tri_cos_mat}.

\item Show that 
\begin{equation}
\label{eq:tri_cos_form}
\cos A = \frac{b^2+c^2-a^2}{2bc}
\end{equation}
%
\solution 
Using the properties of determinants,
%
\begin{align}
\cos A = \frac{
\begin{vmatrix}
a & c & b \\
b & 0 & a \\
c & a & 0
\end{vmatrix}
	}
	{
\begin{vmatrix}
0 & c & b \\
c & 0 & a \\
b & a & 0
\end{vmatrix}
	}
	=\frac{ab^2 + ac^2 - a^3}{abc + abc} 
= \frac{b^2 + c^2 - a^2}{2abc}
\end{align}
\end{enumerate}
\section{Area of a Triangle: Cross Product}
\begin{enumerate}[label=\thesection.\arabic*.,ref=\thesection.\theenumi]
\numberwithin{equation}{enumi}
\item The {\em cross product} or {\em vector product} defined as $\vec{A}\times \vec{B}$ is given by  
  \eqref{eq:det2d} for  $2 \times 1$ vectors.
  \iffalse
\item For  $3 \times 1$ vectors, 
\begin{align}
  \label{eq:cross3d}
	\vec{A} \times \vec{B} = \myvec{ \mydet{\vec{A}_{23} & \vec{B}_{23}} \\[10pt] \mydet{\vec{A}_{31} & \vec{B}_{31}} \\[10pt] \mydet{\vec{A}_{12}  & \vec{B}_{12}}}
\end{align}
\fi
\item The area of the triangle with vertices $\vec{A}, \vec{B}, \vec{C}$ is given by 
	\label{prop:area2d}
\begin{align}
  \label{eq:area2d}
	\frac{1}{2}\norm{\brak{\vec{A}-\vec{B}} \times \brak{\vec{A}-\vec{C}}}
 = 
 \frac{1}{2}\norm{\vec{A} \times \vec{B}+\vec{B} \times \vec{C}+\vec{C} \times \vec{A}}
  \end{align}
  \item If 
  \label{prop:area2d-norm}
\begin{align}
  \label{eq:area2d-norm}
	\norm{\vec{A}\times\vec{B}}  &= \norm{\vec{C}\times \vec{D}}, \quad \text{then}
	\\
	\vec{A}\times\vec{B}  &= \pm\brak{\vec{C}\times \vec{D}}
  \end{align}
  where the sign depends on the orientation of the vectors.
  \item The median divides the triangle into two triangles of equal area.
	  \label{prop:two-median-area}
\item Let $\vec{x}$ be equidistant from the points $\vec{A}$ and $\vec{B}$.  Then 
  \begin{align}
	  \brak{\vec{A}-\vec{B}}^{\top}{\vec{x}} 
	  =  \frac{\norm{\vec{A}}^2 - \norm{\vec{B}}^2}{2}
  \label{eq:norm2d_equidist}
  \end{align}
  \solution 
\begin{align}
	\norm{\vec{x}-\vec{A}} &=
\norm{\vec{A}-\vec{B}} 
\\
	\implies \norm{\vec{x}-\vec{A}}^2 &=
\norm{\vec{x}-\vec{B}}^2 
\end{align}
which can be expressed as 
\begin{multline}
%  \label{eq:norm2d_dist}
	\brak{\vec{x}-\vec{A}}^{\top} \brak{\vec{x}-\vec{A}}=
	\brak{\vec{x}-\vec{B}}^{\top} 
\brak{\vec{x}-\vec{B}}
\\
	\implies	\norm{\vec{x}}^2-2{\vec{x}}^{\top}\vec{A} + \norm{\vec{A}}^2
	\\= \norm{\vec{x}}^2-2{\vec{x}}^{\top}\vec{B} + \norm{\vec{B}}^2
\end{multline}
which can be simplified to obtain
  \eqref{eq:norm2d_equidist}.
\item If $\vec{x}$ lies on the  $x$-axis and is  equidistant from the points $\vec{A}$ and $\vec{B}$, 
  \begin{align}
	  \vec{x} &=
	   x\vec{e}_1
  \end{align}
  where 
  \begin{align}
	  x &=\frac{\norm{\vec{A}}^2 -\norm{\vec{B}}^2 }{2\brak{\vec{A}-\vec{B}}^{\top }\vec{e}_1
}
	  \label{eq:cbse_10_x}
  \end{align}
  \solution 
  From \eqref{eq:norm2d_equidist}.
  \begin{align}
	   x\brak{\vec{A}-\vec{B}}^{\top }\vec{e}_1
		  &=
	  \frac{\norm{\vec{A}}^2 -\norm{\vec{B}}^2 }{2}
   \end{align}
	  yielding \eqref{eq:cbse_10_x}.
  \item The angle between two vectors is given by 
    \label{prop:angle2d}
  \begin{align}
    \label{eq:angle2d}
    \theta = \cos^{-1}\frac{\vec{A}^{\top} \vec{B}}{\norm{A}\norm{B}}
  \end{align}
  \item If two vectors are orthogonal (perpendicular), 
  \begin{align}
    \label{eq:angle2d_orth}
\vec{A}^{\top} \vec{B} = 0
  \end{align}
  \item For an isoceles triangle $ABC$ ith $AB = AC$, the median $AD \perp BC$.
    \label{prop:two-isosc}
%  \begin{align}
%    \label{eq:two-isosc}
%\vec{A}^{\top} \vec{B} = 0
%  \end{align}

  \item The {\em direction vector} of the line joining two points $\vec{A},\vec{B}$ is given by 
  \begin{align}
    \label{eq:dir_vec}
    \vec{m} = \vec{A}-\vec{B}
  \end{align}
  \item The points $\vec{A}\vec{A}\vec{A}$
\item The unit vector in the direction of $\vec{m}$ is defined as
\begin{align}
    \frac{\vec{m}}{\norm{\vec{m}}}
\end{align}
\item If the direction vector of a line is expressed as 
		\label{prop:two-dir-vec}
	\begin{align}
		\label{eq:two-dir-vec}
    \vec{m} = \myvec{1\\m},
\end{align}
 the $m$ is defined to be the {\em} slope of the line. 
  \item $AB \parallel CD$ if 
	  \label{prop:two-par-dir-vec}
  \begin{align}
	  \vec{A}- \vec{B}= k\brak{\vec{C}- \vec{D}}
	  \label{eq:two-par-dir-vec}
  \end{align}
  \item The {\em normal vector} to $\vec{m}$ is defined by 
  \begin{align}
    \label{eq:normal_vec}
    \vec{m}^{\top}  \vec{n} = 0
  \end{align}
  \item  If
	  \label{prop:two-orth-para}
\begin{align}
	\vec{m}^{\top}  \vec{n}_1 &= 0
	\\
	\vec{m}^{\top}  \vec{n}_2 &= 0,
	\\
	\vec{n}_1 &\parallel \vec{n}_2
	  \label{eq:two-orth-para}
\end{align}
\item  The standard basis vectors are defined as 
	\label{def:matrix-two}

  \begin{align}
  \vec{e}_1&= \myvec{1\\0}, 
  \\
  \vec{e}_2&= \myvec{0\\1}.
  \end{align}
\end{enumerate}
\section{Parallelogram}
\begin{enumerate}[label=\thesection.\arabic*.,ref=\thesection.\theenumi]
\numberwithin{equation}{enumi}
  \item If $ABCD$ be a parallelogram,
	  \label{prop:two-pgm}
  \begin{align}
	  \label{eq:two-pgm}
 \vec{B}-\vec{A} = \vec{C} -\vec{D}
  \end{align}
%  \item Diagonals of a parallelogram bisect each other.
%	  \label{prop:two-pgm-diag-bisect}
\item The area of the parallelogram with vertices $\vec{A}, \vec{B}, \vec{C}$ and $\vec{D}$ is given by 
  \label{prop:pgm2d}
\begin{align}
  \label{eq:pgm2d}
	\norm{\brak{\vec{A}-\vec{B}} \times \brak{\vec{A}-\vec{D}}}
 = 
 \norm{\vec{A} \times \vec{B}+\vec{B} \times \vec{C}+\vec{C} \times \vec{A}}
  \end{align}
\end{enumerate}
