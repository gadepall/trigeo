\begin{enumerate}[label=\thesection.\arabic*.,ref=\thesection.\theenumi]
\numberwithin{equation}{enumi}
  \item In 
	\figref{fig:tri_med_isect}	
	\begin{align}
	AF = BF, \,
	AE = BE, 
	\end{align}
	and the medians $BE$ and $CF$ meet at $\vec{G}$.
	Show that
\begin{align}
	ar\brak{BEC}
	=ar\brak{BFC} = \frac{1}{2}ar\brak{ABC}
	\label{eq:median-area}
\end{align}
\solution
	From \eqref{tri:area-sin},
\begin{align}
	ar\brak{BEC} &= 
	\frac{1}{2}a\brak{\frac{b}{2}}\sin C 
	\\
	ar\brak{BFC}&=
	\frac{1}{2}a\brak{\frac{c}{2}}\sin  B
\end{align}
yielding
	\eqref{eq:median-area}.
\item 
	Show that
\begin{align}
	ar\brak{CGE}
	=ar\brak{BGF} 
	\label{eq:median-sub-area}
\end{align}
\solution 
From 
	\figref{fig:tri_med_isect}	
	and 
	\eqref{eq:median-area},
\begin{align}
	ar\brak{BGF}
	+
	ar\brak{BGC}
	=
	ar\brak{CGE}
	+
	ar\brak{BGC}
\end{align}
yielding 
	\eqref{eq:median-sub-area}.
\item 
	If $\vec{G}$ divides $BE$ and $CF$ in the ratios $k_1$ and $k_2$ respectively, 
	show that
%	Using Fig. \ref{ch2_median_ratio_val}, 
	\begin{align}
\label{eq:tri_med_centroid_ratio}
k_1 = k_2 
	\end{align}
%
\begin{figure}[!ht]
	\begin{center}
		\resizebox{\columnwidth}{!}{%Code by GVV Sharma
%July 8, 2023
%released under GNU GPL
%Drawing the medians

\begin{tikzpicture}
[scale=2,>=stealth,point/.style={draw,circle,fill = black,inner sep=0.5pt},]

%Triangle sides
\def\a{5}
\def\b{6}
\def\c{4}
 
%Coordinates of A
\def\p{2.25}
\def\q{{sqrt(\c^2-\p^2)}}

%Labeling points
\node (A) at (\p,\q)[point,label=above right:$A$] {};
\node (B) at (0, 0)[point,label=below left:$B$] {};
\node (C) at (\a, 0)[point,label=below right:$C$] {};

%Foot of median

%\node (D) at ($(B)!0.5!(C)$)[point,label=below:$D$] {};
\node (E) at ($(A)!0.5!(C)$)[point,label=right:$E$] {};
\node (F) at ($(B)!0.5!(A)$)[point,label=left:$F$] {};

%Drawing triangle ABC
\draw (A) -- node[] {} (B) -- node[below, yshift=-5mm] {$\textrm{a}$} (C) -- node[] {} (A);

%Drawing medians AD, BE and CF
\draw (B) -- (E);
\draw (C) -- (F);
%\draw (A) -- (D);

%Drawing EF
%\draw [dashed] (E) -- (F);

%Centroid
%\node (G) at ($(B)!0.67!(E)$)[label={[shift={(0.8,-0.5)}]$G$}] {};
\node (G) at ($(B)!0.67!(E)$)[point, label=below:$G$] {};

%Labeling sides
\node [right] at ($(A)!0.5!(E)$) {$\frac{b}{2}$};
\node [right] at ($(C)!0.5!(E)$) {$\frac{b}{2}$};
\node [left] at ($(B)!0.5!(F)$) {$\frac{c}{2}$};
\node [left] at ($(A)!0.5!(F)$) {$\frac{c}{2}$};
% Adding new labels
\node [above] at ($(G)!0.5!(E)$) {$p$};
\node [above] at ($(B)!0.5!(G)$) {$k_1p$};
\node [above] at ($(G)!0.5!(F)$) {$q$};
\node [above] at ($(C)!0.5!(G)$) {$k_2q$};
%\node [below] at ($(E)!0.5!(G)$) {$1$};
%\node [below] at ($(B)!0.5!(G)$) {$k_1$};
%\node [below] at ($(F)!0.5!(G)$) {$1$};
%\node [below] at ($(C)!0.5!(G)$) {$k_2$};
\tkzLabelAngle[pos=0.5](C,G,E){$\theta$}
\tkzLabelAngle[pos=0.5](F,G,B){$\theta$}
\tkzFillAngle[fill=red!20,size=.3](C,G,E)
\tkzFillAngle[fill=red!20,size=.3](F,G,B)
%\tkzMarkAngle[color=red, fill=red!40,size=0.3](C,G,E)
%\tkzMarkAngle[color=red, fill=red!40,size=0.3](F,G,B)
%\tkzMarkAngle[fill=blue!50,size=.3](C,G,E)
\iffalse
\node [above right] at ($(F)!0.5!(E)$) {$P$};
\fi

%\node (G) at ($(B)!0.67!(E)$)[label={[shift={(-0.8,-0.5)}]$G_1$}] {};

%
\end{tikzpicture}
}
%		\resizebox{\columnwidth}{!}{%Code by GVV Sharma
%December 10, 2019
%released under GNU GPL
%Drawing the median

\begin{tikzpicture}
[scale=2,>=stealth,point/.style={draw,circle,fill = black,inner sep=0.5pt},]

%Triangle sides
\def\a{5}
\def\b{6}
\def\c{4}
 
%Coordinates of A
\def\p{2.25}
\def\q{{sqrt(\c^2-\p^2)}}

%Labeling points
\node (A) at (\p,\q)[point,label=above right:$A$] {};
\node (B) at (0, 0)[point,label=below left:$B$] {};
\node (C) at (\a, 0)[point,label=below right:$C$] {};

%Foot of median

\node (D) at ($(B)!0.5!(C)$)[point,label=below:$D$] {};
\node (E) at ($(A)!0.5!(C)$)[point,label=right:$E$] {};
\node (F) at ($(B)!0.5!(A)$)[point,label=left:$F$] {};

%Drawing triangle ABC
\draw (A) -- node[] {} (B) -- node[below, yshift=-5mm] {$\textrm{a}$} (C) -- node[] {} (A);

%Drawing medians AD, BE and CF
\draw (B) -- (E);
\draw (C) -- (F);
\draw (A) -- (D);

%Drawing EF
%\draw [dashed] (E) -- (F);

%Centroid
\node (G) at ($(B)!0.67!(E)$)[label={[shift={(0.8,-0.5)}]$G$}] {};

%Labeling sides
\node [right] at ($(A)!0.5!(E)$) {$\frac{b}{2}$};
\node [right] at ($(C)!0.5!(E)$) {$\frac{b}{2}$};
\node [left] at ($(B)!0.5!(F)$) {$\frac{c}{2}$};
\node [left] at ($(A)!0.5!(F)$) {$\frac{c}{2}$};
%\node [below] at ($(E)!0.5!(G)$) {$1$};
%\node [below] at ($(B)!0.5!(G)$) {$2$};
%\node [below] at ($(F)!0.5!(G)$) {$1$};
%\node [below] at ($(C)!0.5!(G)$) {$2$};
%\node [right] at ($(D)!0.5!(G)$) {$1$};
\node [right] at ($(A)!0.5!(G)$) {$k_3$};
%\node [below] at ($(D)!0.5!(C)$) {$1$};
\node [below] at ($(B)!0.5!(D)$) {$k_4$};
\iffalse
\node [above right] at ($(F)!0.5!(E)$) {$P$};
\fi

%\node (G) at ($(B)!0.67!(E)$)[label={[shift={(-0.8,-0.5)}]$G_1$}] {};

%
\end{tikzpicture}

}
	\end{center}
	\caption{$k_1=k_2$.}
	\label{fig:tri_med_isect}	
	%\label{fig:tri_med_meet}	
\end{figure}
\solution 
Let 
\begin{align}
	GE = l_1, GF = l_2
\end{align}
From 
	\eqref{tri:area-sin}
	and 
	\eqref{eq:median-sub-area},
\begin{align}
	\frac{1}{2}l_1 \brak{k_2l_2} \sin \theta
	= \frac{1}{2}l_2 \brak{k_1l_1}\sin \theta
\end{align}
yielding
\eqref{eq:tri_med_centroid_ratio}.
\item Show that
\begin{align}
	k_1 = k_2 = 2
\end{align}
\solution 
Let 
\begin{align}
	  \label{eq:section_formula-G-2}
	k_1 = k_2 = k
\end{align}
Using 
	  \eqref{eq:section_formula},
  \begin{align}
	  \label{eq:section_formula-G}
\vec{G} = 
	   \frac{k\vec{E}+ \vec{B}}{k+1}
	  &= \frac{k\vec{F}+ \vec{C}}{k+1}
	  \\
	  \implies 
	   k\brak{\frac{\vec{A}+\vec{C}}{2}}+ \vec{B}
	  &= k\brak{\frac{\vec{A}+\vec{B}}{2}}+ \vec{C}
	  \label{eq:section_formula-G-val}
	  \\
	  \implies 
	   k\brak{\vec{B}-\vec{C}}
	  &= 
	   2\brak{\vec{B}-\vec{C}}
  \end{align}
  resulting in 
	  \eqref{eq:section_formula-G-2}.
  \item Substituting $k = 2$ in \eqref{eq:section_formula-G-val},
\begin{align}
	\vec{G} = \frac{\vec{A}+\vec{B}+\vec{C}}{3}
	  \label{eq:centroid-G}
\end{align}
\item 
In	\figref{fig:tri_med_meet},	
$AG$ is extended to join $BC$ at $\vec{D}$.  Show that $AD$ is also a median.
\begin{figure}[!ht]
	\begin{center}
%		\resizebox{\columnwidth}{!}{%Code by GVV Sharma
%July 8, 2023
%released under GNU GPL
%Drawing the medians

\begin{tikzpicture}
[scale=2,>=stealth,point/.style={draw,circle,fill = black,inner sep=0.5pt},]

%Triangle sides
\def\a{5}
\def\b{6}
\def\c{4}
 
%Coordinates of A
\def\p{2.25}
\def\q{{sqrt(\c^2-\p^2)}}

%Labeling points
\node (A) at (\p,\q)[point,label=above right:$A$] {};
\node (B) at (0, 0)[point,label=below left:$B$] {};
\node (C) at (\a, 0)[point,label=below right:$C$] {};

%Foot of median

%\node (D) at ($(B)!0.5!(C)$)[point,label=below:$D$] {};
\node (E) at ($(A)!0.5!(C)$)[point,label=right:$E$] {};
\node (F) at ($(B)!0.5!(A)$)[point,label=left:$F$] {};

%Drawing triangle ABC
\draw (A) -- node[] {} (B) -- node[below, yshift=-5mm] {$\textrm{a}$} (C) -- node[] {} (A);

%Drawing medians AD, BE and CF
\draw (B) -- (E);
\draw (C) -- (F);
%\draw (A) -- (D);

%Drawing EF
%\draw [dashed] (E) -- (F);

%Centroid
%\node (G) at ($(B)!0.67!(E)$)[label={[shift={(0.8,-0.5)}]$G$}] {};
\node (G) at ($(B)!0.67!(E)$)[point, label=below:$G$] {};

%Labeling sides
\node [right] at ($(A)!0.5!(E)$) {$\frac{b}{2}$};
\node [right] at ($(C)!0.5!(E)$) {$\frac{b}{2}$};
\node [left] at ($(B)!0.5!(F)$) {$\frac{c}{2}$};
\node [left] at ($(A)!0.5!(F)$) {$\frac{c}{2}$};
% Adding new labels
\node [above] at ($(G)!0.5!(E)$) {$p$};
\node [above] at ($(B)!0.5!(G)$) {$k_1p$};
\node [above] at ($(G)!0.5!(F)$) {$q$};
\node [above] at ($(C)!0.5!(G)$) {$k_2q$};
%\node [below] at ($(E)!0.5!(G)$) {$1$};
%\node [below] at ($(B)!0.5!(G)$) {$k_1$};
%\node [below] at ($(F)!0.5!(G)$) {$1$};
%\node [below] at ($(C)!0.5!(G)$) {$k_2$};
\tkzLabelAngle[pos=0.5](C,G,E){$\theta$}
\tkzLabelAngle[pos=0.5](F,G,B){$\theta$}
\tkzFillAngle[fill=red!20,size=.3](C,G,E)
\tkzFillAngle[fill=red!20,size=.3](F,G,B)
%\tkzMarkAngle[color=red, fill=red!40,size=0.3](C,G,E)
%\tkzMarkAngle[color=red, fill=red!40,size=0.3](F,G,B)
%\tkzMarkAngle[fill=blue!50,size=.3](C,G,E)
\iffalse
\node [above right] at ($(F)!0.5!(E)$) {$P$};
\fi

%\node (G) at ($(B)!0.67!(E)$)[label={[shift={(-0.8,-0.5)}]$G_1$}] {};

%
\end{tikzpicture}
}
		\resizebox{\columnwidth}{!}{%Code by GVV Sharma
%December 10, 2019
%released under GNU GPL
%Drawing the median

\begin{tikzpicture}
[scale=2,>=stealth,point/.style={draw,circle,fill = black,inner sep=0.5pt},]

%Triangle sides
\def\a{5}
\def\b{6}
\def\c{4}
 
%Coordinates of A
\def\p{2.25}
\def\q{{sqrt(\c^2-\p^2)}}

%Labeling points
\node (A) at (\p,\q)[point,label=above right:$A$] {};
\node (B) at (0, 0)[point,label=below left:$B$] {};
\node (C) at (\a, 0)[point,label=below right:$C$] {};

%Foot of median

\node (D) at ($(B)!0.5!(C)$)[point,label=below:$D$] {};
\node (E) at ($(A)!0.5!(C)$)[point,label=right:$E$] {};
\node (F) at ($(B)!0.5!(A)$)[point,label=left:$F$] {};

%Drawing triangle ABC
\draw (A) -- node[] {} (B) -- node[below, yshift=-5mm] {$\textrm{a}$} (C) -- node[] {} (A);

%Drawing medians AD, BE and CF
\draw (B) -- (E);
\draw (C) -- (F);
\draw (A) -- (D);

%Drawing EF
%\draw [dashed] (E) -- (F);

%Centroid
\node (G) at ($(B)!0.67!(E)$)[label={[shift={(0.8,-0.5)}]$G$}] {};

%Labeling sides
\node [right] at ($(A)!0.5!(E)$) {$\frac{b}{2}$};
\node [right] at ($(C)!0.5!(E)$) {$\frac{b}{2}$};
\node [left] at ($(B)!0.5!(F)$) {$\frac{c}{2}$};
\node [left] at ($(A)!0.5!(F)$) {$\frac{c}{2}$};
%\node [below] at ($(E)!0.5!(G)$) {$1$};
%\node [below] at ($(B)!0.5!(G)$) {$2$};
%\node [below] at ($(F)!0.5!(G)$) {$1$};
%\node [below] at ($(C)!0.5!(G)$) {$2$};
%\node [right] at ($(D)!0.5!(G)$) {$1$};
\node [right] at ($(A)!0.5!(G)$) {$k_3$};
%\node [below] at ($(D)!0.5!(C)$) {$1$};
\node [below] at ($(B)!0.5!(D)$) {$k_4$};
\iffalse
\node [above right] at ($(F)!0.5!(E)$) {$P$};
\fi

%\node (G) at ($(B)!0.67!(E)$)[label={[shift={(-0.8,-0.5)}]$G_1$}] {};

%
\end{tikzpicture}

}
	\end{center}
	\caption{$k_3 = 2, k_4 =1$}
%	\label{fig:tri_med_isect}	
	\label{fig:tri_med_meet}	
\end{figure}
	\\
	\solution Considering the ratios in 
	\figref{fig:tri_med_meet},	
  \begin{align}
\vec{G} = 
	  \frac{k_3\vec{D}+\vec{A} }{k_3+1} 
	  \\
	\vec{D}  =\frac{k_4\vec{C}+\vec{B} }{k_4+1} 
  \end{align}
  Substituting from 
	  \eqref{eq:centroid-G}
	  in the above, 
  \begin{align}
	  \brak{k_3+1}\brak{\frac{\vec{A}+\vec{B} + \vec{C}}{3}}
 = 
	  {k_3\brak{\frac{k_4\vec{C}+\vec{B} }{k_4+1}} +\vec{A} } 
  \end{align}
  which can be expressed as
\begin{multline}
\brak{k_3+1}\brak{k_4+1}\brak{{\vec{A}+\vec{B} + \vec{C}}}
 = 
 \\
	  {3} \cbrak{ {k_3\brak{{k_4\vec{C}+\vec{B} }} +\brak{k_4+1}\vec{A} }} 
  \end{multline}
  which can be expressed as
  \begin{multline}
	  \brak{k_3k_4+k_3-2k_4-2}\vec{A}
	  \\
	-  \brak{-k_3k_4-k_4+2k_3-1}\vec{B}
	  \\
	  - \brak{-k_3-k_4 - 1 
+2k_3k_4} \vec{C} = \vec{0}
  \end{multline}
  Comparing the above with 
	  \eqref{eq:two-tri-indep},
  \begin{align}
	  p = {-k_3k_4-k_4+2k_3-1}, q = {-k_3-k_4 - 1 
+2k_3k_4}
  \end{align}
  yielding 
  \begin{align}
	  \label{eq:centroid-G-meet-1}
	   {-k_3k_4-k_4+2k_3-1} = 0
	   \\ {-k_3-k_4 - 1 
+2k_3k_4} = 0
	  \label{eq:centroid-G-meet-2}
  \end{align}
  Subtracting 
	  \eqref{eq:centroid-G-meet-1}
	  from
	  \eqref{eq:centroid-G-meet-2},
  \begin{align}
	  3k_3\brak{k_4-1} &= 0
	  \\
	  \implies k_4&=1
  \end{align}
  which upon substituting in 
	  \eqref{eq:centroid-G-meet-1}
	  yields
  \begin{align}
	  k_3 = 2
  \end{align}
\end{enumerate}
