\subsection{Properties}
%
\renewcommand{\theequation}{\theenumi}
\begin{enumerate}[label=\arabic*.,ref=\thesubsection.\theenumi]
\numberwithin{equation}{enumi}
%
\item
	Fig. \ref{ch4_circle_def} represents a circle, which passes through the vertices $A B, C$ of  $\triangle ABC$ in Fig.  	\eqref{ch3_perp_bisector}	
 The points in the circle are at a distance $R$ from the {\em centre} $O$.  $R$ is known as the {\em radius}. The line  joining any two points on a circle is known as a {\em chord}.  Thus, the sides of $\triangle ABC$ are chords.

\begin{figure}[!ht]
	\begin{center}
		
		%\includegraphics[width=\columnwidth]{./figs/ch4_circle_def}
		%\vspace*{-10cm}
%		\resizebox{\columnwidth}{!}{\begin{tikzpicture}
[
scale =3,
>=stealth,
point/.style = {draw, circle, fill = black, inner sep = 1pt},
]
\def\rad{1}
\coordinate [point, label={above : $O$ }] (O) at (0, 0);
\draw (O) circle (\rad);
\node (A) at (-0.7,-0.7)[point,label=above :$A$] {};
\node (B) at (0.7,-0.7)[point,label=above :$B$] {};

\draw (A)--(O);
\draw (O)--(B);
\draw (B)--(A);

\node [above] at (-0.4,-0.4) {$r$};
\node [above] at (0.4,-0.4) {$r$};

\end{tikzpicture}
}
		\resizebox{\columnwidth}{!}{\begin{tikzpicture}
[
 scale=2,
  >=stealth,
  point/.style = {draw, circle, fill = black, inner sep = 1pt},
]
\node (B) at (-2,-2)[point,label = below left:${B}$] {};
\node (A) at (1,3)[point,label = above left:${A}$] {};
\node (C) at (4,-1)[point,label = below right:${C}$] {};
\draw (A) -- (B) -- (C) -- (A);

\node (D) at (1,-1.5)[point,label = below:$D$] {};

\node (F) at (-0.5,0.5)[point,label = left:$F$] {};

\node (E) at (2.5,1)[point,label = right:$E$] {};

\node (O) at (0.7962963,-0.27777778) [point,label = right:$O$] {};

\draw[thick, dashed] (D) -- (O);
\draw (F) -- (O);
\draw (E) -- (O);
\draw (A) -- (O);
\draw (B) -- (O);
\draw (C) -- (O);

\tkzMarkRightAngle[fill=blue!20,size=.3](A,E,O)
\tkzMarkRightAngle[fill=blue!20,size=.3](A,F,O)
\tkzMarkRightAngle[fill=blue!20,size=.3](O,D,C)


\node [below] at (3.65,0) {\strut $\frac{b}{2}$};
\node [below] at (2.2,2) {\strut  $\frac{b}{2}$};


\node [below] at (2.5,-1.2) {\strut  $\frac{a}{2}$};    
\node [below] at (0,-1.65) {\strut  $\frac{a}{2}$};  
      
\node [below] at (-1.7,-1) {\strut  $\frac{c}{2}$}; 
\node [below] at (0,2) {\strut  $\frac{c}{2}$};   

\def\rad{3.284101453883}
%\coordinate [point, label={above : $O$ }] (O) at (0, 0);
\draw (O) circle (\rad);

\end{tikzpicture}
}
%		\resizebox{\columnwidth}{!}{\begin{tikzpicture}
[
scale =3,
>=stealth,
point/.style = {draw, circle, fill = black, inner sep = 1pt},
]
\def\rad{1}
\coordinate [point, label={above : $O$ }] (O) at (0, 0);
\draw (O) circle (\rad);
\node (A) at (-0.7,-0.7)[point,label=above :$A$] {};
\node (B) at (0.7,-0.7)[point,label=above :$B$] {};

\draw (A)--(O);
\draw (O)--(B);
\draw (B)--(A);

\node [above] at (-0.4,-0.4) {$r$};
\node [above] at (0.4,-0.4) {$r$};

\end{tikzpicture}
}
	\end{center}
	\caption{Circle Definitions}
	\label{ch4_circle_def}	
\end{figure}
%\item
%	In Fig. \ref{ch4_circle_def}, $A$ and $B$ are points on the circle.  The line $AB$ is known as a chord of the circle.

%
%
%\item
%	\label{ch4_prob_circle_subtend}
%	In Fig. \ref{ch4_circle_subtend}  
%%Show that $\angle AOB = 2\angle ACB $.
%
%\begin{figure}[!ht]
%	\begin{center}
%		
%		%\includegraphics[width=\columnwidth]{./figs/ch4_circle_subtend}
%		%\vspace*{-10cm}
%		\resizebox{\columnwidth}{!}{\begin{tikzpicture}
[
 scale=2,
  >=stealth,
  point/.style = {draw, circle, fill = black, inner sep = 1pt},
]
\node (B) at (-2,-2)[point,label = below left:${B}$] {};
\node (A) at (1,3)[point,label = above left:${A}$] {};
\node (C) at (4,-1)[point,label = below right:${C}$] {};
\draw (A) -- (B) -- (C) -- (A);

\node (D) at (1,-1.5)[point,label = below:$D$] {};

\node (F) at (-0.5,0.5)[point,label = left:$F$] {};

\node (E) at (2.5,1)[point,label = right:$E$] {};

\node (O) at (0.7962963,-0.27777778) [point,label = right:$O$] {};

\draw[thick, dashed] (D) -- (O);
\draw (F) -- (O);
\draw (E) -- (O);
\draw (A) -- (O);
\draw (B) -- (O);
\draw (C) -- (O);

\tkzMarkRightAngle[fill=blue!20,size=.3](A,E,O)
\tkzMarkRightAngle[fill=blue!20,size=.3](A,F,O)
\tkzMarkRightAngle[fill=blue!20,size=.3](O,D,C)


\node [below] at (3.65,0) {\strut $\frac{b}{2}$};
\node [below] at (2.2,2) {\strut  $\frac{b}{2}$};


\node [below] at (2.5,-1.2) {\strut  $\frac{a}{2}$};    
\node [below] at (0,-1.65) {\strut  $\frac{a}{2}$};  
      
\node [below] at (-1.7,-1) {\strut  $\frac{c}{2}$}; 
\node [below] at (0,2) {\strut  $\frac{c}{2}$};   

\def\rad{3.284101453883}
%\coordinate [point, label={above : $O$ }] (O) at (0, 0);
\draw (O) circle (\rad);

\end{tikzpicture}
}
%%		\resizebox{\columnwidth}{!}{\begin{tikzpicture}
[scale =2,>=stealth,point/.style = {draw, circle, fill = black, inner sep = 1pt},]

\def\rad{3}
\coordinate [point, label={above right: $O$ }] (O) at (0, 0);
\draw (O) circle (\rad);
\node (A) at (-2.24,-2)[point,label=below left :$A$] {};
\node (B) at (2.24,-2)[point,label=below right :$B$] {};
\node (C) at (0,3)[point,label=above :$C$] {};
\node (D) at (0,-2)[point,label=below :$D$] {};

\draw (A)--(B);
\draw (B)--(C);
\draw (C)--(A);
\draw (O)--(A);
\draw (O)--(B);
\draw [thick,dashed] (C)--(D);
\tkzMarkAngle[size=0.5](A,C,O)
\tkzMarkAngle[size=0.56](O,C,B)
\tkzMarkAngle[size=0.3](A,O,D)
\tkzMarkAngle[size=0.36](D,O,B)

\draw (-0.1,2.4) node{$\theta_1$};
\draw (0.15,2.3) node{$\theta_2$};

\node [above] at (-0.1,1){$r$};
\node [above] at (-1.1,-1){$r$};
\node [above] at (1.12,-1){$r$};
\node [above] at (-0.2,-0.5){$\alpha$};
\node [above] at (0.2,-0.6){$\beta$};

\end{tikzpicture}
}
%	\end{center}
%	\caption{Angle subtended by chord $AB$ at the centre $O$ is twice the angle subtended at $P$. }
%	\label{ch4_circle_subtend}	
%\end{figure}

%\solution In Fig. \ref{ch4_circle_subtend}, the triangleles $OPA$ and $OPB$ are isosceles. Hence,
%%
%\begin{align}
%\angle OCA = \angle OAC &= \theta_1 \\
%\angle OCB = \angle OBC &= \theta_2
%\end{align}
%%
%Also, $\alpha$ and $\beta$ are exterior angles corresponding to the triangle $AOC$ and $BOC$ respectively. Hence
%%
%\begin{align}
%\alpha &= 2\theta_1 \\
%\beta &= 2\theta_2
%\end{align}
%%
%Thus,
%%
%\begin{align}
%\angle AOB &= \alpha + \beta \\
%&= 2\brak{\theta_1 + \theta_2} \\
%&= 2\angle ACB
%\end{align}
%

%
\item
In Fig. \ref{ch4_circle_dia}, $AB$ is the diameter and passes through the centre $O$.  show that $\angle APB = 90^{\degree}$ .

%
\begin{figure}[!ht]
	\begin{center}
		
		%\includegraphics[width=\columnwidth]{./figs/ch4_circle_dia}
		%\vspace*{-10cm}
		\resizebox{\columnwidth}{!}{\begin{tikzpicture}
[scale =3,>=stealth,point/.style = {draw, circle, fill = black, inner sep = 1pt},]

\def\rad{2}
\coordinate [point, label={above : $O$ }] (O) at (0, 0);
\draw (O) circle (\rad);
\node (A) at (-2,0)[point,label=above left :$A$] {};
\node (B) at (2,0)[point,label=above right :$B$] {};
\node (P) at (0,2)[point,label=above :$P$] {};


\draw (A)--(B);
\draw (P)--(A);
\draw (P)--(B);

\tkzMarkRightAngle[size=.2](B,P,A)

\end{tikzpicture}}
	\end{center}
	\caption{Diameter of a circle.}
	\label{ch4_circle_dia}	
\end{figure}
\item
	In Fig. \ref{ch4_chord_product}, show that 
	\begin{equation}
	\begin{split}
\angle ABD &= \angle ACD \\
\angle CAB &= \angle CDB	
	\end{split}
	\end{equation}

\begin{figure}[!ht]
	\begin{center}
		
		%\includegraphics[width=\columnwidth]{./figs/ch4_chord_product}
		%\vspace*{-10cm}
		\resizebox{\columnwidth}{!}{\begin{tikzpicture}
[scale =2,>=stealth,point/.style = {draw, circle, fill = black, inner sep = 1pt},]

\def\rad{3}
\coordinate [point, label={right: $P$ }] (P) at (0, 0);
\draw (P) circle (\rad);
\node (A) at (-2.24,-2)[point,label=below left :$A$] {};
\node (C) at (2.24,-2)[point,label=below right :$C$] {};
\node (B) at (2.24,2)[point,label=above :$B$] {};
\node (D) at (-2.24,2)[point,label=below :$D$] {};

\draw (A)--(B);
\draw (C)--(D);
\draw [thick,dashed] (A)--(C);
\draw [thick,dashed] (B)--(D);
\tkzMarkAngle[size=0.3](P,D,B)
\tkzMarkAngle[size=0.3](D,B,P)
\tkzMarkAngle[size=0.3](B,P,D)
\tkzMarkAngle[size=0.3](A,P,C)
\tkzMarkAngle[size=0.3](C,A,P)
\tkzMarkAngle[size=0.3](P,C,A)

\draw (1.8,1.8) node{$\alpha$};
\draw (-1.84,1.84) node{$\beta$};
\draw (-1.8,-1.9) node{$\beta$};
\draw (1.8,-1.9) node{$\alpha$};
\draw (0,0.4) node{$\theta$};
\draw (0,-0.4) node{$\theta$};

\end{tikzpicture}}
	\end{center}
	\caption{$PA.PB = PC.PD$}
	\label{ch4_chord_product}	
\end{figure}
%
%
\solution Use Problem \ref{ch4_prob_circle_subtend}.
%
\item
	In Fig. \ref{ch4_chord_product}, show that the triangles $PAB$ and $PBD$ are similar

\solution Trivial using previous problem
\item
	In Fig. \ref{ch4_chord_product}, show that 
	\begin{equation}
	PA.PB = PC.PD
	\end{equation}

%
\solution Since triangles $PAC$ and $PBD$ are similar, 
%
\begin{align}
\frac{PA}{PD} &= \frac{PC}{PB} \\
\Rightarrow PA.PB &= PC.PD
\end{align}
%
\item
	Fig. \ref{fig:incirc_def} touches the sides of $\triangle ABC$ \eqref{ch3_angle_bisector} and is known as  the {\em incircle}.  The sides of the $\triangle$	are known as the {\em tangents} of the circle.

\begin{figure}[!ht]
	\begin{center}
		
		%\includegraphics[width=\columnwidth]{./figs/ch4_circle_def}
		%\vspace*{-10cm}
%		\resizebox{\columnwidth}{!}{\begin{tikzpicture}
[
scale =3,
>=stealth,
point/.style = {draw, circle, fill = black, inner sep = 1pt},
]
\def\rad{1}
\coordinate [point, label={above : $O$ }] (O) at (0, 0);
\draw (O) circle (\rad);
\node (A) at (-0.7,-0.7)[point,label=above :$A$] {};
\node (B) at (0.7,-0.7)[point,label=above :$B$] {};

\draw (A)--(O);
\draw (O)--(B);
\draw (B)--(A);

\node [above] at (-0.4,-0.4) {$r$};
\node [above] at (0.4,-0.4) {$r$};

\end{tikzpicture}
}
		\resizebox{\columnwidth}{!}{\begin{tikzpicture}
[
 scale=2,
  >=stealth,
  point/.style = {draw, circle, fill = black, inner sep = 1pt},
]

\node (B) at (-2,-2)[point,label = below left:${B}$] {};
\node (A) at (1,3)[point,label = above left:${A}$] {};
\node (C) at (4,-1)[point,label = below right:${C}$] {};
\draw (A) -- (B) -- (C) -- (A);
\node (O) at (1.14738665, 0.14292163) [point,label = right:$O$] {};

\node (D) at (1.40982295, -1.43169617)[point,label = below right:$D$] {};
\draw (O) -- (D);

\node (E) at (2.42445681, 1.10072425)[point,label = below:$E$] {};
\draw (O) -- (E);
%\draw[thick, dashed] (A) -- (E);


\node (F) at (-0.22146164,  0.9642306)[point,label = left:$F$] {};
\draw (O) -- (F);
\draw (O) -- (C);
\draw (O) -- (B);
\draw[thick, dashed] (A) -- (O);
%\draw (O) -- (A);


\tkzMarkAngle[fill=blue!60,size=.3](O,B,F)
\tkzMarkAngle[fill=blue!40,size=.3](D,B,O)


\tkzMarkAngle[fill=red!60,size=.3](F,A,O)
\tkzMarkAngle[fill=red!40,size=.3](O,A,E)


\tkzMarkAngle[fill=orange!60,size=.3](E,C,O)
\tkzMarkAngle[fill=orange!40,size=.3](O,C,D)

\tkzMarkRightAngle[fill=blue!20,size=.3](A,E,O)
\tkzMarkRightAngle[fill=blue!20,size=.3](A,F,O)
\tkzMarkRightAngle[fill=blue!20,size=.3](O,D,C)

\def\rad{1.596337700952456}
%%\coordinate [point, label={above : $O$ }] (O) at (0, 0);
\draw (O) circle (\rad);

\end{tikzpicture}
}
%		\resizebox{\columnwidth}{!}{\begin{tikzpicture}
[
scale =3,
>=stealth,
point/.style = {draw, circle, fill = black, inner sep = 1pt},
]
\def\rad{1}
\coordinate [point, label={above : $O$ }] (O) at (0, 0);
\draw (O) circle (\rad);
\node (A) at (-0.7,-0.7)[point,label=above :$A$] {};
\node (B) at (0.7,-0.7)[point,label=above :$B$] {};

\draw (A)--(O);
\draw (O)--(B);
\draw (B)--(A);

\node [above] at (-0.4,-0.4) {$r$};
\node [above] at (0.4,-0.4) {$r$};

\end{tikzpicture}
}
	\end{center}
	\caption{Incircle and Tangent}
	\label{fig:incirc_def}	
\end{figure}
%
\item Tangents to a circle from any point outside the circle are equal.
\\
\solution See Fig. \ref{fig:incirc_def} and use \eqref{eq:tang_eq}.

\item
	Show that 
	\begin{equation}
	\label{ch5_sin_90}
	\sin 90^{\degree} = 1
	\end{equation}
\solution From 	Prolem \ref{prob:ch2_triang_area}
and Problem \ref{prob:tri_area_sin}, the area of the right angled $\triangle ABC$ in Fig. \ref{ch2_sq_ar} is 	
	\begin{equation}
	\label{ch5_sin_90_der}
	\frac{1}{2}ab = \frac{1}{2}ab\sin 90^{\degree} 
	\end{equation}
%
resulting in 	\eqref{ch5_sin_90}.
\item
	Show that 
	\begin{equation}
	\label{ch5_cos_90}
	\cos 90^{\degree} = 0
	\end{equation}

\solution 
Follows from the fact that $\sin 90\degree = 1$ and \eqref{eq:tri_sin_cos_id}.



%
% 	
\item
	The line $PX$ in Fig. \ref{ch4_tangent_def} touches the circle at exactly one  point $P$. 
%It is known as the tangent to the circle.
%
%
	Show that $OP \perp PX$.
% is the perpendicular to the line $PX$ as shown in the Fig. \ref{ch4_short_dist}. Show that $OP$ is the shortest distance between the point $O$ and the line $PX$. 

\solution Without loss of generality, let $0 \le \theta \le 90^{\degree}$. Using the cosine formula in $\triangle OPP_n$,\begin{align}
\brak{r+d_n}^2 > r^2,
\end{align}
%Let $P_1$ be a point on the line $PX$. Then $OPP_1$ is a right angled triangle.  Using Budhayana's theorem,
%
\begin{figure}[!ht]
	\begin{center}
		
		%\includegraphics[width=\columnwidth]{./figs/ch4_tangent_def}
		%\vspace*{-10cm}
		\resizebox{\columnwidth}{!}{\begin{tikzpicture}
[scale =2,>=stealth,point/.style = {draw, circle, fill = black, inner sep = 1pt},]

\def\rad{2}
\coordinate [point, label={right: $O$ }] (O) at (0, 2);
\draw (O) circle (\rad);
\node (P) at (0,0)[point,label=below :$P$] {};
\node (X) at (2,0)[point,label=right :$X$] {};
\node (Y) at (-4,0)[point,label=left :$Y$] {};
\node (P_n) at (-1,0)[point,label=below  :$P_n$] {};
\node (P_2) at (-2,0)[point,label=below  :$P_2$] {};
\node (P_1) at (-3,0)[point,label=below  :$P_1$] {};

\draw (O)--(P);
\draw (X)--(Y);
\draw (O)--(P_n);
\draw (O)--(P_2);
\draw (O)--(P_1);

\tkzMarkAngle[size=.2](O,P,P_1);
\draw (-0.2,0.2) node{$\theta$};

%\tkzMarkRightAngle[size=.2](O,P,P_1);
%\node [above] at (-2.3,0.24){\rotatebox{45}{$90-A$}};

\node [above] at (-0.6,0.8){\rotatebox{59}{$r+d_n$}};
\node [above] at (-1.1,0.8){\rotatebox{45}{$r+d_2$}};
\node [above] at (-1.35,1){\rotatebox{35}{$r+d_1$}};
\node [above] at (0.1,0.8){$r$};
\node [above] at (-0.9,0.8){$\dots$};
\node [above] at (-0.5,-0.3){$x_n$};
\node [above] at (-1.6,-0.3){$\dots$};
\node [above] at (-2.5,-0.3){$x_1$};

\end{tikzpicture}}
	\end{center}
	\caption{Tangent to a Circle.}
	\label{ch4_tangent_def}	
\end{figure}
%
%\begin{figure}[!ht]
%	\begin{center}
%		
%		%\includegraphics[width=\columnwidth]{./figs/ch4_short_dist}
%		%\vspace*{-10cm}
%		\resizebox{\columnwidth}{!}{\begin{tikzpicture}
[scale =2,>=stealth,point/.style = {draw, circle, fill = black, inner sep = 1pt},]


\node (O) at (0,3)[point,label=above :$O$] {};
\node (P) at (0,0)[point,label=below :$P$] {};
\node (P_1) at (-1.5,0)[point,label=below :$P_1$] {};
\node (P_2) at (-3,0)[point,label=below :$P_2$] {};
\node (X) at (2,0)[point,label=right :$X$] {};
\node (Y) at (-4,0)[point,label=left :$Y$] {};
\draw (O)--(P);
\draw (X)--(Y);
\draw (O)--(P_1);
\draw (O)--(P_2);

\tkzMarkRightAngle[size=.2](O,P,X);

\end{tikzpicture}}
%	\end{center}
%	\caption{Shortest distance from $O$ to line $PX$}
%	\label{ch4_short_dist}	
%\end{figure}

%
\begin{align}
%\begin{split}
\brak{r+d_n}^2 = r^2 + x_n^2 - 2rx_n\cos\theta > r^2 
\\
\implies  0 <\cos\theta < \frac{x_n}{2r},
%OP_1^2 &= OP^2 + PP_1^2 \\
%\Rightarrow OP_1 > OP
%\end{split}
\end{align}
%
where $x_n$ can be made as small as we choose.  Thus, 
%
\begin{align}
\cos \theta = 0 \implies \theta  = 90 ^{\degree}
\end{align}
from \eqref{ch5_cos_90}.

%\solution In Fig. \ref{ch4_tangent_def}, we can see that $OP$ is is the radius of the circle and the length of all line segments from $O$ to the line $PX > r$.  Using the result of the previous 
%problem, it is obvious that $OP \perp PX$. 
%
	%
\item
In Fig. \ref{ch4_tangent_prod} show that 
%
\begin{equation}
\angle PCA = \angle PBC
\end{equation}
%
$O$ is the centre of the circle and $PC$ is the tangent.

	\begin{figure}[!ht]
		\begin{center}
			
			%\includegraphics[width=\columnwidth]{./figs/ch4_tangent_prod}
			%\vspace*{-10cm}
			\resizebox{\columnwidth}{!}{\begin{tikzpicture}
[scale =2,>=stealth,point/.style = {draw, circle, fill = black, inner sep = 1pt},]

\def\rad{2}
\coordinate [point, label={above: $O$ }] (O) at (0, 2);
\draw (O) circle (\rad);
\node (P) at (-4,0)[point,label=below :$P$] {};
\node (C) at (0,0)[point,label=below :$C$] {};
\node (A) at (-1.92,1.45)[point,label=above left :$A$] {};
\node (B) at (1.2,3.6)[point,label=above right :$B$] {};
\draw (O)--(C);
\draw (P)--(C);
\draw (P)--(B);
\draw (A)--(C);
\draw (B)--(C);

\draw [thick,dashed](A)--(O);
\tkzMarkRightAngle[size=.2](P,C,O);
\tkzMarkAngle[size=.3](A,B,C);
\tkzMarkAngle[size=.4](O,C,A);
%\tkzMarkAngle[size=.5](B,C,O);
%\tkzMarkAngle[size=.3](P,A,C);
\tkzMarkAngle[size=.2](C,A,O);
\tkzMarkAngle[size=.2](A,O,C);

\node [above] at (0.65,1.5){$r$};
\node [above] at (-0.9,1.7){$r$};
\node [above] at (0.1,1){$r$};
%\draw (-1.9,1) node{$\theta$};

\draw (0.95,3.3) node{$\alpha$};
\draw (-0.2,1.7) node{$2\alpha$};
\draw (-0.2,0.5) node{$90-\alpha$};
\draw (-1.4,1.4) node{$90-\alpha$};
%\draw (.1,.6) node{$\phi$};

\end{tikzpicture}}
		\end{center}
		\caption{$PA.PB = PC^2$.}
		\label{ch4_tangent_prod}	
	\end{figure}
	%

%
\solution Obvious from the figure once we observe that $\triangle OAC$ is isosceles.
%
%
\item
	In Fig. \ref{ch4_tangent_prod}, show that the triangles $PAC$ and $PBC$ are similar.

\solution From the previous problem, it is obvious that corresponding angles of both triangles are equal.  Hence they are similar.
%
\item
	Show that $PA.PB = PC^2$

\solution Since $\Delta PAC \sim \Delta PBC$, their sides are in the same ratio.  Hence,
%
\begin{align}
\frac{PA}{PC} &= \frac{PC}{PB} \\
\Rightarrow PA.PB &=PC^2
\end{align}
%
\item
Given that $PA.PB = PC^2$, show that $PC$ is a tangent to the circle.

%
\item
	In Fig. \ref{ch4_chord_tangent_prod}, show that\begin{equation}
	PA.PB = PC.PD
	\end{equation}

%
\begin{figure}[!ht]
	\begin{center}
		
		%\includegraphics[width=\columnwidth]{./figs/ch4_chord_tangent_prod}
		%\vspace*{-10cm}
		\resizebox{\columnwidth}{!}{\begin{tikzpicture}
[scale =2,>=stealth,point/.style = {draw, circle, fill = black, inner sep = 1pt},]

\def\rad{2}
\coordinate [point, label={above: $O$ }] (O) at (0, 2);
\draw (O) circle (\rad);
\node (P) at (-5,0)[point,label=below :$P$] {};
\node (C) at (-1.65,0.9)[point,label=below :$C$] {};
\node (A) at (-2,2.3)[point,label=left :$A$] {};
\node (B) at (0.3,4)[point,label=above :$B$] {};
\node (D) at (2,2)[point,label=right :$D$] {};
\draw (C)--(D);
\draw (A)--(B);
\draw (A)--(P);
\draw (P)--(C);
\end{tikzpicture}}
	\end{center}
	\caption{$PA.PB = PC^2$.}
	\label{ch4_chord_tangent_prod}	
\end{figure}

\solution Draw a tangent and use the previous problem.
\end{enumerate}
