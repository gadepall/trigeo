%
\begin{enumerate}[label=\thesection.\arabic*.,ref=\thesection.\theenumi]
%
\item
In  \figref{fig:tri_baudh}, 
show that 
%
\begin{equation}
\label{ch1_budh_basic}
b = a \cos \theta + c \sin \theta
\end{equation}
%
\begin{figure}[!ht]
	\begin{center}
		\resizebox{\columnwidth}{!}{%Code by GVV Sharma
%December 7, 2019
%released under GNU GPL
%Proof of Baudhyana Theorem

\begin{tikzpicture}
[scale=2,>=stealth,point/.style={draw,circle,fill = black,inner sep=0.5pt},]

%Triangle sides
\def\a{4}
\def\c{3}
\def\b{sqrt(\a^2+\c^2)}

%Trigonometric ratios
\def\ct{\a/\b}
\def\st{\c/\b}

%perp distance
\def\r{\a*\st}

%Section Ratio
\def\k{1.2}


%Labeling points
\node (A) at (0,\c)[point,label=above right:$A$] {};
\node (B) at (0, 0)[point,label=below left:$B$] {};
\node (C) at (\a, 0)[point,label=below right:$C$] {};

%Foot of perpendicular

\node (D) at ($({\r*\st}, {\r*\ct})$)[point,label=above right:$D$] {};


%Drawing triangle ABC
\draw (A) -- node[left] {$\textrm{c}$} (B) -- node[below] {$\textrm{a}$} (C) -- node[above,xshift=2mm] {$\textrm{b}$} (A);

%Joining BD
\draw (B)--(D);

%Drawing and marking angles
\tkzMarkAngle[fill=orange!40,size=0.5cm,mark=](A,C,B)
\tkzMarkAngle[fill=orange!40,size=0.4cm,mark=](D,B,A)
\tkzMarkAngle[fill=green!40,size=0.5cm,mark=](B,A,C)
\tkzMarkAngle[fill=green!40,size=0.5cm,mark=](C,B,D)
\tkzMarkRightAngle[fill=blue!20,size=.2](A,B,C)
\tkzMarkRightAngle[fill=blue!20,size=.2](B,D,A)
\tkzLabelAngle[pos=0.65](A,C,B){$\theta$}
\tkzLabelAngle[pos=0.65](A,B,D){$\theta$}
\tkzLabelAngle[pos=1](B,A,C){\rotatebox{-45}{$\alpha = 90\degree -\theta$}}
\tkzLabelAngle[pos=0.65](C,B,D){$\alpha$}

\end{tikzpicture}
}
	\end{center}
	\caption{Baudhayana Theorem}
	\label{fig:tri_baudh}	
\end{figure}
\solution We observe that
%
\begin{align}
CD &= a \cos \theta \\
AD &= c \cos\alpha = c \sin \theta \quad \brak{\text{From} \quad 
	\eqref{eq:tri_baudh_comp}	
	%\eqref{eq:tri_90-ang}
}
\end{align}
%
Thus,
\begin{equation}
CD + AD = b = a \cos \theta + c \sin \theta
\end{equation}
\item
From \eqref{ch1_budh_basic}, show that
%
\begin{equation}
%
\label{eq:tri_sin_cos_id}
\sin ^2 \theta + \cos ^2 \theta = 1
\end{equation}
%
\solution Dividing both sides of \eqref{ch1_budh_basic} by $b$, 
\begin{align}
1 &= \frac{a}{b}\cos\theta + \frac{c}{b}\sin\theta\\
\Rightarrow &\sin ^2 \theta + \cos ^2 \theta = 1 \quad \brak{\text{from} \quad \eqref{eq:tri_trig_defs}}
\end{align}

\item
	Using \eqref{ch1_budh_basic}, show that
	\begin{equation}
	\label{eq:tri_baudh}
	b^2 = a^2 + c^2
	\end{equation}
	\eqref{eq:tri_baudh} is known as the Baudhayana theorem.  It is also known as the Pythagoras theorem.
\\
\solution From \eqref{ch1_budh_basic},
\begin{align}
b &= a\frac{a}{b} + c \frac{c}{b} \quad \brak{\text{from} \quad \eqref{eq:tri_trig_defs}}\\
\implies b^2 &= a^2 + c^2
\end{align}
%
\item
\label{prob:tri_area_sin}
	Show that the area of $\Delta ABC$ in Fig. 	\ref{fig:tri_sss}	is $\frac{1}{2}ab \sin C$.
\begin{figure}[!ht]
	\begin{center}
			\resizebox{\columnwidth}{!}{%Code by GVV Sharma
%December 7, 2019
%released under GNU GPL
%Drawing a triangle given 3 sides

\begin{tikzpicture}
[scale=2,>=stealth,point/.style={draw,circle,fill = black,inner sep=0.5pt},]

%Triangle sides
\def\a{6}
\def\b{5}
\def\c{4}
 
%Coordinates of A
%\def\p{{\a^2+\c^2-\b^2}/{(2*\a)}}
\def\p{2.25}
\def\q{{sqrt(\c^2-\p^2)}}

%Labeling points
\node (A) at (\p,\q)[point,label=above right:$A$] {};
\node (B) at (0, 0)[point,label=below left:$B$] {};
\node (C) at (\a, 0)[point,label=below right:$C$] {};

%Foot of perpendicular

\node (D) at (\p,0)[point,label=above right:$D$] {};

%Drawing triangle ABC
\draw (A) -- node[left] {$\textrm{c}$} (B) -- node[below] {$\textrm{a}$} (C) -- node[above,xshift=2mm] {$\textrm{b}$} (A);

%Drawing altitude AD
\draw (A) -- node[left] {$\textrm{h}$}(D);

%Drawing and marking angles
%\tkzMarkAngle[fill=orange!40,size=0.5cm,mark=](A,C,B)
%\tkzMarkAngle[fill=orange!40,size=0.4cm,mark=](D,B,A)
%\tkzMarkAngle[fill=green!40,size=0.5cm,mark=](B,A,C)
%\tkzMarkAngle[fill=green!40,size=0.5cm,mark=](C,B,D)
\tkzMarkRightAngle[fill=blue!20,size=.2](A,D,B)
%\tkzMarkRightAngle[fill=blue!20,size=.2](B,D,A)
%\tkzLabelAngle[pos=0.65](A,C,B){$\theta$}
%\tkzLabelAngle[pos=0.65](A,B,D){$\theta$}
%\tkzLabelAngle[pos=1](B,A,C){\rotatebox{-45}{$\alpha = 90\degree -\theta$}}
%\tkzLabelAngle[pos=0.65](C,B,D){$\alpha$}

\end{tikzpicture}
}
	\end{center}
	\caption{Area of a Triangle}
	\label{fig:tri_sss}	
\end{figure}

\solution We have
%
\begin{equation}
ar\brak{\Delta ABC} = \frac{1}{2}ah = \frac{1}{2}ab\sin C \quad \brak{\because \quad h = b \sin C}.
\label{eq:tri_area_sin}
\end{equation}

\item
	Show that 
	\begin{equation}
	\frac{\sin A}{a} = \frac{\sin B}{b} = \frac{\sin C}{c}
	\end{equation}

\solution Fig. \ref{fig:tri_sss} can be suitably modified to obtain 
\begin{align}
ar\brak{\Delta ABC} = 
\frac{1}{2}ab\sin C = \frac{1}{2}bc\sin A = \frac{1}{2}ca\sin B
	\label{tri:area-sin}
\end{align}
Dividing the above by $abc$, we obtain
	\begin{equation}
\label{eq:tri_sin_form}
	\frac{\sin A}{a} = \frac{\sin B}{b} = \frac{\sin C}{c}
	\end{equation}
This is known as the sine formula.	
%
%
\item Show that 
%
\begin{align}
\label{eq:trig_id_sin_inc}
\alpha > \beta \implies \sin \alpha > \sin \beta
\end{align}
%

\begin{figure}[!ht]
	\begin{center}
		
		%\includegraphics[width=\columnwidth]{./figs/fig:tri_sin_inc}
		%\vspace*{-10cm}
		\resizebox{\columnwidth}{!}{\begin{tikzpicture}
[scale =3,>=stealth,point/.style = {draw, circle, fill = black, inner sep = 1pt},]

\node (A) at (0,3)[point,label=above :$A$] {};
\node (B) at (3,0)[point,label=below :$B$] {};
\node (C) at (0,0)[point,label=below :$C$] {};
\node (D) at (0,1.5)[point,label=left :$D$] {};
\draw (A)--(B);
\draw (C)--(B);
\draw (A)--(C);
\draw (B)--(D);
\tkzMarkAngle[size=.4](A,B,D);
\tkzMarkAngle[size=.3](D,B,C);
\tkzMarkRightAngle[size=.15](A,C,B);

\node [above] at (1.6,1.5){$c$};
\node [below] at (1.6,0){$a$};
\node [below] at (1.6,1){$l$};
\node [above] at (-0.2,1.5){$b$};
\node [above] at (2.5,0){$\theta_2$};
\node [above] at (2.5,0.3){$\theta_1$};
\end{tikzpicture}}
	\end{center}
	\caption{}
	%\caption{$\sin \brak{\theta_1+\theta_2} = \sin\theta_1\cos\theta_2 + \cos\theta_1\sin\theta_2$}
	\label{fig:tri_sin_inc}	
\end{figure}
\solution In Fig. \ref{fig:tri_sin_inc}, 	
%
\begin{align}
ar\brak{\triangle ABD} &< ar \brak{\triangle ABC}
\\
\implies \frac{1}{2}lc \sin \theta_1 &<  \frac{1}{2}ac \sin \brak{\theta_1 + \theta_2 }
\\
\implies \frac{l}{a} &< \frac{\sin \brak{\theta_1 + \theta_2 }}{\sin \theta_1}
\\
\text{or, } 1 < \frac{l}{a} &< \frac{\sin \brak{\theta_1 + \theta_2 }}{\sin \theta_1}
\end{align}
%
from Theorem \ref{them:hyp_largest}, yielding 
\begin{align}
\implies \frac{\sin \brak{\theta_1 + \theta_2 }}{\sin \theta_1} > 1.
\end{align}

This proves \eqref{eq:trig_id_sin_inc}.
%From \eqref{eq:trig_id_sum_diff3},
%%
%\begin{multline}
% \sin \theta_1 - \sin \theta_2 = 2\sin\brak{\frac{\theta_1-\theta_2}{2}}
%\\
%\times \cos\brak{\frac{\theta_1+\theta_2}{2}} > 0, \because \theta_1-\theta_2 > 0
%\end{multline}
%
\item
	Using 
	\figref{fig:tri_sin_inc},
%Fig. \ref{trig_id_sin_theta}, 
show that 
	%
\begin{equation}
\label{trig_id_sin_theta_eq}
\sin  \theta_1 = \sin \brak{\theta_1 + \theta_2}\cos \theta_2 - \cos\brak{\theta_1+\theta_2}\sin\theta_2
\end{equation}	
	%
\iffalse
\begin{figure}[!ht]
	\begin{center}
		
		%\includegraphics[width=\columnwidth]{./figs/trig_id_sin_theta}
		%\vspace*{-10cm}
		\resizebox{\columnwidth}{!}{\begin{tikzpicture}
[scale =3,>=stealth,point/.style = {draw, circle, fill = black, inner sep = 1pt},]

\node (A) at (0,3)[point,label=above :$A$] {};
\node (B) at (3,0)[point,label=below :$B$] {};
\node (C) at (0,0)[point,label=below :$C$] {};
\node (D) at (0,1.5)[point,label=left :$D$] {};
\draw (A)--(B);
\draw (C)--(B);
\draw (A)--(C);
\draw (B)--(D);
\tkzMarkAngle[size=.4](A,B,D);
\tkzMarkAngle[size=.3](D,B,C);
\tkzMarkRightAngle[size=.15](A,C,B);

\node [above] at (1.6,1.5){$c$};
\node [below] at (1.6,0){$a$};
\node [below] at (1.6,1){$l$};
\node [above] at (-0.2,1.5){$b$};
\node [above] at (2.5,0){$\theta_2$};
\node [above] at (2.5,0.3){$\theta_1$};
\end{tikzpicture}}
	\end{center}
	\caption{$\sin \brak{\theta_1+\theta_2} = \sin\theta_1\cos\theta_2 + \cos\theta_1\sin\theta_2$}
	\label{trig_id_sin_theta}	
\end{figure}
\fi
%

\solution The following equations can be obtained from the figure using the forumula for the area of a triangle
%
\begin{align}
ar \brak{\Delta ABC} &= \frac{1}{2}ac \sin\brak{\theta_1 + \theta_2} \\
&= ar \brak{\Delta BDC} + ar \brak{\Delta ADB} \\
&= \frac{1}{2}cl \sin{\theta_1} + \frac{1}{2}al \sin{\theta_2} \\ 
&= \frac{1}{2}ac \sin{\theta_1} \sec \theta_2 + \frac{1}{2}a^2 \tan{\theta_2} 
\end{align}
$\brak{\because
	l = a \sec \theta_2}$.  From the above,
\begin{align}
\sin\brak{\theta_1 + \theta_2} &=  \sin{\theta_1} \sec \theta_2 + \frac{a}{c} \tan{\theta_2} \\
	&=  \sin{\theta_1} \sec \theta_2 
+ \cos\brak{\theta_1 + \theta_2} \tan{\theta_2} 
\end{align}
Multiplying both sides by $\cos \theta_2$,
\begin{align}
\sin\brak{\theta_1 + \theta_2}\cos{\theta_2} =  \sin{\theta_1}  
+ \cos\brak{\theta_1 + \theta_2} \sin\theta_2  
\end{align}
%
resulting in
\eqref{trig_id_sin_theta_eq}.
\item Find Hero's formula for the area of a triangle.
\\
\solution 
%In Fig. \ref{fig:rt_triangle}, from Baudhayana's theorem, 
%\begin{align}
%\label{eq:tri_geo_baudh}
%b^2 = a^2+c^2 &
%\\
%=b^2\cos^2C+b^2\sin^2C &
%\\
%\implies \cos^2C+\sin^2C &= 1
%\end{align}
%
%In Fig. \ref{fig:tri_const_ex_cos_form}, 
From \eqref{prob:tri_area_sin}, the area of $\triangle ABC$ is 
{\footnotesize
\begin{align}
\label{eq:tri_geo_area_sin_form}
 \frac{1}{2}ab\sin C
%\\
&=\frac{1}{2}ab\sqrt{1-\cos^2C} 
\quad \brak{\text{from } \eqref{eq:tri_sin_cos_id}
%\eqref{eq:tri_geo_baudh}
}
\\
&=\frac{1}{2}ab\sqrt{1-\brak{\frac{a^2+b^2-c^2}{2ab}}^2} \brak{\text{from } \eqref{eq:tri_cos_form}
}
\\
&=\frac{1}{4}\sqrt{\brak{2ab}^2-\brak{a^2+b^2-c^2}}
\\
&=\frac{1}{4}\sqrt{\brak{2ab+a^2+b^2-c^2}\brak{2ab-a^2-b^2+c^2}}
\\
&= \frac{1}{4}\sqrt{\cbrak{\brak{a+b}^2-c^2}\cbrak{c^2-\brak{a-b}^2}}
\\
&= \frac{1}{4}\sqrt{\brak{a+b+c}\brak{a+b-c}\brak{a+c-b}\brak{b+c-a}}
\label{eq:tri_ex_hero_temp}
\end{align}
}
Substituting 
%
\begin{align}
s=\frac{a+b+c}{2}
\end{align}
%
in \eqref{eq:tri_ex_hero_temp}, the area of $\triangle ABC$ is 
%
\begin{align}
\label{eq:tri_area_hero}
\sqrt{s\brak{s-a}\brak{s-b}\brak{s-c}}
\end{align}
%
This is known as Hero's formula.
\item
	Prove the following identities 
	%
	\begin{enumerate}
\item 
\begin{equation}
		\label{trig_id_sin_diff}
\sin\brak{\alpha - \beta} = \sin \alpha \cos \beta - \cos \alpha \sin \beta.
\end{equation}
\item 
\begin{equation}
\cos\brak{\alpha + \beta} = \cos \alpha \cos \beta - \sin \alpha \sin \beta.
		\label{trig_id_cos_diff}
\end{equation}

	\end{enumerate}
	%

\solution In \eqref{trig_id_sin_theta_eq}, let
%
\begin{equation}
\begin{split}
\theta_1 + \theta_2 &= \alpha \\
\theta_2 &=  \beta
\end{split}
\end{equation}
%
This gives \eqref{trig_id_sin_diff}.  In \eqref{trig_id_sin_diff}, replace $\alpha$ by 
%
$90{\degree} - \alpha$.  This results in
%
\begin{align}
\sin\brak{90{\degree} - \alpha - \beta}
	&=
\sin \brak{90{\degree} -\alpha} \cos \beta - \cos \brak{90{\degree} -\alpha} \sin \beta \\
	\implies \cos\brak{\alpha + \beta} &= \cos \alpha \cos \beta - \sin \alpha \sin \beta
\end{align}
% 
\item
	Using \eqref{trig_id_sin_theta_eq} and \eqref{trig_id_cos_diff}, show that
\begin{align}
\label{trig_id_sin_sum}
\sin\brak{\theta_1 + \theta_2} &= \sin\theta_1  \cos\theta_2 + \cos\theta_1\sin\theta_2
\\
\cos\brak{\theta_1 - \theta_2} &= \cos\theta_1  \cos\theta_2  \sin\theta_1\sin\theta_2
\label{trig_id_cos_sum}
\end{align}

%
\solution From \eqref{trig_id_sin_theta_eq},
%
\begin{align}
 \sin \brak{\theta_1 + \theta_2}\cos \theta_2 =\sin  \theta_1 +\cos\brak{\theta_1+\theta_2}\sin\theta_2 
\end{align}
%
Using \eqref{trig_id_cos_diff} in the above,
%
\begin{multline}
\sin \brak{\theta_1 + \theta_2}\cos \theta_2 
=\sin  \theta_1 +\lbrak{\cos \theta_1\cos\theta_2 }
\\	
\rbrak{	- \sin \theta_1\sin\theta_2}\sin\theta_2 
\end{multline}
%
which can be expressed as
%
\begin{multline}
\sin \brak{\theta_1 + \theta_2}\cos \theta_2 
=\sin  \theta_1 
\\
+\cos \theta_1\cos\theta_2 \sin\theta_2 
		- \sin \theta_1\sin^2\theta_2
\end{multline}
%
Since
%
\begin{equation}
\sin^2\theta_2 = 1- \cos^2\theta_2, 
\end{equation}
%
we obtain
%
\begin{multline}
\sin \brak{\theta_1 + \theta_2}\cos \theta_2 
=\cos \theta_1\cos\theta_2 \sin\theta_2 
+ \sin \theta_1\cos^2\theta_2
\end{multline}
%
resulting in
%
\begin{equation}
\sin \brak{\theta_1 + \theta_2}
=\cos \theta_1 \sin\theta_2 
+ \sin \theta_1\cos\theta_2
\end{equation}
%
after factoring out $\cos \theta_2$.  Using a similar approach, \eqref{trig_id_cos_sum} can also be proved.
\item Show that 
\begin{align}
\label{eq:trig_id_sum_diff1}
\sin \theta_1 + \sin \theta_2 &= 2\sin\brak{\frac{\theta_1+\theta_2}{2}}\cos\brak{\frac{\theta_1-\theta_2}{2}}
\\
\label{eq:trig_id_sum_diff2}
\cos \theta_1 + \cos \theta_2 &= 2\cos\brak{\frac{\theta_1+\theta_2}{2}}\cos\brak{\frac{\theta_1-\theta_2}{2}}
\\
\label{eq:trig_id_sum_diff3}
\sin \theta_1 - \sin \theta_2 &= 2\sin\brak{\frac{\theta_1-\theta_2}{2}}\cos\brak{\frac{\theta_1+\theta_2}{2}}
\\
\label{eq:trig_id_sum_diff4}
\cos \theta_1 - \cos \theta_2 &= 2\sin\brak{\frac{\theta_1+\theta_2}{2}}\cos\brak{\frac{\theta_2-\theta_1}{2}}
\end{align}
%
\\
\solution Let 
%
\begin{align}
\label{eq:trig_id_ang_sum_diff}
\begin{split}
\theta_1 = \alpha + \beta
\\
\theta_2 = \alpha - \beta
\end{split}
\end{align}
%
From \eqref{trig_id_sin_sum},
%
\begin{align}
\sin \theta_1 + \sin \theta_2  &= \sin \brak{\alpha + \beta} + \sin \brak{\alpha - \beta}
\\
&= \sin \alpha \cos \beta + \cos \alpha \sin \beta 
\\
&+\sin \alpha \cos \beta - \cos \alpha \sin \beta
\\
&= 2 \sin \alpha \cos \beta
\end{align}
%
resulting in \eqref{eq:trig_id_sum_diff1}
%
\begin{align}
\because \alpha &= \frac{\theta_1 +\theta_2}{2}
\\
\beta &= \frac{\theta_1 -\theta_2}{2}
\end{align}
from \eqref{eq:trig_id_ang_sum_diff}.  Other identities may be proved similarly.
%
\item Show that 
  \begin{align}
\label{eq:trig-id-2A-sin}
	  \sin 2 \theta = 2 \sin \theta \cos \theta
	  \\
	  \cos 2 \theta &= 1 - 2 \sin^2 \theta 
	  =  2 \cos^2 \theta -1
	  \\
	  &= \cos^2 \theta -\sin^2 \theta 
\label{eq:trig-id-2A-cos}
  \end{align}

\end{enumerate}
