%
\subsection{Sine and Cosine Formula}
\begin{enumerate}[label=\thesubsection.\arabic*.,ref=\thesubsection.\theenumi]
%
	\item 
A right angled triangle looks like Fig. \ref{fig:tri_right_angle}.
\begin{figure}[!ht]
\centering
\resizebox{0.6\columnwidth}{!}{%Code by GVV Sharma
%December 6, 2019
%released under GNU GPL
%Drawing a right angled triangle

\begin{tikzpicture}[scale=2]

%Triangle sides
\def\a{4}
\def\c{3}

%Marking coordiantes
\coordinate [label=above:$A$] (A) at (0,\c);
\coordinate [label=left:$B$] (B) at (0,0);
\coordinate [label=right:$C$] (C) at (\a,0);

%Drawing triangle ABC
\draw (A) -- node[left] {$\textrm{c}$} (B) -- node[below] {$\textrm{a}$} (C) -- node[above,,xshift=2mm] {$\textrm{b}$} (A);

%Drawing and marking angles

\tkzMarkAngle[color=red, fill=red!40,size=0.3](A,C,B)
\tkzMarkRightAngle[fill=blue!20,size=.3](A,B,C)
\tkzLabelAngle[pos=0.65](A,C,B){$\theta$}
\end{tikzpicture}
}
\caption{Right Angled Triangle}
\label{fig:tri_right_angle}	
\end{figure}
with angles $\angle A,\angle B$ and $\angle C$ and sides $a, b$ and $c$.  The unique feature of this triangle is $\angle B$ which is defined to be $90\degree$.
\item
	For simplicity, let the greek letter $\theta = \angle C$.  We have the following definitions.
\begin{equation}
\label{eq:tri_trig_defs}
\begin{matrix}
	\sin \theta = \frac{c}{b} & 	\cos \theta = \frac{a}{b} \\[1ex]
	\tan \theta = \frac{c}{a} & \cot \theta = \frac{1}{\tan \theta} \\[1ex]
	\csc \theta = \frac{1}{\sin \theta} & \sec \theta = \frac{1}{\cos \theta}
	\end{matrix}
\end{equation}
\item  
	\begin{equation}
	\cos \theta = \sin \brak{90\degree - \theta}
	\label{eq:tri_baudh_comp}	
	\end{equation}
\item
In  \figref{fig:tri_baudh}, 
show that 
%
\begin{equation}
\label{ch1_budh_basic}
b = a \cos \theta + c \sin \theta
\end{equation}
%
\begin{figure}[!ht]
	\begin{center}
		\resizebox{0.6\columnwidth}{!}{%Code by GVV Sharma
%December 7, 2019
%released under GNU GPL
%Proof of Baudhyana Theorem

\begin{tikzpicture}
[scale=2,>=stealth,point/.style={draw,circle,fill = black,inner sep=0.5pt},]

%Triangle sides
\def\a{4}
\def\c{3}
\def\b{sqrt(\a^2+\c^2)}

%Trigonometric ratios
\def\ct{\a/\b}
\def\st{\c/\b}

%perp distance
\def\r{\a*\st}

%Section Ratio
\def\k{1.2}


%Labeling points
\node (A) at (0,\c)[point,label=above right:$A$] {};
\node (B) at (0, 0)[point,label=below left:$B$] {};
\node (C) at (\a, 0)[point,label=below right:$C$] {};

%Foot of perpendicular

\node (D) at ($({\r*\st}, {\r*\ct})$)[point,label=above right:$D$] {};


%Drawing triangle ABC
\draw (A) -- node[left] {$\textrm{c}$} (B) -- node[below] {$\textrm{a}$} (C) -- node[above,xshift=2mm] {$\textrm{b}$} (A);

%Joining BD
\draw (B)--(D);

%Drawing and marking angles
\tkzMarkAngle[fill=orange!40,size=0.5cm,mark=](A,C,B)
\tkzMarkAngle[fill=orange!40,size=0.4cm,mark=](D,B,A)
\tkzMarkAngle[fill=green!40,size=0.5cm,mark=](B,A,C)
\tkzMarkAngle[fill=green!40,size=0.5cm,mark=](C,B,D)
\tkzMarkRightAngle[fill=blue!20,size=.2](A,B,C)
\tkzMarkRightAngle[fill=blue!20,size=.2](B,D,A)
\tkzLabelAngle[pos=0.65](A,C,B){$\theta$}
\tkzLabelAngle[pos=0.65](A,B,D){$\theta$}
\tkzLabelAngle[pos=1](B,A,C){\rotatebox{-45}{$\alpha = 90\degree -\theta$}}
\tkzLabelAngle[pos=0.65](C,B,D){$\alpha$}

\end{tikzpicture}
}
	\end{center}
	\caption{Baudhayana Theorem}
	\label{fig:tri_baudh}	
\end{figure}
\solution We observe that
%
\begin{align}
CD &= a \cos \theta \\
AD &= c \cos\alpha = c \sin \theta \quad \brak{\text{From} \quad 
	\eqref{eq:tri_baudh_comp}	
	%\eqref{eq:tri_90-ang}
}
\end{align}
%
Thus,
\begin{equation}
CD + AD = b = a \cos \theta + c \sin \theta
\end{equation}
\item
From \eqref{ch1_budh_basic}, show that
%
\begin{equation}
%
\label{eq:tri_sin_cos_id}
\sin ^2 \theta + \cos ^2 \theta = 1
\end{equation}
%
\solution Dividing both sides of \eqref{ch1_budh_basic} by $b$, 
\begin{align}
1 &= \frac{a}{b}\cos\theta + \frac{c}{b}\sin\theta\\
\Rightarrow &\sin ^2 \theta + \cos ^2 \theta = 1 \quad \brak{\text{from} \quad \eqref{eq:tri_trig_defs}}
\end{align}
%
\item 
From \eqref{eq:tri_sin_cos_id}
\begin{align}
\label{eq:tri_sin_cos_minmax}
	\abs{\sin \theta} \le 1,\
	\abs{\cos \theta} \le 1
\end{align}
\item
	Using \eqref{ch1_budh_basic}, show that
	\begin{equation}
	\label{eq:tri_baudh}
	b^2 = a^2 + c^2
	\end{equation}
	\eqref{eq:tri_baudh} is known as the Baudhayana theorem.  It is also known as the Pythagoras theorem.
\\
\solution From \eqref{ch1_budh_basic},
\begin{align}
b &= a\frac{a}{b} + c \frac{c}{b} \quad \brak{\text{from} \quad \eqref{eq:tri_trig_defs}}\\
\implies b^2 &= a^2 + c^2
\end{align}
%
\item In a right angled triangle, the hypotenuse is the longest side.
\label{them:hyp_largest}
\\
\solution From 
	\eqref{eq:tri_baudh},
\begin{align}
	a \le b,\ c \le b.
\end{align}
\item
\label{prob:tri_area_sin}
	Show that the area of $\Delta ABC$ in Fig. 	\ref{fig:tri_sss}	is $\frac{1}{2}ab \sin C$.
\begin{figure}[!ht]
	\begin{center}
			\resizebox{0.6\columnwidth}{!}{%Code by GVV Sharma
%December 7, 2019
%released under GNU GPL
%Drawing a triangle given 3 sides

\begin{tikzpicture}
[scale=2,>=stealth,point/.style={draw,circle,fill = black,inner sep=0.5pt},]

%Triangle sides
\def\a{6}
\def\b{5}
\def\c{4}
 
%Coordinates of A
%\def\p{{\a^2+\c^2-\b^2}/{(2*\a)}}
\def\p{2.25}
\def\q{{sqrt(\c^2-\p^2)}}

%Labeling points
\node (A) at (\p,\q)[point,label=above right:$A$] {};
\node (B) at (0, 0)[point,label=below left:$B$] {};
\node (C) at (\a, 0)[point,label=below right:$C$] {};

%Foot of perpendicular

\node (D) at (\p,0)[point,label=above right:$D$] {};

%Drawing triangle ABC
\draw (A) -- node[left] {$\textrm{c}$} (B) -- node[below] {$\textrm{a}$} (C) -- node[above,xshift=2mm] {$\textrm{b}$} (A);

%Drawing altitude AD
\draw (A) -- node[left] {$\textrm{h}$}(D);

%Drawing and marking angles
%\tkzMarkAngle[fill=orange!40,size=0.5cm,mark=](A,C,B)
%\tkzMarkAngle[fill=orange!40,size=0.4cm,mark=](D,B,A)
%\tkzMarkAngle[fill=green!40,size=0.5cm,mark=](B,A,C)
%\tkzMarkAngle[fill=green!40,size=0.5cm,mark=](C,B,D)
\tkzMarkRightAngle[fill=blue!20,size=.2](A,D,B)
%\tkzMarkRightAngle[fill=blue!20,size=.2](B,D,A)
%\tkzLabelAngle[pos=0.65](A,C,B){$\theta$}
%\tkzLabelAngle[pos=0.65](A,B,D){$\theta$}
%\tkzLabelAngle[pos=1](B,A,C){\rotatebox{-45}{$\alpha = 90\degree -\theta$}}
%\tkzLabelAngle[pos=0.65](C,B,D){$\alpha$}

\end{tikzpicture}
}
	\end{center}
	\caption{Area of a Triangle}
	\label{fig:tri_sss}	
\end{figure}

\solution We have
%
\begin{equation}
ar\brak{\Delta ABC} = \frac{1}{2}ah = \frac{1}{2}ab\sin C \quad \brak{\because \quad h = b \sin C}.
\label{eq:tri_area_sin}
\end{equation}

\item
	Show that 
	\begin{equation}
	\frac{\sin A}{a} = \frac{\sin B}{b} = \frac{\sin C}{c}
	\end{equation}

\solution 
\figref{fig:tri_sss} can be suitably modified to obtain 
\begin{align}
ar\brak{\Delta ABC} = 
\frac{1}{2}ab\sin C = \frac{1}{2}bc\sin A = \frac{1}{2}ca\sin B
	\label{tri:area-sin}
\end{align}
Dividing the above by $abc$, we obtain
	\begin{equation}
\label{eq:tri_sin_form}
	\frac{\sin A}{a} = \frac{\sin B}{b} = \frac{\sin C}{c}
	\end{equation}
This is known as the sine formula.	
%
\item 
	In \figref{fig:tri_isoc}, $AB = AC$.  Show that 
\begin{align}
	\angle B
	= \angle C 
\end{align}
\begin{figure}[H]
	\centering
		\resizebox{0.6\columnwidth}{!}{%Code by GVV Sharma
%July 8, 2023
%released under GNU GPL
%Drawing the triangle and angle bisectors


\begin{tikzpicture}
[scale=2,>=stealth,point/.style={draw,circle,fill=black,inner sep=0.5pt}]

% Coordinates of the triangle vertices
\def\a{5}
\def\b{4}
\def\c{4}

% Coordinates of A
\def\p{2.25}
\def\q{{sqrt(\c^2-\p^2)}}

% Labeling points
\node (A) at (\p,\q)[point,label=above right:$A$] {};
\node (B) at (0,0)[point,label=below left:$B$] {};
\node (C) at (\a,0)[point,label=below right:$C$] {};

% Drawing triangle ABC
\draw (A) -- (B) -- (C) -- (A);

% Calculating the coordinates of E using the angle bisector theorem
%\coordinate (E) at ($(A)!\b/(\b+\c)!(C)$);

% Drawing the angle bisector BE
%\draw[dashed] (B) -- (E);

% Labeling point E
%\node (E) at (E)[point,label=below right:$E$] {};

% Calculating the coordinates of F using the angle bisector theorem
%\coordinate (F) at ($(A)!\c/(\b+\c)!(B)$);

% Drawing the angle bisector CF
%\draw[dashed] (C) -- (F);

% Labeling point F
%\node (F) at (F)[point,label=below left:$F$] {};

% Calculating the intersection point O of BE and CF
%\coordinate (O) at (intersection of B--E and C--F);

% Drawing the intersection point O
%\node (O) at (O)[point,label=above:$O$] {};


\end{tikzpicture}


}
	\caption{}
	\label{fig:tri_isoc}
\end{figure}
\solution Using the sine formula, 
\begin{align}
	\frac{AB}{\sin C}
	&=\frac{AC}{\sin B}
	\\
	\implies \sin B &= \sin C
	\text{ or, } \angle B = \angle C.
\end{align}
%\end{enumerate}
\item
In Fig. \ref{fig:tri_cosine_formula}, show that
%
\begin{equation}
\label{eq:tri_cos_mat}
\begin{pmatrix}
0 & c & b \\
c & 0 & a \\
b & a & 0
\end{pmatrix}
\begin{pmatrix}
\cos A \\
\cos B \\
\cos C
\end{pmatrix}
= 
\begin{pmatrix}
a\\
b\\
c
\end{pmatrix}
\end{equation}
%
%
\begin{figure}[!ht]
	\begin{center}
		
		%\includegraphics[width=0.6\columnwidth]{./figs/ch2_triang_ar}
		%\vspace*{-10cm}
		\resizebox{0.6\columnwidth}{!}{%Code by GVV Sharma
%December 7, 2019
%released under GNU GPL
%Drawing a triangle given 3 sides

\begin{tikzpicture}
[scale=2,>=stealth,point/.style={draw,circle,fill = black,inner sep=0.5pt},]

%Triangle sides
\def\a{6}
\def\b{5}
\def\c{4}
 
%Coordinates of A
%\def\p{{\a^2+\c^2-\b^2}/{(2*\a)}}
\def\p{2.25}
\def\q{{sqrt(\c^2-\p^2)}}

%Labeling points
\node (A) at (\p,\q)[point,label=above right:$A$] {};
\node (B) at (0, 0)[point,label=below left:$B$] {};
\node (C) at (\a, 0)[point,label=below right:$C$] {};

%Foot of perpendicular

\node (D) at (\p,0)[point,label=above right:$D$] {};

%Drawing triangle ABC
\draw (A) -- node[left] {$\textrm{c}$} (B) -- node[below] {$\textrm{a}$} (C) -- node[above,xshift=2mm] {$\textrm{b}$} (A);

%Drawing altitude AD
\draw (A) -- node[left] {$\textrm{h}$}(D);

\tkzMarkRightAngle[fill=blue!20,size=.2](A,D,B)

\node [below] at ($(B)!0.5!(D)$) {$x$};
\node [below] at ($(C)!0.5!(D)$) {$y$};

\end{tikzpicture}
}
	\end{center}
	\caption{The cosine formula}
	\label{fig:tri_cosine_formula}	
\end{figure}
\solution From Fig. \ref{fig:tri_cosine_formula}, 
%
\begin{align}
	a = x + y &= b \cos C + c \cos B = \myvec{  \cos C & \cos B } \myvec{ b \\ c }
	\\
&=\myvec{0 & b & c } \myvec{ \cos A \\ \cos C \\ \cos B } 
\end{align}
%
Similarly,
%
\begin{align}
b &= c \cos A + a \cos C 
=\myvec{c & 0 & a } \myvec{ \cos A \\ \cos C \\ \cos B } 
	\\
c &= b \cos A + a \cos B
=\myvec{b & a & 0 } \myvec{ \cos A \\ \cos C \\ \cos B } 
\end{align}
%
The above equations can be expressed in matrix form as
\eqref{eq:tri_cos_mat}.

\item Show that 
\begin{equation}
\label{eq:tri_cos_form}
\cos A = \frac{b^2+c^2-a^2}{2bc}
\end{equation}
%
\solution 
Using the properties of determinants,
%
\begin{align}
\cos A = \frac{
\begin{vmatrix}
a & c & b \\
b & 0 & a \\
c & a & 0
\end{vmatrix}
	}
	{
\begin{vmatrix}
0 & c & b \\
c & 0 & a \\
b & a & 0
\end{vmatrix}
	}
	=\frac{ab^2 + ac^2 - a^3}{abc + abc} 
= \frac{b^2 + c^2 - a^2}{2abc}
\end{align}
%
\item Find Hero's formula for the area of a triangle.
\\
\solution 
From \eqref{prob:tri_area_sin}, the area of $\triangle ABC$ is 
\begin{align}
\label{eq:tri_geo_area_sin_form}
 \frac{1}{2}ab\sin C
%\\
&=\frac{1}{2}ab\sqrt{1-\cos^2C} 
\quad \brak{\text{from } \eqref{eq:tri_sin_cos_id}
%\eqref{eq:tri_geo_baudh}
}
\\
&=\frac{1}{2}ab\sqrt{1-\brak{\frac{a^2+b^2-c^2}{2ab}}^2} \brak{\text{from } \eqref{eq:tri_cos_form}
}
\\
&=\frac{1}{4}\sqrt{\brak{2ab}^2-\brak{a^2+b^2-c^2}}
\\
&=\frac{1}{4}\sqrt{\brak{2ab+a^2+b^2-c^2}\brak{2ab-a^2-b^2+c^2}}
\\
&= \frac{1}{4}\sqrt{\cbrak{\brak{a+b}^2-c^2}\cbrak{c^2-\brak{a-b}^2}}
\\
&= \frac{1}{4}\sqrt{\brak{a+b+c}\brak{a+b-c}\brak{a+c-b}\brak{b+c-a}}
\label{eq:tri_ex_hero_temp}
\end{align}
Substituting 
%
\begin{align}
s=\frac{a+b+c}{2}
\end{align}
%
in \eqref{eq:tri_ex_hero_temp}, the area of $\triangle ABC$ is 
%
\begin{align}
\label{eq:tri_area_hero}
\sqrt{s\brak{s-a}\brak{s-b}\brak{s-c}}
\end{align}
%
This is known as Hero's formula.
\item Show that 
%
\begin{align}
\label{eq:trig_id_sin_inc}
\alpha > \beta \implies \sin \alpha > \sin \beta
\end{align}
%

\begin{figure}[!ht]
	\begin{center}
		
		%\includegraphics[width=\columnwidth]{./figs/fig:tri_sin_inc}
		%\vspace*{-10cm}
		\resizebox{0.6\columnwidth}{!}{\begin{tikzpicture}
[scale =3,>=stealth,point/.style = {draw, circle, fill = black, inner sep = 1pt},]

\node (A) at (0,3)[point,label=above :$A$] {};
\node (B) at (3,0)[point,label=below :$B$] {};
\node (C) at (0,0)[point,label=below :$C$] {};
\node (D) at (0,1.5)[point,label=left :$D$] {};
\draw (A)--(B);
\draw (C)--(B);
\draw (A)--(C);
\draw (B)--(D);
\tkzMarkAngle[size=.4](A,B,D);
\tkzMarkAngle[size=.3](D,B,C);
\tkzMarkRightAngle[size=.15](A,C,B);

\node [above] at (1.6,1.5){$c$};
\node [below] at (1.6,0){$a$};
\node [below] at (1.6,1){$l$};
\node [above] at (-0.2,1.5){$b$};
\node [above] at (2.5,0){$\theta_2$};
\node [above] at (2.5,0.3){$\theta_1$};
\end{tikzpicture}}
	\end{center}
	\caption{}
	%\caption{$\sin \brak{\theta_1+\theta_2} = \sin\theta_1\cos\theta_2 + \cos\theta_1\sin\theta_2$}
	\label{fig:tri_sin_inc}	
\end{figure}
\solution In Fig. \ref{fig:tri_sin_inc}, 	
%
\begin{align}
ar\brak{\triangle ABD} &< ar \brak{\triangle ABC}
\\
\implies \frac{1}{2}lc \sin \theta_1 &<  \frac{1}{2}ac \sin \brak{\theta_1 + \theta_2 }
\\
\implies \frac{l}{a} &< \frac{\sin \brak{\theta_1 + \theta_2 }}{\sin \theta_1}
\\
\text{or, } 1 < \frac{l}{a} &< \frac{\sin \brak{\theta_1 + \theta_2 }}{\sin \theta_1}
\end{align}
%
from Theorem \ref{them:hyp_largest}, yielding 
\begin{align}
\implies \frac{\sin \brak{\theta_1 + \theta_2 }}{\sin \theta_1} > 1.
\end{align}

This proves \eqref{eq:trig_id_sin_inc}.
\end{enumerate}
%
\subsection{Trigonometric Identities}
\begin{enumerate}[label=\thesubsection.\arabic*.,ref=\thesubsection.\theenumi]
\item
	Using 
	\figref{fig:tri_sin_inc},
%Fig. \ref{trig_id_sin_theta}, 
show that 
	%
\begin{equation}
\label{trig_id_sin_theta_eq}
\sin  \theta_1 = \sin \brak{\theta_1 + \theta_2}\cos \theta_2 - \cos\brak{\theta_1+\theta_2}\sin\theta_2
\end{equation}	
	%
\iffalse
\begin{figure}[!ht]
	\begin{center}
		
		%\includegraphics[width=0.6\columnwidth]{./figs/trig_id_sin_theta}
		%\vspace*{-10cm}
		\resizebox{0.6\columnwidth}{!}{\begin{tikzpicture}
[scale =3,>=stealth,point/.style = {draw, circle, fill = black, inner sep = 1pt},]

\node (A) at (0,3)[point,label=above :$A$] {};
\node (B) at (3,0)[point,label=below :$B$] {};
\node (C) at (0,0)[point,label=below :$C$] {};
\node (D) at (0,1.5)[point,label=left :$D$] {};
\draw (A)--(B);
\draw (C)--(B);
\draw (A)--(C);
\draw (B)--(D);
\tkzMarkAngle[size=.4](A,B,D);
\tkzMarkAngle[size=.3](D,B,C);
\tkzMarkRightAngle[size=.15](A,C,B);

\node [above] at (1.6,1.5){$c$};
\node [below] at (1.6,0){$a$};
\node [below] at (1.6,1){$l$};
\node [above] at (-0.2,1.5){$b$};
\node [above] at (2.5,0){$\theta_2$};
\node [above] at (2.5,0.3){$\theta_1$};
\end{tikzpicture}}
	\end{center}
	\caption{$\sin \brak{\theta_1+\theta_2} = \sin\theta_1\cos\theta_2 + \cos\theta_1\sin\theta_2$}
	\label{trig_id_sin_theta}	
\end{figure}
\fi
%

\solution The following equations can be obtained from the figure using the forumula for the area of a triangle
%
\begin{align}
ar \brak{\Delta ABC} &= \frac{1}{2}ac \sin\brak{\theta_1 + \theta_2} \\
&= ar \brak{\Delta BDC} + ar \brak{\Delta ADB} \\
&= \frac{1}{2}cl \sin{\theta_1} + \frac{1}{2}al \sin{\theta_2} \\ 
&= \frac{1}{2}ac \sin{\theta_1} \sec \theta_2 + \frac{1}{2}a^2 \tan{\theta_2} 
\end{align}
$\brak{\because
	l = a \sec \theta_2}$.  From the above,
\begin{align}
\sin\brak{\theta_1 + \theta_2} &=  \sin{\theta_1} \sec \theta_2 + \frac{a}{c} \tan{\theta_2} \\
	&=  \sin{\theta_1} \sec \theta_2 
+ \cos\brak{\theta_1 + \theta_2} \tan{\theta_2} 
\end{align}
Multiplying both sides by $\cos \theta_2$,
\begin{align}
\sin\brak{\theta_1 + \theta_2}\cos{\theta_2} =  \sin{\theta_1}  
+ \cos\brak{\theta_1 + \theta_2} \sin\theta_2  
\end{align}
%
resulting in
\eqref{trig_id_sin_theta_eq}.
\item
	Prove the following identities 
	%
	\begin{enumerate}
\item 
\begin{equation}
		\label{trig_id_sin_diff}
\sin\brak{\alpha - \beta} = \sin \alpha \cos \beta - \cos \alpha \sin \beta.
\end{equation}
\item 
\begin{equation}
\cos\brak{\alpha + \beta} = \cos \alpha \cos \beta - \sin \alpha \sin \beta.
		\label{trig_id_cos_diff}
\end{equation}

	\end{enumerate}
	%

\solution In \eqref{trig_id_sin_theta_eq}, let
%
\begin{equation}
\begin{split}
\theta_1 + \theta_2 &= \alpha \\
\theta_2 &=  \beta
\end{split}
\end{equation}
%
This gives \eqref{trig_id_sin_diff}.  In \eqref{trig_id_sin_diff}, replace $\alpha$ by 
%
$90{\degree} - \alpha$.  This results in
%
\begin{align}
\sin\brak{90{\degree} - \alpha - \beta}
	&=
\sin \brak{90{\degree} -\alpha} \cos \beta - \cos \brak{90{\degree} -\alpha} \sin \beta \\
	\implies \cos\brak{\alpha + \beta} &= \cos \alpha \cos \beta - \sin \alpha \sin \beta
\end{align}
% 
\item
	Using \eqref{trig_id_sin_theta_eq} and \eqref{trig_id_cos_diff}, show that
\begin{align}
\label{trig_id_sin_sum}
\sin\brak{\theta_1 + \theta_2} &= \sin\theta_1  \cos\theta_2 + \cos\theta_1\sin\theta_2
\\
\cos\brak{\theta_1 - \theta_2} &= \cos\theta_1  \cos\theta_2  \sin\theta_1\sin\theta_2
\label{trig_id_cos_sum}
\end{align}

%
\solution From \eqref{trig_id_sin_theta_eq},
%
\begin{align}
 \sin \brak{\theta_1 + \theta_2}\cos \theta_2 =\sin  \theta_1 +\cos\brak{\theta_1+\theta_2}\sin\theta_2 
\end{align}
%
Using \eqref{trig_id_cos_diff} in the above,
%
\begin{align}
\sin \brak{\theta_1 + \theta_2}\cos \theta_2 
=\sin  \theta_1 +\brak{\cos \theta_1\cos\theta_2 
	- \sin \theta_1\sin\theta_2}\sin\theta_2 
\end{align}
%
which can be expressed as
%
\begin{align}
\sin \brak{\theta_1 + \theta_2}\cos \theta_2 
=\sin  \theta_1 
+\cos \theta_1\cos\theta_2 \sin\theta_2 
		- \sin \theta_1\sin^2\theta_2
\end{align}
%
Since
%
\begin{equation}
\sin^2\theta_2 = 1- \cos^2\theta_2, 
\end{equation}
%
we obtain
%
\begin{align}
\sin \brak{\theta_1 + \theta_2}\cos \theta_2 
=\cos \theta_1\cos\theta_2 \sin\theta_2 
+ \sin \theta_1\cos^2\theta_2
\end{align}
%
resulting in
%
\begin{equation}
\sin \brak{\theta_1 + \theta_2}
=\cos \theta_1 \sin\theta_2 
+ \sin \theta_1\cos\theta_2
\end{equation}
%
after factoring out $\cos \theta_2$.  Using a similar approach, \eqref{trig_id_cos_sum} can also be proved.
\item Show that 
\begin{align}
\label{eq:trig_id_sum_diff1}
\sin \theta_1 + \sin \theta_2 &= 2\sin\brak{\frac{\theta_1+\theta_2}{2}}\cos\brak{\frac{\theta_1-\theta_2}{2}}
\\
\label{eq:trig_id_sum_diff2}
\cos \theta_1 + \cos \theta_2 &= 2\cos\brak{\frac{\theta_1+\theta_2}{2}}\cos\brak{\frac{\theta_1-\theta_2}{2}}
\\
\label{eq:trig_id_sum_diff3}
\sin \theta_1 - \sin \theta_2 &= 2\sin\brak{\frac{\theta_1-\theta_2}{2}}\cos\brak{\frac{\theta_1+\theta_2}{2}}
\\
\label{eq:trig_id_sum_diff4}
\cos \theta_1 - \cos \theta_2 &= 2\sin\brak{\frac{\theta_1+\theta_2}{2}}\cos\brak{\frac{\theta_2-\theta_1}{2}}
\end{align}
%
\\
\solution Let 
%
\begin{align}
\label{eq:trig_id_ang_sum_diff}
\begin{split}
\theta_1 = \alpha + \beta
\\
\theta_2 = \alpha - \beta
\end{split}
\end{align}
%
From \eqref{trig_id_sin_sum},
%
\begin{align}
\sin \theta_1 + \sin \theta_2  &= \sin \brak{\alpha + \beta} + \sin \brak{\alpha - \beta}
\\
&= \sin \alpha \cos \beta + \cos \alpha \sin \beta 
+\sin \alpha \cos \beta - \cos \alpha \sin \beta
\\
&= 2 \sin \alpha \cos \beta
\end{align}
%
resulting in \eqref{eq:trig_id_sum_diff1}
%
\begin{align}
\because \alpha = \frac{\theta_1 +\theta_2}{2}
,\
\beta = \frac{\theta_1 -\theta_2}{2}
\end{align}
from \eqref{eq:trig_id_ang_sum_diff}.  Other identities may be proved similarly.
%
\item Show that 
  \begin{align}
\label{eq:trig-id-2A-sin}
	  \sin 2 \theta &= 2 \sin \theta \cos \theta
	  \\
	  \cos 2 \theta &= 1 - 2 \sin^2 \theta 
	  =  2 \cos^2 \theta -1
	  \\
	  &= \cos^2 \theta -\sin^2 \theta 
\label{eq:trig-id-2A-cos}
  \end{align}
\end{enumerate}
%
\subsection{Incircle}
\begin{enumerate}[label=\thesubsection.\arabic*.,ref=\thesubsection.\theenumi]
\item  In  
	\figref{fig:tri_icentre}, the bisectors of $\angle B$ and $\angle C$	 meet at $\vec{I}$.
Show that $IA$ bisects $\angle A$.
\begin{figure}[!ht]
	\begin{center}
		
		\resizebox{0.6\columnwidth}{!}{%Code by GVV Sharma
%July 4, 2023
%released under GNU GPL
%Angle bisectors are concurrent

\begin{tikzpicture}
[scale=2,>=stealth,point/.style={draw,circle,fill = black,inner sep=0.5pt},]

%Triangle sides
\def\a{5}
\def\b{6}
\def\c{4}
 
%Coordinates of A
%\def\p{{\a^2+\c^2-\b^2}/{(2*\a)}}
\def\p{0.5}
\def\q{{sqrt(\c^2-\p^2)}}

%Labeling points
\node (A) at (\p,\q)[point,label=above right:$A$] {};
\node (B) at (0, 0)[point,label=below left:$B$] {};
\node (C) at (\a, 0)[point,label=below right:$C$] {};

%Circumcentre

\node (I) at (1.5,1.32287566)[point,label=right:$I$] {};
\node (D) at (1.5,0) {};
\node (E) at (2.375,2.3150324) {};
\node (F) at (0.1875,1.48823511) {};

%Drawing triangle ABC
%\draw (A) -- node[left] {$\textrm{c}$} (B) -- node[below] {$\textrm{a}$} (C) -- node[above,yshift=2mm] {$\textrm{b}$} (A);
\draw (A) --   (B) --   (C) --   (A);
%Drawing OA, OB, OC
\draw (I) -- node[right] {$l_1$}(A);
\draw (I) --  node[left,yshift=1mm] {$l_2$}(B);
\draw (I) --  node[right,yshift=2mm] {$l_3$}(C);

\iffalse
%Drawing OD, OE, OF
\draw (I) -- node[right] {$\textrm{r}$} (D);
\draw (I) -- node[below] {$\textrm{r}$} (E);
\draw (I) -- node[below] {$\textrm{r}$} (F);
\fi

\tkzMarkAngle[fill=green!60,size=.3](I,B,F)
\tkzMarkAngle[fill=green!40,size=.3](D,B,I)
%
%
\tkzMarkAngle[size=.3](F,A,I)
\tkzMarkAngle[size=.4](I,A,E)
\iffalse
\tkzMarkAngle[fill=red!60,size=.3](F,A,I)
\tkzMarkAngle[fill=red!40,size=.3](I,A,E)
\fi


\tkzMarkAngle[fill=orange!60](E,C,I)
\tkzMarkAngle[fill=orange!40](I,C,D)
%
\iffalse
\tkzMarkRightAngle[fill=blue!20,size=.3](A,E,I)
\tkzMarkRightAngle[fill=blue!20,size=.3](A,F,I)
\tkzMarkRightAngle[fill=blue!20,size=.3](I,D,C)

%Labeling x,y,z
\node (x1) at ($(B)!0.5!(D)$)[label=below:$x$] {};
\node (x2) at ($(B)!0.5!(F)$)[label=left:$x$] {};
\node (y1) at ($(C)!0.5!(D)$)[label=below:$y$] {};
\node (y2) at ($(C)!0.5!(E)$)[label=right:$y$] {};
\node (z1) at ($(A)!0.5!(E)$)[label=right:$z$] {};
\node (z2) at ($(A)!0.5!(F)$)[label=left:$z$] {};
\fi

\tkzLabelAngle[pos=0.65](B,A,I){$\theta$}
\tkzLabelAngle[pos=0.85,sloped](I,A,C){$A - \theta$}

\end{tikzpicture}
}
	\end{center}
	\caption{Incentre $I$ of $\triangle ABC$}
	\label{fig:tri_icentre}	
\end{figure}
\\
\solution
Using sine formula
in
\eqref{eq:tri_sin_form}
  \begin{align}
  \frac{l_1}{\sin\frac{C}{2}}
  = 
   \frac{l_3}{\sin \brak{A-\theta}}
  ,\
   \frac{l_3}{\sin\frac{B}{2}}
=\frac{l_2}{\sin\frac{C}{2}}
  ,\
\frac{l_2}{\sin \theta}
	  =
  \frac{l_1}{\sin\frac{B}{2}}
  \end{align}
  Multiplying the above equations, 
  \begin{align}
  \sin \theta &= \sin \brak{A - \theta}
  \\
	  \implies \theta &=A - \theta 
	  \text{ or, } \theta = \frac{A}{2}
  \end{align}
  \item 
	\figref{fig:tri_iradius}, 
	is obtained from 
	\figref{fig:tri_icentre}	
	with
  \begin{align}
	  ID \perp BC, \, 
	  IE \perp AC, \, 
	  IF \perp AB.
  \end{align}
  Show that 
  \begin{align}
	  ID=   
	  IE= 
	  IF=r 
	\label{eq:tri_iradius}	
  \end{align}
		\begin{figure}[!ht]
	\begin{center}
		\resizebox{0.6\columnwidth}{!}{%Code by GVV Sharma
%July 4, 2023
%released under GNU GPL
%Distance for the incentre to the sides is the same

\begin{tikzpicture}
[scale=2,>=stealth,point/.style={draw,circle,fill = black,inner sep=0.5pt},]

%Triangle sides
\def\a{5}
\def\b{6}
\def\c{4}
 
%Coordinates of A
%\def\p{{\a^2+\c^2-\b^2}/{(2*\a)}}
\def\p{0.5}
\def\q{{sqrt(\c^2-\p^2)}}

%Labeling points
\node (A) at (\p,\q)[point,label=above right:$A$] {};
\node (B) at (0, 0)[point,label=below left:$B$] {};
\node (C) at (\a, 0)[point,label=below right:$C$] {};

%Incentre

\node (I) at (1.5,1.32287566)[point,label=right:$I$] {};
\node (D) at (1.5,0)[point,label=below:$D$] {};
\node (E) at (2.375,2.3150324)[point,label=above right:$E$] {};
\node (F) at (0.1875,1.48823511)[point,label=left:$F$] {};

%Drawing triangle ABC
%\draw (A) -- node[left] {$\textrm{c}$} (B) -- node[below] {$\textrm{a}$} (C) -- node[above,yshift=2mm] {$\textrm{b}$} (A);
\draw (A) --   (B) --   (C) --   (A);
%Drawing OA, OB, OC
\draw (I) -- node[right] {$l_1$}(A);
\draw (I) --  node[left,yshift=1mm] {$l_2$}(B);
\draw (I) --  node[right,yshift=2mm] {$l_3$}(C);

%Drawing OD, OE, OF
\draw (I) -- node[right] {$\textrm{r}$} (D);
\draw (I) --  (E);
\draw (I) --  (F);

\tkzMarkAngle[fill=green!60,size=.3](I,B,F)
\tkzMarkAngle[fill=green!40,size=.3](D,B,I)
%
%
\iffalse
\tkzMarkAngle[size=.3](F,A,I)
\tkzMarkAngle[size=.4](I,A,E)
\fi
\tkzMarkAngle[fill=red!60,size=.3](F,A,I)
\tkzMarkAngle[fill=red!40,size=.3](I,A,E)


\tkzMarkAngle[fill=orange!60](E,C,I)
\tkzMarkAngle[fill=orange!40](I,C,D)
%
\tkzMarkRightAngle[fill=blue!20,size=.3](A,E,I)
\tkzMarkRightAngle[fill=blue!20,size=.3](A,F,I)
\tkzMarkRightAngle[fill=blue!20,size=.3](I,D,C)
\iffalse

%Labeling x,y,z
\node (x1) at ($(B)!0.5!(D)$)[label=below:$x$] {};
\node (x2) at ($(B)!0.5!(F)$)[label=left:$x$] {};
\node (y1) at ($(C)!0.5!(D)$)[label=below:$y$] {};
\node (y2) at ($(C)!0.5!(E)$)[label=right:$y$] {};
\node (z1) at ($(A)!0.5!(E)$)[label=right:$z$] {};
\node (z2) at ($(A)!0.5!(F)$)[label=left:$z$] {};

\tkzLabelAngle[pos=0.65](B,A,I){$\theta$}
\tkzLabelAngle[pos=0.85,sloped](I,A,C){$A - \theta$}
\fi

\end{tikzpicture}
}
	\end{center}
	\caption{Inradius $r$ of $\triangle ABC$}
	\label{fig:tri_iradius}	
\end{figure}
		\solution
In $\triangle$s $IDC$ and $IEC$, 
		\begin{align}
ID = IE=  \frac{l_3}{\sin\frac{C}{2}}
		\end{align}
		Similarly, 
in $\triangle$s $IEA$ and $IFA$, 
		\begin{align}
IF = IE=  \frac{l_1}{\sin\frac{A}{2}}
		\end{align}
		yielding 
	\eqref{eq:tri_iradius}	
  \item In
	\figref{fig:tri_iradius}, show that
  \begin{align}
	  BD=BF ,\, 
	  AE=AF ,\, 
	  CD=CE 
  \end{align}
  \solution  From 
\figref{fig:tri_iradius}, in $\triangle$s $IBD$ and $IBF$, 
		\begin{align}
			x = BD = BF = r \cot \frac{B}{2}
		\end{align}
		Similarly, other results can be obtained.
\item The circle with centre $\vec{I}$ and radius $r$ in  
	\figref{fig:tri_icircle}	
is known as the {\em incircle}. 
\begin{figure}[!ht]
	\begin{center}
		\resizebox{0.6\columnwidth}{!}{%Code by GVV Sharma
%December 10, 2019
%released under GNU GPL
%Drawing the incircle

\begin{tikzpicture}
[scale=2,>=stealth,point/.style={draw,circle,fill = black,inner sep=0.5pt},]

%Triangle sides
\def\a{5}
\def\b{6}
\def\c{4}

%Inradius
\def\r{1.3228756555322954}
 
%Coordinates of A
%\def\p{{\a^2+\c^2-\b^2}/{(2*\a)}}
\def\p{0.5}
\def\q{{sqrt(\c^2-\p^2)}}

%Labeling points
\node (A) at (\p,\q)[point,label=above right:$A$] {};
\node (B) at (0, 0)[point,label=below left:$B$] {};
\node (C) at (\a, 0)[point,label=below right:$C$] {};

%Circumcentre

\node (I) at (1.5,1.32287566)[point,label=right:$I$] {};
\node (D) at (1.5,0)[point,label=below:$D$] {};
\node (E) at (2.375,2.3150324)[point,label=above right:$E$] {};
\node (F) at (0.1875,1.48823511)[point,label=left:$F$] {};

%Drawing triangle ABC
\draw (A) -- node[left] {$\textrm{c}$} (B) -- node[below] {$\textrm{a}$} (C) -- node[above,yshift=2mm] {$\textrm{b}$} (A);
%Drawing OA, OB, OC
\draw (I) --  (A);
\draw (I) --  (B);
\draw (I) --  (C);

%Drawing OD, OE, OF
\draw (I) -- node[right] {$\textrm{r}$} (D);
\draw (I) -- node[below] {$\textrm{r}$} (E);
\draw (I) -- node[below] {$\textrm{r}$} (F);

%Drawing Incircle
\draw (I) circle (\r);



\tkzMarkAngle[fill=green!60,size=.3](I,B,F)
\tkzMarkAngle[fill=green!40,size=.3](D,B,I)
%
%
\tkzMarkAngle[fill=red!60,size=.3](F,A,I)
\tkzMarkAngle[fill=red!40,size=.3](I,A,E)


\tkzMarkAngle[fill=orange!60](E,C,I)
\tkzMarkAngle[fill=orange!40](I,C,D)
%
\tkzMarkRightAngle[fill=blue!20,size=.3](A,E,I)
\tkzMarkRightAngle[fill=blue!20,size=.3](A,F,I)
\tkzMarkRightAngle[fill=blue!20,size=.3](I,D,C)

%Labeling x,y,z
\node (x1) at ($(B)!0.5!(D)$)[label=below:$x$] {};
\node (x2) at ($(B)!0.5!(F)$)[label=left:$x$] {};
\node (y1) at ($(C)!0.5!(D)$)[label=below:$y$] {};
\node (y2) at ($(C)!0.5!(E)$)[label=right:$y$] {};
\node (z1) at ($(A)!0.5!(E)$)[label=right:$z$] {};
\node (z2) at ($(A)!0.5!(F)$)[label=left:$z$] {};


\end{tikzpicture}
}
	\end{center}
	\caption{Incircle of $\triangle ABC$}
	\label{fig:tri_icircle}	
\end{figure}
\item The lengths of tangents drawn from an external point to a circle are equal.
\item In an isosceles $\triangle ABC$, with $AB = AC$, $BE$ and $CF$ are the bisectors of $\angle B$ and $\angle C$ respectively.   Show that 
\begin{align}
BE = CF
	\label{eq:tri_isoc_ang_bsect}
\end{align}
\begin{figure}[H]
	\centering
		\resizebox{0.6\columnwidth}{!}{%Code by GVV Sharma
%July 8, 2023
%released under GNU GPL
%Drawing the triangle and angle bisectors


\begin{tikzpicture}
[scale=2,>=stealth,point/.style={draw,circle,fill=black,inner sep=0.5pt}]

% Coordinates of the triangle vertices
\def\a{5}
\def\b{4}
\def\c{4}

% Coordinates of A
\def\p{2.25}
\def\q{{sqrt(\c^2-\p^2)}}

% Labeling points
\node (A) at (\p,\q)[point,label=above right:$A$] {};
\node (B) at (0,0)[point,label=below left:$B$] {};
\node (C) at (\a,0)[point,label=below right:$C$] {};

% Drawing triangle ABC
\draw (A) -- (B) -- (C) -- (A);

% Calculating the coordinates of E using the angle bisector theorem
\coordinate (E) at ($(A)!\b/(\b+\c)!(C)$);

% Drawing the angle bisector BE
\draw[dashed] (B) -- (E);

% Labeling point E
\node (E) at (E)[point,label=below right:$E$] {};

% Calculating the coordinates of F using the angle bisector theorem
\coordinate (F) at ($(A)!\c/(\b+\c)!(B)$);

% Drawing the angle bisector CF
\draw[dashed] (C) -- (F);

% Labeling point F
\node (F) at (F)[point,label=below left:$F$] {};

% Calculating the intersection point O of BE and CF
\coordinate (I) at (intersection of B--E and C--F);

% Drawing the intersection point O
%\node (I) at (I)[point,label=above:$I$] {};


\end{tikzpicture}


}
	\caption{}
	\label{fig:tri_isoc_ang_bsect}
\end{figure}
\end{enumerate}
\solution
%
In $\triangle$ s $BEC$ and $BFC$, using the sine formula, 
\begin{align}
	\label{eq:tri_isoc_ang_bsect-inter}
	\frac{BE}{\sin C}
	&=\frac{BC}{\sin \brak{\frac{B}{2}+C}}
	\\
	\frac{CF}{\sin B}
	&=\frac{BC}{\sin \brak{\frac{B}{2}+C}}
\end{align}
$\because B = C$, from the above, we obtain
	\eqref{eq:tri_isoc_ang_bsect}.
\subsection{Circumcircle}
\begin{enumerate}[label=\thesubsection.\arabic*.,ref=\thesubsection.\theenumi]
\item In 
	\figref{fig:tri-isosc},	
\begin{figure}[!ht]
	\begin{center}
		\resizebox{0.6\columnwidth}{!}{%Code by GVV Sharma
%July 6, 2023
%Revised July 7, 2023
%released under GNU GPL
%The Isosceles Triangle

\begin{tikzpicture}
[scale=2,>=stealth,point/.style={draw,circle,fill = black,inner sep=0.5pt},]

%Triangle sides
\def\a{5}
\def\b{6}
\def\c{4}
 
%Coordinates of A
%\def\p{{\a^2+\c^2-\b^2}/{(2*\a)}}
\def\p{0.5}
\def\q{{sqrt(\c^2-\p^2)}}

%Labeling points
%\node (A) at (\p,\q)[point,label=above right:$A$] {};
\node (B) at (0, 0)[point,label=below left:$B$] {};
\node (C) at (\a, 0)[point,label=below right:$C$] {};

%Circumcentre

\node (O) at (2.5,1.70084013)[point,label=above right:$O$] {};

%Drawing triangle OBC
%\draw (A) -- node[left] {$\textrm{c}$} (B) -- node[below] {$\textrm{a}$} (C) -- node[above,yshift=2mm] {$\textrm{b}$} (A);
%Drawing OA, OB, OC
%\draw (O) -- node[left] {$\textrm{R}$} (A);
\draw (O) -- node[below] {${R}$} (B);
\draw (O) -- node[below] {${R}$} (C);
\draw (B) -- node[below] {${a}$} (C);

\tkzMarkAngle[fill=blue!50,size=.3](C,B,O)
\tkzMarkAngle[fill=blue!50,size=.3](O,C,B)


\tkzMarkAngle[fill=red!10](B,O,C)
\tkzLabelAngle[pos=0.3](B,O,C){$\theta$}
%\tkzMarkAngle[fill=red!10](A,C,O)

\iffalse
\tkzMarkAngle[fill=orange!50,size=.3](B,A,O)
\tkzMarkAngle[fill=orange!50,size=.3](O,B,A)

\tkzLabelAngle[pos=0.5](O,C,B){$\theta_1$}
\tkzLabelAngle[pos=0.5](O,B,C){$\theta_1$}
\tkzLabelAngle[pos=0.5](O,A,B){$\theta_2$}
\tkzLabelAngle[pos=0.5](O,B,A){$\theta_2$}
\tkzLabelAngle[pos=1.5](O,A,C){$\theta_3$}
\tkzLabelAngle[pos=1.5](O,C,A){$\theta_3$}
\fi

\end{tikzpicture}
}
	\end{center}
	\caption{Isosceles Triangle}
	\label{fig:tri-isosc}	
\end{figure}
\begin{align}
	OB = OC=R
\end{align}
Such a triangle is known as an isosceles triangle.  Show that
\begin{align}
	\angle B = \angle C
\end{align}
\solution 
Using
\eqref{eq:tri_sin_form},
\begin{align}
	\frac{\sin B}{R} &= \frac{\sin C}{R}
	\\
\implies	{\sin B} &= {\sin C}
\\
	\text{or, } \angle B &= \angle C.
\end{align}
\item In 
	\figref{fig:tri-isosc},	
	show that 
  \begin{align}
	  a = 2R \sin\frac{ \theta }{2}
\label{eq:crad_cos2a}
  \end{align}
		\solution In $\triangle OBC$,  using the cosine formula from
\eqref{eq:tri_cos_form},
\begin{align}
	\cos \theta &= \frac{R^2+R^2 - a^2}{2R^2} = 1 -\frac{a^2}{2R^2}
	\\
	\implies \frac{a^2}{2R^2}&= 2\sin^2\frac{\theta}{2}
\end{align}
yielding 
\eqref{eq:crad_cos2a}.

\item In 
	\label{prob:tri-ccentre-def}
	\figref{fig:tri-perp-bis}, 
\begin{align}
OB = OC=R, 	BD = DC.
\end{align}
Show that $OD \perp BC$.
%
\begin{figure}[!ht]
	\begin{center}
		
		\resizebox{0.6\columnwidth}{!}{%Code by GVV Sharma
%July 7, 2023
%released under GNU GPL
%The perpendicular bisector

\begin{tikzpicture}
[scale=2,>=stealth,point/.style={draw,circle,fill = black,inner sep=0.5pt},]

%Triangle sides
\def\a{5}
\def\b{6}
\def\c{4}
 
%Coordinates of A
%\def\p{{\a^2+\c^2-\b^2}/{(2*\a)}}
\def\p{0.5}
\def\q{{sqrt(\c^2-\p^2)}}

%Labeling points
%\node (A) at (\p,\q)[point,label=above right:$A$] {};
\node (B) at (0, 0)[point,label=below left:$B$] {};
\node (C) at (\a, 0)[point,label=below right:$C$] {};
%Mid point
\node (D) at ($(B)!0.5!(C)$)[point,label=below:$D$] {};

%Circumcentre

\node (O) at (2.5,1.70084013)[point,label=above right:$O$] {};

%Drawing triangle OBC
%\draw (A) -- node[left] {$\textrm{c}$} (B) -- node[below] {$\textrm{a}$} (C) -- node[above,yshift=2mm] {$\textrm{b}$} (A);
%Drawing OA, OB, OC
%\draw (O) -- node[left] {$\textrm{R}$} (A);
\draw (O) -- node[below] {${R}$} (B);
\draw (O) -- node[below] {${R}$} (C);
\draw (B) -- (C);
%\draw (B) -- node[below] {${a}$} (C);
\draw (O) --   (D);

\tkzMarkAngle[fill=blue!50,size=.3](C,B,O)
\tkzMarkAngle[fill=blue!50,size=.3](O,C,B)
\tkzMarkRightAngle[fill=blue!30,size=.2](C,D,O)


%\tkzMarkAngle[fill=red!10](O,A,C)
%\tkzMarkAngle[fill=red!10](A,C,O)

\iffalse
\tkzMarkAngle[fill=orange!50,size=.3](B,A,O)
\tkzMarkAngle[fill=orange!50,size=.3](O,B,A)

\tkzLabelAngle[pos=0.5](O,C,B){$\theta_1$}
\tkzLabelAngle[pos=0.5](O,B,C){$\theta_1$}
\tkzLabelAngle[pos=0.5](O,A,B){$\theta_2$}
\tkzLabelAngle[pos=0.5](O,B,A){$\theta_2$}
\tkzLabelAngle[pos=1.5](O,A,C){$\theta_3$}
\tkzLabelAngle[pos=1.5](O,C,A){$\theta_3$}
\fi

\end{tikzpicture}
}
	\end{center}
	\caption{Perpendicular bisector.}
	\label{fig:tri-perp-bis}	
%github/geometry/figs/
\end{figure}
\item In 
	\figref{fig:tri_ccentre},
$OD$ and $OE$ are the perpendicular bisectors of sides $BC$ and $AC$ respectively.  Show that 
$OA = R$.
\begin{figure}[!ht]
	\begin{center}
		
		\resizebox{0.6\columnwidth}{!}{%Code by GVV Sharma
%December 9, 2019
%released under GNU GPL
%Locating the circumcentre

\begin{tikzpicture}
[scale=2,>=stealth,point/.style={draw,circle,fill = black,inner sep=0.5pt},]

%Triangle sides
\def\a{5}
\def\b{6}
\def\c{4}
 
%Coordinates of A
%\def\p{{\a^2+\c^2-\b^2}/{(2*\a)}}
\def\p{0.5}
\def\q{{sqrt(\c^2-\p^2)}}

%Labeling points
\node (A) at (\p,\q)[point,label=above right:$A$] {};
\node (B) at (0, 0)[point,label=below left:$B$] {};
\node (C) at (\a, 0)[point,label=below right:$C$] {};

\node (D) at ($(B)!0.5!(C)$)[point,label=below:$D$] {};
\node (E) at ($(A)!0.5!(C)$)[point,label=right:$E$] {};
%Circumcentre

\node (O) at (2.5,1.70084013)[point,label=left:$O$] {};

%Drawing triangle ABC
\draw (A) -- node[left] {} (B) -- node[below] {} (C) -- node[above,yshift=2mm] {} (A);
%Drawing OA, OB, OC
\draw (O) -- node[left] {} (A);
\draw (O) -- node[below] {$\textrm{R}$} (B);
\draw (O) -- node[below] {$\textrm{R}$} (C);
\draw (O) --   (D);
\draw (O) --   (E);

\tkzMarkRightAngle[fill=blue!30,size=.2](C,D,O)
\tkzMarkRightAngle[fill=blue!30,size=.2](O,E,C)
%\tkzMarkAngle[fill=blue!50,size=.3](O,C,B)


%\tkzMarkAngle[fill=red!10](O,A,C)
%\tkzMarkAngle[fill=red!10](A,C,O)


%\tkzMarkAngle[fill=orange!50,size=.3](B,A,O)
%\tkzMarkAngle[fill=orange!50,size=.3](O,B,A)
%
%\tkzLabelAngle[pos=0.5](O,C,B){$\theta_1$}
%\tkzLabelAngle[pos=0.5](O,B,C){$\theta_1$}
%\tkzLabelAngle[pos=0.5](O,A,B){$\theta_2$}
%\tkzLabelAngle[pos=0.5](O,B,A){$\theta_2$}
%\tkzLabelAngle[pos=1.5](O,A,C){$\theta_3$}
%\tkzLabelAngle[pos=1.5](O,C,A){$\theta_3$}

\end{tikzpicture}
}
	\end{center}
	\caption{ Perpendicular bisectors of $\triangle ABC$ meet at $\vec{O}$.}
	\label{fig:tri_ccentre}	
\end{figure}
\item In
	\figref{fig:tri_ccentre},
	%\figref{fig:tri_ccircle-ang},
show that 
\begin{align}
\label{eq:tri_crad_R}
\frac{a}{\sin A} = \frac{b}{\sin B} = \frac{c}{\sin C} = 2R.
\end{align}
%
%
\solution
From 
\eqref{eq:ang-subtend-ccentre}
and 
\eqref{eq:crad_cos2a}
  \begin{align}
	  a = 2R \sin A
  \end{align}
  \item 
	\figref{fig:tri_ccircle-ang}
  shows the {\em circumcircle} of $\triangle ABC$.
\begin{figure}[!ht]
	\begin{center}
		\resizebox{0.6\columnwidth}{!}{%Code by GVV Sharma
%July 7, 2023
%released under GNU GPL
%The circumcircle

\begin{tikzpicture}
[scale=2,>=stealth,point/.style={draw,circle,fill = black,inner sep=0.5pt},]

%Triangle sides
\def\a{5}
\def\b{6}
\def\c{4}
\def\R{3.023715784073818}
 
%Coordinates of A
%\def\p{{\a^2+\c^2-\b^2}/{(2*\a)}}
\def\p{0.5}
\def\q{{sqrt(\c^2-\p^2)}}

% Vertices
\node (A) at (\p,\q)[point,label=above right:$A$] {};
\node (B) at (0, 0)[point,label=below left:$B$] {};
\node (C) at (\a, 0)[point,label=below right:$C$] {};
\iffalse
% Mid points
\node (D) at ($(B)!0.5!(C)$)[point,label=below:$D$] {};
\node (E) at ($(C)!0.5!(A)$)[point,label=right:$E$] {};
\node (F) at ($(B)!0.5!(A)$)[point,label=left:$F$] {};
\fi

%Circumcentre

\node (O) at (2.5,1.70084013)[point,label=right:$O$] {};

%Drawing triangle ABC
%\draw (A) -- node[above left, yshift=2mm] {$\textrm{c}$} (B) -- node[below right, xshift = 2mm] {$\textrm{a}$} (C) -- node[above,yshift=2mm] {$\textrm{b}$} (A);
\draw (A) --  (B) --  (C) --  (A);
%Drawing OA, OB, OC
%\draw (O) -- node[left] {$\textrm{R}$} (A);
\draw (O) -- node[below] {$\textrm{R}$} (B);
\draw (O) -- node[below] {$\textrm{R}$} (C);
\iffalse
%Drawing OD, OE, OF
\draw (O) --  (D);
\draw (O) --  (E);
\draw (O) --  (F);
\fi


%Drawing circumcircle
\draw (O) circle (\R);

\iffalse
\tkzMarkRightAngle[fill=blue!20,size=.2](O,D,C)
\tkzMarkRightAngle[fill=blue!20,size=.2](O,E,A)
\tkzMarkRightAngle[fill=blue!20,size=.2](O,F,B)
\fi

\tkzMarkAngle[fill=orange!10](B,O,C)
\tkzMarkAngle[fill=orange!10](B,A,C)
\tkzLabelAngle[pos=0.35](B,O,C){$\theta$}
\end{tikzpicture}
}
	\end{center}
	\caption{Circumcircle of $\triangle ABC$}
	\label{fig:tri_ccircle-ang}	
\end{figure}
\item Any point on the circle can be expressed as 
  \begin{align}
	  \vec{x} = \vec{O} + R\myvec{\cos \theta \\ \sin \theta}, \quad 0 \in \sbrak{0, 2\pi}
\label{eq:polar-ccentre}.
  \end{align}
  where $\vec{O}$ is the centere of the circle.
  \item Let
  \begin{align}
	  R = 1,\,
	  \vec{O} = \vec{0} ,\,
	  \vec{A} = \myvec{\cos \theta_1 \\ \sin \theta_1},\,
	  \vec{B} = \myvec{\cos \theta_2 \\ \sin \theta_2},\,
  \end{align}
Show that the distance
  \begin{align}
	  AB = \norm{\vec{A}-\vec{B}} = 
	   2 \sin \brak{\frac{\theta_1-\theta_2}{2}}
\label{eq:norm-polar-ccentre}
  \end{align}
  \solution 
  From 
\eqref{eq:polar-ccentre}.
  \begin{align}
	  \vec{A}-\vec{B} &= 
\myvec{\cos \theta_1-\cos \theta_2 \\ \sin \theta_1-\sin \theta_2}
\\
\implies 
	  \norm{\vec{A}-\vec{B}}^2 &= 
	  \brak{\vec{A}-\vec{B}}^\top
	  \brak{\vec{A}-\vec{B}} 
	  \\
	  &= 
	  \brak{\cos \theta_1-\cos \theta_2}^2 +\brak{\sin \theta_1-\sin \theta_2}^2
	  \\
	  &= 2\cbrak{1-
	  \cos \brak{\theta_1-\theta_2}} = 4 \sin^2 \brak{\frac{\theta_1-\theta_2}{2}}
  \end{align}
  yielding 
\eqref{eq:norm-polar-ccentre} from
\eqref{eq:trig-id-2A-cos}.
  \item In 
	\figref{fig:tri_ccircle-ang}, show that 
\begin{align}
	\cos A = \frac{
	  \brak{\vec{A}-\vec{B}}^\top
	  \brak{\vec{A}-\vec{B}} 
}
{
	  \norm{\vec{A}-\vec{B}}
	  \norm{\vec{A}-\vec{C}}
}
\label{eq:tri_cos_form-ccentre-norm},
\end{align}
  \item In 
	\figref{fig:tri_ccircle-ang},
show that 
  \begin{align}
	  \theta = 2A
\label{eq:ang-subtend-ccentre}.
  \end{align}
  \solution Let 
  \begin{align}
	  \vec{C} = \myvec{\cos \theta_3 \\ \sin \theta_3}
  \end{align}
  Then, 
  substituting 
  from 
\eqref{eq:norm-polar-ccentre}
in 
\eqref{eq:tri_cos_form},
%\eqref{eq:tri_cos_form-ccentre-norm},
  \begin{align}
	  \cos A &= \frac{4 \sin^2 \brak{\frac{\theta_1-\theta_2}{2}} +4\sin^2 \brak{\frac{\theta_1-\theta_3}{2}}-4 \sin^2 \brak{\frac{\theta_2-\theta_3}{2}}}{8 \sin \brak{\frac{\theta_1-\theta_2}{2}} \sin \brak{\frac{\theta_1-\theta_3}{2}}}
	  \\
	   &= \frac{2 \sin^2 \brak{\frac{\theta_1-\theta_2}{2}} +\cos \brak{{\theta_2-\theta_3}}- \cos \brak{{\theta_1-\theta_3}}}{4 \sin \brak{\frac{\theta_1-\theta_2}{2}} \sin \brak{\frac{\theta_1-\theta_3}{2}}}
  \end{align}
  from 
\eqref{eq:trig-id-2A-cos}. $\therefore$ From 
\eqref{eq:trig_id_sum_diff4},
  \begin{align}
	   \cos A &= \frac{2 \sin^2 \brak{\frac{\theta_1-\theta_2}{2}} +2\sin \brak{\frac{\theta_1-\theta_2}{2}}\sin \brak{\frac{\theta_1+\theta_2}{2}-\theta_3}}{4 \sin \brak{\frac{\theta_1-\theta_2}{2}} \sin \brak{\frac{\theta_1-\theta_3}{2}}}
	  \\
	   &= \frac{ \sin \brak{\frac{\theta_1-\theta_2}{2}} +\sin \brak{\frac{\theta_1+\theta_2}{2}-\theta_3}}{ 2\sin\brak{\frac{\theta_1-\theta_3}{2}}}
  \end{align}
  From 
\eqref{eq:trig_id_sum_diff1}, the above equation can be expressed as
  \begin{align}
\cos A	   &= \frac{ 2\sin \brak{\frac{\theta_1-\theta_3}{2}} \cos\brak{\frac{\theta_2-\theta_3}{2}}}{ 2\sin\brak{\frac{\theta_1-\theta_3}{2}}} = \cos\brak{\frac{\theta_2-\theta_3}{2}}
\label{eq:tri_ccentre_subtend-temp}
	   \\
	   \implies 2A &= \theta_2-\theta_3
\label{eq:tri_ccentre_subtend}
  \end{align}
  Similarly, 
  \begin{align}
	  \cos \theta = \frac{1 + 1 - 4\sin^2\brak{\frac{\theta_2-\theta_3}{2}}}{2} = \cos\brak{{\theta_2-\theta_3}}= \cos 2A
  \end{align}
\item
In Fig. \ref{fig:circ_tang_icept}, show that 
%
\begin{equation}
\theta = \alpha
		\label{fig:circ_tang_icept-equal}	
\end{equation}
%
\label{them:tang_icept_ang}
where $CP$ is the tangent.
	\begin{figure}[!ht]
		\begin{center}
			
			%\includegraphics[width=0.6\columnwidth]{./figs/fig:circ_tang_icept}
			%\vspace*{-10cm}
			\resizebox{0.6\columnwidth}{!}{\begin{tikzpicture}
[scale =2,>=stealth,point/.style = {draw, circle, fill = black, inner sep = 1pt},]

\def\rad{2}
\coordinate [point, label={above: $O$ }] (O) at (0, 2);
\draw (O) circle (\rad);
\node (P) at (-4,0)[point,label=below :$P$] {};
\node (C) at (0,0)[point,label=below :$C$] {};
\node (A) at (-1.92,1.45)[point,label=above left :$A$] {};
\node (B) at (1.2,3.6)[point,label=above right :$B$] {};
\tkzMarkAngle[fill=orange!5](A,B,C)
\tkzLabelAngle[pos=0.35](A,B,C){$\alpha$}
\tkzMarkAngle[fill=orange!5](A,C,P)
\tkzLabelAngle[pos=0.35](A,C,P){$\theta$}
\iffalse
\draw (O)--(C);
\fi
\draw (P)--(C);
\draw (P)--(B);
\draw (A)--(C);
\draw (B)--(C);
\iffalse
\draw [thick,dashed](A)--(O);
\tkzMarkRightAngle[size=.2](P,C,O);
\tkzMarkAngle[size=.4](O,C,A);
\tkzMarkAngle[size=.2](C,A,O);
\tkzMarkAngle[size=.2](A,O,C);
\node [above] at (0.65,1.5){$r$};
\node [above] at (-0.9,1.7){$r$};
\node [above] at (0.1,1){$r$};
\draw (0.95,3.3) node{$\alpha$};
\draw (-0.2,1.7) node{$2\alpha$};
\draw (-0.2,0.5) node{$90-\alpha$};
\draw (-1.4,1.4) node{$90-\alpha$};
\fi
%\tkzMarkAngle[size=.5](B,C,O);
%\tkzMarkAngle[size=.3](P,A,C);

%\draw (-1.9,1) node{$\theta$};

%\draw (.1,.6) node{$\phi$};

\end{tikzpicture}
}
		\end{center}
		\caption{$\theta= \alpha$.}
		\label{fig:circ_tang_icept}	
	\end{figure}
	%
	\solution
    Let
  \begin{align}
	  \vec{O} = \vec{0},\
	  \vec{A} = \myvec{\cos \theta_1 \\ \sin \theta_1},\,
	  \vec{B} =  \myvec{\cos \theta_2 \\ \sin \theta_2},\,
	  \vec{C} =  \myvec{\cos \theta_3 \\ \sin \theta_3}
  \end{align}
  Without loss of generality,  let 
  \begin{align}
	  \theta_3 = \frac{\pi}{2}
		\label{eq:circ_tang-line-t3}	
  \end{align}
  Then, 
  \begin{align}
	  \vec{C}-\vec{O} = \myvec{0 \\ 1}.
\implies	  \vec{C}-\vec{P} \equiv \myvec{1 \\ 0}
		\label{eq:circ_tang-line-pc},	
  \end{align}
  $\because CO \perp CP$.
From   
\eqref{eq:tri_cos_form-ccentre-norm},
and 
		\eqref{eq:circ_tang-line-pc},	
  \begin{align}
	  \cos \theta &= \frac{
		  \myvec{\cos \theta_3-\cos \theta_1 & \sin \theta_3-\sin \theta_1}
		  \myvec{1 \\ 0}
		  }
		  {
	   2 \sin \brak{\frac{\theta_1-\theta_3}{2}}
			  } 
			  \\
			  &=
	    \sin \brak{\frac{\theta_1+\theta_3}{2}}
	    =\cos\brak{\frac{\pi}{2}-\frac{\theta_1+\theta_3}{2}}
	    =\cos\brak{\frac{\pi}{4}-\frac{\theta_1}{2}}
  \end{align}
  upon substituting from 
		\eqref{eq:circ_tang-line-t3}.  Similarly, 	
		from
\eqref{eq:tri_ccentre_subtend-temp},
  \begin{align}
	  \cos \alpha = \cos \brak{\frac{\theta_1-\theta_3}{2}  }
	    =\cos\brak{\frac{\pi}{4}-\frac{\theta_1}{2}}
	  =\cos \theta
  \end{align}
%
\item
	In Fig. \ref{fig:circ_tang_icept}, show that $PA.PB = PC^2$.
\label{them:circ_tang_icept_prod}	
\\
\solution 
In $\triangle$s $APC$ and $BPC$, 
using
		\eqref{fig:circ_tang_icept-equal},	
  \begin{align}
	  \frac{AP}{\sin \theta} &= \frac{AC}{\sin P} 
	  \\
	  \frac{PC}{\sin \theta} &= \frac{BC}{\sin P} 
	  \\
	  \implies \frac{PC}{AP} &= \frac{BC}{AC}  \brak{= \frac{BP}{CP}}
  \end{align}
  which gives the desired result.
$\triangle$s $APC$ and $BPC$ are said to be {\em similar}.
\end{enumerate}
%
\subsection{Medians}
\begin{enumerate}[label=\thesubsection.\arabic*.,ref=\thesubsection.\theenumi]
  \item In 
	\figref{fig:tri_med}	
	\begin{align}
	AF = BF, \,
	AE = BE, 
	\end{align}
	and the medians $BE$ and $CF$ meet at $\vec{G}$.
	Show that
\begin{align}
	ar\brak{BEC}
	=ar\brak{BFC} = \frac{1}{2}ar\brak{ABC}
	\label{eq:median-area}
\end{align}
\begin{figure}[!ht]
	\begin{center}
		\resizebox{0.6\columnwidth}{!}{%Code by GVV Sharma
%July 8, 2023
%released under GNU GPL
%Drawing the medians

\begin{tikzpicture}
[scale=2,>=stealth,point/.style={draw,circle,fill = black,inner sep=0.5pt},]

%Triangle sides
\def\a{5}
\def\b{6}
\def\c{4}
 
%Coordinates of A
\def\p{2.25}
\def\q{{sqrt(\c^2-\p^2)}}

%Labeling points
\node (A) at (\p,\q)[point,label=above right:$A$] {};
\node (B) at (0, 0)[point,label=below left:$B$] {};
\node (C) at (\a, 0)[point,label=below right:$C$] {};

%Foot of median

%\node (D) at ($(B)!0.5!(C)$)[point,label=below:$D$] {};
\node (E) at ($(A)!0.5!(C)$)[point,label=right:$E$] {};
\node (F) at ($(B)!0.5!(A)$)[point,label=left:$F$] {};

%Drawing triangle ABC
\draw (A) -- node[] {} (B) -- node[below, yshift=-5mm] {$\textrm{a}$} (C) -- node[] {} (A);

%Drawing medians AD, BE and CF
\draw (B) -- (E);
\draw (C) -- (F);
%\draw (A) -- (D);

%Drawing EF
%\draw [dashed] (E) -- (F);

%Centroid
%\node (G) at ($(B)!0.67!(E)$)[label={[shift={(0.8,-0.5)}]$G$}] {};
\node (G) at ($(B)!0.67!(E)$)[point, label=below:$G$] {};

%Labeling sides
\node [right] at ($(A)!0.5!(E)$) {$\frac{b}{2}$};
\node [right] at ($(C)!0.5!(E)$) {$\frac{b}{2}$};
\node [left] at ($(B)!0.5!(F)$) {$\frac{c}{2}$};
\node [left] at ($(A)!0.5!(F)$) {$\frac{c}{2}$};
%\node [below] at ($(E)!0.5!(G)$) {$1$};
%\node [below] at ($(B)!0.5!(G)$) {$k_1$};
%\node [below] at ($(F)!0.5!(G)$) {$1$};
%\node [below] at ($(C)!0.5!(G)$) {$k_2$};
%\tkzLabelAnglepos=0.5{$\theta$}
%\tkzLabelAnglepos=0.5{$\theta$}
%\tkzFillAnglefill=red!20,size=.3
%\tkzFillAnglefill=red!20,size=.3
%\tkzMarkAnglecolor=red, fill=red!40,size=0.3
%\tkzMarkAnglecolor=red, fill=red!40,size=0.3
%\tkzMarkAnglefill=blue!50,size=.3
\iffalse
\node [above right] at ($(F)!0.5!(E)$) {$P$};
\fi

%\node (G) at ($(B)!0.67!(E)$)[label={[shift={(-0.8,-0.5)}]$G_1$}] {};

%
\end{tikzpicture}

}
	\end{center}
	\caption{$k_1=k_2$.}
	\label{fig:tri_med}	
\end{figure}
\solution
	From \eqref{tri:area-sin},
\begin{align}
	ar\brak{BEC} &= 
	\frac{1}{2}a\brak{\frac{b}{2}}\sin C 
	\\
	ar\brak{BFC}&=
	\frac{1}{2}a\brak{\frac{c}{2}}\sin  B
\end{align}
yielding
	\eqref{eq:median-area}.
\item The median divides a triangle into two triangle of equal area.
	\label{prob:median-area}.
\item 
	In \figref{fig:tri_med},	
	show that
\begin{align}
	ar\brak{CGE}
	=ar\brak{BGF} 
	\label{eq:median-sub-area}
\end{align}
\solution 
From 
	\figref{fig:tri_med}	
	and 
	\eqref{eq:median-area},
\begin{align}
	ar\brak{BGF}
	+
	ar\brak{BGC}
	=
	ar\brak{CGE}
	+
	ar\brak{BGC}
\end{align}
yielding 
	\eqref{eq:median-sub-area}.
\item In 
	\figref{fig:tri_med_isect},	show that
\begin{align}
	k_1=k_2
	\label{eq:med-ratio-eq}
\end{align}
\begin{figure}[!ht]
	\begin{center}
		\resizebox{0.6\columnwidth}{!}{%Code by GVV Sharma
%July 8, 2023
%released under GNU GPL
%Drawing the medians

\begin{tikzpicture}
[scale=2,>=stealth,point/.style={draw,circle,fill = black,inner sep=0.5pt},]

%Triangle sides
\def\a{5}
\def\b{6}
\def\c{4}
 
%Coordinates of A
\def\p{2.25}
\def\q{{sqrt(\c^2-\p^2)}}

%Labeling points
\node (A) at (\p,\q)[point,label=above right:$A$] {};
\node (B) at (0, 0)[point,label=below left:$B$] {};
\node (C) at (\a, 0)[point,label=below right:$C$] {};

%Foot of median

%\node (D) at ($(B)!0.5!(C)$)[point,label=below:$D$] {};
\node (E) at ($(A)!0.5!(C)$)[point,label=right:$E$] {};
\node (F) at ($(B)!0.5!(A)$)[point,label=left:$F$] {};

%Drawing triangle ABC
\draw (A) -- node[] {} (B) -- node[below, yshift=-5mm] {$\textrm{a}$} (C) -- node[] {} (A);

%Drawing medians AD, BE and CF
\draw (B) -- (E);
\draw (C) -- (F);
%\draw (A) -- (D);

%Drawing EF
%\draw [dashed] (E) -- (F);

%Centroid
%\node (G) at ($(B)!0.67!(E)$)[label={[shift={(0.8,-0.5)}]$G$}] {};
\node (G) at ($(B)!0.67!(E)$)[point, label=below:$G$] {};

%Labeling sides
\node [right] at ($(A)!0.5!(E)$) {$\frac{b}{2}$};
\node [right] at ($(C)!0.5!(E)$) {$\frac{b}{2}$};
\node [left] at ($(B)!0.5!(F)$) {$\frac{c}{2}$};
\node [left] at ($(A)!0.5!(F)$) {$\frac{c}{2}$};
% Adding new labels
\node [above] at ($(G)!0.5!(E)$) {$p$};
\node [above] at ($(B)!0.5!(G)$) {$k_1p$};
\node [above] at ($(G)!0.5!(F)$) {$q$};
\node [above] at ($(C)!0.5!(G)$) {$k_2q$};
%\node [below] at ($(E)!0.5!(G)$) {$1$};
%\node [below] at ($(B)!0.5!(G)$) {$k_1$};
%\node [below] at ($(F)!0.5!(G)$) {$1$};
%\node [below] at ($(C)!0.5!(G)$) {$k_2$};
\tkzLabelAngle[pos=0.5](C,G,E){$\theta$}
\tkzLabelAngle[pos=0.5](F,G,B){$\theta$}
\tkzFillAngle[fill=red!20,size=.3](C,G,E)
\tkzFillAngle[fill=red!20,size=.3](F,G,B)
%\tkzMarkAngle[color=red, fill=red!40,size=0.3](C,G,E)
%\tkzMarkAngle[color=red, fill=red!40,size=0.3](F,G,B)
%\tkzMarkAngle[fill=blue!50,size=.3](C,G,E)
\iffalse
\node [above right] at ($(F)!0.5!(E)$) {$P$};
\fi

%\node (G) at ($(B)!0.67!(E)$)[label={[shift={(-0.8,-0.5)}]$G_1$}] {};

%
\end{tikzpicture}
}
	\end{center}
	\caption{Equal areas.}
	\label{fig:tri_med_isect}	
\end{figure}
\solution
	From \eqref{eq:median-sub-area},
\begin{align}
	\frac{1}{2}p\brak{k_1q} \sin \theta
	&=\frac{1}{2}q\brak{k_2p}\sin \theta
\end{align}
	yielding \eqref{eq:med-ratio-eq}.
\item In 
	\figref{fig:tri_med_meet}, show that 	
\begin{align}
	k_3 = k
\label{eq:med-ratio-eq-k3}
\end{align}
\solution 
	From \probref{prob:median-area},
\begin{align}
\begin{split}
	ar\brak{AGE}
	&=ar\brak{CGE}
	\\
	ar\brak{AGF}
	&=ar\brak{BGF}
\end{split}
\\
\implies
\begin{split}
	\frac{1}{2}p\brak{k_3r}\sin \alpha
	&=\frac{1}{2}p\brak{kq}\sin \theta
	\\
	\frac{1}{2}q\brak{k_3r}\sin \beta
	&=\frac{1}{2}q\brak{kp} \sin \theta
\end{split}
\end{align}
yileding upon division
\begin{align}
	p\sin \alpha &= q \sin \beta
	\\
	\implies 
	\frac{1}{2}kpr\sin \alpha &= \frac{1}{2}kqr \sin \beta
	\\
	\implies ar\brak{BGD}
	&=ar\brak{CGD}
\end{align}
Thus, 
	from \probref{prob:median-area}, $AD$ is also a median.
	Consequently, 
	from
	\eqref{eq:med-ratio-eq} we obtain
\eqref{eq:med-ratio-eq-k3}.
%
\begin{figure}[H]
	\begin{center}
		\resizebox{0.6\columnwidth}{!}{%Code by GVV Sharma
%December 10, 2019
%released under GNU GPL
%Drawing the median

\begin{tikzpicture}
[scale=2,>=stealth,point/.style={draw,circle,fill = black,inner sep=0.5pt},]

%Triangle sides
\def\a{5}
\def\b{6}
\def\c{4}
 
%Coordinates of A
\def\p{2.25}
\def\q{{sqrt(\c^2-\p^2)}}

%Labeling points
\node (A) at (\p,\q)[point,label=above right:$A$] {};
\node (B) at (0, 0)[point,label=below left:$B$] {};
\node (C) at (\a, 0)[point,label=below right:$C$] {};

%Foot of median

\node (D) at ($(B)!0.5!(C)$)[point,label=below:$D$] {};
\node (E) at ($(A)!0.5!(C)$)[point,label=right:$E$] {};
\node (F) at ($(B)!0.5!(A)$)[point,label=left:$F$] {};

%Drawing triangle ABC
\draw (A) -- node[] {} (B) -- node[below, yshift=-5mm] {$\textrm{a}$} (C) -- node[] {} (A);

%Drawing medians AD, BE and CF
\draw (B) -- (E);
\draw (C) -- (F);
\draw (A) -- (D);

%Drawing EF
%\draw [dashed] (E) -- (F);

%Centroid
\node (G) at ($(B)!0.67!(E)$)[label={[shift={(0.8,-0.5)}]$G$}] {};

%Labeling sides
\node [right] at ($(A)!0.5!(E)$) {$\frac{b}{2}$};
\node [right] at ($(C)!0.5!(E)$) {$\frac{b}{2}$};
\node [left] at ($(B)!0.5!(F)$) {$\frac{c}{2}$};
\node [left] at ($(A)!0.5!(F)$) {$\frac{c}{2}$};
%\node [below] at ($(E)!0.5!(G)$) {$1$};
%\node [below] at ($(B)!0.5!(G)$) {$2$};
%\node [below] at ($(F)!0.5!(G)$) {$1$};
%\node [below] at ($(C)!0.5!(G)$) {$2$};
%\node [right] at ($(D)!0.5!(G)$) {$1$};
\node [right] at ($(A)!0.5!(G)$) {$k_3$};
%\node [below] at ($(D)!0.5!(C)$) {$1$};
\node [below] at ($(B)!0.5!(D)$) {$k_4$};
\iffalse
\node [above right] at ($(F)!0.5!(E)$) {$P$};
\fi

%\node (G) at ($(B)!0.67!(E)$)[label={[shift={(-0.8,-0.5)}]$G_1$}] {};

%
\end{tikzpicture}

}
	\end{center}
	\caption{$k_3=k$.}
	\label{fig:tri_med_meet}	
\end{figure}
\item In 
	\figref{fig:tri_med_sim}, show that $k = 2$.
	\\
\solution Using the cosine formula, 
\begin{align}
	DE^2 &= \brak{\frac{b}{2}}^2+\brak{\frac{c}{2}}^2-2\brak{\frac{b}{2}}\brak{\frac{c}{2}}\cos A  
	\\
	a^2&= b^2+b^2-2bc\cos A  
	\\
	\implies DE &= \frac{a}{2}
\end{align}
$\because \triangle EGF \sim \triangle BGC, k = 2$.
\begin{figure}[H]
	\begin{center}
		\resizebox{0.6\columnwidth}{!}{%Code by GVV Sharma
%July 8, 2023
%released under GNU GPL
%Drawing the medians

\begin{tikzpicture}
[scale=2,>=stealth,point/.style={draw,circle,fill = black,inner sep=0.5pt},]

%Triangle sides
\def\a{5}
\def\b{6}
\def\c{4}
 
%Coordinates of A
\def\p{2.25}
\def\q{{sqrt(\c^2-\p^2)}}

%Labeling points
\node (A) at (\p,\q)[point,label=above right:$A$] {};
\node (B) at (0, 0)[point,label=below left:$B$] {};
\node (C) at (\a, 0)[point,label=below right:$C$] {};

%Foot of median

\node (D) at ($(B)!0.5!(C)$)[point,label=below:$D$] {};
\node (E) at ($(A)!0.5!(C)$)[point,label=right:$E$] {};
\node (F) at ($(B)!0.5!(A)$)[point,label=left:$F$] {};

%Drawing triangle ABC
\draw (A) -- node[] {} (B) -- node[below, yshift=-5mm] {$\textrm{a}$} (C) -- node[] {} (A);

%Drawing medians AD, BE and CF
\draw (B) -- (E);
\draw (C) -- (F);
%\draw (A) -- (D);

%Drawing DE
\draw [dashed] (F) -- (E);

%Centroid
\node (G) at ($(B)!0.67!(E)$)[point, label=below:$G$] {};

%Labeling sides
\node [right] at ($(A)!0.5!(E)$) {$\frac{b}{2}$};
\node [right] at ($(C)!0.5!(E)$) {$\frac{b}{2}$};
\node [left] at ($(B)!0.5!(F)$) {$\frac{c}{2}$};
\node [left] at ($(A)!0.5!(F)$) {$\frac{c}{2}$};

% Adding new labels
\node [above] at ($(G)!0.5!(E)$) {$p$};
\node [above] at ($(B)!0.5!(G)$) {$kp$};
\node [above] at ($(G)!0.5!(F)$) {$q$};
\node [above] at ($(C)!0.5!(G)$) {$kq$};

\end{tikzpicture}

}
	\end{center}
	\caption{k = 2}
	\label{fig:tri_med_sim}
\end{figure}


\end{enumerate}
