\begin{enumerate}[label=\thesubsection.\arabic*.,ref=\thesubsection.\theenumi]
	\item 
A right angled triangle looks like Fig. \ref{fig:tri_right_angle}.
\begin{figure}[!ht]
\centering
\resizebox{0.75\columnwidth}{!}{%Code by GVV Sharma
%December 6, 2019
%released under GNU GPL
%Drawing a right angled triangle

\begin{tikzpicture}[scale=2]

%Triangle sides
\def\a{4}
\def\c{3}

%Marking coordiantes
\coordinate [label=above:$A$] (A) at (0,\c);
\coordinate [label=left:$B$] (B) at (0,0);
\coordinate [label=right:$C$] (C) at (\a,0);

%Drawing triangle ABC
\draw (A) -- node[left] {$\textrm{c}$} (B) -- node[below] {$\textrm{a}$} (C) -- node[above,,xshift=2mm] {$\textrm{b}$} (A);

%Drawing and marking angles

\tkzMarkAngle[color=red, fill=red!40,size=0.3](A,C,B)
\tkzMarkRightAngle[fill=blue!20,size=.3](A,B,C)
\tkzLabelAngle[pos=0.65](A,C,B){$\theta$}
\end{tikzpicture}
}
\caption{Right Angled Triangle}
\label{fig:tri_right_angle}	
\end{figure}
with angles $\angle A,\angle B$ and $\angle C$ and sides $a, b$ and $c$.  The unique feature of this triangle is $\angle B$ which is defined to be $90\degree$.
\item
	For simplicity, let the greek letter $\theta = \angle C$.  We have the following definitions.
\begin{equation}
\label{eq:tri_trig_defs}
\begin{matrix}
	\sin \theta = \frac{c}{b} & 	\cos \theta = \frac{a}{b} \\[1ex]
	\tan \theta = \frac{c}{a} & \cot \theta = \frac{1}{\tan \theta} \\[1ex]
	\csc \theta = \frac{1}{\sin \theta} & \sec \theta = \frac{1}{\cos \theta}
	\end{matrix}
\end{equation}
\item  
	\begin{equation}
	\cos \theta = \sin \brak{90\degree - \theta}
	\label{eq:tri_baudh_comp}	
	\end{equation}
\end{enumerate}
