\begin{enumerate}[label=\thesubsection.\arabic*.,ref=\thesubsection.\theenumi]
\numberwithin{equation}{enumi}

\item The equation of the perpendicular bisector of $BC$ is
		\begin{align}
			\label{eq:tri-perp-bisect}
			\brak{\vec{x}-\frac{\vec{B}+\vec{C}}{2}}\brak{\vec{B}-\vec{C}} = 0
		\end{align}
		Substitute numerical values and find the equations of the perpendicular bisectors of $AB, BC$ and $CA$.
	\\	\input{solutions/1/4/1/q1.4.1.tex}
	\item Find the intersection $\vec{O}$ of the perpendicular bisectors of $AB$ and $AC$.
 \\
 \iffalse
\title{Trignometric Functions and Equations}
\author{EE24BTECH11007- ARNAV MAKARAND YADNOPAVIT}
\section{mains}
\fi

	\item Verify that $\vec{O}$ satisfies
			\eqref{eq:tri-perp-bisect}.
$\vec{O}$ is known as the circumcentre.\\
   \iffalse
\title{Trignometric Functions and Equations}
\author{EE24BTECH11007- ARNAV MAKARAND YADNOPAVIT}
\section{mains}
\fi

		\item Verify that 
		\begin{align}
			OA = OB = OC 
		\end{align}
	\item Draw the circle with centre at $\vec{O}$ and radius 
		\begin{align}
			R = OA
		\end{align}
		This is known as the {\em circumradius}. 
  \\  \input{solutions/1/4/5/assignment2.tex}
	\item Verify that 
		\begin{align}
			\angle BOC = 2\angle BAC.
		\end{align}\\
  \input{solutions/1/4/6/Q_1.4.6.tex}
	\item Let 
		\begin{align}
			\vec{P} = \myvec{\cos \theta & -\sin \theta \\ \sin \theta & \cos \theta}
		\end{align}
			where
\begin{align}
	\theta = \angle BOC
\end{align}
Verify that 
		\begin{align}
			\vec{B}-\vec{O}=\vec{P}\brak{\vec{C}-\vec{O}}
		\end{align}
All codes for this section are available at
\begin{lstlisting}
	codes/triangle/perp-bisect.py
\end{lstlisting}
\end{enumerate}
