\begin{enumerate}[label=\thesection.\arabic*.,ref=\thesection.\theenumi]
\numberwithin{equation}{enumi}
\item The equation of a line is given by 
\begin{align}
			\label{eq:line-school}
	y &= mx + c
	\\
	\implies \myvec{x \\ y} &= \myvec{x \\ 
	 mx + c} =\myvec{0 \\ c} + x\myvec{1 \\ m}
\end{align}
			yielding \eqref{eq:geo-param}.
\item 			\eqref{eq:line-school} can also be expressed as
\begin{align}
	y - mx &= c 
	\\
	\implies \myvec{-m & 1}\myvec{x \\ y} &= c
\end{align}
			yielding \eqref{eq:geo-normal}.
  \item From \eqref{eq:geo-param}, 
	  if $\vec{A},\vec{D}$ and $\vec{C}$ are on the same line,
\begin{align}
			\vec{D}=\vec{A}+q\vec{m} 
			\\ 
			\vec{C}=\vec{D}+p\vec{m} \\
			\label{eq:collinear} 
			\implies 	p\brak{\vec{D}-\vec{A}} 
			+ q\brak{\vec{D}-\vec{C}} = 0, \quad p, q \ne 0 \\ 
			\implies \vec{D} = \frac{p\vec{A}+q\vec{C}}{p+q} 
			\end{align} 
			yielding \eqref{eq:section_formula} upon substituting \begin{align} k = \frac{p}{q}. \end{align} 
			$\brak{\vec{D}-\vec{A}}, \brak{\vec{D}-\vec{C}}$ 
		are then said to be {\em linearly dependent}.
	\item If $\vec{A}, \vec{B}, \vec{C}$ are collinear,  from \eqref{eq:geo-normal}, \begin{align}
	 \vec{n}^{\top}\vec{A} &=  c 
	 \\
	 \vec{n}^{\top}\vec{B} &=  c 
	 \\
	 \vec{n}^{\top}\vec{C} &=  c 
\end{align}
which can be expressed as 
\begin{align}
	\myvec{ \vec{A} & \vec{B} & \vec{C}}^{\top}\vec{n} = c\myvec{1 \\ 1 \\ 1}
	\\
	\implies 
	\myvec{ 1 & 1 &1 \\ \vec{A} & \vec{B} & \vec{C}}^{\top}\myvec{\vec{n} \\ -c} &= \vec{0}
\end{align}
yielding
			\eqref{eq:line-rank}.  Rank is defined to be the number of linearly indpendent rows or columns of a matrix.

  \item Consequently, points $\vec{A},\vec{B}$ and $\vec{C}$ form a triangle  if 
	  \label{prop:two-tri-indep}
  \begin{align}
	  p\brak{\vec{A}- \vec{B}} +q\brak{\vec{C} -\vec{B}} 
	  \\
	  =\brak{p+q}\vec{B}- p\vec{A} -q\vec{C} = 0
	  \\
	  \implies p=0, q=0
	  \label{eq:two-tri-indep}
  \end{align}
  \item In 
	\figref{fig:tri_med_isect}	
	\begin{align}
	AF = BF, \,
	AE = BE, 
	\end{align}
	and the medians $BE$ and $CF$ meet at $\vec{G}$.
	Show that 
%	Using Fig. \ref{ch2_median_ratio_val}, 
	\begin{align}
\label{eq:tri_med_centroid_ratio}
	\frac{GB}{GE} = \frac{GC}{GF} = 2
	\end{align}
%
\begin{figure}[!ht]
	\begin{center}
		\resizebox{\columnwidth}{!}{\input{./figs/triangle/tri_med_isect.tex}}
%		\resizebox{\columnwidth}{!}{\input{./figs/coord/tri_med_meet.tex}}
	\end{center}
	\caption{$k_1=k_2=2$.}
	\label{fig:tri_med_isect}	
	%\label{fig:tri_med_meet}	
\end{figure}
\solution From 
	  \eqref{eq:section_formula},
  \begin{align}
	  \label{eq:section_formula-G}
\vec{G} = 
	   \frac{k_1\vec{E}+ \vec{B}}{k_1+1}
	  &= \frac{k_2\vec{F}+ \vec{C}}{k_2+1}
	  \\
	  \implies 
	   \frac{k_1\brak{\frac{\vec{A}+\vec{C}}{2}}+ \vec{B}}{k_1+1}
	  &= \frac{k_2\brak{\frac{\vec{A}+\vec{B}}{2}}+ \vec{C}}{k_2+1}
  \end{align}
\begin{multline}
	  \implies 
	\brak{k_2+1}   \cbrak{k_1\brak{{\vec{A}+\vec{C}}}+ 2\vec{B}}
	  \\= \brak{k_1+1}\cbrak{k_2\brak{{\vec{A}+\vec{B}}}+ 2\vec{C}}
\end{multline}
  which can be expressed as
  \begin{align}
	  \cbrak{2 + k_2- k_1k_2 }\vec{B}-\brak{k_2-k_1}\vec{A}  - \cbrak{k_1 +2 - k_1k_2}\vec{C}
	  =0
  \end{align}
  and is of the form
	  \eqref{eq:two-tri-indep}
	  with 
  \begin{align}
	  p = {k_2-k_1}, q = {k_1 +2 - k_1k_2}.
  \end{align}
  Thus, from 
	  \eqref{eq:two-tri-indep}
  \begin{align}
\label{eq:tri_med_centroid_ratio-1}
	  k_2-k_1 &= 0,
	  \\
	  k_1 +2 - k_1k_2 &=0
\label{eq:tri_med_centroid_ratio-2}
  \end{align}
  Thus, from 
\eqref{eq:tri_med_centroid_ratio-2}
  \begin{align}
	  k_1=k_2
  \end{align}
  and substituting the above in 
\eqref{eq:tri_med_centroid_ratio-2} results in the quadratic
  \begin{align}
	  k_1^2 - k_1-2 &=0
	  \\
	  \implies 
	  \brak{k_1-2}\brak{k_1+1} &=0
  \end{align}
  admitting $k_1=k_2=2$ as the only possible solution.
  \item Substituting $k_1 =2 $ in 
	  \eqref{eq:section_formula-G}
  \begin{align}
	  \vec{G}=\frac{\vec{A}+\vec{B} + \vec{C}}{3}
	  \label{eq:centroid-G}
  \end{align}
\item 
In	\figref{fig:tri_med_meet},	
$AG$ is extended to join $BC$ at $\vec{D}$.  Show that $AD$ is also a median.
\begin{figure}[!ht]
	\begin{center}
%		\resizebox{\columnwidth}{!}{\input{./figs/coord/tri_med_isect.tex}}
		\resizebox{\columnwidth}{!}{\input{./figs/coord/tri_med_meet.tex}}
	\end{center}
	\caption{$k_3 = 2, k_4 =1$}
%	\label{fig:tri_med_isect}	
	\label{fig:tri_med_meet}	
\end{figure}
	\\
	\solution Considering the ratios in 
	\figref{fig:tri_med_meet},	
  \begin{align}
\vec{G} = 
	  \frac{k_3\vec{D}+\vec{A} }{k_3+1} 
	  \\
	\vec{D}  =\frac{k_4\vec{C}+\vec{B} }{k_4+1} 
  \end{align}
  Substituting from 
	  \eqref{eq:centroid-G}
	  in the above, 
  \begin{align}
	  \brak{k_3+1}\brak{\frac{\vec{A}+\vec{B} + \vec{C}}{3}}
 = 
	  {k_3\brak{\frac{k_4\vec{C}+\vec{B} }{k_4+1}} +\vec{A} } 
  \end{align}
\begin{multline}
	  \implies \brak{k_3+1}\brak{k_4+1}\brak{{\vec{A}+\vec{B} + \vec{C}}}
	  \\
 = 
	  {3} \cbrak{ {k_3\brak{{k_4\vec{C}+\vec{B} }} +\brak{k_4+1}\vec{A} }} 
\end{multline}
  which can be expressed as
  \begin{multline}
	  \brak{k_3k_4+k_3-2k_4-2}\vec{A}
	  \\
	-  \brak{-k_3k_4-k_4+2k_3-1}\vec{B}
	  \\
	  - \brak{-k_3-k_4 - 1 
+2k_3k_4} \vec{C} = \vec{0}
  \end{multline}
  Comparing the above with 
	  \eqref{eq:two-tri-indep},
  \begin{align}
	  p = {-k_3k_4-k_4+2k_3-1}, q = {-k_3-k_4 - 1 
+2k_3k_4}
  \end{align}
  yielding 
  \begin{align}
	  \label{eq:centroid-G-meet-1}
	   {-k_3k_4-k_4+2k_3-1} = 0
	   \\ {-k_3-k_4 - 1 
+2k_3k_4} = 0
	  \label{eq:centroid-G-meet-2}
  \end{align}
  Subtracting 
	  \eqref{eq:centroid-G-meet-1}
	  from
	  \eqref{eq:centroid-G-meet-2},
  \begin{align}
	  3k_3\brak{k_4-1} &= 0
	  \\
	  \implies k_4&=1
  \end{align}
  which upon substituting in 
	  \eqref{eq:centroid-G-meet-1}
	  yields
  \begin{align}
	  k_3 = 2
  \end{align}
	  \end{enumerate}
