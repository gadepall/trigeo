\begin{enumerate}[label=\thesubsection.\arabic*,ref=\thesubsection.\theenumi]
    \item Let $ABCD$ be a convex quadrilateral with perpendicular diagonals. If $AB = 20$, $BC = 70$, and $CD = 90$, then what is the value of $DA$?\hfill(PRMO 2014)
    \item In a triangle with integer side lengths, one side is three times as long as a second side, and the length of the third side is $17$. What is the greatest possible perimeter of the triangle? \hfill(PRMO 2014)
    \item In a triangle $ABC$, $X$ and $Y$ are points on the segments $AB$ and $AC$, respectively, such that $ AX : XB = 1 : 2 $ and $ AY : YC = 2 : 1$. If the area of triangle $AXY$ is $10$, then what is the area of triangle $ABC$?\hfill(PRMO 2014)
    \item Let $XOY$ be a triangle with $\angle XOY = 90\degree$. Let $M$ and $N$ be the midpoints of legs $OX$ and $OY$, respectively. Suppose that $XN = 19$ and $YM = 22$. What is $XY$?

	    \hfill(PRMO 2014)
\item In $\triangle ABC$, we have $AC = BC = 7$ and $AB = 2$. Suppose that $D$ is a point on line $AB$ such that $B$ lies between $A$ and $D$ and $CD = 8$. What is the length of the segment $BD$?\hfill(PRMO 2012)
\item In rectangle $ABCD$, $AB = 5$ and $BC = 3$. Points $F$ and $G$ are on line segment $CD$ so that $DF = 1$ and $GC = 2$. Lines $AF$ and $BG$ intersect at $E$. What is the area of $\triangle ABE$?\hfill(PRMO 2012)
\item A triangle with perimeter 7 has integer side lengths. What is the maximum possible area of such a triangle?\hfill(PRMO 2012)
\item $ABCD$ is a square and $AB$ = 1. Equilateral triangles $AYB$ and $CXD$ are drawn such that $X$ and $Y$ are inside the square. What is the length of $XY$?\hfill(PRMO 2012)
    \item A $ 2 \times 3 $ rectangle and a $ 3 \times 4 $ rectangle are contained within a square without overlapping at any interior point, and the sides of the square are parallel to the sides of the two given rectangles. What is the smallest possible area of the square? \hfill(PRMO 2015)

    \item What is the greatest possible perimeter of a right-angled triangle with integer side lengths if one of the sides has length 12? \hfill(PRMO 2015)



\item In the acute-angled triangle $ABC$, let $D$ be the foot of the altitude from $A$, and $E$ be the midpoint of $BC$. Let $F$ be the midpoint of $AC$. Suppose $ \angle BAE = 40\degree $. If $ \angle DAE = \angle DFE $, what is the magnitude of $ \angle ADF $ in degrees?
	\hfill(PRMO 2015)
	\item In an equilateral triangle of side length $6$, pegs are placed at the vertices and also evenly along each side at a distance of $1$ from each other. Four distinct pegs are chosen from the $15$ interior pegs on the sides (that is, the chosen ones are not vertices of the triangle) and each peg is joined to the respective opposite vertex by a line segment. If $N$ denotes the number of ways we can choose the pegs such that the drawn linesegments divide the interior of the triangle into exactly nine regions, find the sum of the squares of the digits of $N$.\hfill(IOQM 2015)		
	\item In a triangle $ABC$, let $E$ be the midpoint of $AC$ and $F$ be the midpoint of $AB$. The medians $BE$ and $CF$ intersect at $G$. Let $Y$ and $Z$ be the midpoints of $BE$ and $CF$, respectively. If the area of triangle $ABC$ is 480, find the area of triangle $GYZ$.

		\hfill(IOQM 2015)
    \item Let $X$ be the set of all even positive integers $n$ such that the measure of the angle of some regular polygon is $n$ degrees. Find the number of elements in $X$.

	    \hfill(IOQM 2015)
    \item Let $ABC$ be a triangle in the $xy$-plane, where $B$ is at the origin $\brak{0, 0}$. Let $BC$ be produced to $D$ such that $BC : CD = 1 : 1$, $CA$ be produced to $E$ such that $CA : AE = 1 : 2$, and $AB$ be produced to $F$ such that $AB : BF = 1 : 3$. Let $G\brak{32, 24}$ be the centroid of triangle $ABC$ and $K$ be the centroid of triangle $DEF$. Find the length $GK$.\hfill(IOQM 2015)
    
    \item In the coordinate plane, a point is called a lattice point if both of its coordinates are integers. Let $A$ be the point $\brak{12, 84}$. Find the number of right-angled triangles $ABC$ in the coordinate plane where $B$ and $C$ are lattice points, having a right angle at the vertex $A$ and whose incenter is at the origin $\brak{0, 0}$.\hfill(IOQM 2015)
    
    \item A trapezium in the plane is a quadrilateral in which a pair of opposite sides are parallel. A trapezium is said to be non-degenerate if it has positive area. Find the number of mutually non-congruent, non-degenerate trapeziums whose sides are four distinct integers from the set $\cbrak{5, 6, 7, 8, 9, 10}$.\hfill(IOQM 2015)
    
\item Consider the convex quadrilateral $ABCD$. The point $P$ is the interior of $ABCD$. The following ratio equalities hold
\begin{align*}
\angle PAD: \angle PBA: \angle DPA =1:2:3 = \angle CBP: \angle BAP: \angle BPC.
\end{align*} 
prove
 that the following three lines meet in a point: the internal bisectors 
of angles $\angle ADP$ and $\angle PCB$ and the perpendicular bisector 
of segment AB
\hfill(IMO 2020)
\item Three points $ X, Y, Z $ are on a straight line such that $ XY = 10 $ and $ XZ = 3 $. What is the product of all possible values of $ YZ $?\hfill(PRMO 2013)

\item Let $ AD $ and $ BC $ be the parallel sides of a trapezium $ ABCD $. Let $ P $ and $ Q $ be the midpoints of the diagonals $ AC $ and $ BD $. If $ AD = 16 $ and $ BC = 20 $, what is the length of $ PQ $?\hfill(PRMO 2013)

\item Let $ ABC $ be an equilateral triangle. Let $ P $ and $ S $ be points on $ AB $ and $ AC $, respectively, and let $ Q $ and $ R $ be points on $ BC $ such that $ PQRS $ is a rectangle. If $ PQ = \sqrt{3} PS $ and the area of $ PQRS $ is $ \frac{28}{3} $, what is the length of $ PC $?\hfill(PRMO 2013)

\item Let $ A_1, B_1, C_1, D_1 $ be the midpoints of the sides of a convex quadrilateral $ ABCD $ and let $ A_2, B_2, C_2, D_2 $ be the midpoints of the sides of the quadrilateral $ A_1B_1C_1D_1 $. If $ A_2B_2C_2D_2 $ is a rectangle with sides 4 and 6, then what is the product of the lengths of the diagonals of $ ABCD $?\hfill(PRMO 2013)


\item In a triangle $ ABC $ with $ \angle BCA = 90\degree $, the perpendicular bisector of $ AB $ intersects segments $ AB $ and $ AC $ at $ X $ and $ Y $, respectively. If the ratio of the area of quadrilateral $ BXYC $ to the area of triangle $ ABC $ is 13:18 and $ BC = 12 $, then what is the length of $ AC $?\hfill(PRMO 2013)
\item $A$ convex hexagon has the property that for any pair of opposite sides the distance between their midpoints is $\frac{\sqrt{3}}{2}$ times the sum of their lengths Show that all the hexagon's angles are equal.\hfill(IMO 2003)
\item In a triangle $ABC$, let $AP$ bisect $\angle B AC$, with $P$ on $BC$, and let $BQ$ bisect $\angle A BC$, with $Q$ on $CA$. It is known that $\angle BAC= 60\degree$ and that $AB+BP=AQ+QB$. What are the possible angles of triangle $ABC$?\hfill(IMO 2001)
\item Let $d$ be the sum of the lengths of all the diagonals of a plane convex polygon with $n$ vertices $\brak{n>3}$, and let $p$ be its perimeter. Prove that\begin{align*}                                    \ln -3<\frac{2d}{p}<\sbrak{\frac{n}{2}}\sbrak{\frac{n+1}{2}}-2,\end{align*}
		Where $\sbrak{x}$ denotes the gratest integer not exceeding $x$.  \hfill(IMO 1984)
\item $P$ is a point inside a given triangle $ABC$. $D, E, F$ are the feet of the perpendiculars from $P$ to the lines $BC, CA, AB$ respectively. Find all $P$ for which $$\frac{BC}{PD}+\frac{CA}{PE}+\frac{AB}{PF}$$ is least. \hfill(IMO  1981)
 \item The diagonals $AC$ and $CE$ of the regular hexagon $ABCDEF$ are divided by the inner points $M$ and $N$, respectively, so that \begin{align*} \frac{AM}{AC}=\frac{CN}{CE}=r.
                   \end{align*}
 Determine $r$ if $B, M$ and $N$ are collinear. \hfill(IMO  1982)
    \item Let $A$, $B$ be adjacent vertices of a regular $n$-gon $\brak{n\leq5}$ in the plane having center at $O$. A triangle $XYZ$, which is congruent to and initially conincides with $OAB$, moves in the plane in such a way that $Y$ and $Z$ each trace out the whole boundary of the polygon, $X$ remaining inside the polygon. Find the locus of $X$.\hfill(IMO  1986)

\item $ABC$ is a triangle right-angled at $A$, and $D$ is the foot of the altitude from $A$. The straight line joining the incenters of the triangles $ABD$, $ACD$ intersects the sides $AB$, $AC$ at the points $K$, $L$ respectively. $S$ and $T$ denote the areas of the triangles $ABC$ and $AKL$ respectively. Show that $S\geq 2T$.\hfill(IMO  1988)
\end{enumerate}
