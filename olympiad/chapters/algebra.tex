\begin{enumerate}
	\item Let $x, y$ be positive integers such that 
		\begin{align}
			x^4 = \brak{x - 1}\brak{y^3 - 23} - 1.
		\end{align}
Find the maximum possible value of $x + y$.\hfill(IOQM 2015)
    
    \item The ex-radii of a triangle are $10\frac{1}{2}$, $12, 12$ and $14$. If the sides of the triangle are the roots of the cubic 
	    \begin{align}
	x^3 - px^2 + qx - r = 0,
	    \end{align}
where $p$, $q$, $r$ are integers, find the integer nearest to $\sqrt{\cbrak{p + q + r}}$.\hfill(IOQM 2015)

	\item Let $P\brak{x} = x^3 + ax^2 + bx + c$ be a polynomial where $a$, $b$, $c$ are integers and $c$ is odd. Let $p_i$ be the value of $P\brak{x}$ at $x = i$. Given that $p_{31} + p_{32} + p_{33} = 3p_1p_2p_3$, find the value of $p_2 + 2p_1 - 3p_0$.\hfill(IOQM 2015)
    
    \item A positive integer $m$ has the property that $m^2$ is expressible in the form $4n^2 - 5n + 16$ where $n$ is an integer (of any sign). Find the maximum possible value of $|m - n|$.\hfill(IOQM 2015)
    
    \item Find the least positive integer $n$ such that there are at least $1000$ unordered pairs of diagonals in a regular polygon with $n$ vertices that intersect at a right angle in the interior of the polygon.\hfill(IOQM 2015)
    
    \item Let $d(m)$ denote the number of positive integer divisors of a positive integer $m$. If $r$ is the number of integers $n \leq 2023$ for which
$
		\sum_{i=1}^{n} d \brak{i}
$
    is odd, find the sum of the digits of $r$.\hfill(IOQM 2015)
    \item Let Z be the set of integers. We want to determine all functions $ f : Z \to Z $ such that for all integers a  and  b :
$
f \brak{2a} + 2f\brak {b} = f\brak {f\brak{ a + b}}
$
\hfill(IMO 2019)
\item. A social network has 2019 users, some pairs of whom are friends. Whenever user A is friends with user B, user B is also friends with user A. Events of the following kind may happen repeatedly, one at a time:
Three users A, B, and C such that A is friends with both B and C, but B and C are not friends, change their friendship statuses such that B and C are now friends, but A is no longer friends with B, and no longer friends with C. All other friendship statuses are unchanged.
Initially, 1010 users have 1009 friends each, and 1009 users have 1010 friends each. Prove that there exists a sequence of such events after which each user is friends with at most one other user.
\hfill(IMO 2019)
\item The Bank of Bath issues coins with an H on one side and a T on the other. Harry has  $n$ of these coins arranged in a line from left to right. He repeatedly performs the following operation: if there are exactly $k > 0$ coins showing H, then he turns over the $ k^th$ coin from the left; otherwise, all coins show T and he stops. For example, if $n = 3$, the process starting with the configuration $THT$ would be: $THT$ $\rightarrow HHT $ $\rightarrow HTT$ $\rightarrow TTT$,which stops after three operations.

\begin{enumerate}
    \item[(a)] Show that, for each initial configuration, Harry stops after a finite number of operations.
    
    \item[(b)] For each initial configuration  $C $, let $L(C)$ be the number of operations before Harry stops. For example, $L\brak{THT} = 3$ and $L\brak{ TTT } = 0$. Determine the average value of $L\brak{C}$ over all $2^n$ possible initial configurations $C$.
    
\end{enumerate}
\hfill(IMO 2019)
\item A deck of n\textgreater1 cards is given. A positive integer is written on each card. The deck
has the property that the arithmetic mean of the numbers on each pair of cards is also the geometric
mean of the numbers on some collection of one or more cards.
For which n does it follow that the numbers on the cards are all equal?
\hfill(IMO 2020)
\item Let $ n\geq 100$ be an integer.Ivan writes the numbers n,n+1,....,2n each on different cards.He then shuffles these n+1 cards,and divides them into two piles.prove that at least one of the piles contains two cards such that the sum of their numbers is a perfect square.
\hfill(IMO 2021)
\item Let $ m \geq 2$ be an integer,A be a finite set of (not necessarily positive )integers,and $B_{1},B_{2},B_{3}...B_{m}$ be subsets of A.Assume that for each k=1,2,....,m the sum of the elements of $B_{k}$ is $m^k$ .Prove that A contains at least m/2 elements
\hfill(IMO 2021)
\item The real numbers a,b,c,d are such that  $a\geq b\geq c\geq d>0$ and a+b+c+d=1. prove that
\begin{align}
    \brak {a+2b+3c+4d}a^a b^b c^c d^d \textless 1
\end{align}
\hfill(IMO 2020)
\item Show that the inequality
$\sum_{i=1}^{n}\sum_{j=1}^{n}\sqrt{|x_{i}-x_{j}|}\le\sum_{i=1}^{n}\sum_{j=1}^{n}\sqrt{|x_{i}+x_{j}|}$
holds for all real numbers x1,....xn
\hfill(IMO 2021)
\item
Find all triples \brak{a,b,p}  of positive integers with  \brak{p} prime and Prove that: 
\begin{align}
\brak{ a^{p} = b! + p }.
\end{align} \hfill(IMO 2022)
\item
Let \( \mathbb{R}^{+} \) denote the set of positive real numbers. Find all functions \( f: \mathbb{R}^{+} \to \mathbb{R}^{+} \) such that for each \( x \in \mathbb{R}^{+} \), there is exactly one \( y \in \mathbb{R}^{+} \) satisfying \\ 
\begin{align} xf\brak{y} + y f\brak{x} \leq 2.\end{align}
\hfill(IMO 2022)
\item
Let  $k$  be a positive integer and let  $S$  be a finite set of odd prime numbers. Prove there is at most one way \brak{\text {up to rotation and reflection}} to place the elements of  $S$  around a circle such that the product of any two neighbours is of the form  $x^{2} + x + k$  for some positive integer  $x$. \hfill(IMO 2022)
\item
Determine all composite integers  $n \geq 1$  that satisfy the following property: if  $d_{1}, d_{2}, ..., d_{k}$  are all the positive divisors of  $n$  with  $1 = d_{1} \le d_{2} \le \cdots \le d_{k} = n$,  then  $d_{i} \text{ divides } d_{i+1} + d_{i+2}$  for every  $1 \leq i \leq k - 2.$ \hfill(IMO 2023)
\item
 For each integer  $k \geq 2$, determine all infinite sequences of positive integers $a_{1}, a_{2}, \ldots$ for which there exists a polynomial  $P$ of the form $ P\brak{x} = x^{k} + c_{k-1}x^{k-1} + \cdots + c_{1}x + c_{0}$  where $c_{0}, c_{1}, \ldots, c_{k-1}$  are non-negative integers, such that
\begin{align}
	P\brak{a_{n}} = a_{n+1}a_{n+2}\cdots a_{n+k}
\end{align} \hfill(IMO 2023)
\item
Let $x_{1}, x_{2},\ldots , x_{2023}$  be pairwise different positive real numbers such that 
\begin{align}
a_{n} = \sqrt{\brak{x_{1} + x_{2} + \cdots + x_{n}} \brak{\frac{1}{x_{1}}+\frac{1}{x_{2}}+\ldots+\frac{1}{x_{n}}}}
\end{align}
 is an integer for every $n = 1, 2,\ldots, 2023.$  Prove that  $a_{2023} \geq 3034$. \hfill(IMO 2023)
\item
Determine all real numbers such that, for every positive integer $n$, the integer 
\begin{align}
\sbrak{\alpha} + \sbrak{2\alpha} + \cdot\cdot\cdot + \sbrak{\alpha} 
\end{align}
is a multiple of  $n$.  Note that $\sbrak{z}$  denotes the greatest integer less than or equal to $z$. For example $\sbrak{-\pi}$ = $-4$  and  $\sbrak{2} = \sbrak{2.9}$ = $2$. \hfill(IMO 2024)
\item
 Let  $\mathbb{Q}$  be the set of rational numbers. A function  $f: \mathbb{Q} \to \mathbb{Q}$  is called aquaesulian if the following property holds: for every  $x, y \in \mathbb{Q},$ 
\begin{align}
f\brak{x + f\brak{y}} = f\brak{x} + y \quad \text{or} \quad f\brak{f\brak{x} + y} = x + f\brak{y}.
\end{align}
Show that there exists an integer  $c$  such that for any aquaesulian function  $f$  there are at most  $c$  different rational numbers of the form  $f\brak{r} + f\brak{-r}$  for some rational number  $r$,  and find the smallest possible value of  $c$. \hfill(IMO 2024)
\item Let
$ S_n = \sum_{k=0}^{n} \frac{1}{\sqrt{k+1} + \sqrt{k}}. $
What is the value of 
$ \sum_{n=1}^{90} \frac{1}{S_n + S_{n-1}}? $\hfill(Prermo 2013)

\item There are $ n-1 $ red balls, $ n $ green balls, and $ n+1 $ blue balls in a bag. The number of ways of choosing two balls from the bag that have different colours is 299. What is the value of $ n $?\hfill(Prermo 2013)

\item To each element of the set $ S = \{1, 2, \dots, 1000\} $ a color is assigned. Suppose that for any two elements $ a, b $ of $ S $, if 15 divides $ a + b $, then they are both assigned the same color. What is the maximum possible number of distinct colors used?\hfill(Prermo 2013)
\item Let Akbar and Birbal together have $ n $ marbles, where $ n > 0 $. Akbar says to Birbal, "If I give you some marbles, then you will have twice as many marbles as I will have." Birbal says to Akbar, "If I give you some marbles, then you will have thrice as many marbles as I will have." What is the minimum possible value of $ n $ for which the above statements are true?\hfill(Prermo 2013)

\item Carol was given three numbers and was asked to add the largest of the three to the product of the other two. Instead, she multiplied the largest with the sum of the other two, but still got the right answer. What is the sum of the three numbers?\hfill(Prermo 2013)

\item Three real numbers $ x, y, z $ are such that
$ x^2 + 6y = -17 $, $ y^2 + 4z = 1 $, and $ x^2 + 2x = 2 $. What is the value of $ x^2 + y^2 + z^2 $?\hfill(Prermo 2013)

\item Let $ f(x) = x^3 - 3x + b $ and $ g(x) = x^2 + bx - 3 $, where $ b $ is a real number. What is the sum of all $ b $ for which $ f(x) = 0 $ and $ g(x) = 0 $ have a common root?\hfill(Prermo 2013)
\item Find all pairs $\brak{m,n}$ of positiv e integers such that $\frac{m^{2}}{2mn^{2}-n^{3}+1}$ is a positive integer.\hfill(IMO 2003)
 \item Given $n \textgreater 2$ and reals $x_{1} \leq x_{2} \leq \ldots\leq x_{n}$, show that $\brak{\sum _{ij} \mydet{x_ix_j}^2} \leq \frac{2}{3}\brak{n^ 2-1}\sum_{ij} \brak{x_ix_j}^2$Show that we have equality iff the sequence is an arithmetic progressi on.\hfill(IMO 2003)
 \item Show that for each prime $p$, there exists a p rime $q$ such that $n^{p}-p$ is not divisible by $q$ for any positive integer $n$.\hfill(IMO 2003)(IMO 2003)
 \item Let $a, b, c,d$ be integers with $a\textless b\textless c\textless d\textless 0$. Suppose that $ac+bd=\brak{b+d+a-c}\brak{b+d-a+c}$. Prove that $ab+cd$ is not prime. \hfill(IMO 2001)
\item Let $n$ be an odd integer greater then $1$, an d let $k_{1}, k_{2}\ldots,k_{n}$ be given integers. For each of the n! permutations  $a=\brak{a_{1}, a _{2}\ldots,a_{n}}$ of 1,2,\ldots, n, let $S\brak{a}=\sum_{i=1}^{n}k_{i}a_{i}$. 83 Prove that there are two permutations $b$ and $c$, $b\neq$, such that n! is a divisor of $S\brak{b}-S\brak{c}$.\hfill(IMO 2001)
\item Prove that 79 $\frac{a}{\sqrt{a^{2}+8bc}}+\frac{b}{\sqrt{b^{2}+8c a}}+\frac{c}{\sqrt{c^{2}+8ab}}\geq{1}$ for all posi tive real numbers a, b and c.\hfill(IMO 2001)
\item Find all pairs of integer $m \textgreater 2, n \textgreater 2$ such that there are infinetely many positive integers $k$ for which $k^{n}+k^{ 2}-1$ divides $k^{m}+k-1$.\hfill(IMO 2002)

\item The positive divisors of the integer $ n\geq 1$ are $d_{1}\leq d_{2}\leq \ldots \leq d_{k}$ so that $d_{1}=1, d_{k}=n$. Let $d=d_{1}d_{2}+d_{2 }d_{3}+\ldots d_{k}-d_{k}$. Show that $d\leq n^{2}$ and find all $n$ for which $d$ divides $n^{2}$.\hfil 1(IMO 2002)
\item Find all real-valued functions on the reals such that $\brak{f\brak{x}+f\brak{y}}\brak{f\b rak{u}+f\brak{v}}=f\brak{xu-yv}=f\brak{xv-yu}$ for a 11 $x,y,u,v$.\hfill(IMO 2002)
\item Let $a,b$ and $c$ be the lengths of the sides of a triangle. Prove that.
	\begin{align*} a^2b \brak {a-b}+b^2c\brak{b-c}+c^2a\brak{c-a}\geq 0 \end{align*}	Determine when quality occurs. \hfill(IMO 1983)

\item Let $ABC$ be an equilateral triangle and $\epsilon$ the set of aLl points contained in the three segments $AB$, $BC$, and $CA$ (including $A$, $B$, and $C$). Determine whether for every partition of $\epsilon n$ into two disjoint subsets, at least one of the two subsets that contains the vertices of a right-angled triangle. Justify your answer.\hfill(IMO 1983)

	\item For any polynomial $P\brak{x} = a_0 + a_1x + ..... + a_kx^k$ with integer coefficients,  the number of coefficients which are odd is denoted by $w\brak{P}$.  For $i = 0, 1, ..., $let $Q_i\brak{x} = \brak{1+x}^i$. Prove that if $i_1i_2, ..., i_n$ are integers such that  $0\leq i_1<i_2<......<i_n$, then \begin{align*}  w (Q_{i1}+Q_{i2},+....+Q_{in})\geq w (Q_{i1}) \end{align*}\hfill(IMO 1985)                                     
\item Let $f(x)=x^{n}+5x^{n-1}+3$, where $n>1$ is an integer. Prove that $f(x)$ cannot be expressed as the product of two nonconstant polynomials with integer coefficients.  \hfill(IMO 1993)
\end{enumerate}
