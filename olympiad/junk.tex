
%    \item Prove that there is no function $f$ from the set of non-negative integers into itself such that $f\brak{f\brak{n}}=n+1987$ for every $n$.\hfill(IMO  1987)
%\item   A configuration of $4027$ points in the plane is called Colombian if it consists of $2013$ red points and $2014$ blue points, and no three of the points of the configuration are collinear. By drawing some lines, the plane is divided into several regions. An arrangement of lines is good for a Colombian configuration if the following two conditions are satisfied:
% * no line passes through any point of the configuration;
%* no region contains points of both colours
%Find the least value of $k$ such that for any Colombian configuration of $4027$ points, there is a good arrangement of $k$ lines \hfill(IMO  2013)
%\item Let $ABC$ be an acute-angled triangle with orthocentre $H$, and let $W$ be a point on the side $BC$, lying strictly between $B$ and $C$. The points $M$ and $N$ are the fect of the altitudes from $B$ and $C$ respectively. Denote by $w_1$ the circumcircle of $BWN$, and let $X$ be the point on wy such that $WX$ is a diameter of $w_1$ Analogously, denote by $w_2$ the circumcircle of $CWM$. and let $Y$ be the point on such that $WY$ is a diameter of Prove that $X$, $Y$ and Hare collinear. \hfill(IMO  2013)
%\item  Let $ Q_{>0}$ be the set of positive rational mumbers. Let $f: Q_{>0} \rightarrow R$ be a function satisfying the following three conditions:
%	\begin{enumerate}
%		\item for all $x,y\epsilon  Q>0$, we have $f\brak{x} f\brak{y} \geq  f\brak{xy}$
%		\item for all $x,y\epsilon Q>0$,we have$f\brak{x+y} \geq f\brak{x}+f\brak{y}$
%		\item there exists a rational number $a> 1$ such that $f\brak{a}=a$.
%			
%		prove that $F\brak{x}=x$ for all $x \epsilon Q>0$.
%	\end{enumerate} \hfill(IMO  2013)			
%\item  9. let $n\geq 2$ be an integer. Consider an $n\times n$ chessboard consisting of $n^2$ unit squares. A configuration of $n$ rooks on this board is peaceful if every row and every column contains exactly one rook. Find the greatest positive integer $k$ such that, for each peaceful configuration of $n$ rooks, there is a $k\times k$ square which does not contain a rook on any of its $k^2$ unit squares. \hfill(IMO  2014)
	
%\item  11. A set of lines in the plane is in general position if no two are parallel and no three pass through the same point. A set of lines in general position cats the plane into regions, some of which have finite area; we call these its finite regions. Prove that for all sufficiently large $n$. in any set of a lines in general position it is possible to colour at least $\sqrt n$ of the lines blue in such a way that none of its finite regions has a completely blue boundary.
%	Note: Results with $\sqrt n$ replaced by $c \sqrt n$  will be awarded points depending on the value of the constant $c$. \hfill(IMO  2014)
%	
%\item  12. We say that a finite set $S$ of points in the plane is balanced if, for any two different points $A and B$ in $S$, there is a point Cin Ssuch that $AC=BC$. We say that $S$ is centre-free if for any three different points $A, B$ and $C$ in $S$, there is no point $P$ in $S$ such that $PA=PB=PC$
%	\begin{enumerate} 
%
%\item  Show that for all integers $n\geq3$, there exists a balanced set consisting of $n$ points.
%
%\item  Determine all integers $n\geq3$ for which there exists a balanced centre-free set consisting of $n$ points.
%	\end{enumerate}	\hfill(IMO  2015)	
%	
%\item  13. Determine all triples $\brak{a, b, c}$ of positive integers such that each of the numbers
%			 $ ab-c, bc-a,ca-b$\\
%		is a  power of $2$\\(A power of 2 is an integer of the form $2^n$,Where $n$ is a non-negative integer). \hfill(IMO  2015)
%		
%\item  16. Let $R$ be the set of real numbers. Determine all functions $f:R\rightarrow R$ satisfying the equation
%	\begin{align*}
%		f\brak{x+f\brak{x+y}}+f\brak{xy}=x+f\brak{x+y}+yf\brak{x}
%	\end{align*}
%for all real numbers $x$ and $y$ \hfill(IMO 2015)
%
%\item 17 the sequence $a_1,a_2, \ldots$ of an integers satisfies the following conditions;
%	\begin{enumerate}
%		\item $1\leq a_{j} \leq2015$ for all $j\geq 1;$
%		\item $k+a_{k} \neq l+a_{l}$ for all $1\leq k \textless l.$
%	\end{enumerate}	
%prove that there exist two positive integers $b and N$ such that
%
%$\mydet {\sum_{j=m+1}^{n} \brak {aj-b} }\leq 1007^2$
%
%for all integers $m and n$ satisfying $n > m\geq N$ \hfill(IMO  2015)
%\item Prove that the set $\cbrak{1,2,.........,1989}$ can be expressed as the disjoint union of subsets $A_i$\brak{i=1,2,........,117} such that :
%\brak{i} Each $A_i$ contains $17$ elements ;
%		\brak{ii} The sum of all the elements in each $A_i$ is the same . \hfill(IMO  1989)
%
%\item Let $n$ and $k$ be positive integers and let $S$ be a set of $n$ points in the plane such that
%
%\brak{i} No three points of $S$ are collinear, and 
%
%\brak{ii} For any point $P$ of $S$ there are at least $k$ points of $S$ equidistant from $P$. \hfill(IMO  1989)
%
%		Prove that: \begin{align*}k < \frac{1}{2} + \sqrt{2n}.\end{align*}
%\item Let $ABC$ be an acute-angled triangle with circumcentre $O$. Let $P$ on $BC$ be the foot of the altitude from $A$. Suppose that $\angle BC S$ $\leq$ $\angle ABC+30\degree$. \\Prove that $\textless CAB+\leq cop \angle 90^o$.\hfill(IMO 2001)
% \item Let $S$ be a square with sides of length $100$, and let $L$ be a path with in $S$ which does not meet itself and which is composed of line segments $A_0A_1, A_1A_2,.... A_{n-1}A_1$ with $A_0 \neq A_n$.     Suppose that for every point $P$ of the boundary of $S$ there is a point of $L$ at a distance from $P$ not greater than $\frac{1}{2}$. Prove that there are two points $X$ and $Y$ in  $\&$ such that the distance between $X$ and $Y$ is not greater than $1$, and the length of that part of $L$ which lies between $X$ and $Y$ is not smaller than $198$.\hfill(IMO  1982)
\iffalse
\item Prove that there exists a positive constant $c$ such that the following statement is true:
Consider an integer $n> 1$, and a set $S$ of $n$ points in the plane such that the distance between
any
 two different points in $S$ is at least 1. It follows that there is a 
line $l$ separating $S$ such that the distance from any point of $S$ to $l$ is 
at least $cn^\frac{-1}{3}$.
(A line $l$ separates a set of points $S$ if some segment joining two points in $S$ crosses $l$.)
Note. Weaker results with  replaced by $cn^\alpha$ may be awarded points depending on the value of the constant $ \alpha$ \textgreater1/3.
   \hfill(IMO 2020)
	\item On each side of an equilateral triangle with side length $n$ units, where $n$ is an integer, $1 \leq n \leq 100$, consider $n - 1$ points that divide the side into $n$ equal segments.Through these points, draw lines parallel to the sides of the triangle, obtaining a net of equilateral triangles of side length one unit. On each of the vertices of these small triangles, place a coin head up. Two coins are said to be adjacent if the distance between them is $1$ unit. A move consists of flipping over any three mutually adjacent coins. Find the number of values of $n$ for which it is possible to turn all coins tail up after a finite number of moves.\hfill(IOQM 2015)
	    \item The six sides of a convex hexagon $A_{1}A_{2}A_{3}A_{4}A_{5}A_{6}$ are colored red. Each of the diagonals of the hexagon is colored either red or blue. If $N$ is the number of colorings such that every triangle $A_{i}A_{j}A_{k}$, where $1 \leq i < j < k \leq 6$, has at least one red side, find the sum of the squares of the digits of $N$.\hfill(IOQM 2015)
		    \fi
%\item $n>2$ circles of radius $1$ are drawn in the plane so that no line meets more than two of the circles. Their centers are $O_{1}, O_{ 2}\dotsO_{n}$. Show that $\sum_{i\textless}$ $1/O_{i }O_{j}\leq \brak{n-1} \frac{\pi}{4}$.\hfill(IMO 2002)
%\item In the plane two different points $O$ and $A$ are given.  For each point $X$ of the plane, other than $O$, denote by $a\brak{X}$  the measure of the angle between $OA$ and $OX$ in radians counterclockwise from $OA\brak {O\leq a\brak{X}<2\pi}$. Let $C\brak{X}$ be the circle  with center $O$ and radius of length $\frac {OX+a\brak{X}}{OX}$. Each  point  of the plane is colored by one of a finite number of colors. Proveoint $Y$ for which $a\brak{y}>0$ such that color appears on  the circumference of the circle $C\brak{Y}$.\hfill(IMO 1984)
%      \item Let $n_3$ and consider a set $E$ of $2_{n-1}$ distinct points on a circle. Suppose that exactly $k$ of these points are to he colored black. Such a coloring is $"good"$ if there is at least  one pair of black points such that the interior of one of the ares between them contains exactly in points from $E$. Find the smallest value of $k$ so that every such coloring of $k$ points of $E$ is good \hfill(IMO  1990)
%     \item Given an initial integer $n_0 > 1$, two players. $A$ a    nd $B$, choose integers $n_1, n_2 , n_3,.......$ alternately accordi    ng to the following rules:
%           Knowing $n_{2k}$, $A$ chooses any integer $n_{2k+2}$ such that \begin{align*} n_{2k}\leq n_{2k+1} \leq n^{2}_2{k} \end{align*}
%   Knowing $n_{2k+1}$ , $B$ chooses any integer $n_{2k+2}$ such that \begin{align*}
%              \frac{n_{2k+1}}{n_{2k+2}}\end{align*}
%  is a prime raised to a positive integer power.
%	Plaver $A$ wins the game by choosing the number $1990$: player $B$ wins by choosing the number $1$. For which $n_0$ does:
%\brak{a}$A$ have a winning strategy?
%\brak{b} $B$ have a winning strategy?
%\brak{c} Neither player have a winning strategy?\hfill(IMO  1990)
%\item $P$ is a given point inside a given sphere.Three mutually perpendic ular rays from Pintersect the sphere at points $U, V$, and $W$; $Q$ denotes the vertex diagonally opposite to $P$ in the parallelepiped determined by $PU, PV$, and $PW$.     Find the locus of $Q$ for all such triads of rays from $P$\hfill(IMO  1978)

%\item A prism with pentagons $A1 A2 A3 A4 A5$ and $B1 B2 B3 B4 B5$, as top and bottom faces is given. Each side of the two pentagons and each of the line- segments $A,B$ for all $i, j = 1,\ldots,5$, is colored either red or green. Every triangle whose vertices are vertices of the prism and whose sides have all been colored has two sides of a different color. Show that all $10$ sides of the top and bottom faces are the same color.\hfill(IMO  1979) 
%\item Given a plane $\pi$, a point $P$ in this plane and a point $Q$ not in $\pi$, find all points$R$ in $\pi$ such that the ratio $\brak{QP+PA}/Q R$ is a maximum. \hfill(IMO  1979)
%\item Determine all finite sets $S$ of at least three points in the plane which satisfy the following condition:\\for any two distinct points $A$ and $B$ in $S$, the perpendicular bisector of the line segment $AB$ is an axis of symmetry for $S$.\hfill( IMO  1999)

%     \item prove that $x,y,z$ be three positve real such that $xyz$ $\geq{1}$.Prove that                                                            \begin{align*}
%\frac{x^5-x^2}{x^5+y^2+z^2} + \frac{y^5-y^2}{x^2+y^5+z^2} + \frac{z^5-z^2} {x^2+y^2+z^5} \geq{0}                                                 \end{align*} \hfill(IMO  2005)
% \item In a mathematical competition, in which $6$ problems were posed to the participants, every two of these problems were solved by more than $\frac{2}{5}$ of the contestants. Moreover, no contestant solved all the $6$ problems. Show that there are at least $2$ contestants who solved exactly $5$ problems each.\hfill(IMO  2005)
%\item  We are given a positive interger $r$ and a rectangular board $ABCD$ with dimensions $\mydet{AB} =20$, $\mydet{BC}=12$. The rectangle is divided into a grid of $20\times12$ unit squares. The following moves are permitted on the board: one can move from one square to another only if the distance between the centers of the two squares is $\sqrt{r}$. The task is to find a sequence of moves leading from the square with $A$ as a vertex to the square with $B$ as a vertex.
%\begin{enumerate}
%\item Show that the task cannot be done if $r$ is divisible by $2$ or $3$.
% \item Prove that the task is possible when $r=73$.
% \item Can the task be done when $r=97$?\hfill(IMO  1996)
%\end{enumerate}
% \item In the plane the points with integer coordinates are the vertices of unit squares. The squares are colored alternately black and white (as on a chessboard).
%For any pair of positive integers $m$ and $n$, consider a right-angled triangle whose vertices have integer coordinates and whose legs, of lengths $m$ and $n$, lie along edges of the square s.
%Let $S_1$ be the total area of the black part of triangle ans $S_2$ be the total area of white part. Let
%  \begin{align*}
%          f(m,n)=\mydet{S_1-S_2}.
%  \end{align*}
%  \begin{enumerate}
%\item calculate $f(m,n)$ for all positive integers $m$ and $n$ which are either both even or both odd.
% \item Prove that $f(m,n) \leq \frac{1}{2}max\cbrak{m,n}$ for all $m$ and $n$
%  \item Show that there is no constant $C$ such that $f(m,n)<c$ for all $m$ and $n$.\hfill(IMO  1997)
%  \end{enumerate}	
%\item Determine all integers $n>3$ for which there exist $n$ points $A_1\dots,A_n$ in the plane, no three collinear, and real numbers $r_1,\dots,r_n$ such that for $1\leq{i}<{j}<{k}\leq{n}$, the area of $\triangle A_iA_jA_k$ is $r_i+r_j+r_k$.\hfill(IMO  1995)
%\item Let $a$, $b$, $c$ be positive real numbers such that $abc=1$. Prove that.
%\begin{align*}
%\frac{1}{a^3(b+c)}+\frac{1}{b^3(c+a)}+\frac{1}{c^3(a+b)}\geq\frac{3}{2}.\hfill(IMO  1995)
% \end{align*}.
%\item $A$ hunter and an invisible rabbit play a game in the Euclidean plane. The rabbit's starting point, $Ag$, and the hunter's starting point, $Bo$, are the same. After $n-1$ rounds of the game, the rabbit is at point $An-$ and the hunter is at point $B-1$. In the $nth$ round of the game, three things occur in order.\hfill (IMO  2017)        
%	(i) The rabbit moves invisibly to a point $A$, such that the distance between $An-1$ and $A$,, is exactly $1$.                    
%	(ii) $A$ tracking device reports a point $P$, to the hunter. The only guarantee provided by the tracking device to the hunter is that the distance between $P$ and $A$, is at most $1$.
% (iii) The hunter moves visibly to a point $B$, such that the distance between $Bu-1$ and $Bn$ is exactly $1$. Is it always possible, no matter how the rabbit moves, and no matter what points are reported by the tracking device, for the hunter to choose her moves so that after $10$ rounds she can ensure that the distance between her and the rabbit is at most $1002.$
%  \begin{enumerate}[label=(\roman*)]
%  \item The rabbit moves invisibly to a point An such that the distance between $An-1$ and An is exactly $1$.
% \item $A$ tracking device reports a point $Pa$ to the hunter. The only guarantee provided by the tracking device to the hunter is that the distance between $    P$, and $An$, is at most $1$.
%\item The hunter moves visibly to a point $B,$ such that the distance between $B-1$ and $B$, is exactly $1$.
% \end{enumerate}
% Is it always possible, no matter how the rabbit moves, and no matter what points are reported by the tracking device, for the hunter to choose her moves so that after $10$ rounds she can ensure that the distance between her and the rabbit is at most $100?$\hfill (IMO  2017)
%\item Let $R$ and $S$ be different points on a circle and such that $RS$ is not a diameter. Let $E$ be the tangent line to $2$ at $R$. Point $T$ is such that $S$ is the midpoint of the line segment $RT$. Point $J$ is chosen on the shorter are $RS$ of $Q$ so that the circumcircle $I$ of triangle $JST$ intersects ( at two distinct points. Let $A$ be the common point of $I$ and that is closer to $R$. Line $AJ$ meets again at $K$. Prove that the line $KT$ is tangent to $\gamma$.\hfill (IMO  2017)
%\item An integer $N\leq2$ is given. A collection of $N(N+1)$ soccer players, no two of whom are of the same h    eight, stand in a row. Sir Alex wants to remove $N(N-1)$ players from this row leaving a new row of $2N$ players in which the following $V$ conditions hold.\hfill (IMO  2017)                                         
%	\begin{enumerate}                
%		\item no one stands between the two tallest players,        
%\item no one stands between the third and fourth tallest players.                                       
%\item no one stands between the two shortest players.                                                           
%	\end{enumerate}                         
%	Show that this is always possible.
%\item An anti-Pascal triangle is an equilateral triangular array of numbers such that, excep    t for the numbers in the bottom row, each number is the absolute value of the difference of the two numbers immediately below it. For example, the following array is an anti-Pascal triangle with four rows which contains every integer from $1$ to $10$.
%	Does there exist an anti-Pascal triangle with $2018$ rows which contains every integer from \begin{align*}1 to 1+2 +....+2018?\end{align*} \hfill (IMO  2018)
%\item Four real constants $a, b, A, B$ are given, and \begin{align*}
%f\brak{\theta} = 1 - a\cos\theta - b \sin \theta     - A \cos 2\theta - B \sin 2\theta  
%\end{align*}. Prove that if 
%\begin{align*}f\brak{\theta} \textgreater= 0 
%	\end{align*} ,for all real $\theta$, then
%\begin{align*} a^{2} + b^{2} \leq 2 and A^{2} + B^{2} \geq = 1 
%\end{align*}\hfill(IMO  1977)
%
%		
%	
