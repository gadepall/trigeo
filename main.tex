\documentclass[journal]{IEEEtran}
\usepackage[a5paper, margin=10mm]{geometry}
%\usepackage{lmodern} % Ensure lmodern is loaded for pdflatex
\usepackage{tfrupee} % Include tfrupee package


\setlength{\headheight}{1cm} % Set the height of the header box
\setlength{\headsep}{0mm}     % Set the distance between the header box and the top of the text


%\usepackage[a5paper, top=10mm, bottom=10mm, left=10mm, right=10mm]{geometry}

%
\usepackage{gvv-book}
\usepackage{gvv}
%\setlength{\intextsep}{10pt} % Space between text and floats

\makeindex

\begin{document}
\bibliographystyle{IEEEtran}
\onecolumn


\title{
	%\begin{flushleft}
	\begin{center}
	%MATRICES \\ In Geometry
	Trigonometry through Geometry
	\\
\rule{0.4\columnwidth}{0.4pt}
%\end{flushleft}
\end{center}
}
\author{
\vspace{11cm}
	%\begin{flushleft}
	\begin{center}
\includegraphics[width=0.2\columnwidth]{figs/logo.jpg}
\\
		{	\huge G. V. V. Sharma}
	\end{center}
	%\end{flushleft}
%\IEEEpubid{\makebox[\columnwidth]{978-1-7281-5966-1/20/\$31.00 ©2020 IEEE \hfill} \hspace{\columnsep}\makebox[\columnwidth]{ }}
}
\maketitle

\newpage
\section*{About this Book}

This book introduces trigonometry through high school geometry. This approach relies more on trigonometric equations than cumbersome constructions which are usually non intuitive.
 All problems in the book are from NCERT mathematics textbooks from Class 9-12.  Exercises are from CBSE board exam papers.   

The content is sufficient for all practical applications of trigonometry.
There is no copyright, so readers are free to print and share.  

This book is dedicated to my high school maths teacher, Dr. G.N. Chandwani. 
\begin{flushright}
\today
\end{flushright}
Github: https://github.com/gadepall/matgeo
		\\
License: https://creativecommons.org/licenses/by-sa/3.0/
\\
and
\\
https://www.gnu.org/licenses/fdl-1.3.en.html

\newpage


\tableofcontents

\newpage
%\twocolumn
\onecolumn


%\renewcommand{\theequation}{\theenumi}
\numberwithin{equation}{enumi}
\numberwithin{figure}{enumi}
%\renewcommand{\thefigure}{\theenumi}
\renewcommand{\thetable}{\theenumi}

\section{Triangle}
\subsection{Formulae}
\begin{enumerate}[label=\thesubsection.\arabic*.,ref=\thesubsection.\theenumi]
	\item 
A right angled triangle looks like Fig. \ref{fig:tri_right_angle}.
\begin{figure}[!ht]
\centering
\resizebox{0.75\columnwidth}{!}{\input{./figs/trig/tri_right_angle.tex}}
\caption{Right Angled Triangle}
\label{fig:tri_right_angle}	
\end{figure}
with angles $\angle A,\angle B$ and $\angle C$ and sides $a, b$ and $c$.  The unique feature of this triangle is $\angle B$ which is defined to be $90\degree$.
\item
	For simplicity, let the greek letter $\theta = \angle C$.  We have the following definitions.
\begin{equation}
\label{eq:tri_trig_defs}
\begin{matrix}
	\sin \theta = \frac{c}{b} & 	\cos \theta = \frac{a}{b} \\[1ex]
	\tan \theta = \frac{c}{a} & \cot \theta = \frac{1}{\tan \theta} \\[1ex]
	\csc \theta = \frac{1}{\sin \theta} & \sec \theta = \frac{1}{\cos \theta}
	\end{matrix}
\end{equation}
\item  
	\begin{equation}
	\cos \theta = \sin \brak{90\degree - \theta}
	\label{eq:tri_baudh_comp}	
	\end{equation}
\end{enumerate}

\subsection{Problems}
\begin{enumerate}[label=\thesubsection.\arabic*.,ref=\thesubsection.\theenumi]
\item $ABCD$ is a trapezium in which $AB  \parallel  DC$ and its diagonals intersect each other at the point $O$. Show
that
$\frac{AO}{ BO}=\frac{CO}{  DO}$
\item In an isosceles $\triangle ABC$, with $AB = AC$, the bisectors of $\angle B$ and $\angle C$ intersect each other at $O$. Join $A$ to $O$. Show that :
\begin{enumerate} 
\item $OB = OC$ 
\item $AO$ bisects $\angle A$
\end{enumerate}
\item $ABC$ is an isosceles triangle in which altitudes $BE$ and $CF$ are drawn to equal sides $AC$ and $AB$ respectively . Show that these altitudes are equal.
%
\item $ABC$ is a triangle in which altitudes $BE$ and $CF$ to sides $AC$ and $AB$ are equal. Show that
%
\begin{enumerate} 
\item $\triangle  ABE \cong  \triangle  ACF $
\item  $AB = AC$, i.e., $ABC$ is an isosceles triangle.
\end{enumerate}
\item $ABCD$ is a trapezium in which $AB  \parallel  DC$ and its diagonals intersect each other at the point $O$. Show
that
$\frac{AO}{ BO}=\frac{CO}{  DO}$
%
\item  $D$ is a point on the side $BC$ of a $\triangle ABC$ such that  $\angle  ADC =  \angle  BAC$. Show that $CA^2 = CB.CD$.
%
\item $BL$ and $CM$ are medians of a $\triangle ABC$ right angled at $A$. Prove that $4 (BL^2 + CM^2
) = 5 BC^2$ .
\item $O$ is any point inside a rectangle $ABCD$. Prove that $OB^2+OD^2 = OA^2+OC^2$.
\item   $D$ is a point on side $BC$ of  $\triangle  ABC$ such that
$\frac{BD}{CD}= \frac{AB}{AC}  $.  Prove that $AD$ is the bisector of  $\angle  BAC$.
\item $\vec{Q}$ is a point on the side $\vec{SR}$ of $\triangle \vec{PSR}$ such that $\vec{PQ=PR}$. Prove that $\vec{PS>PQ}$.
\item $\vec{S}$ is any point on side $\vec{QR}$ of a $\triangle \vec{PQR}$. Show that $\vec{PQ+QR+RP>2PS}$.
\item $\vec{D}$ is any point on side $\vec{AC}$ of a $\triangle \vec{ABC}$ with $\vec{AB=AC}$. Show that $\vec{CD<BD}$.
\item $AD$ is the bisector of $\angle BAC$. Prove that $AB>BD$.
\item Prove that sum of any two sides of a triangle is greater than twice the median with respect to the third side.
\item Prove that in a triangle, other than an equilateral triangle, angle opposite the longest side is greater than $\frac{2}{3}$ of a right angle.
\item P is a point in the interior of a parallelogram $ABCD$. Show that
\begin{enumerate}
\item $ar (APB) + ar (PCD) = \frac{1}{ 2}ar (ABCD)$
\item $ar (APD) + ar (PBC) = ar (APB) + ar (PCD)$
\end{enumerate}
%
\item $PQRS$ and $ABRS$ are parallelograms and $X$ is any point on side $BR$. show that 
\begin{enumerate} 
\item $ar (PQRS) = ar (ABRS)$
\item $ar (AX S) = \frac{1}{ 2} ar (PQRS)$
\end{enumerate}
\item $ABCD$ is a quadrilateral and $BE  \parallel  AC$ and also $BE$ meets $DC$ produced at $E$. Show that area of $ \triangle  ADE$ is equal to the area of the quadrilateral $ABCD$.
%
\item $E$ is any point on median $AD$ of a  $\triangle  ABC$. Show that $ar (ABE) = ar (ACE)$.
\item  In a $\triangle ABC, E$ is the mid-point of median $AD$. Show that $ar (BED) = \frac{1}{ 4}ar(ABC)$ .
%
\item AB is a diameter of the circle, $CD$ is a chord equal to the radius of the circle. $AC$ and $BD$ when extended intersect at a point $E$. Prove that $\angle AEB = 60\degree$.
\item If the non-parallel sides of a trapezium are equal, prove that it is cyclic.
\item Prove that the line of centres of two intersecting circles subtends equal angles at the
two points of intersection.
%
\end{enumerate}

\section{Applications}
\input{./chapters/trig/appl.tex}
%\begin{enumerate}[label=\thesection.\arabic*.,ref=\thesection.\theenumi]
%\input{./chapters/trig/tri_geo_baudh}
%\renewcommand{\theequation}{\theenumi}
\begin{enumerate}[label=\thesection.\arabic*.,ref=\thesection.\theenumi]
\numberwithin{equation}{enumi}
\numberwithin{figure}{enumi}

\end{enumerate}

%\end{enumerate}
\section{Inequalities}
\begin{enumerate}[label=\thesection.\arabic*.,ref=\thesection.\theenumi]
\item $D$ is a point on side $BC$ of $\triangle  ABC$ such that $AD = AC$. Show that $AB > AD$
\item Show that in a right angled triangle, the hypotenuse is the longest side.
\item Sides AB and AC of $\triangle  ABC$ are extended to points P and Q respectively. Also, $\angle  PBC < \angle  QCB$. Show that $AC > AB$.

\item Line segments $AD$ and $BC$ intersect at $O$ and form $\triangle OAB$ and $\triangle ODC$. $\angle  B < \angle  A$ and $\angle  C < \angle  D$. Show that $AD < BC$.

\item $AB$ and $CD$ are respectively the smallest and longest sides of a quadrilateral $ABCD$. Show that $\angle  A > \angle  C$ and $\angle  B > \angle  D$.
%
\item In $\triangle PQR,  PR > PQ$ and $PS$ bisects $\angle  QPR$. Prove that $\angle  PSR > \angle  PSQ$.
\item $\vec{Q}$ is a point on the side $\vec{SR}$ of $\triangle \vec{PSR}$ such that $\vec{PQ=PR}$. Prove that $\vec{PS>PQ}$.
\item $\vec{S}$ is any point on side $\vec{QR}$ of a $\triangle \vec{PQR}$. Show that $\vec{PQ+QR+RP>2PS}$.
\item $\vec{D}$ is any point on side $\vec{AC}$ of a $\triangle \vec{ABC}$ with $\vec{AB=AC}$. Show that $\vec{CD<BD}$.
\item $AD$ is the bisector of $\angle BAC$. Prove that $AB>BD$.
\item Prove that sum of any two sides of a triangle is greater than twice the median with respect to the third side.
\item Prove that in a triangle, other than an equilateral triangle, angle opposite the longest side is greater than $\frac{2}{3}$ of a right angle.
\item AD is a median of the triangle ABC. Is it true that AB + BC + CA $ > $ 2AD? 
\item M is a point on side BC of a triangle ABC such that AM is the bisector of $ \angle{BAC} $. Is it true to say that perimeter of the triangle is greater than 2AM?
\item Parallelogram $ABCD$ and rectangle $ABEF$ are on the same base $AB$ and have equal areas. Show that the perimeter of the parallelogram is greater than that of the rectangle.
\end{enumerate}

\section{Altitudes of a Triangle}
%\subsection{Properties}
%
\renewcommand{\theequation}{\theenumi}
\begin{enumerate}[label=\thesection.\arabic*.,ref=\thesection.\theenumi]
\numberwithin{equation}{enumi}
\numberwithin{figure}{enumi}
%
%
\item 
	In \figref{fig:tri_alt_h},	
$AD \perp BC$ and $BE \perp AC$ are defined to be the altitudes of $\triangle ABC$. 
\item Let $\vec{H}$ be the intersection of the altitudes $AD$ and $BE$ as shown in Fig. \ref{fig:tri_alt_h}.  $CH$ is extended to meet $AB$ at $\vec{F}$.  Show that $CF \perp AB$.
%
\begin{figure}[!ht]
	\begin{center}
		\resizebox{\columnwidth}{!}{\input{./figs/coord/tri_alt_h.tex}}
	\end{center}
	\caption{Altitudes of a triangle meet at the orthocentre $H$}
	\label{fig:tri_alt_h}	
\end{figure}
%
\\
\solution 
From 
\eqref{eq:tri_cos_form-orth},
%
\begin{align}
\brak{\vec{B}-\vec{C}}^{\top}\brak{\vec{H}-\vec{A}} &= 0  
\\
\brak{\vec{C}-\vec{A}}^{\top}\brak{\vec{H}-\vec{B}} &= 0  
\end{align}
%
Adding both the above and simplifying, 
%
\begin{align}
\brak{\vec{B}-\vec{A}}^{\top}\brak{\vec{H}-\vec{C}} &= 0  
\end{align}
%
$\implies CH \perp AB$, or $CF \perp AB$.  
%
\iffalse
The python code for  Fig. \ref{fig:tri_alt_h} is
\begin{lstlisting}
codes/triangle/tri_alt_h.py
\end{lstlisting}
%
and the equivalent latex-tikz code is
%
\begin{lstlisting}
figs/triangle/tri_alt_h.tex
\end{lstlisting}
\fi
\item Altitudes of a $\triangle$ meet at the {\em orthocentre} $H$.
%
\end{enumerate}

\section{Median}
\begin{enumerate}[label=\thesection.\arabic*.,ref=\thesection.\theenumi]
\numberwithin{equation}{enumi}
  \item In 
	\figref{fig:tri_med_isect}	
	\begin{align}
	AF = BF, \,
	AE = BE, 
	\end{align}
	and the medians $BE$ and $CF$ meet at $\vec{G}$.
	Show that
\begin{align}
	ar\brak{BEC}
	=ar\brak{BFC} = \frac{1}{2}ar\brak{ABC}
	\label{eq:median-area}
\end{align}
\solution
	From \eqref{tri:area-sin},
\begin{align}
	ar\brak{BEC} &= 
	\frac{1}{2}a\brak{\frac{b}{2}}\sin C 
	\\
	ar\brak{BFC}&=
	\frac{1}{2}a\brak{\frac{c}{2}}\sin  B
\end{align}
yielding
	\eqref{eq:median-area}.
\item 
	Show that
\begin{align}
	ar\brak{CGE}
	=ar\brak{BGF} 
	\label{eq:median-sub-area}
\end{align}
\solution 
From 
	\figref{fig:tri_med_isect}	
	and 
	\eqref{eq:median-area},
\begin{align}
	ar\brak{BGF}
	+
	ar\brak{BGC}
	=
	ar\brak{CGE}
	+
	ar\brak{BGC}
\end{align}
yielding 
	\eqref{eq:median-sub-area}.
\item 
	If $\vec{G}$ divides $BE$ and $CF$ in the ratios $k_1$ and $k_2$ respectively, 
	show that
%	Using Fig. \ref{ch2_median_ratio_val}, 
	\begin{align}
\label{eq:tri_med_centroid_ratio}
k_1 = k_2 
	\end{align}
%
\begin{figure}[!ht]
	\begin{center}
		\resizebox{\columnwidth}{!}{\input{./figs/coord/tri_med_isect.tex}}
%		\resizebox{\columnwidth}{!}{\input{./figs/coord/tri_med_meet.tex}}
	\end{center}
	\caption{$k_1=k_2$.}
	\label{fig:tri_med_isect}	
	%\label{fig:tri_med_meet}	
\end{figure}
\solution 
Let 
\begin{align}
	GE = l_1, GF = l_2
\end{align}
From 
	\eqref{tri:area-sin}
	and 
	\eqref{eq:median-sub-area},
\begin{align}
	\frac{1}{2}l_1 \brak{k_2l_2} \sin \theta
	= \frac{1}{2}l_2 \brak{k_1l_1}\sin \theta
\end{align}
yielding
\eqref{eq:tri_med_centroid_ratio}.
\item Show that
\begin{align}
	k_1 = k_2 = 2
\end{align}
\solution 
Let 
\begin{align}
	  \label{eq:section_formula-G-2}
	k_1 = k_2 = k
\end{align}
Using 
	  \eqref{eq:section_formula},
  \begin{align}
	  \label{eq:section_formula-G}
\vec{G} = 
	   \frac{k\vec{E}+ \vec{B}}{k+1}
	  &= \frac{k\vec{F}+ \vec{C}}{k+1}
	  \\
	  \implies 
	   k\brak{\frac{\vec{A}+\vec{C}}{2}}+ \vec{B}
	  &= k\brak{\frac{\vec{A}+\vec{B}}{2}}+ \vec{C}
	  \label{eq:section_formula-G-val}
	  \\
	  \implies 
	   k\brak{\vec{B}-\vec{C}}
	  &= 
	   2\brak{\vec{B}-\vec{C}}
  \end{align}
  resulting in 
	  \eqref{eq:section_formula-G-2}.
  \item Substituting $k = 2$ in \eqref{eq:section_formula-G-val},
\begin{align}
	\vec{G} = \frac{\vec{A}+\vec{B}+\vec{C}}{3}
	  \label{eq:centroid-G}
\end{align}
\item 
In	\figref{fig:tri_med_meet},	
$AG$ is extended to join $BC$ at $\vec{D}$.  Show that $AD$ is also a median.
\begin{figure}[!ht]
	\begin{center}
%		\resizebox{\columnwidth}{!}{\input{./figs/coord/tri_med_isect.tex}}
		\resizebox{\columnwidth}{!}{\input{./figs/coord/tri_med_meet.tex}}
	\end{center}
	\caption{$k_3 = 2, k_4 =1$}
%	\label{fig:tri_med_isect}	
	\label{fig:tri_med_meet}	
\end{figure}
	\\
	\solution Considering the ratios in 
	\figref{fig:tri_med_meet},	
  \begin{align}
\vec{G} = 
	  \frac{k_3\vec{D}+\vec{A} }{k_3+1} 
	  \\
	\vec{D}  =\frac{k_4\vec{C}+\vec{B} }{k_4+1} 
  \end{align}
  Substituting from 
	  \eqref{eq:centroid-G}
	  in the above, 
  \begin{align}
	  \brak{k_3+1}\brak{\frac{\vec{A}+\vec{B} + \vec{C}}{3}}
 = 
	  {k_3\brak{\frac{k_4\vec{C}+\vec{B} }{k_4+1}} +\vec{A} } 
  \end{align}
  which can be expressed as
\begin{multline}
\brak{k_3+1}\brak{k_4+1}\brak{{\vec{A}+\vec{B} + \vec{C}}}
 = 
 \\
	  {3} \cbrak{ {k_3\brak{{k_4\vec{C}+\vec{B} }} +\brak{k_4+1}\vec{A} }} 
  \end{multline}
  which can be expressed as
  \begin{multline}
	  \brak{k_3k_4+k_3-2k_4-2}\vec{A}
	  \\
	-  \brak{-k_3k_4-k_4+2k_3-1}\vec{B}
	  \\
	  - \brak{-k_3-k_4 - 1 
+2k_3k_4} \vec{C} = \vec{0}
  \end{multline}
  Comparing the above with 
	  \eqref{eq:two-tri-indep},
  \begin{align}
	  p = {-k_3k_4-k_4+2k_3-1}, q = {-k_3-k_4 - 1 
+2k_3k_4}
  \end{align}
  yielding 
  \begin{align}
	  \label{eq:centroid-G-meet-1}
	   {-k_3k_4-k_4+2k_3-1} = 0
	   \\ {-k_3-k_4 - 1 
+2k_3k_4} = 0
	  \label{eq:centroid-G-meet-2}
  \end{align}
  Subtracting 
	  \eqref{eq:centroid-G-meet-1}
	  from
	  \eqref{eq:centroid-G-meet-2},
  \begin{align}
	  3k_3\brak{k_4-1} &= 0
	  \\
	  \implies k_4&=1
  \end{align}
  which upon substituting in 
	  \eqref{eq:centroid-G-meet-1}
	  yields
  \begin{align}
	  k_3 = 2
  \end{align}
\end{enumerate}

\section{Angle Bisectors}
\input{./chapters/trig/ang.tex}
\section{Perpendicular Bisectors}
%%
%\subsection{Perpendicular Bisectors}
\renewcommand{\theequation}{\theenumi}
\begin{enumerate}[label=\thesection.\arabic*.,ref=\thesection.\theenumi]
\numberwithin{equation}{enumi}
\item In 
	\figref{fig:tri-isosc},	
\begin{figure}[!ht]
	\begin{center}
		\resizebox{\columnwidth}{!}{\input{./figs/trig/tri_isosc.tex}}
	\end{center}
	\caption{Isosceles Triangle}
	\label{fig:tri-isosc}	
\end{figure}
\begin{align}
	OB = OC=R
\end{align}
Such a triangle is known as an isosceles triangle.  Show that
\begin{align}
	\angle B = \angle C
\end{align}
\solution 
Using
\eqref{eq:tri_sin_form},
\begin{align}
	\frac{\sin B}{R} &= \frac{\sin C}{R}
	\\
\implies	{\sin B} &= {\sin C}
\\
	\text{or, } \angle B &= \angle C.
\end{align}
\item In 
	\figref{fig:tri-isosc},	
	show that 
  \begin{align}
	  a = 2R \sin\frac{ \theta }{2}
\label{eq:crad_cos2a}
  \end{align}
		\solution In $\triangle OBC$,  using the cosine formula from
\eqref{eq:tri_cos_form},
\begin{align}
	\cos \theta &= \frac{R^2+R^2 - a^2}{2R^2} = 1 -\frac{a^2}{2R^2}
	\\
	\implies \frac{a^2}{2R^2}&= 2\sin^2\frac{\theta}{2}
\end{align}
yielding 
\eqref{eq:crad_cos2a}.
\item In
	\figref{fig:tri_ccircle-ang},
show that 
\begin{align}
\label{eq:tri_crad_R}
\frac{a}{\sin A} = \frac{b}{\sin B} = \frac{c}{\sin C} = 2R.
\end{align}
%
%
\solution
From 
\eqref{eq:ang-subtend-ccentre}
and 
\eqref{eq:crad_cos2a}
  \begin{align}
	  a = 2R \sin A
  \end{align}

\item In 
	\label{prob:tri-ccentre-def}
	\figref{fig:tri-perp-bis}, 
\begin{align}
OB = OC=R, 	BD = DC.
\end{align}
Show that $OD \perp BC$.
%
\begin{figure}[!ht]
	\begin{center}
		
		\resizebox{\columnwidth}{!}{\input{./figs/coord/tri-perp-bis.tex}}
	\end{center}
	\caption{Perpendicular bisector.}
	\label{fig:tri-perp-bis}	
%github/geometry/figs/
\end{figure}
\\
\solution 
\begin{align}
	\norm{\vec{O}-\vec{C}} &=
\norm{\vec{O}-\vec{B}} =R
\\
	\implies \norm{\vec{O}-\vec{C}}^2 &=
\norm{\vec{O}-\vec{B}}^2 
\end{align}
which can be expressed as 
\begin{align}
%  \label{eq:norm2d_dist}
	\brak{\vec{O}-\vec{C}}^{\top} \brak{\vec{O}-\vec{C}}&=
	\brak{\vec{O}-\vec{B}}^{\top} 
\brak{\vec{O}-\vec{B}}
\\
\norm{\vec{O}}^2-2{\vec{O}}^{\top}\vec{C} + \norm{\vec{C}}^2
	&= \norm{\vec{O}}^2-2{\vec{O}}^{\top}\vec{B} + \norm{\vec{B}}^2
	\\
	\implies 
	  \brak{\vec{B}-\vec{C}}^{\top}{\vec{O}} 
	  &=  \frac{\norm{\vec{B}}^2 - \norm{\vec{C}}^2}{2}
\end{align}
which can be simplified to obtain
  \begin{align}
	  \brak{\vec{B}-\vec{C}}^{\top}\cbrak{{\vec{O}}- 
	    \brak{\frac{{\vec{B}} + {\vec{C}}}{2}}}=0
  \label{eq:norm2d_equidist}
  \\
	  \text{or, }
	  \brak{\vec{B}-\vec{C}}^{\top}\cbrak{{\vec{O}}- \vec{D}}=0
  \label{eq:norm2d_equidist-conv}
  \end{align}
  which proves the give result using 
	  \eqref{eq:section_formula}
	  and 
\eqref{eq:tri_cos_form-orth}.
\item In 
	\figref{fig:tri_ccentre},
$OD$ and $OE$ are the perpendicular bisectors of sides $BC$ and $AC$ respectively.  Show that 
$OA = R$.
\begin{figure}[!ht]
	\begin{center}
		
		\resizebox{\columnwidth}{!}{%Code by GVV Sharma
%December 9, 2019
%released under GNU GPL
%Locating the circumcentre

\begin{tikzpicture}
[scale=2,>=stealth,point/.style={draw,circle,fill = black,inner sep=0.5pt},]

%Triangle sides
\def\a{5}
\def\b{6}
\def\c{4}
 
%Coordinates of A
%\def\p{{\a^2+\c^2-\b^2}/{(2*\a)}}
\def\p{0.5}
\def\q{{sqrt(\c^2-\p^2)}}

%Labeling points
\node (A) at (\p,\q)[point,label=above right:$A$] {};
\node (B) at (0, 0)[point,label=below left:$B$] {};
\node (C) at (\a, 0)[point,label=below right:$C$] {};

\node (D) at ($(B)!0.5!(C)$)[point,label=below:$D$] {};
\node (E) at ($(A)!0.5!(C)$)[point,label=right:$E$] {};
%Circumcentre

\node (O) at (2.5,1.70084013)[point,label=left:$O$] {};

%Drawing triangle ABC
\draw (A) -- node[left] {} (B) -- node[below] {} (C) -- node[above,yshift=2mm] {} (A);
%Drawing OA, OB, OC
\draw (O) -- node[left] {} (A);
\draw (O) -- node[below] {$\textrm{R}$} (B);
\draw (O) -- node[below] {$\textrm{R}$} (C);
\draw (O) --   (D);
\draw (O) --   (E);

\tkzMarkRightAngle[fill=blue!30,size=.2](C,D,O)
\tkzMarkRightAngle[fill=blue!30,size=.2](O,E,C)
%\tkzMarkAngle[fill=blue!50,size=.3](O,C,B)


%\tkzMarkAngle[fill=red!10](O,A,C)
%\tkzMarkAngle[fill=red!10](A,C,O)


%\tkzMarkAngle[fill=orange!50,size=.3](B,A,O)
%\tkzMarkAngle[fill=orange!50,size=.3](O,B,A)
%
%\tkzLabelAngle[pos=0.5](O,C,B){$\theta_1$}
%\tkzLabelAngle[pos=0.5](O,B,C){$\theta_1$}
%\tkzLabelAngle[pos=0.5](O,A,B){$\theta_2$}
%\tkzLabelAngle[pos=0.5](O,B,A){$\theta_2$}
%\tkzLabelAngle[pos=1.5](O,A,C){$\theta_3$}
%\tkzLabelAngle[pos=1.5](O,C,A){$\theta_3$}

\end{tikzpicture}
}
	\end{center}
	\caption{ Perpendicular bisectors of $\triangle ABC$ meet at $\vec{O}$.}
	\label{fig:tri_ccentre}	
\end{figure}
\\
\solution Tracing
  \eqref{eq:norm2d_equidist-conv}
  backwards yields
\begin{align}
	OB = OC, OC = OA = R.
\end{align}
\end{enumerate}

\section{Circumcircle: Circle Equation}
\input{./chapters/coord/tri_geo_ccentre}
\section{Tangent}
\input{./chapters/coord/circ_geo_prop}
\appendices
\section{Collinear Points}
%\renewcommand{\theequation}{\theenumi}
%\begin{enumerate}[label=\arabic*.,ref=\theenumi]
\begin{enumerate}[label=\thesection.\arabic*.,ref=\thesection.\theenumi]
\numberwithin{equation}{enumi}
\item In 
	\figref{fig:line-eq},
\begin{align}
			\label{eq:line-school}
	a &= y \cot \theta + x
	\\
	\implies \vec{D} &= \myvec{-x \\ y} = \myvec{-a + y\cot \theta \\ y}
	\\
	&=\myvec{-a \\ 0}+y\cot \theta \myvec{ 1 \\ \tan \theta }
	\\
	\text{or, }\vec{D}&\equiv \vec{B} + \kappa \vec{m}
\end{align}
The above equation can be generalized for any point on the line $AB$ as
\begin{align}
\vec{x} = \vec{B} + \kappa \vec{m}
\label{eq:geo-param}
\end{align}
		which is known as the {\em parametric} equation of a line.
		$\vec{m}$ is defined to be the {\em direction vector} of $AB$ and
\begin{align}
	m = \tan \theta
\end{align}
		is defined to be the {\em slope}.
\begin{figure}[!ht]
	\begin{center}
		\resizebox{\columnwidth}{!}{%Code by GVV Sharma
%December 7, 2019
%released under GNU GPL
%Proof of Baudhyana Theorem


\begin{tikzpicture}
[scale=2,>=stealth,point/.style={draw,circle,fill = black,inner sep=0.5pt},]

%Triangle sides
\def\a{4}
\def\c{3}
\def\b{sqrt(\a^2+\c^2)}

%Trigonometric ratios
\def\ct{\a/\b}
\def\st{\c/\b}

%perp distance
\def\r{\a*\st}

%Section Ratio
\def\k{1.2}

%Labeling points
\node (A) at (0,\c)[point,label=above left:$A$] {};
\node (B) at (-\a, 0)[point,label=below right:$B$] {};
\node (C) at (0, 0)[point,label=below left:$C$] {};

%Foot of perpendicular
\node (D) at ($({-\r*\st}, {\r*\ct})$)[point,label=above left:$D$] {};
%Coordinates of point E (foot of perpendicular DE on BC)
\coordinate (E) at ($(B)!(D)!(C)$);
%Labeling point E
\node [point,label=below right:$E$] at (E) {};


%Drawing triangle ABC
\draw (A) -- node[left] {$\textrm{c}$} (B) -- node[below] {$\textrm{a}$} (C) -- node[above,xshift=2mm] {$\textrm{b}$} (A);

%Drawing perpendicular DE
\draw[dashed] (D) -- node[right] {} (E);

%Adding label for DE
\node [right] at ($ (D)!0.5!(E) $) {$ y $};

%Adding label for CE
\node [below] at ($ (C)!0.5!(E) $) {$ x $};


%Drawing and marking angles
\tikzset{my angle/.style={fill=#1!40, size=0.5cm, mark=}}
\tkzMarkRightAngle[fill=blue!20,size=.2](A,C,B)
\tkzLabelAngle[pos=0.65](C,B,A){$\theta$}
\tkzMarkAngle[fill=orange!40,size=0.5cm,mark=](C,B,A)
\end{tikzpicture}


}
%		\resizebox{\columnwidth}{!}{\input{./figs/coord/tri_med_meet.tex}}
	\end{center}
	\caption{$k_1=k_2=2$.}
	\label{fig:line-eq}	
	%\label{fig:tri_med_meet}	
\end{figure}
\item The direction vector of the line $AB$ is
\begin{align}
	\vec{A}-
	\vec{B} \equiv
	\vec{B}-
	\vec{A} \equiv \kappa \myvec{1 \\ m},
\label{eq:dir-vec}
\end{align}
			
\item 			\eqref{eq:line-school} can also be expressed as
\begin{align}
	a &= y \cot \theta + x
	\\
\implies 	\myvec{-\tan \theta & 1}\myvec{-x \\ y} &= b
  \label{eq:dot2d}
	\\
	 \text{or, }\vec{n}^{\top}\vec{x} &= b
\label{eq:geo-normal}
\end{align}
		which is known as the {\em normal} equation of a line.
		Here, 
\begin{align}
	\vec{n} = \myvec{-m \\ 1} 
\end{align}
		is defined to be the {\em normal vector} of the line.
		%	
The vector product in 
  \eqref{eq:dot2d}
  is known as the 
{\em inner product} or {\em dot product} 
%
\item It is easy to verify that
%
\begin{align}
\label{eq:dir_normal_orth}
\vec{n}^{\top}\vec{m} &= 0
\end{align}
%
and
\item 
%
\begin{align}
\vec{n} = \myvec{0 & -1 \\ 1 & 0}\vec{m}
	= \myvec{\cos\brak{\frac{\pi}{2}} & \sin\brak{\frac{\pi}{2}} \\  \sin\brak{\frac{\pi}{2}}& \cos\brak{\frac{\pi}{2}}}\vec{m}
\label{eq:dir_normal_orth-rot}
\end{align}
The matrix 
%
\begin{align}
	\vec{R}_{\theta} 
	= \myvec{\cos	\theta & \sin	\theta \\  \sin	\theta& \cos	\theta}
\label{eq:rot}
\end{align}
is defined to be the {\em rotation matrix}.
\eqref{eq:dir_normal_orth-rot} implies that $\vec{n}$ can be obtained from $\vec{m}$ through a $90 \degree$ clockwise rotation.
  \item From \eqref{eq:geo-param}, 
	  since $\vec{A},\vec{D}$ and $\vec{C}$ are on the same line,
\begin{align}
\begin{split}
	\vec{D}&=\vec{A}+q\vec{m} 
			\\ 
			\vec{B}&=\vec{D}+p\vec{m} 
\end{split}
	\\
			\label{eq:collinear} 
			\implies 	p\brak{\vec{D}-\vec{A}} 
			+ q\brak{\vec{D}-\vec{B}} &= 0, \quad p, q \ne 0 \\ 
			\implies \vec{D} &= \frac{k\vec{A}+\vec{B}}{k+1}, \quad k = \frac{p}{q}.
	  \label{eq:section_formula}
			\end{align} 
	which is known as {\em section formula}. $\brak{\vec{D}-\vec{A}}, \brak{\vec{D}-\vec{B}}$ 
		are then said to be {\em linearly dependent}.
  \item Consequently, points $\vec{A},\vec{B}$ and $\vec{C}$ form a triangle  if 
	  \label{prop:two-tri-indep}
  \begin{align}
	  p\brak{\vec{A}- \vec{B}} +q\brak{\vec{C} -\vec{B}} 
	  \\
	  =\brak{p+q}\vec{B}- p\vec{A} -q\vec{C} = 0
	  \\
	  \implies p=0, q=0
	  \label{eq:two-tri-indep}
  \end{align}
\end{enumerate}
\section{Cosine Formula}
\begin{enumerate}[label=\thesection.\arabic*.,ref=\thesection.\theenumi]
\numberwithin{equation}{enumi}
%
\item
In Fig. \ref{fig:tri_cosine_formula}, show that
%
\begin{equation}
\label{eq:tri_cos_mat}
\begin{pmatrix}
0 & c & b \\
c & 0 & a \\
b & a & 0
\end{pmatrix}
\begin{pmatrix}
\cos A \\
\cos B \\
\cos C
\end{pmatrix}
= 
\begin{pmatrix}
a\\
b\\
c
\end{pmatrix}
\end{equation}
%
%
\begin{figure}[!ht]
	\begin{center}
		
		%\includegraphics[width=\columnwidth]{./figs/ch2_triang_ar}
		%\vspace*{-10cm}
		\resizebox{\columnwidth}{!}{\input{./figs/triangle/tri_cosine_formula.tex}}
	\end{center}
	\caption{The cosine formula}
	\label{fig:tri_cosine_formula}	
\end{figure}
\solution From Fig. \ref{fig:tri_cosine_formula}, 
%
\begin{align}
	a &= x + y = b \cos C + c \cos B = \myvec{  \cos C & \cos B } \myvec{ b \\ c }
	\\
&=\myvec{0 & b & c } \myvec{ \cos A \\ \cos C \\ \cos B } 
\end{align}
%
Similarly,
%
\begin{align}
b &= c \cos A + a \cos C 
=\myvec{c & 0 & a } \myvec{ \cos A \\ \cos C \\ \cos B } 
	\\
c &= b \cos A + a \cos B
=\myvec{b & a & 0 } \myvec{ \cos A \\ \cos C \\ \cos B } 
\end{align}
%
The above equations can be expressed in matrix form as
\eqref{eq:tri_cos_mat}.

\item Show that 
\begin{equation}
\label{eq:tri_cos_form}
\cos A = \frac{b^2+c^2-a^2}{2bc}
\end{equation}
%
\solution 
Using the properties of determinants,
%
\begin{align}
\cos A = \frac{
\begin{vmatrix}
a & c & b \\
b & 0 & a \\
c & a & 0
\end{vmatrix}
	}
	{
\begin{vmatrix}
0 & c & b \\
c & 0 & a \\
b & a & 0
\end{vmatrix}
	}
	=\frac{ab^2 + ac^2 - a^3}{abc + abc} 
= \frac{b^2 + c^2 - a^2}{2abc}
\end{align}
\item The {\em norm} of $\vec{A}$ is defined as
\begin{align}
  \label{eq:norm2d}
	\norm{\vec{A}} 
  &= \sqrt{\vec{A}^{\top} \vec{A}}= \sqrt{a_1^2+a_2^2}
\end{align}
\item In 
	\figref{fig:tri_baudh}	
%\figref{fig:tri_right_angle},	
it is easy to verify that 
\begin{align}
\norm{\vec{A}-\vec{C}}^2  
  = \myvec{-a & c} \myvec{-a \\ c}
= a^2 + c^2 = b^2
\end{align}
from 
	\eqref{eq:tri_baudh}.
Thus, 
	the distance betwen any two  points $\vec{A}$ and $\vec{B}$ is given by 
\begin{align}
  \label{eq:norm2d_dist}
\norm{\vec{A}-\vec{B}} 
\end{align}
  \item In 
	\figref{fig:tri_cosine_formula}	
	show that 
\begin{equation}
	\cos A= 	\frac{\brak{\vec{A}-
	\vec{B}}^{\top}\brak{\vec{A}-\vec{C}}}{\norm{\vec{A}-\vec{B}}\norm{\vec{A}-\vec{C}}}
\label{eq:tri_cos_form-ccentre}
\end{equation}
\solution
From 
\eqref{eq:tri_cos_form}, using 
  \eqref{eq:norm2d_dist},
\begin{align}
\label{eq:tri_cos_form-ccentre-norm}
	\cos A&= 	\frac{\norm{\vec{A}-\vec{B}}^2+\norm{\vec{A}-\vec{C}}^2-\norm{\vec{B}-\vec{C}}^2}{2\norm{\vec{A}-\vec{B}}\norm{\vec{A}-\vec{C}}}
	\\
	&= 	\frac{\norm{\vec{A}}^2-\vec{A}^{\top}\vec{B}-\vec{A}^{\top}\vec{C}+\vec{B}^{\top}\vec{C}}{\norm{\vec{A}-\vec{B}}\norm{\vec{A}-\vec{C}}}
\end{align}
which can be expressed as 
\eqref{eq:tri_cos_form-ccentre}.
\item For $A = 90 \degree$, 
\begin{align}
	\cos A&= 0
	\\
	\implies 
 	\brak{\vec{A}-
	\vec{B}}^{\top}\brak{\vec{A}-\vec{C}} &=0
\label{eq:tri_cos_form-orth}
\end{align}
from 
\eqref{eq:tri_cos_form-ccentre}.
\end{enumerate}

\section{Trigonometric Identities}
%
\begin{enumerate}[label=\thesection.\arabic*.,ref=\thesection.\theenumi]
%
\item
In  \figref{fig:tri_baudh}, 
show that 
%
\begin{equation}
\label{ch1_budh_basic}
b = a \cos \theta + c \sin \theta
\end{equation}
%
\begin{figure}[!ht]
	\begin{center}
		\resizebox{\columnwidth}{!}{\input{./figs/trig/tri_baudh.tex}}
	\end{center}
	\caption{Baudhayana Theorem}
	\label{fig:tri_baudh}	
\end{figure}
\solution We observe that
%
\begin{align}
CD &= a \cos \theta \\
AD &= c \cos\alpha = c \sin \theta \quad \brak{\text{From} \quad 
	\eqref{eq:tri_baudh_comp}	
	%\eqref{eq:tri_90-ang}
}
\end{align}
%
Thus,
\begin{equation}
CD + AD = b = a \cos \theta + c \sin \theta
\end{equation}
\item
From \eqref{ch1_budh_basic}, show that
%
\begin{equation}
%
\label{eq:tri_sin_cos_id}
\sin ^2 \theta + \cos ^2 \theta = 1
\end{equation}
%
\solution Dividing both sides of \eqref{ch1_budh_basic} by $b$, 
\begin{align}
1 &= \frac{a}{b}\cos\theta + \frac{c}{b}\sin\theta\\
\Rightarrow &\sin ^2 \theta + \cos ^2 \theta = 1 \quad \brak{\text{from} \quad \eqref{eq:tri_trig_defs}}
\end{align}

\item
	Using \eqref{ch1_budh_basic}, show that
	\begin{equation}
	\label{eq:tri_baudh}
	b^2 = a^2 + c^2
	\end{equation}
	\eqref{eq:tri_baudh} is known as the Baudhayana theorem.  It is also known as the Pythagoras theorem.
\\
\solution From \eqref{ch1_budh_basic},
\begin{align}
b &= a\frac{a}{b} + c \frac{c}{b} \quad \brak{\text{from} \quad \eqref{eq:tri_trig_defs}}\\
\implies b^2 &= a^2 + c^2
\end{align}
%
\item
\label{prob:tri_area_sin}
	Show that the area of $\Delta ABC$ in Fig. 	\ref{fig:tri_sss}	is $\frac{1}{2}ab \sin C$.
\begin{figure}[!ht]
	\begin{center}
			\resizebox{\columnwidth}{!}{\input{./figs/triangle/tri_sss.tex}}
	\end{center}
	\caption{Area of a Triangle}
	\label{fig:tri_sss}	
\end{figure}

\solution We have
%
\begin{equation}
ar\brak{\Delta ABC} = \frac{1}{2}ah = \frac{1}{2}ab\sin C \quad \brak{\because \quad h = b \sin C}.
\label{eq:tri_area_sin}
\end{equation}

\item
	Show that 
	\begin{equation}
	\frac{\sin A}{a} = \frac{\sin B}{b} = \frac{\sin C}{c}
	\end{equation}

\solution Fig. \ref{fig:tri_sss} can be suitably modified to obtain 
\begin{align}
ar\brak{\Delta ABC} = 
\frac{1}{2}ab\sin C = \frac{1}{2}bc\sin A = \frac{1}{2}ca\sin B
	\label{tri:area-sin}
\end{align}
Dividing the above by $abc$, we obtain
	\begin{equation}
\label{eq:tri_sin_form}
	\frac{\sin A}{a} = \frac{\sin B}{b} = \frac{\sin C}{c}
	\end{equation}
This is known as the sine formula.	
%
%
\item Show that 
%
\begin{align}
\label{eq:trig_id_sin_inc}
\alpha > \beta \implies \sin \alpha > \sin \beta
\end{align}
%

\begin{figure}[!ht]
	\begin{center}
		
		%\includegraphics[width=\columnwidth]{./figs/fig:tri_sin_inc}
		%\vspace*{-10cm}
		\resizebox{\columnwidth}{!}{\input{./figs/triangle/tri_sin_inc.tex}}
	\end{center}
	\caption{}
	%\caption{$\sin \brak{\theta_1+\theta_2} = \sin\theta_1\cos\theta_2 + \cos\theta_1\sin\theta_2$}
	\label{fig:tri_sin_inc}	
\end{figure}
\solution In Fig. \ref{fig:tri_sin_inc}, 	
%
\begin{align}
ar\brak{\triangle ABD} &< ar \brak{\triangle ABC}
\\
\implies \frac{1}{2}lc \sin \theta_1 &<  \frac{1}{2}ac \sin \brak{\theta_1 + \theta_2 }
\\
\implies \frac{l}{a} &< \frac{\sin \brak{\theta_1 + \theta_2 }}{\sin \theta_1}
\\
\text{or, } 1 < \frac{l}{a} &< \frac{\sin \brak{\theta_1 + \theta_2 }}{\sin \theta_1}
\end{align}
%
from Theorem \ref{them:hyp_largest}, yielding 
\begin{align}
\implies \frac{\sin \brak{\theta_1 + \theta_2 }}{\sin \theta_1} > 1.
\end{align}

This proves \eqref{eq:trig_id_sin_inc}.
%From \eqref{eq:trig_id_sum_diff3},
%%
%\begin{multline}
% \sin \theta_1 - \sin \theta_2 = 2\sin\brak{\frac{\theta_1-\theta_2}{2}}
%\\
%\times \cos\brak{\frac{\theta_1+\theta_2}{2}} > 0, \because \theta_1-\theta_2 > 0
%\end{multline}
%
\item
	Using 
	\figref{fig:tri_sin_inc},
%Fig. \ref{trig_id_sin_theta}, 
show that 
	%
\begin{equation}
\label{trig_id_sin_theta_eq}
\sin  \theta_1 = \sin \brak{\theta_1 + \theta_2}\cos \theta_2 - \cos\brak{\theta_1+\theta_2}\sin\theta_2
\end{equation}	
	%
\iffalse
\begin{figure}[!ht]
	\begin{center}
		
		%\includegraphics[width=\columnwidth]{./figs/trig_id_sin_theta}
		%\vspace*{-10cm}
		\resizebox{\columnwidth}{!}{\input{./figs/triangle/tri_sin_inc.tex}}
	\end{center}
	\caption{$\sin \brak{\theta_1+\theta_2} = \sin\theta_1\cos\theta_2 + \cos\theta_1\sin\theta_2$}
	\label{trig_id_sin_theta}	
\end{figure}
\fi
%

\solution The following equations can be obtained from the figure using the forumula for the area of a triangle
%
\begin{align}
ar \brak{\Delta ABC} &= \frac{1}{2}ac \sin\brak{\theta_1 + \theta_2} \\
&= ar \brak{\Delta BDC} + ar \brak{\Delta ADB} \\
&= \frac{1}{2}cl \sin{\theta_1} + \frac{1}{2}al \sin{\theta_2} \\ 
&= \frac{1}{2}ac \sin{\theta_1} \sec \theta_2 + \frac{1}{2}a^2 \tan{\theta_2} 
\end{align}
$\brak{\because
	l = a \sec \theta_2}$.  From the above,
\begin{align}
\sin\brak{\theta_1 + \theta_2} &=  \sin{\theta_1} \sec \theta_2 + \frac{a}{c} \tan{\theta_2} \\
	&=  \sin{\theta_1} \sec \theta_2 
+ \cos\brak{\theta_1 + \theta_2} \tan{\theta_2} 
\end{align}
Multiplying both sides by $\cos \theta_2$,
\begin{align}
\sin\brak{\theta_1 + \theta_2}\cos{\theta_2} =  \sin{\theta_1}  
+ \cos\brak{\theta_1 + \theta_2} \sin\theta_2  
\end{align}
%
resulting in
\eqref{trig_id_sin_theta_eq}.
\item Find Hero's formula for the area of a triangle.
\\
\solution 
%In Fig. \ref{fig:rt_triangle}, from Baudhayana's theorem, 
%\begin{align}
%\label{eq:tri_geo_baudh}
%b^2 = a^2+c^2 &
%\\
%=b^2\cos^2C+b^2\sin^2C &
%\\
%\implies \cos^2C+\sin^2C &= 1
%\end{align}
%
%In Fig. \ref{fig:tri_const_ex_cos_form}, 
From \eqref{prob:tri_area_sin}, the area of $\triangle ABC$ is 
{\footnotesize
\begin{align}
\label{eq:tri_geo_area_sin_form}
 \frac{1}{2}ab\sin C
%\\
&=\frac{1}{2}ab\sqrt{1-\cos^2C} 
\quad \brak{\text{from } \eqref{eq:tri_sin_cos_id}
%\eqref{eq:tri_geo_baudh}
}
\\
&=\frac{1}{2}ab\sqrt{1-\brak{\frac{a^2+b^2-c^2}{2ab}}^2} \brak{\text{from } \eqref{eq:tri_cos_form}
}
\\
&=\frac{1}{4}\sqrt{\brak{2ab}^2-\brak{a^2+b^2-c^2}}
\\
&=\frac{1}{4}\sqrt{\brak{2ab+a^2+b^2-c^2}\brak{2ab-a^2-b^2+c^2}}
\\
&= \frac{1}{4}\sqrt{\cbrak{\brak{a+b}^2-c^2}\cbrak{c^2-\brak{a-b}^2}}
\\
&= \frac{1}{4}\sqrt{\brak{a+b+c}\brak{a+b-c}\brak{a+c-b}\brak{b+c-a}}
\label{eq:tri_ex_hero_temp}
\end{align}
}
Substituting 
%
\begin{align}
s=\frac{a+b+c}{2}
\end{align}
%
in \eqref{eq:tri_ex_hero_temp}, the area of $\triangle ABC$ is 
%
\begin{align}
\label{eq:tri_area_hero}
\sqrt{s\brak{s-a}\brak{s-b}\brak{s-c}}
\end{align}
%
This is known as Hero's formula.
\item
	Prove the following identities 
	%
	\begin{enumerate}
\item 
\begin{equation}
		\label{trig_id_sin_diff}
\sin\brak{\alpha - \beta} = \sin \alpha \cos \beta - \cos \alpha \sin \beta.
\end{equation}
\item 
\begin{equation}
\cos\brak{\alpha + \beta} = \cos \alpha \cos \beta - \sin \alpha \sin \beta.
		\label{trig_id_cos_diff}
\end{equation}

	\end{enumerate}
	%

\solution In \eqref{trig_id_sin_theta_eq}, let
%
\begin{equation}
\begin{split}
\theta_1 + \theta_2 &= \alpha \\
\theta_2 &=  \beta
\end{split}
\end{equation}
%
This gives \eqref{trig_id_sin_diff}.  In \eqref{trig_id_sin_diff}, replace $\alpha$ by 
%
$90{\degree} - \alpha$.  This results in
%
\begin{align}
\sin\brak{90{\degree} - \alpha - \beta}
	&=
\sin \brak{90{\degree} -\alpha} \cos \beta - \cos \brak{90{\degree} -\alpha} \sin \beta \\
	\implies \cos\brak{\alpha + \beta} &= \cos \alpha \cos \beta - \sin \alpha \sin \beta
\end{align}
% 
\item
	Using \eqref{trig_id_sin_theta_eq} and \eqref{trig_id_cos_diff}, show that
\begin{align}
\label{trig_id_sin_sum}
\sin\brak{\theta_1 + \theta_2} &= \sin\theta_1  \cos\theta_2 + \cos\theta_1\sin\theta_2
\\
\cos\brak{\theta_1 - \theta_2} &= \cos\theta_1  \cos\theta_2  \sin\theta_1\sin\theta_2
\label{trig_id_cos_sum}
\end{align}

%
\solution From \eqref{trig_id_sin_theta_eq},
%
\begin{align}
 \sin \brak{\theta_1 + \theta_2}\cos \theta_2 =\sin  \theta_1 +\cos\brak{\theta_1+\theta_2}\sin\theta_2 
\end{align}
%
Using \eqref{trig_id_cos_diff} in the above,
%
\begin{multline}
\sin \brak{\theta_1 + \theta_2}\cos \theta_2 
=\sin  \theta_1 +\lbrak{\cos \theta_1\cos\theta_2 }
\\	
\rbrak{	- \sin \theta_1\sin\theta_2}\sin\theta_2 
\end{multline}
%
which can be expressed as
%
\begin{multline}
\sin \brak{\theta_1 + \theta_2}\cos \theta_2 
=\sin  \theta_1 
\\
+\cos \theta_1\cos\theta_2 \sin\theta_2 
		- \sin \theta_1\sin^2\theta_2
\end{multline}
%
Since
%
\begin{equation}
\sin^2\theta_2 = 1- \cos^2\theta_2, 
\end{equation}
%
we obtain
%
\begin{multline}
\sin \brak{\theta_1 + \theta_2}\cos \theta_2 
=\cos \theta_1\cos\theta_2 \sin\theta_2 
+ \sin \theta_1\cos^2\theta_2
\end{multline}
%
resulting in
%
\begin{equation}
\sin \brak{\theta_1 + \theta_2}
=\cos \theta_1 \sin\theta_2 
+ \sin \theta_1\cos\theta_2
\end{equation}
%
after factoring out $\cos \theta_2$.  Using a similar approach, \eqref{trig_id_cos_sum} can also be proved.
\item Show that 
\begin{align}
\label{eq:trig_id_sum_diff1}
\sin \theta_1 + \sin \theta_2 &= 2\sin\brak{\frac{\theta_1+\theta_2}{2}}\cos\brak{\frac{\theta_1-\theta_2}{2}}
\\
\label{eq:trig_id_sum_diff2}
\cos \theta_1 + \cos \theta_2 &= 2\cos\brak{\frac{\theta_1+\theta_2}{2}}\cos\brak{\frac{\theta_1-\theta_2}{2}}
\\
\label{eq:trig_id_sum_diff3}
\sin \theta_1 - \sin \theta_2 &= 2\sin\brak{\frac{\theta_1-\theta_2}{2}}\cos\brak{\frac{\theta_1+\theta_2}{2}}
\\
\label{eq:trig_id_sum_diff4}
\cos \theta_1 - \cos \theta_2 &= 2\sin\brak{\frac{\theta_1+\theta_2}{2}}\cos\brak{\frac{\theta_2-\theta_1}{2}}
\end{align}
%
\\
\solution Let 
%
\begin{align}
\label{eq:trig_id_ang_sum_diff}
\begin{split}
\theta_1 = \alpha + \beta
\\
\theta_2 = \alpha - \beta
\end{split}
\end{align}
%
From \eqref{trig_id_sin_sum},
%
\begin{align}
\sin \theta_1 + \sin \theta_2  &= \sin \brak{\alpha + \beta} + \sin \brak{\alpha - \beta}
\\
&= \sin \alpha \cos \beta + \cos \alpha \sin \beta 
\\
&+\sin \alpha \cos \beta - \cos \alpha \sin \beta
\\
&= 2 \sin \alpha \cos \beta
\end{align}
%
resulting in \eqref{eq:trig_id_sum_diff1}
%
\begin{align}
\because \alpha &= \frac{\theta_1 +\theta_2}{2}
\\
\beta &= \frac{\theta_1 -\theta_2}{2}
\end{align}
from \eqref{eq:trig_id_ang_sum_diff}.  Other identities may be proved similarly.
%
\item Show that 
  \begin{align}
\label{eq:trig-id-2A-sin}
	  \sin 2 \theta = 2 \sin \theta \cos \theta
	  \\
	  \cos 2 \theta &= 1 - 2 \sin^2 \theta 
	  =  2 \cos^2 \theta -1
	  \\
	  &= \cos^2 \theta -\sin^2 \theta 
\label{eq:trig-id-2A-cos}
  \end{align}

\end{enumerate}

%
\end{document}

