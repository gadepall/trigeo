%\documentclass[12pt,-letter paper]{article}
%\usepackage{siunitx}         
%\usepackage{setspace}        
%\usepackage{gensymb}        
%\usepackage{xcolor}          
%\usepackage{caption}
%%\usepackage{subcaption}
%\doublespacing
%\singlespacing
%\usepackage[none]{hyphenat}  
%\usepackage{amssymb}         
%\usepackage{relsize}         
%\usepackage[cmex10]{amsmath} 
%\usepackage{mathtools}       
%\usepackage{amsmath}
%\usepackage{amsfonts}        
%\usepackage{amssymb}        
%\usepackage{commath}
%\usepackage{amsthm}
%\interdisplaylinepenalty=2500
%%\savesymbol{iint}
%\usepackage{txfonts}%\restoresymbol{TXF}{iint}
%\usepackage{wasysym}
%\usepackage{amsthm}
%\usepackage{mathrsfs}        
%\usepackage{txfonts}
%\let\vec\mathbf{}
%\usepackage{stfloats}
%\usepackage{float}
%\usepackage{cite}
%\usepackage{cases}
%\usepackage{subfig}          %\usepackage{xtab}
%\usepackage{longtable}
%\usepackage{multirow}
%%\usepackage{algorithm}
%\usepackage{amssymb}
%%\usepackage{algpseudocode}
%\usepackage{enumitem}
%\usepackage{mathtools}
%%\usepackage{eenrc}
%%\usepackage[framemethod=tikz]{mdframed}  \usepackage{listings}                
%%\usepackage{listings}
%\usepackage[latin1]{inputenc}
%%%\usepackage{color}{
%%%\usepackage{lscape}
%\usepackage{textcomp}
%\usepackage{titling}
%\usepackage{hyperref}
%%\usepackage{fulbigskip}
%\usepackage{tikz}
%\usepackage{graphicx}
%%%\lstset{frame=single, \breaklines=true}
%\let\vec\mathbf{}
%\usepackage{enumitem}
%\usepackage{amsmath}
%\usepackage{graphicx}        
%\usepackage{tfrupee}
%\usepackage{amsmath}         
%\usepackage{amssymb}
%\usepackage{mwe} % for blindtext and example-image-a in example
%\usepackage{wrapfig}
%\usepackage{enumitem}
%\providecommand{\qfunc}[1]{\ensuremath{Q\left(#1\right)}}
%\providecommand{\sbrak}[1]{\ensuremath{{}\left[#1\right]}}
%\providecommand{\lsbrak}[1]{\ensuremath{{}\left[#1\right]}}
%\providecommand{\rsbrak}[1]{\ensuremath{{}\left[#1\right]}}
%\providecommand{\brak}[1]{\ensuremath{\left(#1\right)}}
%\providecommand{\lbrak}[1]{\ensuremath{\left(#1\right.}}
%\providecommand{\rbrak}[1]{\ensuremath{\left.#1\right)}}
%\providecommand{\cbrak}[1]{\ensuremath{\left\{#1\right\}}}
%\providecommand{\lcbrak}[1]{\ensuremath{\left\{#1\right.}}
%\providecommand{\rcbrak}[1]{\ensuremath{\left.#1\right\}}}
%\title{FUNCTIONS}
%\author{KATTELA SHREYA}
%\date{December 2023}        
%\begin{document}             
%\maketitle                   
\begin{enumerate}
\item The interval in which the function $f$ given by $f\brak{x}=x^{2}e^{-x}$ is strictly increasing, is         
\begin{enumerate}            
\item$\brak{-\infty, \infty}$     
\item$\brak{-\infty, 0}$      
\item$\brak{2 , \infty}$         
\item$\brak{0,2}$
\end{enumerate}             
\item The function $f\brak{x}=\frac{x-1} {x\brak{x^2-1}}$ is discontinuous at       
\begin{enumerate}
\item exactly one point     
\item exactly two points  
\item exactly three points 
\item no points            
\end{enumerate}            
\item The function $f:\mathbb{R}\to$ \sbrak{-1,1} defined by $f\brak{x}=\cos{x}$ is
\begin{enumerate}           
\item both one-one and onto
\item not one-one, but onto
\item one-one, but onto     
\item neither one-one, nor onto
\end{enumerate}
\item The range of the principal value branch of the function $y= \sec^{-1}x$ is 
\item The principal value of $\cos^{-1} \brak{\frac{-1}{2}}$ is 
\item Find the value of $k$, so that the function $f\brak{x} =
\begin{cases}kx^{2}+ 5  & \text{if } x\leq 1,  \\ 2  & \text{if } x > 1
\end{cases}$  is continuous at $x=1$.
\item Check whether the relation $\mathbb{R}$ in the set $\mathbb{N}$ of natural numbers given by
\begin{align}
	\mathbb{R} = \cbrak{\brak{a, b} :\text{a is divisor of b}}
\end{align}
is reflexive, symmetric or transitive.Also determine whether $\mathbb{R}$ is an equivalence relation.
\item Prove that:
\begin{align}
\tan^{-1}\frac{1}{4}+\tan^{-1}\frac{2}{9}=\frac{1}{2}\sin^{-1}\brak{\frac{4}{5}}
\end{align}
\end{enumerate}
%\end{document}
