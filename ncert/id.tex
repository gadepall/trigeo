\begin{enumerate}[label=\thesubsection.\arabic*.,ref=\thesubsection.\theenumi,itemsep=1ex]
%\numberwithin{equation}{enumi}
%
\item If $\cos x = -\frac{3}{5}, x$ lies in the third quadrant, find the values of other five trigonometric function.
%
\item If $\cot x = - \frac{5}{12}, x$ lies in the second quadrant, find the values of other five trigonometric function.
%
\item Find the value of $\sin \frac{31\pi}{3}$.
%
\item Find the value of $\cos(-1710\degree)$.
%
\item Prove that $3\sin\frac{\pi}{6}\sec\frac{\pi}{3}-4\sin\frac{5\pi}{6}\cot\frac{\pi}{4} = 1.$
%
\item Find the value of $\sin 15\degree$.
%
\item Find the value of $\tan\frac{13\pi}{12}$.
%
%
\item Prove that $\frac{\sin(x+y)}{\sin(x-y)} = \frac{\tan x + \tan y}{\tan x - \tan y}$
%
%
\item Show that
$\tan3x\tan2x\tan x = \tan3x-\tan2x-\tan x$.
%
%
\item Prove that
$\cos\brak{\frac{\pi}{4}+x} + \cos\brak{\frac{\pi}{4}-x} = \sqrt 2\cos x$
%
%
\item Prove that $\frac{\cos7x+\cos5x}{\cos7x-\cos5x} = \cot x$
%
%
\item Prove that $\frac{\sin5x-2\sin3x+\sin x}{\cos5x-\cos x} = \tan x$
%
%
\item If $\sin x=\frac{3}{5}, \cos y=-\frac{12}{13}$, where x and y
both lies in second quadrant, find the value of
$\sin(x+y)$.
%
%
\item Prove that
$\cos2x\cos\frac{x}{2}-\cos3x\cos\frac{9x}{2}=\sin5x\sin\frac{5x}{2}$.
%
%
\item Find the value of $\tan\frac{\pi}{8}$.
%
%
\item If $\tan x=\frac{3}{4}, \pi<x<\frac{3\pi}{2}$, find the value of $\sin\frac{x}{2},\cos\frac{x}{2}$ and $\tan\frac{x}{2}$
%
%
\item Prove that
$\cos^{2}x+\cos^{2}(x+\frac{\pi}{3})+\cos^{2}(x-\frac{\pi}{3})=\frac{3}{2}$
%
\item Find the radian measures corresponding to the following meausres:
(i) $25\degree $
(ii) $-47\degree 30^{'}$
(iii) $240\degree$
(iv) $520\degree$
%
\item Find the degree measures corresponding to the following radian measures(use $\pi$=3.14)
(i) $\frac{11}{16}$
(ii) -4
(iii) $\frac{5\pi}{3}$
(iv) $\frac{7\pi}{6}$
%
\item A wheel makes 360 revolutions in one minute. Through how many radians does it turn in one second?
%
\item Find the degree measure of the angle subtended at the centre of a circle of radius 100 cm by an arc of length 22 cm?
%
\item In a circle of diameter 40 cm, the length of a chord is 20 cm. Find the length of minor arc of the chord.
%
\item If in two circles, arcs of the same length subtend angles $60\degree$ and $75\degree$ at the centre, find the ratio of their radii?
%
\item Find the angle in radian through which a pendulum swings if its length is 75 cm and the tip describes an arc of length
(i) 10 cm
(ii) 15 cm
(iii) 21 cm
%
\item Find the values of other five trigonometric functions 
1. $\cos x=-\frac{1}{2}$, x lies in third quadrant.
2. $\sin x= \frac{3}{5}$, x lies in second quadrant.
3. $\cot x= \frac{3}{4}$, x lies in third quadrant.
4. $\sec x= \frac{13}{5}$, x lies in fourth quadrant.
5. $\tan x=-\frac{5}{12}$, x lies in second quadrant.
%
\item Find the values of the trigonometric functions
1. $\sin765\degree$
2. $cosec(-1410\degree)$
3. $\tan\frac{19\pi}{3}$
4. $\sin\frac{-11\pi}{3}$
5. $\cot\frac{-15\pi}{4}$
%
\item Prove that
1. $\sin^{2}\frac{\pi}{6}+\cos^{2}\frac{\pi}{3}-\tan^{2}\frac{\pi}{4}=-\frac{1}{2}$
2. $2\sin^{2}\frac{\pi}{6}+cosec^{2}\frac{7\pi}{6}\cos^{2}\frac{\pi}{3}=-\frac{3}{2}$
3. $\cot^{2}\frac{\pi}{6}+cosec^{2}\frac{5\pi}{6}+3\tan^{2}\frac{\pi}{6}$=6
4. $2\sin^{2}\frac{3\pi}{4}+2\cos^{2}\frac{\pi}{4}+2\sec^{2}\frac{\pi}{3}$=10
%
\item Find the value of
(i) $\sin75\degree$
(ii) $\tan15\degree$
%
\item Prove that 
 $\cos(\frac{\pi}{4}-x)\cos(\frac{\pi}{4}-y)-\sin(\frac{\pi}{4}-x)\sin(\frac{\pi}{4}-y)=\sin(x+y)$
%
\item Prove that 
$\frac{\tan(\frac{\pi}{4}+x)}{\tan(\frac{\pi}{4}-x)}=(\frac{1+\tan x}{1-\tan x})^{2}$
%
\item Prove that
$\frac{\cos(\pi+x)\cos(-x)}{\sin(\pi-x)\cos(\frac{\pi}{2}+x)}=\cot^{2}x$
%
\item Prove that
$\cos(\frac{3\pi}{2}+x)\cos(2\pi+x)[\cot(\frac{3\pi}{2}-x)+\cot(2\pi +x)]=1$
%
\item Prove that
$\sin(n+1)x\sin(n+2)x+\cos(n+1)x\cos(n+2)x=\cos x$
%
\item Prove that
$\cos(\frac{3\pi}{4}+x)-\cos(\frac{3\pi}{4}-x)=-\sqrt 2\sin x $
%
\item Prove that
$\sin^{2}6x-\sin^{2}4x=\sin2x\sin10x$
%
\item Prove that
$\cos^{2}2x-\cos^{2}6x=\sin4x\sin8x$
%
\item Prove that
$\sin2x+2\sin4x+\sin6x=4\cos^{2}x\sin4x$
%
\item Prove that
$\cot4x(\sin5x+\sin3x)= \cot x(\sin5x-\sin3x)$
%
\item Prove that
$\frac{\cos9x-\cos5x}{\sin17x-\sin3x}=-\frac{\sin2x}{\cos10x}$
%
\item Prove that
$\frac{\sin5x+\sin3x}{\cos5x+\cos3x}=\tan4x$
%
\item Prove that
$\frac{\sin x+\sin y}{\cos x+\cos y}=\tan(\frac{x-y}{2})$
%
\item Prove that
$\frac{\sin x+\sin3x}{\cos x+\cos3x}=\tan2x$
%
\item Prove that
$\frac{\sin x-\sin3x}{\sin^{2}x-\cos^{2}x}=2\sin x$
%
\item Prove that
$\frac{\cos4x+\cos3x+\cos2x}{\sin4x+\sin3x+\sin2x}=\cot3x$
%
\item Prove that
$\cot x\cot2x-\cot2x\cot3x-\cot3x\cot x=1$
%
\item Prove that
$\tan4x=\frac{4\tan x(1-\tan^{2}x)}{1-6\tan^{2}x+\tan^{4}x}$
%
\item Prove that
$\cos4x=1-8\sin^{2}x\cos^{2}x$
%
\item Prove that
$\cos6x=32\cos^{6}x-48\cos^{4}x+18\cos^{2}x-1$
%
\item Find the principle and general solutions of the following equations:
1. $\tan x=\sqrt 3$
2. $\sec x=2$
3. $\cot x=-\sqrt 3$
4. $cosec x=-2$
%
\item Find the general solution for each of the following equations:
1. $\cos4x=\cos2x$
2. $\cos3x+\cos x-\cos2x=0$
3. $\sin2x+\cos x=0$
4. $\sec^{2}2x=1-\tan2x$
5. $\sin x+\sin3x+\sin5x=0$
%
\item Prove that
1. 2$\cos\frac{\pi}{13}\cos\frac{9\pi}{13}+\cos\frac{3\pi}{13}+\cos\frac{5\pi}{13}=0$
2. $(\sin3x+\sin x)\sin x+(\cos3x-\cos x)\cos x=0$
3. $(\cos x+\cos y)^{2}+(\sin x-\sin y)^{2}=4\cos^{2}(\frac{x+y}{2})$
4. $(\cos x-\cos y)^{2}+(\sin x-\sin y)^{2}=4\sin^{2}(\frac{x-y}{2})$
5. $\sin x+\sin3x+\sin5x+\sin7x=4\cos x\cos2x\sin4x$
6. $\frac{(\sin7x+\sin5x)+(\sin9x+\sin3x)}{(\cos7x+\cos5x)+(\cos9x+\cos3x)}=\tan6x$
7. $\sin3x+\sin2x-\sin x=4\sin x\cos\frac{x}{2\cos\frac{3x}{2}}$
%
\item Find $\sin\frac{x}{2},\cos\frac{x}{2}$ and $\tan\frac{x}{2}$ in each of the following:
1. $\tan x=-\frac{4}{3}$, x in second quadrant. 
2. $\sin x=\frac{1}{4}$, x in second quadrant.
3. $\cos x=-\frac{1}{3}$, x in third quadrant.
    
\end{enumerate}
    
