\begin{enumerate}[label=\thesubsection.\arabic*.,ref=\thesubsection.\theenumi]

\item A ladder is placed against a wall such that its foot is at a distance of 2.5$m$ from the wall and its top reaches a window 6$m$ above the ground. Find the length of the ladder.
\item  A ladder 10$m$ long reaches a window 8$m$ above the ground. Find the distance of the foot of the ladder from base of the wall.
\item  A guy wire attached to a vertical pole of height 18$m$ is 24$m$ long and has a stake attached to the other end. How far from the base of the pole should the stake be driven so that the wire will be taut?
\item  An aeroplane leaves an airport and flies due north at a speed of 1000$km$ per hour. At the same time, another aeroplane leaves the same airport and flies due west at a speed of 1200$km$ per hour. How far apart will be the two planes after $1\frac{1}{2}$ hours?
\item  Two poles of heights 6$m$ and 11$m$ stand on a plane ground. If the distance between the feet of the poles is $12m$, find the distance between their tops.
\item  $D$ is a point on the side $BC$ of a $\triangle ABC$ such that  $\angle  ADC =  \angle  BAC$. Show that 
\begin{align}
	\label{eq:tri-sim}	
	CA^2 = CB.CD
\end{align}
%
	\\
		\solution
	See \figref{fig:tri-sim}.	
\begin{align}
	\frac{x}{\sin\brak{A+C}} &= 
	\frac{b}{\sin A}  
\quad 
		\brak{\triangle ADC},
		\\
	\implies	\frac{x}{\sin B} &= 
	\frac{b}{\sin A}  
	\\
\implies	\frac{x}{b} = 
	\frac{\sin B}{\sin A}  &= 
	\frac{b}{a}  \quad \brak{\text{ sine formula }}
\end{align}
yielding
	\eqref{eq:tri-sim}.	
\begin{figure}[H]
	\begin{center}
			\resizebox{0.6\columnwidth}{!}{%Code by GVV Sharma
%October 21, 2024
%released under GNU GPL
%Drawing a triangle given 3 sides

\begin{tikzpicture}
[scale=2,>=stealth,point/.style={draw,circle,fill = black,inner sep=0.5pt},]

%Triangle sides
\def\a{6}
\def\b{5}
\def\c{4}
 
%Coordinates of A
%\def\p{{\a^2+\c^2-\b^2}/{(2*\a)}}
\def\p{2.25}
\def\q{{sqrt(\c^2-\p^2)}}
% Calculate x = b^2 / a
\pgfmathsetmacro\x{\b^2/\a}

%Labeling points
\node (A) at (\p,\q)[point,label=above right:$A$] {};
\node (B) at (0, 0)[point,label=below left:$B$] {};
\node (C) at (\a, 0)[point,label=below right:$C$] {};



%Foot of perpendicular

\node (D) at (\a-\x,0)[point,label=below :$D$] {};

%Drawing triangle ABC
\draw (A) -- node[left] {$c$} (B) -- node[below] {} (C) -- node[above,xshift=2mm] {${b}$} (A);

%Drawing altitude AD
\draw (A) -- node[left] {}(D);
%\draw (A) -- node[left] {$\textrm{h}$}(D);
%\draw (C) -- node[left] {}(C);
\draw[<->] (\a - \x,-0.3) -- (\a,-0.3) node[midway, above] {$x$};

%Drawing and marking angles
\tkzLabelAngle[pos=0.4](C,D,A){$\theta$}
\tkzLabelAngle[pos=0.4](B,A,C){ $\theta$}
\tkzFillAngle[fill=red!20,size=.3](C,D,A)
\tkzFillAngle[fill=red!20,size=.3](B,A,C)
%\tkzLabelAngle[pos=1](B,A,C){\rotatebox{-45}{$\alpha = 90\degree -\theta$}}
%\tkzLabelAngle[pos=0.65](C,B,D){$\alpha$}

% Drawing arrows to mark length of BC
\draw[<->] (\a,-0.5) -- (0,-0.5) node[midway, above] {$a$};

\end{tikzpicture}
}
	\end{center}
	\caption{}
	\label{fig:tri-sim}	
\end{figure}
\item   $D$ is a point on side $BC$ of  $\triangle  ABC$ such that
$\frac{BD}{CD}= \frac{AB}{AC}  $.  Prove that $AD$ is the bisector of  $\angle  BAC$.
\\
\solution 
	See \figref{fig:tri-ang-bis}.	
\begin{align}
	\frac{x}{a-x} &= 
	\frac{c}{b}  \quad \brak{\text{ given }}
	\\
	\frac{c}{\sin\phi} &= 
	\frac{x}{\sin \theta}  \quad \brak{\triangle ABD}
	\\
	\frac{a-x}{\sin\brak{A-\theta}} &= 
	\frac{b}{\sin 180-\phi}  \quad \brak{\triangle ACD}
	\\
	&=\frac{b}{\sin \phi}
\end{align}
using the sine formula.  Multiplying all the above equations yields
\begin{align}
	\sin \brak{A-\theta}
	=\sin \theta \implies \theta = \frac{A}{2}
\end{align}
\begin{figure}[H]
	\begin{center}
			\resizebox{0.6\columnwidth}{!}{%Code by GVV Sharma
%July 8, 2023
%released under GNU GPL
%Drawing the triangle and angle bisectors


\begin{tikzpicture}
[scale=2,>=stealth,point/.style={draw,circle,fill=black,inner sep=0.5pt}]

% Coordinates of the triangle vertices
\def\a{6}
\def\b{5}
\def\c{4}

% Coordinates of A
\pgfmathsetmacro\p{(\a^2+\c^2-\b^2)/(2*\a)}
\pgfmathsetmacro\q{sqrt(\c^2-\p^2)}
%\def\p{{(\a^2+\c^2-\b^2)/(2*\a)}}
%\def\q{{sqrt(\c^2-\p^2)}}

% Labeling points
\node (A) at (\p,\q)[point,label=above right:$A$] {};
\node (B) at (0,0)[point,label=below left:$B$] {};
\node (C) at (\a,0)[point,label=below right:$C$] {};

% Drawing triangle ABC
\draw (A) -- node[left] {$c$} (B) -- node[below] {} (C) -- node[above,xshift=2mm] {${b}$} (A);

\node (D) at ($(B)!\c/(\a+\c)!(C)$)[point,label=below :$D$] {};

%Drawing altitude AD
\draw (A) -- node[left] {}(D);


\draw[<->] (\a,-0.5) -- (0,-0.5) node[midway, above] {$a$};
\node at ($(B)!0.5!(D)$) [below] {$x$};
\node at ($(D)!0.5!(C)$) [below] {$a-x$};

%Drawing and marking angles
\tkzLabelAngle[pos=1](D,A,C){$A-\theta$}
\tkzLabelAngle[pos=0.4](B,A,D){ $\theta$}
\tkzLabelAngle[pos=0.4](A,D,B){ $\phi$}
\tkzFillAngle[fill=red!20,size=.6](D,A,C)
\tkzFillAngle[fill=blue!20,size=.3](B,A,D)
\tkzFillAngle[fill=green!20,size=.3](A,D,B)

\end{tikzpicture}


}
	\end{center}
	\caption{}
	\label{fig:tri-ang-bis}	
\end{figure}
%
\item  $ABC$ is a triangle in which  $\angle  ABC > 90\degree$ and $AD  \perp  CB$ produced. Prove that
\begin{align}
	\label{eq:tri-obtuse}
 AC^2= AB^2 + BC^2 + 2 BC . BD.
\end{align}
\solution
	See \figref{fig:tri-obtuse}.	
\begin{align}
	\label{eq:tri-obtuse-1}
	\cos B &= \frac{x}{c} \quad \brak{\triangle ADB}
	\\
	b^2 &= a^2 + c^2 -2ac \cos \brak{180-B} \quad \brak{\triangle ABC}
	\\
	&= a^2 + c^2 +2ac \cos B 
	\label{eq:tri-obtuse-2}
\end{align}
using the cosine formula.
Substituting from 
	\eqref{eq:tri-obtuse-1}
	in
	\eqref{eq:tri-obtuse-2}
	yields 
	\eqref{eq:tri-obtuse}.
\begin{figure}[H]
	\begin{center}
			\resizebox{0.6\columnwidth}{!}{%Code by GVV Sharma
%October 22, 2024
%released under GNU GPL
%Drawing a right angled triangle

\begin{tikzpicture}[scale=2]

%Triangle sides
\def\a{4}
\def\c{3}

%Marking coordiantes
\coordinate [label=above:$A$] (A) at (0,\c);
\coordinate [label=left:$D$] (D) at (0,0);
\coordinate [label=right:$C$] (C) at (\a,0);
\coordinate [label=below:$B$] (B) at (\a/3,0);

%Drawing triangle ABC
\draw (A) -- node[left] {${c}$} (B) -- node[below] {${a}$} (C) -- node[above,,xshift=2mm] {${b}$} (A);
\draw[dashed] (A) -- (D);
\draw[dashed] (B) -- (D);
\node at ($(B)!0.5!(D)$) [below] {$x$};

%Drawing and marking angles
\tkzMarkRightAngle[fill=blue!20,size=.3](A,D,C)
\end{tikzpicture}
}
	\end{center}
	\caption{}
	\label{fig:tri-obtuse}	
\end{figure}
\item In a right triangle, prove that the line-segment joining the mid-point of the hypotenuse to the opposite vertex is half the hypotenuse.
	\\
	\solution
	In \figref{fig:tri-hyp}	
\begin{align}
	\label{eq:tri-hyp}	
	\frac{x}{\sin C} &= \frac{b/2}{\sin \theta} \quad \brak{\triangle BDC}
	\\
	\frac{x}{\sin A} &= \frac{b/2}{\sin \brak{90-\theta}} \quad \brak{\triangle BDA}
	\\
\implies	\frac{x}{\cos C} &= \frac{b/2}{\cos \theta} 
	\label{eq:tri-hyp-1}	
\end{align}
From 
	\eqref{eq:tri-hyp}	
	and
	\eqref{eq:tri-hyp-1},
\begin{align}
	\brak{\frac{\sin C}{x}}^2
	+
	\brak{\frac{\cos C}{x}}^2
	&= 
	\brak{\frac{\cos \theta}{\frac{b}{2}}}^2
	+
	\brak{\frac{\sin \theta}{\frac{b}{2}}}^2
	\\
	\implies x &= \frac{b}{2} 
\end{align}
using \eqref{eq:tri_sin_cos_id}.
\begin{figure}[H]
	\begin{center}
			\resizebox{0.6\columnwidth}{!}{%Code by GVV Sharma
%December 6, 2019
%released under GNU GPL
%Drawing a right angled triangle

\begin{tikzpicture}[scale=2]

%Triangle sides
\def\a{4}
\def\c{3}

%Marking coordiantes
\coordinate [label=above:$A$] (A) at (0,\c);
\coordinate [label=left:$B$] (B) at (0,0);
\coordinate [label=right:$C$] (C) at (\a,0);
\coordinate [label=right:$D$] (D) at ($(A)!0.5!(C)$);

%Drawing triangle ABC
\draw (A) -- node[left] {${c}$} (B) -- node[below] {${a}$} (C) -- node[above,,xshift=2mm] {} (A);
\draw (B) -- (D); 
\node at ($(A)!0.5!(D)$) [above right] {$\frac{b}{2}$};
\node at ($(C)!0.5!(D)$) [above right] {$\frac{b}{2}$};
\node at ($(B)!0.5!(D)$) [right] {$x$};

%Drawing and marking angles
\tkzFillAngle[fill=orange!40,size=0.5cm](C,B,D)
\tkzMarkRightAngle[fill=blue!20,size=.3](A,B,C)
\tkzLabelAngle[pos=0.65](C,B,D){$\theta$}
\end{tikzpicture}
}
	\end{center}
	\caption{}
	\label{fig:tri-hyp}	
\end{figure}
\item $ABCD$ is a trapezium in which $AB  \parallel  DC$ and its diagonals intersect each other at the point $O$. Show
that
\begin{align}
	\label{eq:tri-trap}	
\frac{AO}{ BO}=\frac{CO}{  DO}
\end{align}
\begin{figure}[H]
	\begin{center}
			\resizebox{0.6\columnwidth}{!}{%Code by GVV Sharma
%October 23, 2019
%released under GNU GPL
%Drawing a trapezium

\begin{tikzpicture}
[scale=2,>=stealth,point/.style={draw,circle,fill = black,inner sep=0.5pt},]

%Triangle sides
\def\a{6}
\def\b{5}
\def\c{4}
 
%Coordinates of A
%\def\p{{\a^2+\c^2-\b^2}/{(2*\a)}}
\def\p{2.25}
\def\q{{sqrt(\c^2-\p^2)}}

%Labeling points
\node (A) at (\p,\q)[point,label=above right:$D$] {};
\node (B) at (0, 0)[point,label=below left:$A$] {};
\node (C) at (\a, 0)[point,label=below right:$B$] {};
\node (D) at ({(\a+\p)/2},\q)[point,label=above right:$C$] {};

%\node (D) at ($(B)!0.5!(D)$) [below] {$x$};
%Foot of perpendicular

%\node (D) at (\p,0)[point,label=above right:$D$] {};

%Drawing trapezium ABC
\draw (A) -- node[left] {} (B) -- node[below] {} (C) -- node[above,xshift=2mm] {} (D)
-- node[above,xshift=2mm] {} (A);
\draw (A) --  (C); 
\draw (B) --  (D); 

% Labeling point O
\node (O) at (3, 2.558)[point,label=above right:$O$] {};

%Drawing and marking angles
\tkzLabelAngle[pos=0.4](C,B,D){$\theta$}
\tkzLabelAngle[pos=0.4](A,D,B){$\theta$}
\tkzLabelAngle[pos=0.4](A,C,B){ $\phi$}
\tkzLabelAngle[pos=0.4](C,A,D){ $\phi$}
\tkzFillAngle[fill=red!20,size=.3](C,B,D)
\tkzFillAngle[fill=red!20,size=.3](A,D,B)
\tkzFillAngle[fill=blue!20,size=.3](A,C,B)
\tkzFillAngle[fill=blue!20,size=.3](C,A,D)


\end{tikzpicture}
}
	\end{center}
	\caption{}
	\label{fig:tri-trap}	
\end{figure}
\solution 
	In \figref{fig:tri-trap}, $\because AB \parallel CD$	
\begin{align}
	\frac{AO}{\sin \phi} &= \frac{BO}{\sin \theta} \quad \brak{\triangle OAB}
	\\
	\frac{CO}{\sin \phi} &= \frac{DO}{\sin \theta} \quad \brak{\triangle ODC}
\end{align}
yielding
	\eqref{eq:tri-trap}	
	after simplification.
\item $O$ is any point inside a rectangle $ABCD$. Prove that 
\begin{align}
	OB^2+OD^2 = OA^2+OC^2
	\label{eq:tri-rect}	
\end{align}
	\solution
	In 
	\figref{fig:tri-rect},	
from \eqref{ch1_budh_basic}
\begin{align}
	p \cos \theta_1+q \sin \theta_2 = a \quad \brak{\triangle OAB}
	\\
	r \cos \theta_3+s \sin \theta_4 = a \quad \brak{\triangle OAB}
	\\
	p \cos \theta_1+s \sin \theta_4 = b \quad \brak{\triangle OAB}
	\\
	r \cos \theta_3+q \sin \theta_2 = b \quad \brak{\triangle OAB}
\end{align}
Subtracting the first two and second two equations respectively,
\begin{align}
	p \cos \theta_1 
	-s \sin \theta_4  
= r \cos \theta_3-q \sin \theta_2
\\
	p \cos \theta_1+s \sin \theta_4 = 
	r \cos \theta_3+q \sin \theta_2  
\end{align}
Squaring and adding and using 
\eqref{eq:tri_sin_cos_id}
yields
	\eqref{eq:tri-rect}.	
\begin{figure}[H]
	\begin{center}
			\resizebox{0.6\columnwidth}{!}{%Code by GVV Sharma
%October 23, 2024
%released under GNU GPL
%Drawing a rectangle

\begin{tikzpicture}[scale=2]

%Triangle sides
\def\a{4}
\def\c{3}

%Marking coordiantes
\coordinate [label=above:$D$] (D) at (0,\c);
\coordinate [label=left:$A$] (A) at (0,0);
\coordinate [label=right:$B$] (B) at (\a,0);
\coordinate [label=right:$C$] (C) at (\a,\c);
\coordinate [label=right:$O$] (O) at ({(3*\a/4)},{(\c/4)});

%
\draw (A) -- node[below] {$a$} (B) -- node[right] {$b$} (C) -- node[above,xshift=2mm] {$a$} (D)
-- node[above,xshift=2mm] {$b$} (A);
\draw (O) --  node[above] {$p$}(A); 
\draw (O) -- node[below left] {$q$}(B); 
\draw (O) -- node[left] {$r$}(C); 
\draw (O) -- node[left] {$s$}(D); 
%%Drawing triangle ABC
%\draw (A) -- node[left] {${c}$} (B) -- node[below] {${a}$} (C) -- node[above,,xshift=2mm] {} (A);
%\draw (B) -- (D); 
%\node at ($(A)!0.5!(D)$) [above right] {$\frac{b}{2}$};
%\node at ($(C)!0.5!(D)$) [above right] {$\frac{b}{2}$};
%\node at ($(B)!0.5!(D)$) [right] {$x$};
%
%%Drawing and marking angles
%\tkzFillAngle[fill=orange!40,size=0.5cm](C,B,D)
%\tkzMarkRightAngle[fill=blue!20,size=.3](A,B,C)
\tkzLabelAngle[pos=0.65](B,A,O){$\theta_1$}
\tkzLabelAngle[pos=0.65](C,B,O){$\theta_2$}
\tkzLabelAngle[pos=0.65](D,C,O){$\theta_3$}
\tkzLabelAngle[pos=0.65](A,D,O){$\theta_4$}
\tkzMarkAngle[fill=red!10](B,A,O)
\tkzMarkAngle[fill=red!10](C,B,O)
\tkzMarkAngle[fill=red!10](D,C,O)
\tkzMarkAngle[fill=red!10](A,D,O)
\end{tikzpicture}
}
	\end{center}
	\caption{}
	\label{fig:tri-rect}	
\end{figure}
\item  In  $\triangle  ABC, AB = 6\sqrt{3} cm, AC = 12 cm$ and $BC = 6 cm$. Find the angle $B$.
	\\
	\solution Using 
\eqref{eq:tri_cos_form},
\begin{align}
	\cos B &= 
 \frac{c^2+a^2-b^2}{2ca}=0
	\\
	\implies B &= 90\degree
\end{align}
\end{enumerate}
