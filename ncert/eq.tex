\begin{enumerate}[label=\thesubsection.\arabic*,ref=\thesubsection.\theenumi]
\item Find the principal solutions of the equation $\sin x = \frac{\sqrt 3}{2}$.
%
	\\
		\solution 
\begin{align}
	x = \frac{\pi}{3}, \frac{2\pi}{3}
\end{align}
%
\item Find the principal solutions of the equation $\tan x = -\frac{1}{\sqrt 3}$.
%
	\\
\solution
\begin{align}
	x = \frac{5\pi}{6}, -\frac{\pi}{6}
\end{align}
%
\item Find the solution of $\sin x = -\frac{\sqrt 3}{2}$.
%
	\\
\solution
\begin{align}
	x = \frac{4\pi}{3}
\end{align}
%
\item Solve $\cos x = \frac{1}{2}$.
%
	\\
\solution
%
\begin{align}
	x = 2k\pi \pm \frac{\pi}{3}
\end{align}
\item Solve 
\begin{align}
\tan2x=-\cot\brak{x+\frac{\pi}{3}}
	\label{eq:eq-tan2x}
\end{align}
%
	\\
		\solution
	\eqref{eq:eq-tan2x}
	can be expressed as
\begin{align}
\frac{\sin2x}{\cos2x}=-\frac{\cos\brak{x+\frac{\pi}{3}}}{\sin\brak{x+\frac{\pi}{3}}}
\\
	\implies \cos\brak{x-\frac{\pi}{3}} = 0
	\\
	\text{or, } x = 2k \pi \pm \frac{\pi}{2} + \frac{\pi}{3}
\end{align}
\item Solve $\sin2x-\sin4x+\sin6x=0$.
%
	\\
		\solution
\begin{align}
	LHS &= 
\sin2x+\sin6x-\sin4x
\\
	&=
2\sin4x\cos2x-\sin4x
\\
	&=\sin 4x \brak{2\cos2x - 1}
	\\
	\implies \sin 4x &= \sin 0
	\cos 2x = \cos \frac{\pi}{3},\,
\end{align}
Therefore, the possible solutions are
\begin{align}
	x = k \frac{\pi}{4},\,
	x = k\pi \pm \frac{\pi}{6}.
\end{align}
%
\item Solve 2$\cos^{2}x+3\sin x=0$.
	\\
	\solution
\begin{align}
	LHS &= 2\brak{1-\sin^2x}+3\sin x
\end{align}
%
yielding the quadratic
\begin{align}
	 2\sin^2x-3\sin x -2= 0
\end{align}
with roots
\begin{align}
	\sin x = 2, -\frac{1}{2}
\end{align}
Thus, the only possible solution is
\begin{align}
	\sin x &= \sin \brak{\pi + \frac{\pi}{6}}
\\
	\implies x &= n\pi + \brak{-1}^n\frac{7\pi}{6}
\end{align}
\item Find the general solution for each of the following equations
\begin{enumerate}
\item $\cos4x=\cos2x$.
\item $\cos3x+\cos x-\cos2x=0$.
\item $\sin2x+\cos x=0$.
\item $\sec^{2}2x=1-\tan2x$.
\item $\sin x+\sin3x+\sin5x=0$.
\end{enumerate}
%
\solution
\begin{enumerate}
\item The solution is
\begin{align}
	4x &= 2k\pi \pm 2x
	\\
	\implies x &= \frac{k\pi}{3}, k\pi
\end{align}
\item 
\begin{align}
	LHS &=\cos3x+\cos x-\cos2x
	\\
	&=2\cos2x\cos x-\cos2x=0
	\\
	\implies \cos 2x= 0, \cos x = \frac{1}{2}
\end{align}
Thus, 
\begin{align}
	x = k\pi \pm \frac{\pi}{4}, 2k\pi \pm \frac{\pi}{3}
\end{align}
\item 
\begin{align}
	LHS = \cos\brak{\frac{\pi}{2}-2x}+\cos x = 0
	\\
	\implies 
\cos x =	 \cos\brak{\frac{\pi}{2}-2x} 
	 \\
	 \implies x = 2k\pi \pm \brak{\frac{\pi}{2}-2x} 
\end{align}
yielding
\begin{align}
	x = 
	  \frac{2k\pi }{3}+\frac{\pi}{6},\,
	x = 
	  2k\pi +\frac{\pi}{2}
\end{align}
\item 
\begin{align}
LHS = 1+\tan^2 2x=1-\tan2x
\\
\implies 
	\tan 2x\brak{1+\tan2x} = 0
	\\
	\text{or, }
	\tan 2x = 0, \tan2x = -1= \tan \frac{3\pi}{4} 
\end{align}
yielding
\begin{align}
	x = \frac{k\pi}{2}, \frac{k\pi}{2}+\frac{3\pi}{8} 
\end{align}
\item 
\begin{align}
	LHS &= \sin x+\sin5x+\sin3x
	\\
	&= 2\sin 3x\cos 2x+\sin3x
	\\
	&= \sin 3x\brak{2\cos 2x+1}=0
\end{align}
yielding
\begin{align}
 \sin 3x=\sin 0, \cos 2x=\cos \frac{2\pi}{3}
 \\
	\text{or, }
	x = \frac{k\pi}{3}, x = k\pi \pm \frac{\pi}{3}
\end{align}
\end{enumerate}
\item Find the principal and general solutions of the following equations
\begin{enumerate}
\item $\tan x=\sqrt 3$.
\item $\sec x=2$.
\item $\cot x=-\sqrt 3$.
\item $\csc x=-2$.
\end{enumerate}
\solution
\begin{enumerate}
\item 
\begin{align}
\tan x=\tan \frac{\pi}{3}
\\
\implies x = k\pi + \frac{\pi}{3}
\end{align}
\item 
\begin{align}
\cos x=\cos \frac{\pi}{3}
\\
\implies x = 2k\pi \pm \frac{\pi}{3}
\end{align}
\item 
\begin{align}
\tan x=\tan \frac{5\pi}{6}
\\
\implies
x = k\pi + \frac{5\pi}{6}
\end{align}
\item 
\begin{align}
\sin x=\sin \frac{7\pi}{6}
\\
	\implies x = k\pi + \brak{-1}^k\frac{7\pi}{6}
\end{align}
\end{enumerate}
\item 	Solve $\tan^{-1}2x +\tan^{-1}3x=\frac{\pi}{4}$.
	\item If 
		$\sin\brak{\sin^{-1}\brak{\frac{1}{5}} + \cos^{-1}x}=1$, then find the value of $x$.
	\item If 
		$\tan^{-1}\brak{\frac{x-1}{x-2}} +\tan^{-1}\brak{\frac{x+1}{x+2}}=\frac{\pi}{4}$, then find the value of $x$.
\end{enumerate}
