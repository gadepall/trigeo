\begin{enumerate}[label=\thesubsection.\arabic*,ref=\thesubsection.\theenumi]
\item $AB$ is a diameter of the circle, $CD$ is a chord equal to the radius of the circle. $AC$ and $BD$ when extended intersect at a point $E$. Prove that $\angle AEB = 60\degree$.
\item If the non-parallel sides of a trapezium are equal, prove that it is cyclic.
	\iffalse
\begin{enumerate}

\begin{figure}[!ht]
\centering
\resizebox{\columnwidth}{!}{\begin{tikzpicture}[scale=1.5,>=stealth,point/.style={draw,circle,fill = black,inner sep=0.5pt},]
      
%Labeling points
\node (A) at (0, 0)[point,label=below left:$A$] {};
\node (B) at (5, 0)[point,label=below right:$B$] {};
\node (C) at (3.9, 3)[point,label=above right:$C$] {};
\node (D) at (1.1, 3)[point,label=above left:$D$] {};
\node (P1) at (1.1, 0)[point,label=below right:$P_1$] {};
\node (P2) at (3.9, 0)[point,label=below left:$P_2$] {};
\node (O) at (2.5, 0.79)[point,label=below:$O$]{};

%Drawing quad ABCD
\draw (A) -- node[below=6pt]{$b$}(B) -- (C) -- (D) -- (A);
\draw[dotted] (D) -- node[right = 7pt]{$h$}(P1)(P2)--(C);
\draw[dotted] (O) circle(2.62);
%marking line segment
\tkzMarkSegments[mark=|,size=6pt](A,D C,B)
\tkzMarkSegments[mark=s||,size=6pt](P1,D C,P2)

%marking angles
\tkzMarkAngle[fill=orange!40,size=0.5cm,mark=](P1,A,D)
\tkzMarkAngle[fill=orange!40,size=0.5cm,mark=](D,C,B)
\tkzMarkRightAngle[fill=blue!20](D,P1,A)
\tkzMarkRightAngle[fill=blue!20](C,P2,B)
\tkzLabelAngle[pos=0.75](P1,A,D){$\theta$}
\tkzLabelAngle[pos=0.75](D,C,B){$\alpha$}

\end{tikzpicture} }
\caption{}
\label{fig:8.5.43_trapezium}	
\end{figure}
%
\item {\em Construction: }See Fig. \ref{fig:8.5.43_trapezium}
The input parameters are
\begin{align}
\vec{A} &=\myvec{0\\0},
\vec{B} &= \myvec{b\\0}, \label{eq:8.5.43_constr_b}
\\
\vec{C} &= \myvec{b - h\cot{\theta}\\h}\label{eq:8.5.43_constr_c}\\ 
\vec{D} &= h\myvec{\cot{\theta}\\ 1}\label{eq:8.5.43_constr_d}	\end{align}
%
which are sufficient to draw the trapezium.  The circumcircle of $\triangle ABC$ is then drawn.  This circle passes through $\vec{D}$.

From \eqref{eq:8.5.43_constr_b} - \eqref{eq:8.5.43_constr_d}
\begin{align}
\vec{B} - \vec{A} &= \myvec{b\\0}\label{eq:8.5.43_dir1}\\
\vec{D} - \vec{C} &= \myvec{2h\cot{\theta}-b\\0}\label{eq:8.5.43_dir2}\\
\vec{D} - \vec{A} &= \myvec{h\cot{\theta}\\h}\label{eq:8.5.43_dir3}\\
\vec{B} - \vec{C} &= \myvec{h\cot{\theta}\\-h}\label{eq:8.5.43_dir4}
\end{align}

Finding the scalar products:
\begin{align}
\brak{\vec{D} - \vec{A}}^T\brak{\vec{B} - \vec{A}} &= \norm{\vec{D}-\vec{A}} 
\norm{ \vec{B}-\vec{A}} \cos{\theta}\label{eq:8.5.43_scalar1}\\
\brak{\vec{D} - \vec{C}}^T\brak{\vec{B} - \vec{C}} &= \norm{\vec{D}-\vec{C}} 
\norm{ \vec{B}-\vec{C}} \cos{\alpha}\label{eq:8.5.43_scalar2}
\end{align}

Dividing \eqref{eq:8.5.43_scalar2} with \eqref{eq:8.5.43_scalar1},
\begin{align}
\frac{\brak{\vec{D} - \vec{A}}^T\brak{\vec{B} - \vec{A}}}{\brak{\vec{D} - \vec{C}}^T\brak{\vec{B} - \vec{C}}} &= \frac{\norm{\vec{D}-\vec{A}}\ 
\norm{ \vec{B}-\vec{A}} \cos{\theta}}{\norm{\vec{D}-\vec{C}} 
\norm{ \vec{B}-\vec{C}} \cos{\alpha}}\label{eq:8.5.43_scalar3}
\end{align}\\
$\because \norm{ \vec{D} - \vec{A} } = \norm{ \vec{B} - \vec{C}} $, \eqref{eq:8.5.43_scalar3} can be simplified to the form
\begin{align}
\frac{\brak{\vec{D} - \vec{A}}^T\brak{\vec{B} - \vec{A}}}{\brak{\vec{D} - \vec{C}}^T\brak{\vec{B} - \vec{C}}} &= \frac{\norm{ \vec{B}-\vec{A}} \cos{\theta}}{\norm{\vec{D}-\vec{C}}cos{\alpha}}
\end{align}
%
Substituting values from \eqref{eq:8.5.43_dir1}, \eqref{eq:8.5.43_dir2}, \eqref{eq:8.5.43_dir3} and \eqref{eq:8.5.43_dir4}:
\begin{align}
\frac{bh\cot{\theta}}{\brak{2h\cot{\theta}-b}h\cot{\theta}} &= \frac{b\cos{\theta}}{b-2h\cot{\theta}}\\
\implies\cos{\alpha} &= -\cos{\theta}\\
\implies \alpha + \theta &= 180\degree
\end{align}

$\therefore ABCD$ is a cyclic quadilateral.

\end{enumerate}
\fi
\item  The sum of either pair of opposite angles of a cyclic quadrilateral is 180$\degree$.
\item AB is a diameter of the circle, $CD$ is a chord equal to the radius of the circle. $AC$ and $BD$ when extended intersect at a point $E$. Prove that $\angle AEB = 60\degree$.
\item Two circles intersect at two points $A and B$. $AD$ and $AC$ are diameters to the two circles. Prove that $B$ lies on the line segment $DC$.
\item Prove that the quadrilateral formed (if possible) by the internal angle bisectors of any quadrilateral is cyclic.
\item If a line intersects two concentric circles (circles with the same centre) with centre O at A, B, C and D, prove that AB = CD.
	\iffalse
\begin{enumerate}
\begin{figure}[!ht]
\centering
\resizebox{\columnwidth}{!}{\begin{tikzpicture}[scale = 0.5,>=stealth,point/.style={draw,circle,fill=black, inner sep=0.5pt},]
\node (O) at (0, 0)[point,label=above:$O$] {};
\node (A) at (-8.66, -5)[point,label=below:$A$] {};
\node (B) at (-4.89, -5)[point,label=below:$B$] {};
\node (C) at (4.89, -5)[point,label=below:$C$] {};
\node (D) at (8.66, -5)[point,label=below:$D$] {};
\node (M) at (0,-5)[point,label=below:$M$] {};

\draw (0,0) circle(7cm);
\draw (0,0) circle(10cm);
\draw (A) -- node[below=5pt]{}(B) -- (C) -- (D);
\draw[dotted] (O) -- node[below=5pt]{}(M);
\draw[dotted] (O) -- node[below=5pt]{}(A);
\draw[dotted] (O) -- node[below=5pt]{}(B);
\draw[dotted] (O) -- node[below=5pt]{}(C);
\draw[dotted] (O) -- node[below=5pt]{}(D);

\tkzMarkRightAngle[fill=blue!20, mark=|](O,M,B)

\end{tikzpicture}}
\caption{}
\label{fig:8.5.40_circle_1}	
\end{figure}

\item  {\em Construction: } See Fig. \ref{fig:8.5.40_circle_1}.	 The input parameters are
\begin{align}
\vec{O} &= \myvec{0\\0},
\vec{A} &= r_1\myvec{\cos \theta_1\\ \sin \theta_1}
\vec{D} &= r_2\myvec{\cos \theta_2\\ \sin \theta_2}
\end{align}
The two concentric circles  with centre $\vec{O}$ and radii $r_1$ and $r_2$ are drawn.

\subitem The equation of $AD$ is 
\begin{align}
\label{eq:8.1.40_line}
\vec{x} = \vec{A} + \lambda \brak{\vec{A}-\vec{D}}
\end{align}
Points $\vec{B}$ and $\vec{D}$ are obtained as the intersection of $AB$ and the circle with equation
\begin{align}
\label{eq:8.1.40_circle}
\norm{\vec{x}}^2 = r_2^2
\end{align}
From \eqref{eq:8.1.40_circle}
and \eqref{eq:8.1.40_circle}, 
\begin{align}
\norm{\vec{A} + \lambda \brak{\vec{A}-\vec{D}}}^2 = r_2^2
\end{align}
\begin{multline}
\lambda^2\norm{\vec{A}-\vec{D}}^2 + 2\lambda \vec{A}^T\brak{\vec{A}-\vec{D}} 
\\
+ \norm{\vec{A}}^2 -r_2^2 =0
\label{eq:8.1.40_quad}
\end{multline}
Solving the above quadratic equation yields two values for $\lambda$ yielding $\vec{B}$ and $\vec{C}$.  
\begin{align}
\vec{M} = \frac{\vec{B}+\vec{C}}{2}
\end{align}


\item {\em Proof: } In Fig. \ref{fig:8.5.40_circle_1}, 
\begin{align}
\triangle OMB \cong \triangle OMC
\\
\triangle OMA \cong \triangle OMD
\end{align}
resulting in 
\begin{align}
AM = DM
\\
BM = CM
\\
\implies AB = CD
\end{align}

\end{enumerate}
\fi
\item A chord of a circle is equal to the radius of the
circle. Find the angle subtended by the chord at
a point on the minor arc and also at a point on the
major arc.
\item If diagonals of a cyclic quadrilateral are diameters of the circle through the vertices of
the quadrilateral, prove that it is a rectangle.
\item Two circles intersect at two points $B$ and $C$.
Through $B$, two line segments $ABD$ and $PBQ$
are drawn to intersect the circles at $A, D$ and $P$,
$Q$ respectively. Prove that
$\angle ACP = \angle QCD$.
\item If circles are drawn taking two sides of a triangle as diameters, prove that the point of
intersection of these circles lie on the third side.
\item Prove that the line of centres of two intersecting circles subtends equal angles at the
two points of intersection.
\item Let the vertex of an angle $ABC$ be located outside a circle and let the sides of the angle
intersect equal chords $AD$ and $CE$ with the circle. Prove that $\angle ABC$ is equal to half the
difference of the angles subtended by the chords $AC$ and $DE$ at the centre.
\item Prove that the circle drawn with any side of a rhombus as diameter, passes through
the point of intersection of its diagonals.
\item $ABCD$ is a parallelogram. The circle through $A, B$ and $C$ intersect $CD$ (produced if
necessary) at $E$. Prove that $AE = AD$.
\item $AC$ and $BD$ are chords of a circle which bisect each other. Prove that (i) $AC$ and $BD$ are
diameters, (ii) $ABCD$ is a rectangle.
\item Bisectors of angles $A, B$ and $C$ of a $\triangle ABC$ intersect its circumcircle at $D, E$ and
$F$ respectively. Prove that the angles of the $\triangle DEF$ are $90\degree – \frac{A}{2}, 90\degree – \frac{B}{2}$ and $90\degree – \frac{C}{2}$.

\end{enumerate}
