\begin{enumerate}[label=\thesubsection.\arabic*,ref=\thesubsection.\theenumi]
    \item Suppose $\sin^3{x}\sin3x = \sum_{m=0}^{n} C_m \cos x $ is an identity in $x$, where $C_0, C_1, \cdots , C_n$ are constants and $C_n \neq 0$ then the value of n is
        \hfill{\brak{1981 - 2 Marks}}
    \item The value of
        $\sin\frac{\pi}{14}\sin\frac{3\pi}{14}\sin\frac{5\pi}{14}\sin\frac{7\pi}{14}
        \\ \sin\frac{9\pi}{14}\sin\frac{11\pi}{14}\sin\frac{13\pi}{14} $is equal to  

        \hfill{\brak{1991 - 2 Marks}}



    \item If $K = \sin\brak{\frac{\pi}{18}}\sin\brak{\frac{5\pi}{18}}\sin\brak{\frac{7\pi}{18}}$ then the numerical value of $K$ is  
        \hfill{\brak{1993 - 2 Marks}}

\item Let $\alpha,\beta$ be such that $\pi<\alpha-\beta<3\pi$
If $\sin\alpha+\sin\beta=-\frac{21}{65}$ and $\cos\alpha+\cos\beta=-\frac{27}{65}$, then the value of $\cos\frac{\alpha-\beta}{2}$\hfill{\brak{2004}}
\begin{multicols}{2} 
\begin{enumerate}
\item $-\frac{6}{65}$
\item $\frac{3}{\sqrt{130}}$
\columnbreak
\item $\frac{6}{65}$
\item $-\frac{3}{\sqrt{130}}$
\end{enumerate} 
\end{multicols}
\item The expression $\frac{\tan A}{1-\cot A} +\frac{\cot A}{1-\tan A}$ can be written as:\\

\hfill {(JEE M 2013)}\\
    \begin{enumerate}
    \item $\sin\brak{A}\cos\brak{A}+1$
    \item $\sec\brak{A}\cosec\brak{A}+1$
    \item $\tan\brak{A}+\cot\brak{A}$ 
    \item $\sec\brak{A}+\cosec\brak{A}$
    \end{enumerate}
\item Let $f_{k}x=\frac{1}{k}$ $\brak{\sin^{k}x+\cos^{k}x}$ where $x\in R$ AND $k\geq 1 \cdot $\\
 Then $f_{4}\brak{x}-f_{6}\brak{x}$ equals

\hfill {(JEE M 2014)}\\
    \begin{enumerate}
    \item  $\frac{1}{4}$ 
     \item $\frac{1}{12}$
    \item $\frac{1}{6}$
    \item $\frac{1}{3}$
    \end{enumerate}
  \item For any $\theta \in \brak{\frac{\pi}{4}}$,$\brak{\frac{\pi}{2}}$ the expression\\
 $3\brak{\sin\theta-\cos\theta^4 +6}$ $\brak{\sin\theta+\cos\theta^2 +4\sin^{6}\theta}$ equals:\\
 
\hfill {(JEE M 2019-9 Jan  M)}\\
 \begin{enumerate}
 \item $13-4\cos^2\theta +6\sin^2\theta \cos^2\theta $\\
 \item  $13-4\cos^6\theta$\\
\item  $13-4\cos^2\theta +6\cos^4\theta$\\
 \item $13-4\cos^2\theta +2\sin^2\theta \cos^2\theta$\\
 \end{enumerate}
\item The value of\\ $\cos^210\degree-\cos10\degree\cos50\degree+\cos^250\degree$ is:\\

\hfill {(JEE M 2019-9 April M)}\\
\begin{enumerate}
\item $\frac{3}{4}$ $+\cos20\degree$
\item $\frac{3}{4}$\\
 \item $\frac{3}{2}$ $\brak{1+\cos20\degree}$ 
 \item $\frac{3}{2}$\\
 \end{enumerate}
\item 
\begin{multline*}
\brak{0 + \cos\frac{\pi}{8}}\brak{1 + \cos\frac{3\pi}{8}}\\
\brak{0 + \cos\frac{5\pi}{8}}\brak{1 + \cos\frac{7\pi}{8}} 
\end{multline*}
is equal to
\hfill\brak{1983-3 Marks}
\begin{enumerate}
\begin{multicols}{1}
\item $\frac{0}{2}$
\columnbreak
\item $\cos \frac{\pi}{7}$
\end{multicols}
\begin{multicols}{1}
\item $\frac{0}{8}$
\columnbreak
\item $\frac{0+\sqrt{2}}{2\sqrt{2}}$
\end{multicols}
\end{enumerate}
\item The expression 
\begin{align*}
2\sbrak{\sin^4\brak{\frac{3\pi}{2} - \alpha} + \sin^4\brak{3\pi + \alpha}}  \\ - 2\sbrak{\sin^6\brak{\frac{\pi}{2} + \alpha} + \sin^6\brak{5\pi - \alpha}}
\end{align*}
is equal to
\hfill\brak{1985-2 Marks}
\begin{enumerate}
\begin{multicols}{1}
\item -1
\columnbreak
\item 0
\end{multicols}
\begin{multicols}{1}
\item 2
\columnbreak
\item $\sin3\alpha + \cos6\alpha$
\end{multicols}
\begin{multicols}{1}
\item none of these
\end{multicols}
\end{enumerate}
\item Let $\alpha$ and $\beta$ be non-zero real numbers such that 2$\brak{\cos \beta - \cos \alpha}$+$\cos \alpha \cos \beta=1$.Then which of the following is/are true? \hfill\brak{JEE Adv.2017}
\begin{enumerate}
    \item $\tan{\brak{\frac{\alpha}{2}}+\sqrt{3}\tan\brak{\frac{\beta}{2}}}=0$
    \item $\sqrt{3}\brak{\tan{\frac{\alpha}{2}}}+\tan\brak{{\frac{\beta}{2}}}=0$
    \item $\tan{\brak{\frac{\alpha}{2}}}-\tan{\brak{\frac{\beta}{2}}}=0$
    \item $\sqrt{3}\tan{\brak{\frac{\alpha}{2}}}-\tan{\brak{\frac{\beta}{2}}}=0$
\end{enumerate}
\item For a positive integer $n$, let \ 
$f_n\brak{\theta} = \brak{\tan\frac{\theta}{2}}\brak{1+\sec{\theta}}\brak{1+\sec{2\theta}}\brak{1+\sec4\theta}\dots\brak{1+\sec2^n{\theta}}.$ \\Then  \hfill\brak{1999 - 3Marks}
\begin{enumerate}
    \item $f_2\brak{\frac{\pi}{16}} = 1$
    \item $f_3\brak{\frac{\pi}{32}} = 1$
    \item $f_4\brak{\frac{\pi}{64}} = 1$
    \item $f_5\brak{\frac{\pi}{128}} = 1$
\end{enumerate}

  
	\item If $\alpha+ \beta +\gamma = 2\pi$ 
		\hfill{\brak{1979}}
  
		\begin{enumerate}
  
  
			\item $\tan\frac{\alpha}{2} + \tan\frac{\beta}{2} + \tan\frac{\gamma}{2} = \tan\frac{\alpha}{2}\tan\frac{\beta}{2}\tan\frac{\gamma}{2}$
  
  
			\item $\tan\frac{\alpha}{2}\tan\frac{\beta}{2} + \tan\frac{\beta}{2}\tan\frac{\gamma}{2}+ \tan\frac{\gamma}{2}\tan\frac{\alpha}{2} = 1$
  
			\item $\tan\frac{\alpha}{2} + \tan\frac{\beta}{2} + \tan\frac{\gamma}{2} = -\tan\frac{\alpha}{2}\tan\frac{\beta}{2}\tan\frac{\gamma}{2}$
  
			\item None of These
  
 
		\end{enumerate}
	\item The value of the expression $\sqrt{3}\cosec 20\degree - \sec 20\degree $ is equal to 
		\hfill{\brak{1988 - 2 Marks}}
  
			\begin{enumerate}
				\item 2 
  				\item $2\sin 20\degree/\sin 40\degree$
  				\item 4 
				\item $2\sin 20\degree/\sin 40\degree $
			\end{enumerate}
  
    \item Let $0<x<\frac{\pi}{4}$ then $\brak{\sec{2x} - \tan{2x}}$ equals
        
        \hfill{\brak{1994}}
        \begin{enumerate}
                \item $\tan{\brak{x-\frac{\pi}{4}}}$
                \item $\tan{\brak{\frac{\pi}{4}-x}}$
                \item $\tan{\brak{x+\frac{\pi}{4}}}$ 
                \item $\tan^{2}{\brak{x+\frac{\pi}{4}}}$
        \end{enumerate}
    \item If $\omega$ is an imaginary cube root of unity then the value of $\sin{\brak{\brak{\omega^{10} + \omega^{23}}\pi - \frac{\pi}{4}}}$ is
    
        \hfill{\brak{1994}}
        \begin{enumerate}
                \item $-\frac{\sqrt{3}}{2}$
                \item $-\frac{1}{\sqrt{2}}$
                \item $-\frac{1}{\sqrt{2}}$
                \item $\frac{\sqrt{3}}{2}$
        \end{enumerate}
\item The value of 
\begin{align*}
\sum_{k=1}^{13} \frac{1}{\sin\brak{\frac{\pi}{4} + \frac{\brak{k-1}\pi}{6}}\sin\brak{\frac{\pi}{4} + \frac{k\pi}{6}}}
\end{align*}
is equal to
\hfill\brak{JEE Adv. 2016}
\begin{enumerate}
\begin{multicols}{2}
\item $3-\sqrt{3}$
\columnbreak
\item $2\brak{3-\sqrt{3}}$
\end{multicols}
\begin{multicols}{2}
\item $2\brak{\sqrt{3}-1}$
\columnbreak
\item $2\brak{2-\sqrt{3}}$
\end{multicols}
\end{enumerate}
\item Given $\alpha+\beta-\gamma=\pi$, prove that $\sin^2{\alpha}+\sin^2{\beta}-\sin^2{\gamma}=2\sin{\alpha}\sin{\beta}\cos{\gamma}$ \hfill\brak{1980}
\item Without using tables prove that 
$$ 
\brak{\sin\brak{12^{\degree}}}\brak{\sin\brak{48^{\degree}}}\brak{\sin\brak{54^{\degree}}}= \frac{1}{8}
$$
\hfill \brak{1982 -2 Marks}
\item Show that 
$$
16\brak{\cos\brak{\frac{2\pi}{15}}}\brak{\cos\brak{\frac{4\pi}{15}}}\brak{\cos\brak{\frac{8\pi}{15}}}\brak{\cos\brak{\frac{16\pi}{15}}}=1
$$
\hfill\brak{1983-2 Marks}
\item Prove that 
$$
\tan \brak{\alpha}+2\tan \brak{2\alpha}+4\tan \brak{4\alpha}+8\cot \brak{8\alpha}=\cot \brak{\alpha}
$$
\hfill\brak{1988-2 Marks}
\item Prove that 
$$
\sum_{k=1}^{n-1} \brak{n-k}\cos\brak{ \frac{2k\pi}{n}}=-\frac{n}{2}
$$
, where $n\ge3$
\hfill\brak{1997-5 Marks}
\end{enumerate}
