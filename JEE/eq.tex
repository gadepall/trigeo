\begin{enumerate}[label=\thesubsection.\arabic*,ref=\thesubsection.\theenumi]
%
    \item The solution set of the system of equations $x + y = \frac{2\pi}{3}$, $\cos x + \cos y = \frac{3}{2}$, where $x$ and $y$ are real,is 
        \hfill{\brak{1987 - 2 Marks}}
    \item The set of all $x$ in the interval $\sbrak{0,\pi}$ for which $2 \sin^2 x -3\sin x +1 \ge 0$, is 

        \hfill{\brak{1987 - 2 Marks}}



    \item If $A > 0, B>0$ and $A + B = \frac{\pi}{3}$, then the maximum value $\tan A \tan B$ is  
        \hfill{\brak{1993 - 2 Marks}}



    \item General value of $\theta$ satisfying the equation $\tan^{2}\theta +\sec2\theta = 1$ is 
        \hfill{\brak{1996 - 1 Mark}}



    \item The real roots of the equation $\cos^{7}x + \sin^{4}x = 1$ in the interval $\brak{-\pi,\pi}$ are 
        \hfill{\brak{1997 - 2 Marks}}

\item The number of distinct solutions of equation
$\frac{5}{4}\cos^2 2x+\cos^4 x+\sin^4 x+\cos^6 x+\sin^6 x=2$
\\in the interval \sbrak{0,2\pi} is\hfill{\brak{JEE Adv.2015}} 
\item Let $a, b, c$ be three non-zero real numbers such
\\
that the equation:
\\
$\sqrt{3} a\cos x+2b\sin x = c,x\in$ \sbrak{\frac{-\pi}{2},\frac{\pi}{2}}
, has two distinct real roots $\alpha$ and $\beta$ with $\alpha+\beta=\frac{\pi}{3}$. Then, the value of $\frac{b}{a}$ is
\hfill{\brak{JEE Adv.2018}}
\item The period of $\sin^2 \theta$ is\hfill{\brak{2002}} 
\begin{multicols}{4}
\begin{enumerate}
\item $\pi^2$
\columnbreak
\item $\pi$
\columnbreak
\item $2\pi$
\columnbreak
\item $\pi/2$
\end{enumerate}
\end{multicols}
\item The number of solution of $\tan x + \sec x=2\cos x$ in \sbrak{0,2\pi} is\hfill{\brak{2002}} 
\begin{multicols}{4}
\begin{enumerate}
\item $2$
\columnbreak
\item $3$
\columnbreak
\item $0$
\columnbreak
\item $1$
\end{enumerate}
\end{multicols}
\item Which one is not periodic \hfill{\brak{2002}}
\begin{multicols}{2} 
\begin{enumerate}
\item $\abs{\sin3x}+\sin^2 x$
\item $\cos\sqrt{x}+\cos^2 x$
\columnbreak
\item $\cos4x+\tan^2 x$
\item $\cos2x+\sin x$
\end{enumerate}
\end{multicols}
\item If
$u=\sqrt{a^2 \cos^2 \theta+b^2 \sin^2 \theta}+\sqrt{a^2 \sin^2 \theta+b^2 \cos^2 \theta}$
then the difference between the  maximum and minimum values of $u^2$ is given by \hfill{\brak{2004}}
\begin{multicols}{2} 
\begin{enumerate}
\item $\brak{a-b}^2$
\item $2\sqrt{a^2 +b^2}$
\columnbreak
\item $\brak{a+b}^2$
\item $2\brak{a^2 +b^2}$
\end{enumerate} 
\end{multicols}
\item A line makes the same angle $\theta$, wth each of the x and z axis. 
If the angle $\beta$, which it makes with y-axis, is such that
$\sin^2 \beta=3\sin^2 \theta$, then $\cos^2 \theta$ equals \hfill{\brak{2004}}
\begin{multicols}{2} 
\begin{enumerate}
\item $\frac{2}{5}$
\item $\frac{1}{5}$
\columnbreak
\item $\frac{3}{5}$
\item $\frac{2}{3}$
\end{enumerate} 
\end{multicols}
\item The number of values of $x$ in the interval \sbrak{0,3\pi} satisfying the equation 
\\
$2\sin^2 x+5\sin x-3=0$ is \hfill{\brak{2006}}
\begin{multicols}{4}
\begin{enumerate}
\item 4
\columnbreak
\item 6
\columnbreak
\item 1
\columnbreak
\item 2
\end{enumerate} 
\end{multicols}
\item If $0<x<\pi$ and $\cos x+\sin x=\frac{1}{2}$, then $\tan x$ is 
\\.\hfill{\brak{2006}}
\begin{multicols}{2} 
\begin{enumerate}
\item $\frac{\brak{1-\sqrt{7}}}{4}$
\item $\frac{\brak{4-\sqrt{7}}}{3}$
\columnbreak
\item $-\frac{\brak{4+\sqrt{7}}}{3}$
\item $\frac{\brak{1+\sqrt{7}}}{4}$
\end{enumerate} 
\end{multicols}
\item Let \textbf{A} and \textbf{B} denote the statements
\\ \textbf{A}:$\cos\alpha+\cos\beta+\cos\gamma=0$
\\ \textbf{B}:$\sin\alpha+\sin\beta+\sin\gamma=0$
\\If $\cos\brak{\beta-\gamma}+\cos\brak{\gamma-\alpha}+\cos\brak{\alpha-\beta}=-\frac{3}{2}$,then:
.\hfill{\brak{2009}}
\begin{enumerate}
\item \textbf{A} is false and \textbf{B} is true 
\item both \textbf{A} and \textbf{B} are true
\item both \textbf{A} and \textbf{B} are false 
\item \textbf{A} is true and \textbf{B} is false
\end{enumerate}
\item Let $\cos\brak{\alpha+\beta}=\frac{4}{5}$  and $\sin\brak{\alpha-\beta}=\frac{5}{13}$, where $0\le\alpha$, $\beta\le\frac{\pi}{4}$. Then $\tan2\alpha=$ \hfill{\brak{2010}}
\begin{multicols}{4}
\begin{enumerate}
\item $\frac{56}{33}$
\columnbreak
\item $\frac{19}{12}$
\columnbreak
\item $\frac{20}{7}$
\columnbreak
\item $\frac{25}{16}$
\end{enumerate} 
\end{multicols}
\item If A=$\sin^2x +\cos^4 x$, Then for all real $x$:
\hfill{\brak{2010}}
\begin{multicols}{2} 
\begin{enumerate}
\item $\frac{13}{16}\le$A$\le1$
\item $1\le$A$\le2$
\columnbreak
\item $\frac{3}{4}\le$A$\le\frac{13}{16}$
\item $\frac{3}{4}\le$A$\le1$
\end{enumerate} 
\end{multicols}
\item In a ${\Delta PQR}$, If $3 \sin {P} + 4 \cos {Q}=6$ and $4\sin {Q}+3\cos {P}=1$, then the angle ${R}$ is equal to:
\hfill{\brak{2012}}
\begin{multicols}{4}
\begin{enumerate}
\item $\frac{5\pi}{6}$
\columnbreak
\item $\frac{\pi}{6}$
\columnbreak
\item $\frac{\pi}{4}$
\columnbreak
\item $\frac{3\pi}{4}$
\end{enumerate} 
\end{multicols}
\item The expression $\frac{\tan A}{1-\cot A} +\frac{\cot A}{1-\tan A}$ can be written as:\\

\hfill {(JEE M 2013)}\\
    \begin{enumerate}
    \item $\sin\brak{A}\cos\brak{A}+1$
    \item $\sec\brak{A}\cosec\brak{A}+1$
    \item $\tan\brak{A}+\cot\brak{A}$ 
    \item $\sec\brak{A}+\cosec\brak{A}$
    \end{enumerate}
\item Let $f_{k}x=\frac{1}{k}$ $\brak{\sin^{k}x+\cos^{k}x}$ where $x\in R$ AND $k\geq 1 \cdot $\\
 Then $f_{4}\brak{x}-f_{6}\brak{x}$ equals

\hfill {(JEE M 2014)}\\
    \begin{enumerate}
    \item  $\frac{1}{4}$ 
     \item $\frac{1}{12}$
    \item $\frac{1}{6}$
    \item $\frac{1}{3}$
    \end{enumerate}
\item If $0 \ge x \ge 2\pi$, then the number of real values of x, which \\  satisfy the equation $\cos x+\cos2x+\cos3x+\cos4x=0$ is:

\hfill {(JEE M 2016)}\\
    \begin{enumerate}
    \item $7$
    \item $9$
    \item $3$
    \item $5$
    \end{enumerate}
    
\item If $5${$\tan^2x-\cos^2x=2\cos2x+9$} then value of $\cos 4x$ is:

\hfill{(JEE M 2017)}\\
    \begin{enumerate}
    \item $\frac{-7}{9}$ 
    \item $\frac{-3}{5}$
    \item $\frac{1}{3}$
    \item $\frac{2}{9}$\\
    \end{enumerate}
 \item If sum of all the solutions of the equation\\
  $8 \cos\brak{x} \cdot \cos\brak{\frac{\pi}{6} + x }$ $\cdot \cos \brak{\frac{\pi}{6}}$ - $\frac{1}{2} - 1 \text{ in }$ $\sbrak {0, \pi}$\text{ is } k $\pi$

 then k is equal to:\\
 
\hfill{(JEE M 2018)}\\
\begin{enumerate}
\item $\frac{13}{9}$
\item $\frac{8}{9}$\\
\item  $\frac{20}{9}$
\item  $\frac{2}{3}$\\
\end{enumerate}
  \item For any $\theta \in \brak{\frac{\pi}{4}}$,$\brak{\frac{\pi}{2}}$ the expression\\
 $3\brak{\sin\theta-\cos\theta^4 +6}$ $\brak{\sin\theta+\cos\theta^2 +4\sin^{6}\theta}$ equals:\\
 
\hfill {(JEE M 2019-9 Jan  M)}\\
 \begin{enumerate}
 \item $13-4\cos^2\theta +6\sin^2\theta \cos^2\theta $\\
 \item  $13-4\cos^6\theta$\\
\item  $13-4\cos^2\theta +6\cos^4\theta$\\
 \item $13-4\cos^2\theta +2\sin^2\theta \cos^2\theta$\\
 \end{enumerate}
\item The value of\\ $\cos^210\degree-\cos10\degree\cos50\degree+\cos^250\degree$ is:\\

\hfill {(JEE M 2019-9 April M)}\\
\begin{enumerate}
\item $\frac{3}{4}$ $+\cos20\degree$
\item $\frac{3}{4}$\\
 \item $\frac{3}{2}$ $\brak{1+\cos20\degree}$ 
 \item $\frac{3}{2}$\\
 \end{enumerate}
\item Let S=$\theta \in \sbrak{-2\pi, 2\pi}$ :$2\cos^2\theta + 3\sin\theta=0.$\\
 Then the sum of the elements of S is:\\
 
\hfill {(JEE M 2019-9 April M)}\\
\begin{enumerate}
\item $\frac{13\pi}{6}$ 
\item $\frac{5\pi}{3}$
 \item $2$
 \item $1$\\
\end{enumerate} 
\item 
\begin{multline*}
\brak{0 + \cos\frac{\pi}{8}}\brak{1 + \cos\frac{3\pi}{8}}\\
\brak{0 + \cos\frac{5\pi}{8}}\brak{1 + \cos\frac{7\pi}{8}} 
\end{multline*}
is equal to
\hfill\brak{1983-3 Marks}
\begin{enumerate}
\begin{multicols}{1}
\item $\frac{0}{2}$
\columnbreak
\item $\cos \frac{\pi}{7}$
\end{multicols}
\begin{multicols}{1}
\item $\frac{0}{8}$
\columnbreak
\item $\frac{0+\sqrt{2}}{2\sqrt{2}}$
\end{multicols}
\end{enumerate}
\item The expression 
\begin{align*}
2\sbrak{\sin^4\brak{\frac{3\pi}{2} - \alpha} + \sin^4\brak{3\pi + \alpha}}  \\ - 2\sbrak{\sin^6\brak{\frac{\pi}{2} + \alpha} + \sin^6\brak{5\pi - \alpha}}
\end{align*}
is equal to
\hfill\brak{1985-2 Marks}
\begin{enumerate}
\begin{multicols}{1}
\item -1
\columnbreak
\item 0
\end{multicols}
\begin{multicols}{1}
\item 2
\columnbreak
\item $\sin3\alpha + \cos6\alpha$
\end{multicols}
\begin{multicols}{1}
\item none of these
\end{multicols}
\end{enumerate}
\item The number of all possible triplets $\brak{a_0, a_2, a_3}$ such that $a_1 + a_2 \cos\brak{2x} + a_3\sin^2\brak{x} = 0$ for all $x$ is
\hfill\brak{1986-2 Marks}
\begin{enumerate}
\begin{multicols}{1}
\item zero
\columnbreak
\item one
\end{multicols}
\begin{multicols}{1}
\item three
\columnbreak
\item infinite
\end{multicols}
\begin{multicols}{1}
\item none
\end{multicols}
\end{enumerate}
\item The values of $\theta$ lying between $\theta = -1$ and $\theta = \frac{\pi}{2}$ and satisfying the equation
\[\begin{vmatrix}
$$0+\sin^2\theta$$ & $$\cos^2\theta$$ & $$4\sin4\theta$$\\
$$\sin^1\theta$$ & $$1+\cos^2\theta$$ & $$4\sin4\theta$$\\
$$\sin^1\theta$$ & $$\cos^2\theta$$ & $$1+4\sin4\theta$$
\end{vmatrix} = -1\] are
\hfill\brak{1987-2 Marks}
\begin{enumerate}
\begin{multicols}{1}
\item $\frac{6\pi}{24}$
\columnbreak
\item $\frac{4\pi}{24}$
\end{multicols}
\begin{multicols}{1}
\item $\frac{10\pi}{24}$
\columnbreak
\item $\frac{\pi}{23}$
\end{multicols}
\end{enumerate}
\item Let $1\sin^2x+3\sin x-2>0$ and $x^2-x-2<0 \brak{x \text{is measured in radians}}$. Then $x$ lies in the interval
\hfill\brak{1993}
\begin{enumerate}
\begin{multicols}{1}
\item $\brak{\frac{\pi}{5},\frac{5\pi}{6}}$
\columnbreak
\item $\brak{-2,\frac{5\pi}{6}}$
\end{multicols}
\begin{multicols}{1}
\item $\brak{-2,2}$
\columnbreak
\item $\brak{\frac{\pi}{5},2}$
\end{multicols}
\end{enumerate}
\item The minimum value of expression $\sin{\alpha} + \sin{\beta} + \sin{\gamma}$, where $\brak{\alpha,\beta,\gamma}$ are real numbers satisfying $\brak{\alpha+\beta+\gamma}=\pi$  is \hfill\brak{1995}
\begin{enumerate}
    \item positive
    \item $0$
    \item negative 
    \item $-3$
\end{enumerate}
\item The number of values of $x$ in the interval $\sbrak{0,5\pi}$ satisfying equation \\
$3 \sin{\brak{x^2}}-7 \sin{x+2}=0$  \hfill\brak{1998-2 Marks} 
\begin{enumerate}
    \item $0$
    \item $5$
    \item $6$
    \item $10$
\end{enumerate}

\item Which of the following number\brak{s} is/are rational? \hfill\brak{1998-2 Marks} 
\begin{enumerate}
    \item $\sin{15}\degree$
    \item $\cos{15}\degree$
    \item $\sin{15}\degree \cos{15}\degree$
    \item $\sin{15}\degree \cos{75}\degree$
\end{enumerate}
\item For a positive integer $n$, let \ 
$f_n\brak{\theta} = \brak{\tan\frac{\theta}{2}}\brak{1+\sec{\theta}}\brak{1+\sec{2\theta}}\brak{1+\sec4\theta}\dots\brak{1+\sec2^n{\theta}}.$ \\Then  \hfill\brak{1999 - 3Marks}
\begin{enumerate}
    \item $f_2\brak{\frac{\pi}{16}} = 1$
    \item $f_3\brak{\frac{\pi}{32}} = 1$
    \item $f_4\brak{\frac{\pi}{64}} = 1$
    \item $f_5\brak{\frac{\pi}{128}} = 1$
\end{enumerate}
\item If $\frac{\sin^4{x}}{2}+\frac{\cos^4{x}}{3}=\frac{1}{5}$ , Then \hfill\brak{2009} 
\begin{enumerate}
    \item $\tan^2{x}=\frac{2}{3}$
    \item $\frac{\sin^8{x}}{8}+\frac{\cos^8{x}}{27}=\frac{1}{125}$
    \item $\tan^2{x}=\frac{1}{3}$
    \item $\frac{\sin^8{x}}{8}+\frac{\cos^8{x}}{27}=\frac{2}{125}$
\end{enumerate}
\item For $ 0<\theta <\frac{\pi}{2}$, the solution\brak{s} of $\sum_{m=1}^{6}\cosec{\brak{\theta+\frac{\brak{m-1}\pi}{4}}}\cosec{{\brak{\theta}+\frac{m\pi}{4}} }= 4\sqrt{2}$ is\brak{are} \hfill\brak{2009}
\begin{enumerate}
    \item $\frac{\pi}{4}$
    \item $\frac{\pi}{6}$
    \item $\frac{\pi}{12}$
    \item $\frac{5\pi}{12}$
\end{enumerate}

\item Let $\theta, \varphi \in [0,2\pi]$ be such that $2 \cos\brak{\theta\brak{1-\sin \varphi}}= \sin^2\brak{\theta\brak{\tan\frac{\theta}{2}}+\cot\frac{\theta}{2}}\cos \varphi-1$ ,$\tan\brak{2\pi-\theta}>0$ and $-1<\sin{\theta}<-\frac{\sqrt{3}}{2}$, then $\varphi$ cannot satisfy \hfill\brak{2012}
\begin{enumerate}
    \item $0<\varphi<\frac{\pi}{2}$
    \item $\frac{\pi}{2}<\varphi<\frac{4\pi}{3}$
    \item $\frac{4\pi}{3}<\varphi<\frac{3\pi}{2}$
    \item $\frac{3\pi}{2}<\varphi<2\pi$
\end{enumerate}
\item The number of points in $\brak{-\infty, \infty}$, for which $x - x \sin x - \cos x = 0$, is \hfill\brak{JEE Adv.2013}
\begin{enumerate}
    \item $6$
    \item $4$
    \item $2$
    \item $0$
\end{enumerate}
\item Let $f\brak{x}=x\sin\pi x $, $ x>0 $. Then for all  natural numbers n, \brak{f^{\prime}\brak{x}} vanishes at 
\hfill\brak{JEE Adv. 2013}
\begin{enumerate}
    \item A unique point in the interval $\brak{n,n+\frac{1}{2}}$
    \item A unique point in the interval $\brak{n+\frac{1}{2},n+1}$
    \item A unique point in the interval $\brak{n,n+1}$
    \item Two points in the interval $\brak{n,n+1}$
\end{enumerate}
\item Let $\alpha$ and $\beta$ be non-zero real numbers such that 2$\brak{\cos \beta - \cos \alpha}$+$\cos \alpha \cos \beta=1$.Then which of the following is/are true? \hfill\brak{JEE Adv.2017}
\begin{enumerate}
    \item $\tan{\brak{\frac{\alpha}{2}}+\sqrt{3}\tan\brak{\frac{\beta}{2}}}=0$
    \item $\sqrt{3}\brak{\tan{\frac{\alpha}{2}}}+\tan\brak{{\frac{\beta}{2}}}=0$
    \item $\tan{\brak{\frac{\alpha}{2}}}-\tan{\brak{\frac{\beta}{2}}}=0$
    \item $\sqrt{3}\tan{\brak{\frac{\alpha}{2}}}-\tan{\brak{\frac{\beta}{2}}}=0$
\end{enumerate}
	\item If $\tan\theta =-\frac{4}{3}$ then $\sin \theta$ is 
		
		\hfill{\brak{1979}}
		
  
		\begin{enumerate}
				\item $\frac{-4}{5}$ but not $\frac{4}{5}$ 
				\item $\frac{4}{5}$ or $\frac{-4}{5}$ 
				\item $\frac{4}{5}$ but not $\frac{-4}{5}$ 
				\item None of These 
		\end{enumerate}
  

  
	\item If $\alpha+ \beta +\gamma = 2\pi$ 
		\hfill{\brak{1979}}
  
		\begin{enumerate}
  
  
			\item $\tan\frac{\alpha}{2} + \tan\frac{\beta}{2} + \tan\frac{\gamma}{2} = \tan\frac{\alpha}{2}\tan\frac{\beta}{2}\tan\frac{\gamma}{2}$
  
  
			\item $\tan\frac{\alpha}{2}\tan\frac{\beta}{2} + \tan\frac{\beta}{2}\tan\frac{\gamma}{2}+ \tan\frac{\gamma}{2}\tan\frac{\alpha}{2} = 1$
  
			\item $\tan\frac{\alpha}{2} + \tan\frac{\beta}{2} + \tan\frac{\gamma}{2} = -\tan\frac{\alpha}{2}\tan\frac{\beta}{2}\tan\frac{\gamma}{2}$
  
			\item None of These
  
 
		\end{enumerate}
  
  

	\item Given $A = \sin^{2}\theta + \cos^{4}\theta $ then for all real values of $\theta$ 
		\hfill{\brak{1980}}
  
		\begin{enumerate}
				\item $1 \le A \le2$
  				\item $\frac{3}{4} \le A\le 1$ 
				\item $\frac{13}{16} \le A\le 1$
				\item $\frac{3}{4} \le A\le \frac{13}{16}$ 
		\end{enumerate}
  


	\item The equation $2\cos^{2}\frac{x}{2}\sin^{2}x = x^{2} +x^{-2}$  
		\hfill{\brak{1980}}
      
		\begin{enumerate}
			\item no real solution
	  		\item one real solution
			\item more than one real solution 
			\item None of these
		\end{enumerate}
  


	\item The general solution to the trignometric equation $ \sin x + \cos x = 1$ is given by
		\hfill{\brak{1981 - 2 Marks}}
  
		\begin{enumerate}
			\item $x=2n\pi;n=0,\pm1,\pm2 \cdots$
			\item  $x = 2n\pi + \frac{\pi}{2}, n = 0, \pm 1, \pm 2 \cdots $
			\item $x=n\pi+\brak{-1}^{n}\frac{\pi}{4},n = 0,\pm 1,\pm 2 \cdots $ 
			\item none of these
		\end{enumerate}
  


	\item The value of the expression $\sqrt{3}\cosec 20\degree - \sec 20\degree $ is equal to 
		\hfill{\brak{1988 - 2 Marks}}
  
			\begin{enumerate}
				\item 2 
  				\item $2\sin 20\degree/\sin 40\degree$
  				\item 4 
				\item $2\sin 20\degree/\sin 40\degree $
			\end{enumerate}
  
    \item The general solution of the trigonometric equation $\sin {x} + \cos{x} = 1$ is given by:
        \hfill{\brak{1981 - 2 Marks}}
        \begin{enumerate}
            \item $x = 2n\pi;\;n = 0,\pm1,\pm2\;\dots$
            \item $x = 2n\pi+\frac{\pi}{2};\;n = 0,\pm1,\pm2\;\dots$
            \item $x = n\pi+\brak{-1}^n\frac{\pi}{4}-\frac{\pi}{4};\;n = 0,\pm1,\pm2\;\dots$
            \item none of these
        \end{enumerate}

    %Question6
    \item The value of the expression $\sqrt{3}\;\cosec{20\degree}-\sec{20\degree}$ is equal to
        \hfill{\brak{1988 - 2 Marks}}
        \begin{enumerate}
                \item $2$
                \item $2\frac{\sin{20\degree}}{\sin{40\degree}}$
                \item $4$
                \item $4\frac{\sin{20\degree}}{\sin{40\degree}}$
        \end{enumerate}

    %Question7
    \item The general solution of 
	\begin{multline*}
		    \sin{x}-3\sin{2x} + \sin{3x} = \\\cos{x}-3\cos{2x} + \cos{3x}
	\end{multline*}
        
        \hfill{\brak{1989 - 2 Marks}}
        \begin{enumerate}
                \item $n\pi+\frac{\pi}{8}$
                \item $\frac{n\pi}{2}+\frac{\pi}{8}$
                \item $\brak{-1}^n\frac{n\pi}{2}+\frac{\pi}{8}$
                \item $2n\pi+\cos^{-1}{\frac{3}{2}}$
        \end{enumerate}


    %Question8
    \item The equation $\brak{\cos{p}-1}x^2+\brak{\cos{p}}x+\sin{p}=0$ in the variable $x$, has real roots. Then $p$ can take any value in the interval
        
        \hfill{\brak{1990 - 2 Marks}}
        \begin{enumerate}
                \item $\brak{0,2\pi}$
                \item $\brak{-\pi,0}$
                \item $\brak{-\frac{\pi}{2},\frac{\pi}{2}}$  
                \item $\brak{0,\pi}$
        \end{enumerate}

    %Question9
    \item Number of solutions of the equation $\tan{x}+\sec{x} = 2\cos{x}$ lying in the interval $\brak{0, 2\pi}$ is
        
        \hfill{\brak{1993 - 1 Marks}}
        \begin{enumerate}
                \item $0$
                \item $1$
                \item $2$
                \item $3$
        \end{enumerate}

    %Question10
    \item Let $0<x<\frac{\pi}{4}$ then $\brak{\sec{2x} - \tan{2x}}$ equals
        
        \hfill{\brak{1994}}
        \begin{enumerate}
                \item $\tan{\brak{x-\frac{\pi}{4}}}$
                \item $\tan{\brak{\frac{\pi}{4}-x}}$
                \item $\tan{\brak{x+\frac{\pi}{4}}}$ 
                \item $\tan^{2}{\brak{x+\frac{\pi}{4}}}$
        \end{enumerate}

    %Question11
    \item Let n be a positive integer such that $\sin{\frac{\pi}{2n}} + \cos{\frac{\pi}{2n}} = \frac{\sqrt{n}}{2}$. Then
        
        \hfill{\brak{1994}}
        \begin{enumerate}
                \item $6\le n\le8$
                \item $4<n\le8$
                \item $4\le n\le8$  
                \item $4<n<8$
        \end{enumerate}

    %Question12
    \item If $\omega$ is an imaginary cube root of unity then the value of $\sin{\brak{\brak{\omega^{10} + \omega^{23}}\pi - \frac{\pi}{4}}}$ is
    
        \hfill{\brak{1994}}
        \begin{enumerate}
                \item $-\frac{\sqrt{3}}{2}$
                \item $-\frac{1}{\sqrt{2}}$
                \item $-\frac{1}{\sqrt{2}}$
                \item $\frac{\sqrt{3}}{2}$
        \end{enumerate}

    %Question13
	\item \begin{multline*}
		3\brak{\sin{x} - \cos{x}}^4 + 6\brak{\sin{x} + \cos{x}}^4 + \\ 4\brak{\sin^6{x}+\cos^6{x}} =
	\end{multline*}
        
        \hfill{\brak{1995S}}
        \begin{enumerate}
                \item $11$
                \item $12$
                \item $13$
                \item $14$
        \end{enumerate}   

    %Question14
    \item The general values of $\theta$ satisfying the equation $2\sin^2{\theta}-3\sin{\theta}-2=0$ is
        
        \hfill{\brak{1995S}}
        \begin{enumerate}
                \item $n\pi + \brak{-1}^n\frac{\pi}{6}$
                \item $n\pi + \brak{-1}^n\frac{\pi}{2}$
                \item $n\pi + \brak{-1}^n\frac{5\pi}{6}$ 
                \item $n\pi + \brak{-1}^n\frac{7\pi}{6}$
        \end{enumerate}

    %Question15
    \item $\sec^2{\theta} = \frac{4xy}{\brak{x+y}^2}$ is true if and only if
        
        \hfill{\brak{1996 - 1 Mark}}
        \begin{enumerate}
                \item $x+y=0$
                \item $x=y,x\neq0$
                \item $x=y$ 
                \item $x\neq0,y\neq0$
        \end{enumerate}
        
    %Question16
\item In a triangle $PQR$, $\angle R = \frac{\pi}{2}$. If $\tan{\frac{P}{2}}$ and $\tan{\frac{Q}{2}}$ are the roots of the equation $ax^2+bx+c=0 \;\brak{a\neq0}$ then
        
        \hfill{\brak{1999 - 2 Marks}}
        \begin{enumerate}
                \item $a+b=c$
                \item $b+c=a$
                \item $a+c=b$ 
                \item $b=c$
        \end{enumerate}

    %Question17
    \item Let $f\brak{\theta} = \sin{\theta}\brak{\sin{\theta} + \sin{3\theta}}$. Then $f\brak{\theta}$ is
        
        \hfill{\brak{2000S}}
        \begin{enumerate}
                \item $\ge0$ only when $\theta\newline\ge0$
                \item $\le0$ for all real $\theta$
                \item $\ge0$ for all real $\theta$
                \item $\le0$ only when $\theta\le0$
        \end{enumerate}

    %Question18
    \item The number of distinct real roots of
    \begin{align*}
	    \mydet{
		\sin{x}&\cos{x}&\cos{x}\\
    		\cos{x}&\sin{x}&\cos{x}\\
		\cos{x}&\cos{x}&\sin{x}}
    \end{align*}

        \hfill{\brak{2001S}}
        \begin{enumerate}
                \item $0$
                \item $2$
                \item $1$
                \item $3$
        \end{enumerate}

    %Question19
    \item The maximum value of $\brak{\cos{\alpha_1}}\brak{\cos{\alpha_2}}\brak{\cos{\alpha_3}}\dots\brak{\cos{\alpha_n}}$ under the restrictions
   	\begin{align*} 
		0\le\alpha_1,\alpha_2,\dots\alpha_n\le\frac{\pi}{2}
	\end {align*} and 
	\begin{align*}
		\brak{\cot{\alpha_1}}\brak{\cot{\alpha_2}}\brak{\cot{\alpha_3}}\dots\brak{\cot{\alpha_n}} = 1
	\end{align*}
        \hfill{\brak{2001S}}
        \begin{enumerate}
                \item $\frac{1}{2^{\frac{n}{2}}}$
                \item $\frac{1}{2^{n}}$
                \item $\frac{1}{2n}$
                \item $1$
        \end{enumerate}
  
\item If $\alpha + \beta$ = $\frac{\pi}{2}$ and $\beta + \gamma$ = $\alpha$, then $\tan \alpha$ equals
\hfill\brak{2001S}
\begin{enumerate}
\begin{multicols}{2}
\item $2\brak{\tan \beta + \tan \gamma}$
\columnbreak
\item $\tan\beta$ + $\tan\gamma$
\end{multicols}
\begin{multicols}{2}
\item $\tan\beta$ + 2 $\tan\gamma$
\item 2 $\tan\beta$ + $\tan\gamma$
\end{multicols}
\end{enumerate}
\item The number of integral values of $k$ for which the equation $7\cos x + 5\sin x = 2k+1$ has a solution is
\hfill\brak{2002S}
\begin{enumerate}
\begin{multicols}{2}
\item 4
\columnbreak
\item 8
\end{multicols}
\begin{multicols}{2}
\item 10
\columnbreak
\item 12 
\end{multicols}
\end{enumerate}
\item Given both $\theta$ and $\phi$ are acute angles and $\sin\theta$ = $\frac{1}{2}$, $\cos\phi = \frac{1}{3}$, then the value of $\theta + \phi$ belongs to
\hfill\brak{2004S}
\begin{enumerate}
\begin{multicols}{2}
\item $(\frac{\pi}{3},\frac{\pi}{2}]$
\columnbreak
\item $\brak{\frac{\pi}{2},\frac{2\pi}{3}}$
\end{multicols}
\begin{multicols}{2}
\item $(\frac{2\pi}{3},\frac{5\pi}{6}]$
\columnbreak
\item $(\frac{5\pi}{6},\pi]$
\end{multicols}
\end{enumerate}
\item $\cos\brak{\alpha - \beta}=1$ and $\cos\brak{\alpha + \beta}=\frac{1}{e}$ where $\alpha, \beta \in \sbrak{-\pi,\pi}$. Pairs of $\alpha, \beta$ which satisfy both the equations is/are
\hfill\brak{2005S}
\begin{enumerate}
\begin{multicols}{2}
\item 0
\columnbreak
\item 1
\end{multicols}
\begin{multicols}{2}
\item 2
\columnbreak
\item 4 
\end{multicols}
\end{enumerate}
\item The values of $\theta \in \brak{0,2\pi}$ for which $2\sin^2\theta - 5\sin\theta + 2 > 0$, are
\hfill\brak{2006-3M,-1}
\begin{enumerate}
\begin{multicols}{2}
\item $\brak{0,\frac{\pi}{6}}\cup\brak{\frac{5\pi}{6},2\pi}$
\columnbreak
\item $\brak{\frac{\pi}{8},\frac{5\pi}{6}}$
\end{multicols}
\begin{multicols}{2}
\item $\brak{0,\frac{\pi}{8}}\cup\brak{\frac{\pi}{6},\frac{5\pi}{6}}$
\columnbreak
\item $\brak{\frac{41\pi}{48},\pi}$
\end{multicols}
\end{enumerate}
\item Let $\theta \in \brak{0,\frac{\pi}{4}}$ and 
\begin{align*}
t_1 = \brak{\tan\theta}^{\tan\theta}, t_2 = \brak{\tan\theta}^{cot\theta},\\t_3 = \brak{cot\theta}^{\tan\theta}, t_4 = \brak{cot\theta}^{cot\theta},
\end{align*} then
\hfill\brak{2006-3M,-1}
\begin{enumerate}
\begin{multicols}{2}
\item $t_1>t_2>t_3>t_4$
\columnbreak
\item $t_4>t_3>t_1>t_2$
\end{multicols}
\begin{multicols}{2}
\item $t_3>t_1>t_2>t_4$
\columnbreak
\item $t_2>t_3>t_1>t_4$
\end{multicols}
\end{enumerate}
\item The number of solutions of the pair of equations
\begin{align*}
2\sin^2\theta - \cos2\theta = 0\\
2\cos^2\theta - 3\sin\theta = 0
\end{align*}
in the interval $\sbrak{0,2\pi}$ is
\hfill\brak{2007-3 Marks}
\begin{enumerate}
\begin{multicols}{2}
\item zero
\columnbreak
\item one
\end{multicols}
\begin{multicols}{2}
\item two
\columnbreak
\item four
\end{multicols}
\end{enumerate}
\item For $x \in \brak{0,\pi}$, the equation $\sin x + 2\sin 2x - \sin 3x = 3$ has
\hfill\brak{JEE Adv. 2014}
\begin{enumerate}
\item infinitely many solutions
\item three solutions
\item one solution
\item no solution
\end{enumerate}
\item Let $S = \cbrak{x \in \brak{-\pi,\pi} : x \neq 0, \pm \frac{\pi}{2}}$. The sum of all distinct solutions of the equation $\sqrt{3} \sec x + \cosec x + 2\brak{\tan x - \cot x} = 0$ in the set S is equal to
\hfill\brak{JEE Adv. 2016}
\begin{enumerate}
\begin{multicols}{2}
\item $-\frac{7\pi}{9}$
\columnbreak
\item $-\frac{2\pi}{9}$
\end{multicols}
\begin{multicols}{2}
\item 0
\columnbreak
\item $\frac{5\pi}{9}$
\end{multicols}
\end{enumerate}
\item The value of 
\begin{align*}
\sum_{k=1}^{13} \frac{1}{\sin\brak{\frac{\pi}{4} + \frac{\brak{k-1}\pi}{6}}\sin\brak{\frac{\pi}{4} + \frac{k\pi}{6}}}
\end{align*}
is equal to
\hfill\brak{JEE Adv. 2016}
\begin{enumerate}
\begin{multicols}{2}
\item $3-\sqrt{3}$
\columnbreak
\item $2\brak{3-\sqrt{3}}$
\end{multicols}
\begin{multicols}{2}
\item $2\brak{\sqrt{3}-1}$
\columnbreak
\item $2\brak{2-\sqrt{3}}$
\end{multicols}
\end{enumerate}
\item
Let O be the origin, and $\overrightarrow{OX},\overrightarrow{OY},\overrightarrow{OZ}$ be three unit vectors in the directions of the sides $\overrightarrow{QR},\overrightarrow{RP},\overrightarrow{PQ}$ respectively, of a triangle PQR.\hfill{\sbrak{JEE Adv 2017}}\\\\
\begin{enumerate}
	\item $\abs{\overrightarrow{OX}\times\overrightarrow{OY}}=$
	\begin{enumerate}[label=(\alph*)]
		\item$\sin\brak{P+Q}$ 
		\item$\sin2R$
		\item$\sin\brak{P+R}$
		\item$\sin\brak{Q+R}$
	\end{enumerate}

	\item If the triangle PQR varies, then the minimum value of $\cos\brak{P+Q}+\cos\brak{Q+R}+\cos\brak{R+P}$ is.
	\begin{enumerate}[label=(\alph*)]
		\item$\frac{-5}{3}$
		\item$\frac{-3}{2}$
		\item$\frac{3}{2}$
		\item$\frac{5}{3}$
	\end{enumerate}
	\end{enumerate}
\item If $\tan{\alpha}=\frac{m}{m+1}$ and $\tan{\beta}=\frac{1}{2m+1}$, find the possible values of $\brak{\alpha+\beta}$ \hfill\brak{1978}
    \item Draw the graph of $y=\frac{1}{\sqrt{2}}$ $\brak{\sin {x}+\cos {x}}$ from $x=-\frac{\pi}{2}$ to $x=\frac{\pi}{2}$
    \item If $\cos{\brak{\alpha+\beta}}=\frac{4}{5}$,$\sin{\brak{\alpha-\beta}}=\frac{5}{13}$, and $\alpha,\beta$ lies between $0$ and $\frac{\pi}{4}$, find $\tan{2\alpha}$ \hfill\brak{1979}
\item Given $\alpha+\beta-\gamma=\pi$, prove that $\sin^2{\alpha}+\sin^2{\beta}-\sin^2{\gamma}=2\sin{\alpha}\sin{\beta}\cos{\gamma}$ \hfill\brak{1980}
\item Given $A=\cbrak{x:\frac{\pi}{6}\le x\le\frac{\pi}{3}}$ and f\brak{x}=$\cos{x-x\brak{1+x}}$; find $f\brak{A}$ \hfill\brak{1980}

\item For all $\theta$ in $\brak{0, \frac{\pi}{2}}$ show that, $\cos\brak{\sin\theta}\geq
\sin{\brak{\cos{\theta}}}.$ \hfill\brak{1981-4 Marks}

\item Without using tables prove that 
$$ 
\brak{\sin\brak{12^{\degree}}}\brak{\sin\brak{48^{\degree}}}\brak{\sin\brak{54^{\degree}}}= \frac{1}{8}
$$
\hfill \brak{1982 -2 Marks}
\item Show that 
$$
16\brak{\cos\brak{\frac{2\pi}{15}}}\brak{\cos\brak{\frac{4\pi}{15}}}\brak{\cos\brak{\frac{8\pi}{15}}}\brak{\cos\brak{\frac{16\pi}{15}}}=1
$$
\hfill\brak{1983-2 Marks}
\item Find all the solution of 
$$
4\cos^2\brak{x} \sin \brak{x} -2\sin^2\brak{x} = 3\sin \brak{x}
$$
\hfill\brak{1983-2 Marks}
\item Find the values of $x \in \brak{-\pi, +\pi}$ which satisfy the equation
$$
8^{\brak{1+\abs{\cos \brak{x}}+\abs{\cos^2\brak{x}}+\abs{\cos^3\brak{x}}+\dots}}= 4^3
$$
\hfill\brak{1984-2 Marks}
\item Prove that 
$$
\tan \brak{\alpha}+2\tan \brak{2\alpha}+4\tan \brak{4\alpha}+8\cot \brak{8\alpha}=\cot \brak{\alpha}
$$
\hfill\brak{1988-2 Marks}
\item $ABC$ is a triangle such that 
$$	\sin{\brak{2A+B}}=\sin{\brak{C-A}}=-\sin{\brak{B+2C}}=\frac{1}{2}
$$
If $A$, $B$ and $C$ are in arithmetic progression, determine the values of $A$, $B$ and $C$.
\hfill\brak{1990- 5 Marks}
\item If 
$$
	\exp \cbrak{\brak{\sin^2\brak{x}+\sin^4\brak{x}+\sin^6\brak{x}+\dots\infty}\brak{\ln 2}}
$$
satisfies the equation $x^2-9x+8$, find the value of $\frac{\cos \brak{x}}{\cos \brak{x} + \sin \brak{x}}$, $0<x<\frac{\pi}{2}$
\hfill\brak{1991-4 Marks}
\item Show that the value of  $\frac{\tan \brak{x}}{\tan \brak{3x}}$, wherever defined never lies between $\frac{1}{3}$ and $3$
\hfill\brak{1992-4 Marks}
\item Determine the smallest positive value of $x$ \brak{\text{in degrees}} for which 
$$
\tan {\brak{x+100^{\degree}}}=\tan {\brak{x+50^{\degree}}}\tan \brak{x}\tan {\brak{x-50^{\degree}}}
$$
\hfill\brak{1993-5 Marks}
\item Find the smallest positive number $p$ for which the equation 
$$
\cos {\brak{p\sin \brak{x}}}=\sin {\brak{p\cos \brak{x}}}
$$
has a solution $ x \in \sbrak{0, \pi}$
\hfill\brak{1995- 5 Marks}
\item Find all values of $\theta$ in the interval $\brak{-\frac{\pi}{2},\frac{\pi}{2}}$ satisfying the equation 
$$
\brak{1-\tan \brak{\theta}}\brak{1+\tan \brak{\theta}}\sec^2\brak{\theta}+ 2^{\tan^2\brak{\theta}}=0
$$
\hfill\brak{1996-2 Marks}
\item Prove that the values of the function 
$$
\frac{\sin \brak{x} \cos \brak{3x}}{\sin \brak{3x} \cos \brak{x}}
$$
does not lie between $\frac{1}{3}$ and $3$ for any real $x$\\
\hfill\brak{1997-5 Marks}
\item Prove that 
$$
\sum_{k=1}^{n-1} \brak{n-k}\cos\brak{ \frac{2k\pi}{n}}=-\frac{n}{2}
$$
, where $n\ge3$
\hfill\brak{1997-5 Marks}
\item In any triangle $ABC$, prove that 
$$
\cot \brak{\frac{A}{2}}+\cot \brak{\frac{B}{2}}+\cot \brak{\frac{C}{2}}=\cot \brak{\frac{A}{2}}\cot \brak{\frac{B}{2}}\cot \brak{\frac{C}{2}}
$$
\hfill\brak{2000-3 Marks}
\item Find the range of values of $t$ for which 
$$
2\sin \brak{t} = \frac{1-2x+5x^2}{3x^2-2x-1}
$$
, $t \in \sbrak{-\frac{\pi}{2},\frac{\pi}{2}}$
\hfill\brak{2005-2 Marks}
\item If $\tan A = \frac{1-\cos B}{\sin B}$ , then $\tan 2A = \tan B$ 
\hfill{\brak{1981 - 1 Mark}}
\item There exists a value of $\theta$ between $0$ and $2\pi$ that satisfies the equation $\sin^{4}\theta -2\sin^{2}\theta-1=0$. 
\hfill{\brak{1984 - 1 Mark}}

\end{enumerate}
