
\begin{enumerate}[label=\thesubsection.\arabic*,ref=\thesubsection.\theenumi]
    \item In a $\Delta ABC$, $\angle A=90\degree$ and $AD$ is an altitude. Complete the relation\\
    $\frac{BD}{BA} = \frac{AB}{(\dots)}$
    \hfill (1980)
    
    \item $ABC$ is a triangle, $P$ is a point on $AB$, and $Q$ is point on $AC$ such that $\angle AQP = \angle ABC$. Complete the relation
    $\frac{area\ of \Delta APQ}{area\ of \Delta ABC} =\frac{(\dots)}{AC^2}$
    \hfill (1980)
    
    \item $ABC$ is a triangle with $\angle B $ greater than $\angle C$ 
    $D$ and $E$ are the points on $BC$ such that $AD$ is perpendicular to $BC$ and $AE$ is the bisector of angle $A$ .Complete the relation
    $\angle DAE = \frac{1}{2} [( ) - \angle C]$
    \hfill (1980)
    \item the set of all real numbers $a$ such that $a^2 + 2a, 2a + 3$ and $a^2 + 3a + 8$ are the sides of a triangle is \dots
    \hfill (1985 - 2 Marks)
    \item In a triangle $ABC$, if $\cot A$,$\cot B$,$\cot C$ are in A.P. ,then $a^2$,$b^2$,$c^2$,are in \dots progression \hfill (1985 - 2 Marks)
    \item A polygon of nine sides, each of length $2$, is inscribed in a circle. The radius of the circle is \dots \hfill (1987 - 2 Marks) 
    \item If the angles of a triangle are $30\degree$ and $45\degree$ and the included side is $(\sqrt{3} + 1) cms$, then the area of the triangle is \dots \hfill (1988 - 2 Marks)
    \item If the triangle $ABC$, $\frac{2\cos A}{a} + \frac{2\cos B}{b} + \frac{2\cos C}{c} = \frac{a}{bc} +  \frac{b}{ac}$, then the value of the angle $A$ is \dots degrees. \hfil (1993 - 2 Marks)
    \item In the triangle $ABC$, $AD$ is the altitude from $A$. Given $b>c$, $\angle C=23 \degree$ and $AD = \frac{abc}{b^2 - c^2}$ then $\angle B = $ \dots \hfill (1994 - 2 Marks)
    \item A circle is inscribed in a equilateral triangle of a side $a$. The area of any square inscribed in this circle is \dots \hfill (1994 - 2 Marks)  
    \item In a triangle $ABC$, $a:b:c = 4:5:6$. The ratio of the radius of the circumstances to that of the incircle is \dots \hfill (1996 - 1 Marks) 
        \item The sum of the radii of inscribed and circumscribed circles for an $n$ sided regular polygon of side $a$, is \hfill\brak{2003}
\begin{multicols}{4}
\begin{enumerate}
        \item $\frac{a}{4} \cot{\brak{\frac{\pi}{2n}}}$         
        \item $ a \cot{\brak{\frac{\pi}{n}}}$ 
        \item $\frac{a}{2} \cot{\brak{\frac{\pi}{2n}}}$ 
        \item $ a \cot{\brak{\frac{\pi}{2n}}}$
\end{enumerate}
\end{multicols}

\item In a triangle $\Delta{ABC}$, medians AD and BE are drawn. If AD$=4,\angle{DAB}=\frac{\pi}{6} \text{ and } \angle{ABE}=\frac{\pi}{3},$ then the area of the $\Delta{ABC}$ is \hfill\brak{2003}
\begin{multicols}{4}
\begin{enumerate}
        \item $\frac{64}{3}$                    
        \item $\frac{8}{3}$ 
        \item $\frac{16}{3}$ 
        \item $\frac{32}{3\sqrt{3}}$
\end{enumerate}
\end{multicols} 

\item If in $\Delta{ABC}\;a\cos^2\brak{\frac{\vec{C}}{2}} + c\cos^2\brak{\frac{\vec{A}}{2}} = \frac{3b}{2},$ then the sides $a,b\text{ and }c$ \hfill\brak{2003}
\begin{multicols}{4}
\begin{enumerate}
        \item satisfy $a+b=c$                    
        \item are in A.P. 
        \item are in G.P. 
        \item are in H.P.
\end{enumerate}
\end{multicols} 

\item The sides of a triangle are $\sin\alpha,\cos\alpha \text{ ad } \sqrt{1+\sin\alpha\cos\alpha}$ for some $0<\alpha<\frac{\pi}{2}.$ Then the greatest angle of the triangle is \hfill\brak{2004}
\begin{multicols}{4}
\begin{enumerate}
        \item $150\degree$                    
        \item $90\degree$ 
        \item $120\degree$
        \item $60\degree$
\end{enumerate}
\end{multicols} 

\item In a triangle $\Delta{ABC}$, let $\angle{C}=\frac{\pi}{2}.$ If $r$ is the inradius and $R$ is the circumradius of the triangle $\Delta{ABC},$ then $2\brak{R+r}$ equals \hfill\brak{2005}
\begin{multicols}{4}
\begin{enumerate}
        \item $b+c$                    
        \item $a+b$ 
        \item $a+b+c$ 
        \item $c+a$
\end{enumerate}
\end{multicols} 

\item If in a $\Delta ABC,$ let the altitudes from the vertices $\vec{A,B,C}$ on opposite sides are in H.P., then $\sin \vec{A},\sin \vec{B},\sin \vec{C}$ are in \hfill\brak{2005}
\begin{multicols}{4}
\begin{enumerate}
        \item $G.P.$                    
        \item $A.P.$ 
        \item $A.P.-G.P.$ 
        \item $H.P.$
\end{enumerate}
\end{multicols} 


\item For a regular polygon, let $r$ and $R$ be the radii of the inscribed and the circumscribed circles. A false statement among the following is \hfill\brak{2010}
\begin{enumerate}
        \item There is a regular polygon with $\frac{r}{R}=\frac{1}{\sqrt{2}}$                    
        \item There is a regular polygon with $\frac{r}{R}=\frac{2}{3}$ 
        \item There is a regular polygon with $\frac{r}{R}=\frac{\sqrt{3}}{2}$ 
        \item There is a regular polygon with $\frac{r}{R}=\frac{1}{2}$
\end{enumerate}

    \item There exists a triangle $ABC$ satisfying the conditions
    \hfill{(1986 - 2 mark)}
    \begin{enumerate}
    	\item $b\sin{A} = a, A <\pi/2$
    	\item $b\sin{A} > a, A >\pi/2$
    	\item $b\sin{A} > a, A <\pi/2$
    	\item $b\sin{A} < a, A <\pi/2$, $b > a$
    	\item $b\sin{A} < a, A >\pi/2$, $b = a$
    \end{enumerate}
    \item In a triangle, the lengths of two larger sides are $10$ and $9$ respectively. If the angles are in AP, Then length of third side is
    \hfill{(1987 - 2 mark)}
    \begin{multicols}{2}
    	\begin{enumerate}
    		\item $5-\sqrt{6}$ 
    		\item $3\sqrt{3}$
    		\item $3$
    		\item $5+\sqrt{6}$ 
    		\item none
    	\end{enumerate}
    \end{multicols}
    \item If in a triangle $PQR$, $\sin{P}, \sin{Q}, \sin{R}$ are in AP, then
    \hfill{(1998 - 2 mark)}
    \begin{enumerate}
    	\item The altitudes are in AP
    	\item The altitudes are in HP
    	\item The medians are in GP
    	\item The medians are in AP
    \end{enumerate}
    \item Let $A_{0}A_{1}A_{2}A_{3}A_{4}A_{5}$ be a regular hexagon inscribed in a circle of unit radius. Then the product of the lengths of the line segments $A_{0}A_{1}$,$A_{0}A_{2}$ and $A_{0}A_{4}$ is 
    \hfill{(1998 - 2 mark)}
    \begin{multicols}{2}
    	\begin{enumerate}
    		\item ${\frac{3}{4}}$
    		\item $3\sqrt{3}$
    		\item $3$
    		\item ${\frac{3\sqrt{3}}{2}}$
    	\end{enumerate}
    \end{multicols}
    \item In $\Delta ABC$, internal angle bisector of $\angle A$ meets side $BC$ in $\vec{D}$. $DE \perp AD$ meets $AC$ in $\vec{E}$ and $AB$ in $\vec{F}$. Then
    \hfill{(2006-5M,-1)}
    \begin{enumerate}
    	\item $AE$ is HM of $b \text{ and } c$
    	\item $AD$ = ${\frac{2bc}{b+c}}cos{\frac{A}{2}}$
    	\item $EF$ = ${\frac{4bc}{b+c}}sin{\frac{A}{2}}$
    	\item $\Delta AEF$ is isosceles
    \end{enumerate}
    \item Let $ABC$ be a triangle such that $\angle ACB = \pi/6$ and let $a,b \text{ and } c$ denote lengths of the sides opposite to $\vec{A}$,$\vec{B} \text{ and } \vec{C}$ respectively. The value(s) of $x$ for which $a = x^{2}+x+1, b = x^{2}-1, c = 2x+1$ is(are)
    \hfill{(2010)}
    \begin{multicols}{2}
    	\begin{enumerate}
    		\item $-(2+\sqrt{3})$
    		\item $1+\sqrt{3}$
    		\item $2+\sqrt{3}$
    		\item $4\sqrt{3}$
    	\end{enumerate}
    \end{multicols}
    \item In a triangle $PQR$, $\vec{P}$ is the largest angle and $\cos{P} = \frac{1}{3}$. Further the incircle of the triangle touches the sides $PQ,QR \text{ and } RP$ at $\vec{N},\vec{L} \text{ and } \vec{M}$ respectively, such that the lengths of $PN, QL \text{ and } RM$ are consecutive even integers. Then possible length(s) of the side(s) of the triangle is(are)
    \hfill{(Jee Adv. 2013)}
    \begin{multicols}{2}
    	\begin{enumerate}
    		\item $16$
    		\item $24$
    		\item $18$
    		\item $22$
    	\end{enumerate}
    \end{multicols}
    \item In a triangle $XYZ$, let $x,y,z$ be the lengths of sides opposite to angles $\vec{X},\vec{Y},\vec{Z} \text{ and } 2s = x+y+z$. If ${\frac{s-x}{4}}={\frac{s-y}{3}}={\frac{s-z}{2}}$ and area of the incircle of the triangle $XYZ$ is ${\frac{8\pi}{3}}$
    \hfill{(Jee Adv. 2016)}
    \begin{enumerate}
    	\item area of the triangle is $6\sqrt{6}$
    	\item the radius of circumcirle of $XYZ$ is ${\frac{35\sqrt{6}}{6}}$
    	\item $sin\frac{X}{2}sin\frac{Y}{2}sin\frac{Z}{2} = \frac{4}{35}$
    	\item $sin^{2}\brak{\frac{X+Y}{2}}$ = $\frac{3}{5}$
    \end{enumerate}
    \item In a triangle $PQR$, let $\angle PQR = 30\degree$ and the sides $PQ \text{ and } QR$ have lengths $10\sqrt{3}$ and $10$ respectively. Then which of the following statements is(are) TRUE?
    \hfill{(Jee Adv. 2018)}
    \begin{enumerate}
    	\item $\angle QPR = 45\degree$
    	\item the area of the triangle $PQR$ is $25\sqrt{3}$ and $\angle QRP = 120\degree$
    	\item the radius of the incircle of triangle $PQR$ is $10\sqrt{3}-15$
    	\item the radius of circumcirle $PQR$ is $100\pi$
    \end{enumerate}
    \item In a non-right-angle triangle $\Delta PQR$, let $p,q,r$ denote the lengths of the sides opposite to the angles at $\vec{P},\vec{Q},\vec{R}$ respectively. The median from $\vec{R}$ meets the side $PQ$ at $\vec{S}$, the perpendicular from $\vec{P}$ meets the side $QR$ at $\vec{E}$, $RS \text{ and } PE$ intersect at $\vec{O}$. If $p = \sqrt{3}$, $q = 1$ and the radius of the circumcircle at $\Delta PQR$ equals $1$, then which of the following options is(are)\\ correct.
    \hfill{(Jee Adv. 2018)}
    \begin{enumerate}
    	\item Radius of incircle of $\Delta PQR$ = $\frac{\sqrt{3}}{2}\brak{2-\sqrt{3}}$
    	\item Area of $\Delta SOE = \frac{\sqrt{3}}{12}$
    	\item Length of $OE = \frac{1}{6}$
    	\item Length of $RS = \frac{\sqrt{7}}{2}$
    \end{enumerate}
    \item If the bisector of the angle $P$ of a triangle $PQR$ meets $QR$ in $S$, then
\begin{multicols}{4}
    \begin{enumerate}
        \item $QS = SR$
        \item $QS : SR = PR : PQ$
        \item $QS : SR = PQ : PR$
        \item None of these \hfill (1979)
    \end{enumerate}
\end{multicols}
    \item In the triangle $ABC$, angle $A$ is the greater than angle $B$. If the measures of the angles $A$ and $B$ satisfies the equation $3\sin x - 4 \sin^3 x - k = 0, 0<k<1$ , then the measure of the angle $C$ is 
	    \begin{multicols}{2}
	    \begin{enumerate}
     \item $\frac{\pi}{3}$
     \item $\frac{\pi}{2}$
     \item $\frac{2\pi}{3}$
     \item $\frac{5\pi}{6}$ 
\end{enumerate}
	    \end{multicols}
		\hfill (1985 - 2 Marks)
    \item If the lengths of the sides of triangles are $3,5,7$ then the largest angles of the triangle is
	    \begin{multicols}{2}
	    \begin{enumerate}
     \item $\frac{\pi}{2}$
     \item $\frac{5\pi}{6}$
     \item $\frac{2\pi}{3}$
     \item $\frac{3\pi}{4}$ 
	    \end{enumerate}
	    \end{multicols}
		\hfill (1986 - 2 Marks)
\item In a triangle $ABC$, $\angle B = \frac{\pi}{3}$ and $\angle C = \frac{\pi}{4}$. Let $\vec{D}$ divide $BC$ internally in the ratio $1\colon3$ then $\frac{\sin\angle BAD}{\sin \angle CAD}$ is equal to
\begin{enumerate}
\item $\frac{1}{\sqrt6}$
\item $\frac{1}{3}$
\item $\frac{1}{\sqrt3}$
\item $\sqrt{\frac{2}{3}}$
\end{enumerate}
\hfill (1995S)

\item In a triangle $ABC$, $2ac\sin\frac{1}{2}\brak{A-B+C} = $
\begin{enumerate}
\item $a^2 + b^2 - c^2$
\item $c^2 + a^2 - b^2$
\item $b^2 - c^2 - a^2$
\item $c^2 - a^2 - b^2$
\end{enumerate}
\hfill (2000S)

\item In a triangle $ABC$, let $\angle C = \frac{\pi}{2}$. If $\vec{r}$ is the inradius and $\vec{R}$ is the circumradius of the triangle, then $2\brak{r+R}$ is equal to
\begin{enumerate}
\item $a+b$
\item $b+c$
\item $c+a$
\item $a+b+c$
\end{enumerate}
\hfill (2000S)


\item Which of the following pieces of data does NOT uniquely determine an acute-angled triangle $\triangle ABC$ ($\vec{R}$ being the radius of the circumcircle)?
\begin{enumerate}
\item $a, \sin A, \sin B$
\item $a, b, c$
\item $a, \sin B, R$
\item $a, \sin A, R$
\end{enumerate}
\hfill (2002S)

\item If the angles of a triangle are in the ratio $4\colon1\colon1$, then the ratio of the longest side to the perimeter is
\begin{enumerate}
\item $\sqrt{3}\colon2+\sqrt{3}$
\item $1\colon6$
\item $1\colon2+\sqrt{3}$
\item $2\colon3$
\end{enumerate}
\hfill (2003S)

\item The sides of a triangle are in the ratio $1\colon\sqrt{3}\colon2$, then the angles of the triangle are in the ratio
\begin{enumerate}
\item $1\colon3\colon5$
\item $2\colon3\colon4$
\item $3\colon2\colon1$
\item $1\colon2\colon3$
\end{enumerate}
\hfill (2004S)

\item In an equilateral triangle, $3$ coins of radii $1$ unit each are kept so they touch each other and also the sides of the triangle. Area of the triangle is 
\begin{figure}[htp]
    \centering
    \includegraphics[width=2.5cm]{figs/figure.png}
    \label{fig:figure}
\end{figure}
\begin{enumerate}
\item $4+2\sqrt{3}$
\item $6+4\sqrt{3}$
\item $12+\frac{7\sqrt{3}}{4}$
\item $3+\frac{7\sqrt{3}}{4}$
\end{enumerate}
\hfill (2005S)

\item In a triangle $ABC$, $a, b, c$  are the lengths of its sides and $A, B, C$ are the angles of triangle $ABC$. The correct relation is given by
\begin{enumerate}
\item $\brak{b-c} \sin \brak{\frac{B-C}{2}} = a \cos \brak{\frac{A}{2}}$
\item $\brak{b-c} \cos \brak{\frac{A}{2}} = a \sin \brak{\frac{B-C}{2}}$
\item $\brak{b-c} \sin \brak{\frac{B+C}{2}} = a \cos \brak{\frac{A}{2}}$
\item $\brak{b-c} \cos \brak{\frac{A}{2}} = a \sin \brak{\frac{B+C}{2}}$
\end{enumerate}
\hfill (2005S)

\item One angle of an isosceles $\triangle$ is $120\degree$ and radius of its incircle $= \sqrt{3}$. Then the area of the triangle in sq. units is 
\begin{enumerate}
\item $7+12\sqrt{3}$
\item $12-7\sqrt{3}$
\item $12+7\sqrt{3}$
\item $4\pi$
\end{enumerate}
\hfill (2006 - 3M, -1)

\item Let $ABCD$ be a quadrilateral with area $18$, with side $AB$ parallel to the side $CD$ and $2AB = CD$. Let $AD$ be perpendicular to $AB$ and $CD$. If a circle is drawn inside the quadrilateral $ABCD$ touching all the sides, then the radius is
\begin{enumerate}
\item $3$
\item $2$
\item $\frac{3}{2}$
\item $1$
\end{enumerate}
\hfill (2007 - 3 Marks)

\item If the angles $A, B$ and $C$ of a triangle are in an arithmetic progression and if $\vec{a}, \vec{b}$ and $\vec{c}$ denote the lengths of the sides opposite to $A, B$ and $C$ respectively, then the value of the expression $\frac{a}{c}\sin 2C + \frac{c}{a} \sin 2A$ is
\begin{enumerate}
\item $\frac{1}{2}$
\item $\frac{\sqrt{3}}{2}$
\item $1$
\item $\sqrt{3}$
\end{enumerate}
\hfill (2010)

\item Let $PQR$ be a triangle of area $\Delta$ with $a=2, b= \frac{7}{2}$ and $c=\frac{5}{2}$, where $\vec{a}, \vec{b}$ and $\vec{c}$ are the lengths of the sides of the triangle opposite to the angles at $P, Q$ and $R$ respectively. Then $\frac{2\sin P - \sin 2P}{2\sin P + \sin 2P}$ equals
\begin{enumerate}
\item $\frac{3}{4\Delta}$
\item $\frac{45}{4\Delta}$
\item $\brak{\frac{3}{4\Delta}}^2$
\item $\brak{\frac{45}{4\Delta}}^2$
\end{enumerate}
\hfill (2012)

\item In a triangle the sum of two sides is $\vec{x}$ and the product of the same sides is $\vec{y}$. If $x^2-c^2=y$, where $\vec{c}$ is the third side of the triangle, then the ratio of the inradius to the circum-radius of the triangle is
\begin{enumerate}
\item $\frac{3y}{2\brak{x+c}}$
\item $\frac{3y}{2c\brak{x+c}}$
\item $\frac{3y}{4x\brak{x+c}}$
\item $\frac{3y}{4c\brak{x+c}}$
\end{enumerate}
\hfill (JEE Adv. 2014)
     \item A triangle $ABC$ has sides $AB=AC=5 cm$ and $BC =6 cm$. Triangle $A'B'C'$ is the reflection of the triangle $ABC$ in a line parallel to $AB$ placed at a distance of $2$ cm from $AB$, outside the triangle $ABC$. Triangle $A"B"C"$ is the reflection of the triangle $A'B'C'$ in a line parallel $B'C'$ placed at a distance of $2 cm$ from $B'C'$ outside the triangle $A'B'C'$. Find the distance between $\vec{A}$ and $\vec{A'}$.\hfill {(1978)}
     \item 
     \begin{enumerate}
     	\item If a circle is inscribed in a right angled triangle $ABC$ right angled at $\vec{B}$, show that the diameter of the circle is equal to $AB+BC-AC$.
     	\item If a triangle is inscribed in a circle, then the product of any two sides of the triangle is equal to the product of the diameter and perpendicular distance of the third side from the opposite vertex. Prove the above statement.
     \end{enumerate}
     \hfill {(1979)}
     \item
     	\item Find the area of the smaller part of a disc of radius $10 cm$, cut off by a chord $AB$ which subtends an angle of $22\frac{1}{2} \degree$ at the circumference.
     \hfill {(1980)}
     \item $ABC$ is a triangle. $\vec{D}$ is the middle point of $BC$. If $AD$ is perpendicular to $AC$, then prove that $\cos{A}\cos{C} = \frac{2(c^{2}-a^{2}}{3ac}$.
     \hfill {(1980)}
     \item $ABC$ is a triangle with $AB=AC$. $\vec{D}$ is any point on the side $BC$. $\vec{E}$ and $\vec{F}$ are points on the side $AB \text{ and } AC$, respectively, such that $DE$ is parallel to $AC, \text{ and } DF$ is parallel to $AB$. Prove that \\
     $DF + FA + AE + ED = AB+AC$
     \hfill {(1980)} 

\item Let the angles $A$, $B$, $C$ of a triangle $ABC$ be in A.P. and let $b:c=\sqrt{3}:\sqrt{2}$. Find the angle $A$. 

\hfill{\brak{1981 - 2 Marks}}

\item The ex-radii $r_1$, $r_2$, $r_3$ of $\Delta ABC$ are in H.P. Show that its sides $a$, $b$, $c$ are in A.P.

\hfill{\brak{1983 - 3 Marks}} 

\item For a triangle $ABC$ it is given that $\cos{A}+\cos{B}+\cos{C}=\frac{3}{2}$. Prove that the triangle is equilateral. 

\hfill{\brak{1984 - 4 Marks}}

\item With usual notation, if in a triangle $ABC$; $\frac{b+c}{11}=\frac{c+a}{12}=\frac{a+b}{13}$ then prove that $\frac{\cos{A}}{7}=\frac{\cos{B}}{19}=\frac{\cos{C}}{25}$. 

\hfill{\brak{1984 - 4 Marks}}


\item In a triangle $ABC$, the median to the side $BC$ is of length $\frac{1}{\sqrt{11-6\sqrt{3}}}$ and it divides the angle $A$ into angles $30\degree$ and $45\degree$. Find the length of the side $BC$.

\hfill{\brak{1985 - 5 Marks}}

\item If in a triangle $ABC$, $\cos{A}\cos{B}+\sin{A}\sin{B}\sin{C}=1$, show that $a:b:c=1:1:\sqrt{2}$.

\hfill{\brak{1986 - 5 Marks}}

\item The sides of a triangle are three consecutive natural numbers and its largest angle is twice the smallest one. Determine the sides of the triangle.

\hfill{\brak{1991 - 4 Marks}}

\item In a triangle of base $a$ the ratio of the other two sides is $r\brak{<1}$. Show that the altitude of the triangle is less than or equal to $\frac{ar}{1-r^2}$.

\hfill{\brak{1991 - 4 Marks}}


\end{enumerate}
