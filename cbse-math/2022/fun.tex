%\documentclass[12pt,-letter paper]{article}
%\usepackage{siunitx}
%\usepackage{setspace}
%\usepackage{gensymb}
%\usepackage{xcolor}
%\usepackage{caption}
%\usepackage{subcaption}
%\doublespacing
%\singlespacing
%\usepackage[none]{hyphenat}
%\usepackage{amssymb}
%\usepackage{relsize}
%\usepackage[cmex10]{amsmath}
%\usepackage{mathtools}
%\usepackage{amsmath}
%\usepackage{commath}
%\usepackage{amsthm}
%\interdisplaylinepenalty=2500
%\savesymbol{iint}
%\usepackage{txfonts}
%\restoresymbol{TXF}{iint}
%\usepackage{wasysym}
%\usepackage{amsthm}
%\usepackage{mathrsfs}
%\usepackage{txfonts}
%\let\vec\mathbf{}
%\usepackage{stfloats}
%\usepackage{float}
%\usepackage{cite}
%\usepackage{cases}
%\usepackage{subfig}
%\usepackage{xtab}
%\usepackage{longtable}
%\usepackage{multirow}
%\usepackage{algorithm}
%\usepackage{amssymb}
%\usepackage{algpseudocode}
%\usepackage{enumitem}
%\usepackage{mathtools}
%\usepackage{eenrc}
%\usepackage[framemethod=tikz]{mdframed}
%\usepackage{listings}
%\usepackage{listings}
%\usepackage[latin1]{inputenc}
%%\usepackage{color}{   
%%\usepackage{lscape}
%\usepackage{textcomp}
%\usepackage{titling}
%\usepackage{hyperref}
%\usepackage{fulbigskip}   
%\usepackage{tikz}
%\usepackage{graphicx}
%\lstsetframe=single,breaklines=t}
%\let\vec\mathbf{}
%\usepackage{enumitem}
%\usepackage{graphicx}
%\usepackage{siunitx}
%\let\vec\mathbf{}
%\usepackage{enumitem}
%\usepackage{graphicx}
%\usepackage{enumitem}
%\usepackage{tfrupee}
%\usepackage{amsmath}
%\usepackage{amssymb}
%\usepackage{mwe} % for blindtext and example-image-a in example
%\usepackage{wrapfig}
%\graphicspath{{figs/}}
%\providecommand{\mydet}[1]{\ensuremath{\begin{vmatrix}#1\end{vmatrix}}}
%\providecommand{\myvec}[1]{\ensuremath{\begin{bmatrix}#1\end{bmatrix}}}
%\providecommand{\cbrak}[1]{\ensuremath{\left\{#1\right\}}}
%\providecommand{\sbrak}[1]{\ensuremath{{}\left[#1\right]}}
%\providecommand{\brak}[1]{\ensuremath{\left(#1\right)}}


%\begin{document}
\begin{enumerate}
\item Let $R$ be the relation defined in $N$, as
$R = \cbrak{(x, y) : 2x + 3y = 15, x, y \in N}$, then $R$=  \cbrak{\underline{\hspace{1cm}},\underline{\hspace{1cm}}}.
\item If the function $f(x)=\begin{cases}\frac{k\cos{x}}{\pi - 2x}, & \text{if} x \neq \frac{\pi}{2}\\\text{2},&\text{if} x=\frac{\pi}{2}\end{cases}$  is continuous  at $x=\frac{\pi}{2}$, then the value of $k$ is {\underline{\hspace{1cm}}}.
\item Show that the relation $R$ in the set $\mathbb{R}$ of all real numbers,defined as $\mathbb{R}=\cbrak{(a, b) : a \leq b^2}$ is neither reflexive nor symmetric.
\item Find the value of $\tan^{-1}\sbrak{{2\cos}\brak{2 \sin^{-1}\brak{\frac{1}{2}}}}$
\item Let a function $f:\mathbb{R}-\cbrak{\frac{-4}{3}}\ \to \mathbb{R}$ be defined as $f(x)=\frac{4x}{3x+4}$.To show that $f$ is one-one function. Hence, find the inverse of the function $f:\mathbb{R}-\cbrak{\frac{-4}{3}} \ \to$ Range\hspace{0.25em} of $f$.
\item If $f:R\ \to R$ be given by $f(x)=\brak{3-x^3}^{1/3}$, then find $\brak{fof}\brak{x}$.
\item Let $W$ denote the set of words in the English dictionary. Define the relation $R$ by
$R={\brak{x,y}}\in W \times W$  such $x$ and $y$ have at least one letter in common. Show that this relation $R$ is reflexive and symmetric, but not transitive.
\item Find the inverse of the function $f(x)=\brak{\frac{4x}{3x+4}}$


\end{enumerate}

%\end{document}
